\documentclass[a4j,onecolumn]{tarticle}
\usepackage[dvipdfmx]{graphicx}
%\usepackage{otf}

\makeatletter
\renewcommand{\normalsize}{\@setfontsize\normalsize{12pt}{21.0}}
\renewcommand{\tiny}{\@setfontsize\tiny{6.0pt}{10.5}}
\renewcommand{\huge}{\@setfontsize\huge{24.0pt}{42.0}}
\makeatother

\normalsize
\voffset=-1in
\hoffset=-1in
\paperwidth=558.72pt
\paperheight=419.04pt
\textwidth=392.584128pt
\textheight=516.709992pt
\topmargin=13.227936pt
\oddsidemargin=21.0050039999999pt
\columnsep=24pt
\headheight=0mm
\headsep=0mm
\footskip=1000mm   % don't use footer
%\kanjiskip=0zw plus .0625zw minus .01zw
\kanjiskip=0zw plus .0625zw minus .01zw
\xkanjiskip=0.25em plus 0.15em minus 0.06em
\setlength\parindent{1zw}

\normalsize
\usepackage{furikana}
\usepackage{burasage}


\begin{document}

{\huge \noindent{}こころ} \\
\hfill 夏目漱石 \\

\kana{私}{わたくし}はその人を常に先生と呼んでいた。\hbox{}だからここでもただ先生と書くだけで本名は打ち明けない。これは世間を\kana{憚}{はば}かる遠慮というよりも、\hbox{}その方が私にとって自然だからである。\hbox{}私はその人の記憶を呼び起すごとに、\hbox{}すぐ「先生」といいたくなる。\hbox{}筆を\kana{執}{と}っても心持は同じ事である。\hbox{}よそよそしい\Kana[3]{頭,文,字}{かしら,も,じ}などはとても使う気にならない。\hbox{}

私が先生と知り合いになったのは\Kana{鎌,倉}{かま,くら}である。その時私はまだ若々しい書生であった。\hbox{}暑中休暇を利用して海水浴に行った友達からぜひ来いという\Kana{端,書}{は,がき}を受け取ったので、私は多少の金を\Kana{工,面}{く,めん}して、出掛ける事にした。私は金の工面に\kana{二}{に}、\kana{三日}{さんち}を費やした。ところが私が鎌倉に着いて三日と\kana{経}{た}たないうちに、私を呼び寄せた友達は、急に国元から帰れという電報を受け取った。電報には母が病気だからと断ってあったけれども友達はそれを信じなかった。友達はかねてから国元にいる親たちに\kana{勧}{すす}まない結婚を\kana{強}{し}いられていた。彼は現代の習慣からいうと結婚するにはあまり年が若過ぎた。それに\Kana{肝,心}{かん,じん}の当人が気に入らなかった。それで夏休みに当然帰るべきところを、わざと避けて東京の近くで遊んでいたのである。彼は電報を私に見せてどうしようと相談をした。私にはどうしていいか分らなかった。けれども実際彼の母が病気であるとすれば彼は\kana{固}{もと}より帰るべきはずであった。それで彼はとうとう帰る事になった。せっかく来た私は一人取り残された。

学校の授業が始まるにはまだ\Kana{大,分}{だい,ぶ}\Kana{日,数}{ひ,かず}があるので鎌倉におってもよし、帰ってもよいという境遇にいた私は、当分元の宿に\kana{留}{と}まる覚悟をした。友達は中国のある資産家の\Kana{息,子}{むす,こ}で金に不自由のない男であったけれども、学校が学校なのと年が年なので、生活の程度は私とそう変りもしなかった。したがって\kana{一人}{ひとり}ぼっちになった私は別に\Kana{恰好}{かっこう}な宿を探す面倒ももたなかったのである。

宿は鎌倉でも\Kana{辺,鄙}{へん,ぴ}な方角にあった。\Kana{玉,突}{たま,つ}きだのアイスクリームだのというハイカラなものには長い\kana{畷}{なわて}を一つ越さなければ手が届かなかった。車で行っても二十銭は取られた。けれども個人の別荘はそこここにいくつでも建てられていた。それに海へはごく近いので海水浴をやるには至極便利な地位を占めていた。

\newlength{\hi}
\setlength{\hi}{1zw}
\the\hi%


私は毎日海へはいりに出掛けた。古い\kana{燻}{くす}ぶり返った\Kana{藁,葺}{わら,ぶき}の\kana{間}{あいだ}を通り抜けて\kana{磯}{いそ}へ下りると、この\kana{辺}{へん}にこれほどの都会人種が住んでいるかと思うほど、避暑に来た男や女で砂の上が動いていた。ある時は海の中が\Kana{銭,湯}{せん,とう}のように黒い頭でごちゃごちゃしている事もあった。その中に知った人を一人ももたない私も、こういう\kana{賑}{にぎ}やかな景色の中に\kana{裹}{つつ}まれて、砂の上に\kana{寝}{ね}そべってみたり、\Kana[3]{膝,頭}{ひざ,がしら}を波に打たしてそこいらを\kana{跳}{は}ね\kana{廻}{まわ}るのは愉快であった。

%私は実に先生をこの雑沓(ざっとう)の間(あいだ)に見付け出したのである。その時海岸には掛茶屋(かけぢゃや)が二軒あった。私はふとした機会(はずみ)からその一軒の方に行き慣(な)れていた。長谷辺(はせへん)に大きな別荘を構えている人と違って、各自(めいめい)に専有の着換場(きがえば)を拵(こしら)えていないここいらの避暑客には、ぜひともこうした共同着換所といった風(ふう)なものが必要なのであった。彼らはここで茶を飲み、ここで休息する外(ほか)に、ここで海水着を洗濯させたり、ここで鹹(しお)はゆい身体(からだ)を清めたり、ここへ帽子や傘(かさ)を預けたりするのである。海水着を持たない私にも持物を盗まれる恐れはあったので、私は海へはいるたびにその茶屋へ一切(いっさい)を脱(ぬ)ぎ棄(す)てる事にしていた。

二

%私(わたくし)がその掛茶屋で先生を見た時は、先生がちょうど着物を脱いでこれから海へ入ろうとするところであった。私はその時反対に濡(ぬ)れた身体(からだ)を風に吹かして水から上がって来た。二人の間(あいだ)には目を遮(さえぎ)る幾多の黒い頭が動いていた。特別の事情のない限り、私はついに先生を見逃したかも知れなかった。それほど浜辺が混雑し、それほど私の頭が放漫(ほうまん)であったにもかかわらず、私がすぐ先生を見付け出したのは、先生が一人の西洋人を伴(つ)れていたからである。

%その西洋人の優れて白い皮膚の色が、掛茶屋へ入るや否(いな)や、すぐ私の注意を惹(ひ)いた。純粋の日本の浴衣(ゆかた)を着ていた彼は、それを床几(しょうぎ)の上にすぽりと放(ほう)り出したまま、腕組みをして海の方を向いて立っていた。彼は我々の穿(は)く猿股(さるまた)一つの外(ほか)何物も肌に着けていなかった。私にはそれが第一不思議だった。私はその二日前に由井(ゆい)が浜(はま)まで行って、砂の上にしゃがみながら、長い間西洋人の海へ入る様子を眺(なが)めていた。私の尻(しり)をおろした所は少し小高い丘の上で、そのすぐ傍(わき)がホテルの裏口になっていたので、私の凝(じっ)としている間(あいだ)に、大分(だいぶ)多くの男が塩を浴びに出て来たが、いずれも胴と腕と股(もも)は出していなかった。女は殊更(ことさら)肉を隠しがちであった。大抵は頭に護謨製(ゴムせい)の頭巾(ずきん)を被(かぶ)って、海老茶(えびちゃ)や紺(こん)や藍(あい)の色を波間に浮かしていた。そういう有様を目撃したばかりの私の眼(め)には、猿股一つで済まして皆(みん)なの前に立っているこの西洋人がいかにも珍しく見えた。




\end{document}
