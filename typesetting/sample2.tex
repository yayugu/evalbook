\documentclass[a4j,twocolumn]{tarticle}
\usepackage[dvipdfmx]{graphicx}
\usepackage{wallpaper}
\usepackage{multicol}
\usepackage{otf}

\makeatletter
\renewcommand{\normalsize}{\@setfontsize\normalsize{12pt}{21.0}}
\renewcommand{\tiny}{\@setfontsize\tiny{6.0pt}{10.5}}
\renewcommand{\huge}{\@setfontsize\huge{24.0pt}{42.0}}
\renewcommand{\footnotesize}{\@setfontsize\footnotesize{10pt}{10}}

\def\ps@myhead{%
  \let\@evenfoot\@empty
  \let\@oddfoot\@empty
  \def\@evenhead{% Even Head
  }%
  \def\@oddhead{% Odd Head
  \kern 20.200499029601pt \footnotesize こころ \hfill
  }%
  \let\@mkboth\markboth
}

\makeatother

\normalsize
\voffset=-1in
\hoffset=-1in
\paperwidth=424.325182059202pt
\paperheight=554.716774462811pt
\textwidth=519.59664pt
\textheight=383.924184pt
\topmargin=17.5600672314055pt
\oddsidemargin=20.200499029601pt
\columnsep=24pt
\headheight=5mm
\headsep=5mm
\footskip=1000mm   % don't use footer
%\kanjiskip=0zw plus .0625zw minus .01zw
\kanjiskip=0zw plus .0625zw minus .01zw
\xkanjiskip=0.25em plus 0.15em minus 0.06em
\setlength\parindent{1zw}

\normalsize
\usepackage{furikana}
\usepackage{burasage}
\AtBeginDvi{\special{pdf:docview <</ViewerPreferences <</Direction /R2L>> >>}}

\begin{document}
\pagestyle{myhead}

{\huge こころ} \\
\hfill 夏目漱石\\

\rensuji{20}世紀少年

\kana[2]{凝}{ぎよう}\kana[1]{視}{し}する

\kana[1]{私}{わたくし}はその人を常に先生と呼んでいた\。だからここでもただ先生と書くだけで本名は打ち明けない\。これは世間を\kana[1]{憚}{はば}かる遠慮というよりも\、その方が私にとって自然だからである\。私はその人の記憶を呼び起すごとに\、すぐ「先生」といいたくなる\。筆を\kana[1]{執}{と}っても心持は同じ事である\。よそよそしい\kana[3]{頭}{かしら}\kana[1]{文}{も}\kana[1]{字}{じ}などはとても使う気にならない\。

私が先生と知り合いになったのは\kana[1]{鎌}{かま}\kana[1]{倉}{くら}である\。その時私はまだ若々しい書生であった\。暑中休暇を利用して海水浴に行った友達からぜひ来いという\kana[1]{端}{は}\kana[1]{書}{がき}を受け取ったので\、私は多少の金を\kana[1]{工}{く}\kana[1]{面}{めん}して\、出掛ける事にした\。私は金の工面に\kana[1]{二}{に}\、三\kana[1]{日}{さんち}を費やした\。ところが私が鎌倉に着いて三日と\kana[1]{経}{た}たないうちに\、私を呼び寄せた友達は\、急に国元から帰れという電報を受け取った\。電報には母が病気だからと断ってあったけれども友達はそれを信じなかった\。友達はかねてから国元にいる親たちに\kana[1]{勧}{すす}まない結婚を\kana[1]{強}{し}いられていた\。彼は現代の習慣からいうと結婚するにはあまり年が若過ぎた\。それに\kana[1]{肝}{かん}\kana[1]{心}{じん}の当人が気に入らなかった\。それで夏休みに当然帰るべきところを\、わざと避けて東京の近くで遊んでいたのである\。彼は電報を私に見せてどうしようと相談をした\。私にはどうしていいか分らなかった\。けれども実際彼の母が病気であるとすれば彼は\kana[1]{固}{もと}より帰るべきはずであった\。それで彼はとうとう帰る事になった\。せっかく来た私は一人取り残された\。


         \hfil%
         \includegraphics[width=6cm,keepaspectratio,angle=90]{1.pdf}%
         \hfill
        

学校の授業が始まるにはまだ\kana[1]{大}{だい}\kana[1]{分}{ぶ}\kana[1]{日}{ひ}\kana[1]{数}{かず}があるので鎌倉におってもよし\、帰ってもよいという境遇にいた私は\、当分元の宿に\kana[1]{留}{と}まる覚悟をした\。友達は中国のある資産家の\kana[1]{息}{むす}\kana[1]{子}{こ}で金に不自由のない男であったけれども\、学校が学校なのと年が年なので\、生活の程度は私とそう変りもしなかった\。したがって\kana[1]{一人}{ひとり}ぼっちになった私は別に\kana[1]{恰好}{かっこう}な宿を探す面倒ももたなかったのである\。


         \noindent\includegraphics[width=\textheight,%
                                     height=247.79832pt,%
                                     keepaspectratio, angle=90]{1.pdf}%
        


宿は鎌倉でも\kana[1]{辺}{へん}\kana[1]{鄙}{ぴ}な方角にあった\。\kana[1]{玉}{たま}\kana[1]{突}{つ}きだのアイスクリームだのというハイカラなものには長い\kana[1]{畷}{なわて}を一つ越さなければ手が届かなかった\。車で行っても二十銭は取られた\。けれども個人の別荘はそこここにいくつでも建てられていた\。それに海へはごく近いので海水浴をやるには至極便利な地位を占めていた\。


\end{document}

