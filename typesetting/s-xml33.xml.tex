\documentclass[a4j,onecolumn]{tarticle}
\usepackage[dvipdfmx]{graphicx}
\usepackage[deluxe, burasage]{otf}

\makeatletter
\DeclareFontShape{JT1}{hmc}{m}{n}{<-> s * [0.9375] brsgnmlminr-v}{}
\renewcommand{\normalsize}{\@setfontsize\normalsize{12pt}{21.0}}
\renewcommand{\tiny}{\@setfontsize\tiny{6.0pt}{10.5}}
\renewcommand{\huge}{\@setfontsize\huge{33.0pt}{21.0}}
\makeatother

\normalsize
\voffset=-1in
\hoffset=-1in
\paperwidth=143.908251026047pt
\paperheight=206.868110849943pt
\textwidth=191.25pt
\textheight=126.0pt
\topmargin=7.80905542497138pt
\oddsidemargin=2.95412551302357pt
\columnsep=24pt
\headheight=0mm
\headsep=0mm
\topskip=0mm
\footskip=1000mm   % don't use footer
%\kanjiskip=0zw plus .1zw minus .01zw
\kanjiskip=0zw plus .5zw minus .01zw
\xkanjiskip=0.25em plus 0.15em minus 0.06em
\setlength\parindent{0.0zw}

\prebreakpenalty`っ=0

\normalsize
\usepackage{furikana}
\usepackage{burasage}
\AtBeginDvi{\special{pdf:docview <</ViewerPreferences <</Direction /R2L>> >>}}

\begin{document}






\raisebox{0pt}[6.0pt][6.0pt]{\Huge\mcfamily\bfseries こ}
\raisebox{0pt}[6.0pt][6.0pt]{\Huge\mcfamily\bfseries こ}
\raisebox{0pt}[6.0pt][6.0pt]{\Huge\mcfamily\bfseries ろ}



\hfill 夏目漱石







\par{}\par{}



\par{}  上 先生と私
\par{}\par{}     一
\par{}
{
  \leftskip=5zw
 \kana{私}{わたくし}はその人を常に先生と呼んでいた。\hbox{}だからここでもただ先生と書くだけで本名は打ち明けない。\hbox{}これは世間を\kana{憚}{はば}かる遠慮というよりも、\hbox{}その方が私にとって自然だからである。\hbox{}私はその人の記憶を呼び起すごとに、\hbox{}すぐ「先生」といいたくなる。\hbox{}筆を\kana{執}{と}っても心持は同じ事である。\hbox{}よそよそしい\kana{頭文字}{かしらもじ}などはとても使う気にならない。\hbox{}\par{}
 私が先生と知り合いになったのは\kana{鎌倉}{かまくら}である。\hbox{}その時私はまだ若々しい書生であった。\hbox{}暑中休暇を利用して海水浴に行った友達からぜひ来いという\kana{端書}{はがき}を受け取ったので、\hbox{}私は多少の金を\kana{工面}{くめん}して、\hbox{}出掛ける事にした。\hbox{}私は金の工面に\kana{二}{に}、\hbox{}\kana{三日}{さんち}を費やした。\hbox{}ところが私が鎌倉に着いて三日と\kana{経}{た}たないうちに、\hbox{}私を呼び寄せた友達は、\hbox{}急に国元から帰れという電報を受け取った。\hbox{}電報には母が病気だからと断ってあったけれども友達はそれを信じなかった。\hbox{}友達はかねてから国元にいる親たちに\kana{勧}{すす}まない結婚を\kana{強}{し}いられていた。\hbox{}彼は現代の習慣からいうと結婚するにはあまり年が若過ぎた。\hbox{}それに\kana{肝心}{かんじん}の当人が気に入らなかった。\hbox{}それで夏休みに当然帰るべきところを、\hbox{}わざと避けて東京の近くで遊んでいたのである。\hbox{}彼は電報を私に見せてどうしようと相談をした。\hbox{}私にはどうしていいか分らなかった。\hbox{}けれども実際彼の母が病気であるとすれば彼は\kana{固}{もと}より帰るべきはずであった。\hbox{}それで彼はとうとう帰る事になった。\hbox{}せっかく来た私は一人取り残された。\hbox{}\par{}
}
 学校の授業が始まるにはまだ\kana{大分}{だいぶ}\kana{日数}{ひかず}があるので鎌倉におってもよし、\hbox{}帰ってもよいという境遇にいた私は、\hbox{}当分元の宿に\kana{留}{と}まる覚悟をした。\hbox{}友達は中国のある資産家の\kana{息子}{むすこ}で金に不自由のない男であったけれども、\hbox{}学校が学校なのと年が年なので、\hbox{}生活の程度は私とそう変りもしなかった。\hbox{}したがって\kana{一人}{ひとり}ぼっちになった私は別に\kana{恰好}{かっこう}な宿を探す面倒ももたなかったのである。\hbox{}\par{}
 宿は鎌倉でも\kana{辺鄙}{へんぴ}な方角にあった。\hbox{}\kana{玉突}{たまつ}きだのアイスクリームだのというハイカラなものには長い\kana{畷}{なわて}を一つ越さなければ手が届かなかった。\hbox{}車で行っても二十銭は取られた。\hbox{}けれども個人の別荘はそこここにいくつでも建てられていた。\hbox{}それに海へはごく近いので海水浴をやるには至極便利な地位を占めていた。\hbox{}\par{}
 私は毎日海へはいりに出掛けた。\hbox{}古い\kana{燻}{くす}ぶり返った\kana{藁葺}{わらぶき}の\kana{間}{あいだ}を通り抜けて\kana{磯}{いそ}へ下りると、\hbox{}この\kana{辺}{へん}にこれほどの都会人種が住んでいるかと思うほど、\hbox{}避暑に来た男や女で砂の上が動いていた。\hbox{}ある時は海の中が\kana{銭湯}{せんとう}のように黒い頭でごちゃごちゃしている事もあった。\hbox{}その中に知った人を一人ももたない私も、\hbox{}こういう\kana{賑}{にぎ}やかな景色の中に\kana{裹}{つつ}まれて、\hbox{}砂の上に\kana{寝}{ね}そべってみたり、\hbox{}\kana{膝頭}{ひざがしら}を波に打たしてそこいらを\kana{跳}{は}ね\kana{廻}{まわ}るのは愉快であった。\hbox{}\par{}
 私は実に先生をこの\kana{雑沓}{ざっとう}の\kana{間}{あいだ}に見付け出したのである。\hbox{}その時海岸には\kana{掛茶屋}{かけぢゃや}が二軒あった。\hbox{}私はふとした\kana{機会}{はずみ}からその一軒の方に行き\kana{慣}{な}れていた。\hbox{}\kana{長谷辺}{はせへん}に大きな別荘を構えている人と違って、\hbox{}\kana{各自}{めいめい}に専有の\kana{着換場}{きがえば}を\kana{拵}{こしら}えていないここいらの避暑客には、\hbox{}ぜひともこうした共同着換所といった\kana{風}{ふう}なものが必要なのであった。\hbox{}彼らはここで茶を飲み、\hbox{}ここで休息する\kana{外}{ほか}に、\hbox{}ここで海水着を洗濯させたり、\hbox{}ここで\kana{鹹}{しお}はゆい\kana{身体}{からだ}を清めたり、\hbox{}ここへ帽子や\kana{傘}{かさ}を預けたりするのである。\hbox{}海水着を持たない私にも持物を盗まれる恐れはあったので、\hbox{}私は海へはいるたびにその茶屋へ\kana{一切}{いっさい}を\kana{脱}{ぬ}ぎ\kana{棄}{す}てる事にしていた。\hbox{}\par{}\par{}     二
\par{}
 \kana{私}{わたくし}がその掛茶屋で先生を見た時は、\hbox{}先生がちょうど着物を脱いでこれから海へ入ろうとするところであった。\hbox{}私はその時反対に\kana{濡}{ぬ}れた\kana{身体}{からだ}を風に吹かして水から上がって来た。\hbox{}二人の\kana{間}{あいだ}には目を\kana{遮}{さえぎ}る幾多の黒い頭が動いていた。\hbox{}特別の事情のない限り、\hbox{}私はついに先生を見逃したかも知れなかった。\hbox{}それほど浜辺が混雑し、\hbox{}それほど私の頭が\kana{放漫}{ほうまん}であったにもかかわらず、\hbox{}私がすぐ先生を見付け出したのは、\hbox{}先生が一人の西洋人を\kana{伴}{つ}れていたからである。\hbox{}\par{}
 その西洋人の優れて白い皮膚の色が、\hbox{}掛茶屋へ入るや\kana{否}{いな}や、\hbox{}すぐ私の注意を\kana{惹}{ひ}いた。\hbox{}純粋の日本の\kana{浴衣}{ゆかた}を着ていた彼は、\hbox{}それを\kana{床几}{しょうぎ}の上にすぽりと\kana{放}{ほう}り出したまま、\hbox{}腕組みをして海の方を向いて立っていた。\hbox{}彼は我々の\kana{穿}{は}く\kana{猿股}{さるまた}一つの\kana{外}{ほか}何物も肌に着けていなかった。\hbox{}私にはそれが第一不思議だった。\hbox{}私はその二日前に\kana{由井}{ゆい}が\kana{浜}{はま}まで行って、\hbox{}砂の上にしゃがみながら、\hbox{}長い間西洋人の海へ入る様子を\kana{眺}{なが}めていた。\hbox{}私の\kana{尻}{しり}をおろした所は少し小高い丘の上で、\hbox{}そのすぐ\kana{傍}{わき}がホテルの裏口になっていたので、\hbox{}私の\kana{凝}{じっ}としている\kana{間}{あいだ}に、\hbox{}\kana{大分}{だいぶ}多くの男が塩を浴びに出て来たが、\hbox{}いずれも胴と腕と\kana{股}{もも}は出していなかった。\hbox{}女は\kana{殊更}{ことさら}肉を隠しがちであった。\hbox{}大抵は頭に\kana{護謨製}{ゴムせい}の\kana{頭巾}{ずきん}を\kana{被}{かぶ}って、\hbox{}\kana{海老茶}{えびちゃ}や\kana{紺}{こん}や\kana{藍}{あい}の色を波間に浮かしていた。\hbox{}そういう有様を目撃したばかりの私の\kana{眼}{め}には、\hbox{}猿股一つで済まして\kana{皆}{みん}なの前に立っているこの西洋人がいかにも珍しく見えた。\hbox{}\par{}
 彼はやがて自分の\kana{傍}{わき}を顧みて、\hbox{}そこにこごんでいる日本人に、\hbox{}\kana{一言}{ひとこと}\kana{二言}{ふたこと}\kana{何}{なに}かいった。\hbox{}その日本人は砂の上に落ちた\kana{手拭}{てぬぐい}を拾い上げているところであったが、\hbox{}それを取り上げるや否や、\hbox{}すぐ頭を包んで、\hbox{}海の方へ歩き出した。\hbox{}その人がすなわち先生であった。\hbox{}\par{}
 私は単に好奇心のために、\hbox{}並んで浜辺を下りて行く二人の\kana{後姿}{うしろすがた}を見守っていた。\hbox{}すると彼らは\kana{真直}{まっすぐ}に波の中に足を踏み込んだ。\hbox{}そうして\kana{遠浅}{とおあさ}の\kana{磯近}{いそちか}くにわいわい騒いでいる\kana{多人数}{たにんず}の\kana{間}{あいだ}を通り抜けて、\hbox{}比較的広々した所へ来ると、\hbox{}二人とも泳ぎ出した。\hbox{}彼らの頭が小さく見えるまで沖の方へ向いて行った。\hbox{}それから引き返してまた一直線に浜辺まで戻って来た。\hbox{}掛茶屋へ帰ると、\hbox{}井戸の水も浴びずに、\hbox{}すぐ\kana{身体}{からだ}を\kana{拭}{ふ}いて着物を着て、\hbox{}さっさとどこへか行ってしまった。\hbox{}\par{}
 彼らの出て行った\kana{後}{あと}、\hbox{}私はやはり元の\kana{床几}{しょうぎ}に腰をおろして\kana{烟草}{タバコ}を吹かしていた。\hbox{}その時私はぽかんとしながら先生の事を考えた。\hbox{}どうもどこかで見た事のある顔のように思われてならなかった。\hbox{}しかしどうしてもいつどこで会った人か\kana{想}{おも}い出せずにしまった。\hbox{}\par{}
 その時の私は\kana{屈托}{くったく}がないというよりむしろ\kana{無聊}{ぶりょう}に苦しんでいた。\hbox{}それで\kana{翌日}{あくるひ}もまた先生に会った時刻を見計らって、\hbox{}わざわざ\kana{掛茶屋}{かけぢゃや}まで出かけてみた。\hbox{}すると西洋人は来ないで先生一人\kana{麦藁帽}{むぎわらぼう}を\kana{被}{かぶ}ってやって来た。\hbox{}先生は\kana{眼鏡}{めがね}をとって台の上に置いて、\hbox{}すぐ\kana{手拭}{てぬぐい}で頭を包んで、\hbox{}すたすた浜を下りて行った。\hbox{}先生が\kana{昨日}{きのう}のように騒がしい\kana{浴客}{よくかく}の中を通り抜けて、\hbox{}一人で泳ぎ出した時、\hbox{}私は急にその\kana{後}{あと}が追い掛けたくなった。\hbox{}私は浅い水を頭の上まで\kana{跳}{はね}かして相当の深さの所まで来て、\hbox{}そこから先生を\kana{目標}{めじるし}に\kana{抜手}{ぬきで}を切った。\hbox{}すると先生は昨日と違って、\hbox{}一種の\kana{弧線}{こせん}を\kana{描}{えが}いて、\hbox{}妙な方向から岸の方へ帰り始めた。\hbox{}それで私の目的はついに達せられなかった。\hbox{}私が\kana{陸}{おか}へ上がって\kana{雫}{しずく}の垂れる手を振りながら掛茶屋に入ると、\hbox{}先生はもうちゃんと着物を着て入れ違いに外へ出て行った。\hbox{}\par{}\par{}     三
\par{}
 \kana{私}{わたくし}は次の日も同じ時刻に浜へ行って先生の顔を見た。\hbox{}その次の日にもまた同じ事を繰り返した。\hbox{}けれども物をいい掛ける機会も、\hbox{}\kana{挨拶}{あいさつ}をする場合も、\hbox{}二人の間には起らなかった。\hbox{}その上先生の態度はむしろ非社交的であった。\hbox{}一定の時刻に超然として来て、\hbox{}また超然と帰って行った。\hbox{}周囲がいくら\kana{賑}{にぎ}やかでも、\hbox{}それにはほとんど注意を払う様子が見えなかった。\hbox{}最初いっしょに来た西洋人はその\kana{後}{ご}まるで姿を見せなかった。\hbox{}先生はいつでも一人であった。\hbox{}\par{}
 \kana{或}{あ}る時先生が例の通りさっさと海から上がって来て、\hbox{}いつもの場所に\kana{脱}{ぬ}ぎ\kana{棄}{す}てた\kana{浴衣}{ゆかた}を着ようとすると、\hbox{}どうした訳か、\hbox{}その浴衣に砂がいっぱい着いていた。\hbox{}先生はそれを落すために、\hbox{}後ろ向きになって、\hbox{}浴衣を二、\hbox{}三度\kana{振}{ふる}った。\hbox{}すると着物の下に置いてあった眼鏡が板の\kana{隙間}{すきま}から下へ落ちた。\hbox{}先生は\kana{白絣}{しろがすり}の上へ\kana{兵児帯}{へこおび}を締めてから、\hbox{}眼鏡の\kana{失}{な}くなったのに気が付いたと見えて、\hbox{}急にそこいらを探し始めた。\hbox{}私はすぐ\kana{腰掛}{こしかけ}の下へ首と手を突ッ込んで眼鏡を拾い出した。\hbox{}先生は有難うといって、\hbox{}それを私の手から受け取った。\hbox{}\par{}
 次の日私は先生の\kana{後}{あと}につづいて海へ飛び込んだ。\hbox{}そうして先生といっしょの方角に泳いで行った。\hbox{}二\kana{丁}{ちょう}ほど沖へ出ると、\hbox{}先生は後ろを振り返って私に話し掛けた。\hbox{}広い\kana{蒼}{あお}い海の表面に浮いているものは、\hbox{}その近所に私ら二人より\kana{外}{ほか}になかった。\hbox{}そうして強い太陽の光が、\hbox{}眼の届く限り水と山とを照らしていた。\hbox{}私は自由と歓喜に\kana{充}{み}ちた筋肉を動かして海の中で\kana{躍}{おど}り狂った。\hbox{}先生はまたぱたりと手足の運動を\kana{已}{や}めて仰向けになったまま\kana{浪}{なみ}の上に寝た。\hbox{}私もその\kana{真似}{まね}をした。\hbox{}青空の色がぎらぎらと眼を射るように痛烈な色を私の顔に投げ付けた。\hbox{}「愉快ですね」と私は大きな声を出した。\hbox{}\par{}
 しばらくして海の中で起き上がるように姿勢を改めた先生は、\hbox{}「もう帰りませんか」といって私を促した。\hbox{}比較的強い体質をもった私は、\hbox{}もっと海の中で遊んでいたかった。\hbox{}しかし先生から誘われた時、\hbox{}私はすぐ「ええ帰りましょう」と快く答えた。\hbox{}そうして二人でまた元の\kana{路}{みち}を浜辺へ引き返した。\hbox{}\par{}
 私はこれから先生と懇意になった。\hbox{}しかし先生がどこにいるかはまだ知らなかった。\hbox{}\par{}
 それから\kana{中}{なか}二日おいてちょうど三日目の午後だったと思う。\hbox{}先生と\kana{掛茶屋}{かけぢゃや}で出会った時、\hbox{}先生は突然私に向かって、\hbox{}「君はまだ\kana{大分}{だいぶ}長くここにいるつもりですか」と聞いた。\hbox{}考えのない私はこういう問いに答えるだけの用意を頭の中に蓄えていなかった。\hbox{}それで「どうだか分りません」と答えた。\hbox{}しかしにやにや笑っている先生の顔を見た時、\hbox{}私は急に\kana{極}{きま}りが悪くなった。\hbox{}「先生は?」と聞き返さずにはいられなかった。\hbox{}これが私の口を出た先生という言葉の始まりである。\hbox{}\par{}
 私はその晩先生の宿を尋ねた。\hbox{}宿といっても普通の旅館と違って、\hbox{}広い寺の\kana{境内}{けいだい}にある別荘のような建物であった。\hbox{}そこに住んでいる人の先生の家族でない事も\kana{解}{わか}った。\hbox{}私が先生先生と呼び掛けるので、\hbox{}先生は苦笑いをした。\hbox{}私はそれが年長者に対する私の\kana{口癖}{くちくせ}だといって弁解した。\hbox{}私はこの間の西洋人の事を聞いてみた。\hbox{}先生は彼の風変りのところや、\hbox{}もう\kana{鎌倉}{かまくら}にいない事や、\hbox{}色々の話をした末、\hbox{}日本人にさえあまり\kana{交際}{つきあい}をもたないのに、\hbox{}そういう外国人と\kana{近付}{ちかづ}きになったのは不思議だといったりした。\hbox{}私は最後に先生に向かって、\hbox{}どこかで先生を見たように思うけれども、\hbox{}どうしても思い出せないといった。\hbox{}若い私はその時\kana{暗}{あん}に相手も私と同じような感じを持っていはしまいかと疑った。\hbox{}そうして腹の中で先生の返事を予期してかかった。\hbox{}ところが先生はしばらく\kana{沈吟}{ちんぎん}したあとで、\hbox{}「どうも君の顔には\kana{見覚}{みおぼ}えがありませんね。\hbox{}人違いじゃないですか」といったので私は変に一種の失望を感じた。\hbox{}\par{}\par{}     四
\par{}
 \kana{私}{わたくし}は月の末に東京へ帰った。\hbox{}先生の避暑地を引き上げたのはそれよりずっと前であった。\hbox{}私は先生と別れる時に、\hbox{}「これから折々お\kana{宅}{たく}へ伺っても\kana{宜}{よ}ござんすか」と聞いた。\hbox{}先生は\kana{単簡}{たんかん}にただ「ええいらっしゃい」といっただけであった。\hbox{}その時分の私は先生とよほど懇意になったつもりでいたので、\hbox{}先生からもう少し\kana{濃}{こまや}かな言葉を予期して\kana{掛}{かか}ったのである。\hbox{}それでこの物足りない返事が少し私の自信を\kana{傷}{いた}めた。\hbox{}\par{}
 私はこういう事でよく先生から失望させられた。\hbox{}先生はそれに気が付いているようでもあり、\hbox{}また全く気が付かないようでもあった。\hbox{}私はまた軽微な失望を繰り返しながら、\hbox{}それがために先生から離れて行く気にはなれなかった。\hbox{}むしろそれとは反対で、\hbox{}不安に\kana{揺}{うご}かされるたびに、\hbox{}もっと前へ進みたくなった。\hbox{}もっと前へ進めば、\hbox{}私の予期するあるものが、\hbox{}いつか眼の前に満足に現われて来るだろうと思った。\hbox{}私は若かった。\hbox{}けれどもすべての人間に対して、\hbox{}若い血がこう素直に働こうとは思わなかった。\hbox{}私はなぜ先生に対してだけこんな心持が起るのか\kana{解}{わか}らなかった。\hbox{}それが先生の亡くなった\kana{今日}{こんにち}になって、\hbox{}始めて解って来た。\hbox{}先生は始めから私を嫌っていたのではなかったのである。\hbox{}先生が私に示した時々の\kana{素気}{そっけ}ない\kana{挨拶}{あいさつ}や冷淡に見える動作は、\hbox{}私を遠ざけようとする不快の表現ではなかったのである。\hbox{}\kana{傷}{いた}ましい先生は、\hbox{}自分に近づこうとする人間に、\hbox{}近づくほどの価値のないものだから\kana{止}{よ}せという警告を与えたのである。\hbox{}\kana{他}{ひと}の懐かしみに応じない先生は、\hbox{}\kana{他}{ひと}を\kana{軽蔑}{けいべつ}する前に、\hbox{}まず自分を軽蔑していたものとみえる。\hbox{}\par{}
 私は無論先生を訪ねるつもりで東京へ帰って来た。\hbox{}帰ってから授業の始まるまでにはまだ二週間の\kana{日数}{ひかず}があるので、\hbox{}そのうちに一度行っておこうと思った。\hbox{}しかし帰って二日三日と\kana{経}{た}つうちに、\hbox{}\kana{鎌倉}{かまくら}にいた時の気分が段々薄くなって来た。\hbox{}そうしてその上に\kana{彩}{いろど}られる大都会の空気が、\hbox{}記憶の復活に伴う強い\kana{刺戟}{しげき}と共に、\hbox{}濃く私の心を染め付けた。\hbox{}私は往来で学生の顔を見るたびに新しい学年に対する希望と緊張とを感じた。\hbox{}私はしばらく先生の事を忘れた。\hbox{}\par{}
 授業が始まって、\hbox{}一カ月ばかりすると私の心に、\hbox{}また一種の\kana{弛}{たる}みができてきた。\hbox{}私は何だか不足な顔をして往来を歩き始めた。\hbox{}物欲しそうに自分の\kana{室}{へや}の中を\kana{見廻}{みまわ}した。\hbox{}私の頭には再び先生の顔が浮いて出た。\hbox{}私はまた先生に会いたくなった。\hbox{}\par{}
 始めて先生の\kana{宅}{うち}を訪ねた時、\hbox{}先生は留守であった。\hbox{}二度目に行ったのは次の日曜だと覚えている。\hbox{}晴れた空が身に\kana{沁}{し}み込むように感ぜられる\kana{好}{い}い\kana{日和}{ひより}であった。\hbox{}その日も先生は留守であった。\hbox{}鎌倉にいた時、\hbox{}私は先生自身の口から、\hbox{}いつでも\kana{大抵}{たいてい}宅にいるという事を聞いた。\hbox{}むしろ外出嫌いだという事も聞いた。\hbox{}二度来て二度とも会えなかった私は、\hbox{}その言葉を思い出して、\hbox{}\kana{理由}{わけ}もない不満をどこかに感じた。\hbox{}私はすぐ玄関先を去らなかった。\hbox{}\kana{下女}{げじょ}の顔を見て少し\kana{躊躇}{ちゅうちょ}してそこに立っていた。\hbox{}この前名刺を取り次いだ記憶のある下女は、\hbox{}私を待たしておいてまた\kana{内}{うち}へはいった。\hbox{}すると奥さんらしい人が代って出て来た。\hbox{}美しい奥さんであった。\hbox{}\par{}
 私はその人から\kana{鄭寧}{ていねい}に先生の出先を教えられた。\hbox{}先生は例月その日になると\kana{雑司ヶ谷}{ぞうしがや}の墓地にある\kana{或}{あ}る仏へ花を\kana{手向}{たむ}けに行く習慣なのだそうである。\hbox{}「たった今出たばかりで、\hbox{}十分になるか、\hbox{}ならないかでございます」と奥さんは気の毒そうにいってくれた。\hbox{}私は\kana{会釈}{えしゃく}して外へ出た。\hbox{}\kana{賑}{にぎや}かな町の方へ一\kana{丁}{ちょう}ほど歩くと、\hbox{}私も散歩がてら雑司ヶ谷へ行ってみる気になった。\hbox{}先生に会えるか会えないかという好奇心も動いた。\hbox{}それですぐ\kana{踵}{きびす}を\kana{回}{めぐ}らした。\hbox{}\par{}\par{}     五
\par{}
 \kana{私}{わたくし}は墓地の手前にある\kana{苗畠}{なえばたけ}の左側からはいって、\hbox{}両方に\kana{楓}{かえで}を植え付けた広い道を奥の方へ進んで行った。\hbox{}するとその\kana{端}{はず}れに見える\kana{茶店}{ちゃみせ}の中から先生らしい人がふいと出て来た。\hbox{}私はその人の\kana{眼鏡}{めがね}の\kana{縁}{ふち}が日に光るまで近く寄って行った。\hbox{}そうして出し抜けに「先生」と大きな声を掛けた。\hbox{}先生は突然立ち留まって私の顔を見た。\hbox{}\par{}
「どうして……、\hbox{}どうして……」\par{}
 先生は同じ言葉を二\kana{遍}{へん}繰り返した。\hbox{}その言葉は\kana{森閑}{しんかん}とした昼の\kana{中}{うち}に異様な調子をもって繰り返された。\hbox{}私は急に何とも\kana{応}{こた}えられなくなった。\hbox{}\par{}
「私の\kana{後}{あと}を\kana{跟}{つ}けて来たのですか。\hbox{}どうして……」\par{}
 先生の態度はむしろ落ち付いていた。\hbox{}声はむしろ沈んでいた。\hbox{}けれどもその表情の\kana{中}{うち}には\kana{判然}{はっきり}いえないような一種の曇りがあった。\hbox{}\par{}
 私は私がどうしてここへ来たかを先生に話した。\hbox{}\par{}
「\kana{誰}{だれ}の墓へ参りに行ったか、\hbox{}\kana{妻}{さい}がその人の名をいいましたか」\par{}
「いいえ、\hbox{}そんな事は何もおっしゃいません」\par{}
「そうですか。\hbox{}――そう、\hbox{}それはいうはずがありませんね、\hbox{}始めて会ったあなたに。\hbox{}いう必要がないんだから」\par{}
 先生はようやく\kana{得心}{とくしん}したらしい様子であった。\hbox{}しかし私にはその意味がまるで\kana{解}{わか}らなかった。\hbox{}\par{}
 先生と私は通りへ出ようとして墓の間を抜けた。\hbox{}\kana{依撒伯拉何々}{イサベラなになに}の墓だの、\hbox{}\kana{神僕}{しんぼく}ロギンの墓だのという\kana{傍}{かたわら}に、\hbox{}\kana{一切衆生悉有仏生}{いっさいしゅじょうしつうぶっしょう}と書いた\kana{塔婆}{とうば}などが建ててあった。\hbox{}全権公使何々というのもあった。\hbox{}私は安得烈と\kana{彫}{ほ}り付けた小さい墓の前で、\hbox{}「これは何と読むんでしょう」と先生に聞いた。\hbox{}「アンドレとでも読ませるつもりでしょうね」といって先生は苦笑した。\hbox{}\par{}
 先生はこれらの墓標が現わす\kana{人種々}{ひとさまざま}の様式に対して、\hbox{}私ほどに\kana{滑稽}{こっけい}もアイロニーも認めてないらしかった。\hbox{}私が丸い\kana{墓石}{はかいし}だの細長い\kana{御影}{みかげ}の\kana{碑}{ひ}だのを指して、\hbox{}しきりにかれこれいいたがるのを、\hbox{}始めのうちは黙って聞いていたが、\hbox{}しまいに「あなたは死という事実をまだ\kana{真面目}{まじめ}に考えた事がありませんね」といった。\hbox{}私は黙った。\hbox{}先生もそれぎり何ともいわなくなった。\hbox{}\par{}
 墓地の区切り目に、\hbox{}大きな\kana{銀杏}{いちょう}が一本空を隠すように立っていた。\hbox{}その下へ来た時、\hbox{}先生は高い\kana{梢}{こずえ}を見上げて、\hbox{}「もう少しすると、\hbox{}\kana{綺麗}{きれい}ですよ。\hbox{}この木がすっかり\kana{黄葉}{こうよう}して、\hbox{}ここいらの地面は\kana{金色}{きんいろ}の落葉で\kana{埋}{うず}まるようになります」といった。\hbox{}先生は月に一度ずつは必ずこの木の下を通るのであった。\hbox{}\par{}
 向うの方で\kana{凸凹}{でこぼこ}の地面をならして新墓地を作っている男が、\hbox{}\kana{鍬}{くわ}の手を休めて私たちを見ていた。\hbox{}私たちはそこから左へ切れてすぐ街道へ出た。\hbox{}\par{}
 これからどこへ行くという\kana{目的}{あて}のない私は、\hbox{}ただ先生の歩く方へ歩いて行った。\hbox{}先生はいつもより口数を\kana{利}{き}かなかった。\hbox{}それでも私はさほどの窮屈を感じなかったので、\hbox{}ぶらぶらいっしょに歩いて行った。\hbox{}\par{}
「すぐお\kana{宅}{たく}へお帰りですか」\par{}
「ええ別に寄る所もありませんから」\par{}
 二人はまた黙って南の方へ坂を下りた。\hbox{}\par{}
「先生のお宅の墓地はあすこにあるんですか」と私がまた口を利き出した。\hbox{}\par{}
「いいえ」\par{}
「どなたのお墓があるんですか。\hbox{}――ご親類のお墓ですか」\par{}
「いいえ」\par{}
 先生はこれ以外に何も答えなかった。\hbox{}私もその話はそれぎりにして切り上げた。\hbox{}すると一\kana{町}{ちょう}ほど歩いた\kana{後}{あと}で、\hbox{}先生が不意にそこへ戻って来た。\hbox{}\par{}
「あすこには私の友達の墓があるんです」\par{}
「お友達のお墓へ\kana{毎月}{まいげつ}お参りをなさるんですか」\par{}
「そうです」\par{}
 先生はその日これ以外を語らなかった。\hbox{}\par{}\par{}     六
\par{}
 私はそれから時々先生を訪問するようになった。\hbox{}行くたびに先生は在宅であった。\hbox{}先生に会う\kana{度数}{どすう}が重なるにつれて、\hbox{}私はますます\kana{繁}{しげ}く先生の玄関へ足を運んだ。\hbox{}\par{}
 けれども先生の私に対する態度は初めて\kana{挨拶}{あいさつ}をした時も、\hbox{}懇意になったその\kana{後}{のち}も、\hbox{}あまり変りはなかった。\hbox{}先生は\kana{何時}{いつ}も静かであった。\hbox{}ある時は静か過ぎて\kana{淋}{さび}しいくらいであった。\hbox{}私は最初から先生には近づきがたい不思議があるように思っていた。\hbox{}それでいて、\hbox{}どうしても近づかなければいられないという感じが、\hbox{}どこかに強く働いた。\hbox{}こういう感じを先生に対してもっていたものは、\hbox{}多くの人のうちであるいは私だけかも知れない。\hbox{}しかしその私だけにはこの直感が\kana{後}{のち}になって事実の上に証拠立てられたのだから、\hbox{}私は若々しいといわれても、\hbox{}\kana{馬鹿}{ばか}げていると笑われても、\hbox{}それを見越した自分の直覚をとにかく頼もしくまた\kana{嬉}{うれ}しく思っている。\hbox{}人間を愛し\kana{得}{う}る人、\hbox{}愛せずにはいられない人、\hbox{}それでいて自分の\kana{懐}{ふところ}に\kana{入}{い}ろうとするものを、\hbox{}手をひろげて抱き締める事のできない人、\hbox{}――これが先生であった。\hbox{}\par{}
 今いった通り先生は始終静かであった。\hbox{}落ち付いていた。\hbox{}けれども時として変な曇りがその顔を横切る事があった。\hbox{}窓に黒い鳥影が\kana{射}{さ}すように。\hbox{}射すかと思うと、\hbox{}すぐ消えるには消えたが。\hbox{}私が始めてその曇りを先生の\kana{眉間}{みけん}に認めたのは、\hbox{}\kana{雑司ヶ谷}{ぞうしがや}の墓地で、\hbox{}不意に先生を呼び掛けた時であった。\hbox{}私はその異様の瞬間に、\hbox{}今まで快く流れていた心臓の潮流をちょっと鈍らせた。\hbox{}しかしそれは単に一時の\kana{結滞}{けったい}に過ぎなかった。\hbox{}私の心は五分と\kana{経}{た}たないうちに平素の弾力を回復した。\hbox{}私はそれぎり暗そうなこの雲の影を忘れてしまった。\hbox{}ゆくりなくまたそれを思い出させられたのは、\hbox{}\kana{小春}{こはる}の尽きるに\kana{間}{ま}のない\kana{或}{あ}る晩の事であった。\hbox{}\par{}
 先生と話していた私は、\hbox{}ふと先生がわざわざ注意してくれた\kana{銀杏}{いちょう}の\kana{大樹}{たいじゅ}を\kana{眼}{め}の前に\kana{想}{おも}い浮かべた。\hbox{}勘定してみると、\hbox{}先生が\kana{毎月例}{まいげつれい}として墓参に行く日が、\hbox{}それからちょうど三日目に当っていた。\hbox{}その三日目は私の課業が\kana{午}{ひる}で\kana{終}{お}える楽な日であった。\hbox{}私は先生に向かってこういった。\hbox{}\par{}
「先生\kana{雑司ヶ谷}{ぞうしがや}の銀杏はもう散ってしまったでしょうか」\par{}
「まだ\kana{空坊主}{からぼうず}にはならないでしょう」\par{}
 先生はそう答えながら私の顔を見守った。\hbox{}そうしてそこからしばし眼を離さなかった。\hbox{}私はすぐいった。\hbox{}\par{}
「今度お\kana{墓参}{はかまい}りにいらっしゃる時にお\kana{伴}{とも}をしても\kana{宜}{よ}ござんすか。\hbox{}私は先生といっしょにあすこいらが散歩してみたい」\par{}
「私は墓参りに行くんで、\hbox{}散歩に行くんじゃないですよ」\par{}
「しかしついでに散歩をなすったらちょうど\kana{好}{い}いじゃありませんか」\par{}
 先生は何とも答えなかった。\hbox{}しばらくしてから、\hbox{}「私のは本当の墓参りだけなんだから」といって、\hbox{}どこまでも\kana{墓参}{ぼさん}と散歩を切り離そうとする\kana{風}{ふう}に見えた。\hbox{}私と行きたくない口実だか何だか、\hbox{}私にはその時の先生が、\hbox{}いかにも子供らしくて変に思われた。\hbox{}私はなおと先へ出る気になった。\hbox{}\par{}
「じゃお墓参りでも\kana{好}{い}いからいっしょに\kana{伴}{つ}れて行って下さい。\hbox{}私もお墓参りをしますから」\par{}
 実際私には墓参と散歩との区別がほとんど無意味のように思われたのである。\hbox{}すると先生の\kana{眉}{まゆ}がちょっと曇った。\hbox{}眼のうちにも異様の光が出た。\hbox{}それは迷惑とも\kana{嫌悪}{けんお}とも\kana{畏怖}{いふ}とも片付けられない\kana{微}{かす}かな不安らしいものであった。\hbox{}私は\kana{忽}{たちま}ち雑司ヶ谷で「先生」と呼び掛けた時の記憶を強く思い起した。\hbox{}二つの表情は全く同じだったのである。\hbox{}\par{}
「私は」と先生がいった。\hbox{}「私はあなたに話す事のできないある理由があって、\hbox{}\kana{他}{ひと}といっしょにあすこへ墓参りには行きたくないのです。\hbox{}自分の\kana{妻}{さい}さえまだ伴れて行った事がないのです」\par{}\par{}     七
\par{}
 \kana{私}{わたくし}は不思議に思った。\hbox{}しかし私は先生を研究する気でその\kana{宅}{うち}へ\kana{出入}{でい}りをするのではなかった。\hbox{}私はただそのままにして打ち過ぎた。\hbox{}今考えるとその時の私の態度は、\hbox{}私の生活のうちでむしろ\kana{尊}{たっと}むべきものの一つであった。\hbox{}私は全くそのために先生と人間らしい温かい\kana{交際}{つきあい}ができたのだと思う。\hbox{}もし私の好奇心が幾分でも先生の心に向かって、\hbox{}研究的に働き掛けたなら、\hbox{}二人の間を\kana{繋}{つな}ぐ同情の糸は、\hbox{}何の容赦もなくその時ふつりと切れてしまったろう。\hbox{}若い私は全く自分の態度を自覚していなかった。\hbox{}それだから\kana{尊}{たっと}いのかも知れないが、\hbox{}もし間違えて裏へ出たとしたら、\hbox{}どんな結果が二人の仲に落ちて来たろう。\hbox{}私は想像してもぞっとする。\hbox{}先生はそれでなくても、\hbox{}冷たい\kana{眼}{まなこ}で研究されるのを絶えず恐れていたのである。\hbox{}\par{}
 私は月に二度もしくは三度ずつ必ず先生の\kana{宅}{うち}へ行くようになった。\hbox{}私の足が段々\kana{繁}{しげ}くなった時のある日、\hbox{}先生は突然私に向かって聞いた。\hbox{}\par{}
「あなたは何でそうたびたび私のようなものの宅へやって来るのですか」\par{}
「何でといって、\hbox{}そんな特別な意味はありません。\hbox{}――しかしお\kana{邪魔}{じゃま}なんですか」\par{}
「邪魔だとはいいません」\par{}
 なるほど迷惑という様子は、\hbox{}先生のどこにも見えなかった。\hbox{}私は先生の交際の範囲の\kana{極}{きわ}めて狭い事を知っていた。\hbox{}先生の元の同級生などで、\hbox{}その\kana{頃}{ころ}東京にいるものはほとんど二人か三人しかないという事も知っていた。\hbox{}先生と同郷の学生などには時たま座敷で同座する場合もあったが、\hbox{}彼らのいずれもは\kana{皆}{みん}な私ほど先生に親しみをもっていないように見受けられた。\hbox{}\par{}
「私は\kana{淋}{さび}しい人間です」と先生がいった。\hbox{}「だからあなたの来て下さる事を喜んでいます。\hbox{}だからなぜそうたびたび来るのかといって聞いたのです」\par{}
「そりゃまたなぜです」\par{}
 私がこう聞き返した時、\hbox{}先生は何とも答えなかった。\hbox{}ただ私の顔を見て「あなたは\kana{幾歳}{いくつ}ですか」といった。\hbox{}\par{}
 この問答は私にとってすこぶる\kana{不得要領}{ふとくようりょう}のものであったが、\hbox{}私はその時\kana{底}{そこ}まで押さずに帰ってしまった。\hbox{}しかもそれから四日と\kana{経}{た}たないうちにまた先生を訪問した。\hbox{}先生は座敷へ出るや\kana{否}{いな}や笑い出した。\hbox{}\par{}
「また来ましたね」といった。\hbox{}\par{}
「ええ来ました」といって自分も笑った。\hbox{}\par{}
 私は\kana{外}{ほか}の人からこういわれたらきっと\kana{癪}{しゃく}に\kana{触}{さわ}ったろうと思う。\hbox{}しかし先生にこういわれた時は、\hbox{}まるで反対であった。\hbox{}癪に触らないばかりでなくかえって愉快だった。\hbox{}\par{}
「私は\kana{淋}{さび}しい人間です」と先生はその晩またこの間の言葉を繰り返した。\hbox{}「私は淋しい人間ですが、\hbox{}ことによるとあなたも淋しい人間じゃないですか。\hbox{}私は淋しくっても年を取っているから、\hbox{}動かずにいられるが、\hbox{}若いあなたはそうは行かないのでしょう。\hbox{}動けるだけ動きたいのでしょう。\hbox{}動いて何かに\kana{打}{ぶ}つかりたいのでしょう……」\par{}
「私はちっとも\kana{淋}{さむ}しくはありません」\par{}
「若いうちほど\kana{淋}{さむ}しいものはありません。\hbox{}そんならなぜあなたはそうたびたび私の\kana{宅}{うち}へ来るのですか」\par{}
 ここでもこの間の言葉がまた先生の口から繰り返された。\hbox{}\par{}
「あなたは私に会ってもおそらくまだ\kana{淋}{さび}しい気がどこかでしているでしょう。\hbox{}私にはあなたのためにその淋しさを\kana{根元}{ねもと}から引き抜いて上げるだけの力がないんだから。\hbox{}あなたは\kana{外}{ほか}の方を向いて今に手を広げなければならなくなります。\hbox{}今に私の宅の方へは足が向かなくなります」\par{}
 先生はこういって淋しい笑い方をした。\hbox{}\par{}\par{}     八
\par{}
 \kana{幸}{さいわ}いにして先生の予言は実現されずに済んだ。\hbox{}経験のない当時の\kana{私}{わたくし}は、\hbox{}この予言の\kana{中}{うち}に含まれている明白な意義さえ了解し得なかった。\hbox{}私は依然として先生に会いに行った。\hbox{}その\kana{内}{うち}いつの間にか先生の食卓で\kana{飯}{めし}を食うようになった。\hbox{}自然の結果奥さんとも口を\kana{利}{き}かなければならないようになった。\hbox{}\par{}
 普通の人間として私は女に対して冷淡ではなかった。\hbox{}けれども年の若い私の今まで経過して来た境遇からいって、\hbox{}私はほとんど交際らしい交際を女に結んだ事がなかった。\hbox{}それが\kana{源因}{げんいん}かどうかは疑問だが、\hbox{}私の興味は往来で出合う知りもしない女に向かって多く働くだけであった。\hbox{}先生の奥さんにはその前玄関で会った時、\hbox{}美しいという印象を受けた。\hbox{}それから会うたんびに同じ印象を受けない事はなかった。\hbox{}しかしそれ以外に私はこれといってとくに奥さんについて語るべき何物ももたないような気がした。\hbox{}\par{}
 これは奥さんに特色がないというよりも、\hbox{}特色を示す機会が来なかったのだと解釈する方が正当かも知れない。\hbox{}しかし私はいつでも先生に付属した一部分のような心持で奥さんに対していた。\hbox{}奥さんも自分の夫の所へ来る書生だからという好意で、\hbox{}私を遇していたらしい。\hbox{}だから中間に立つ先生を取り\kana{除}{の}ければ、\hbox{}つまり二人はばらばらになっていた。\hbox{}それで始めて知り合いになった時の奥さんについては、\hbox{}ただ美しいという\kana{外}{ほか}に何の感じも残っていない。\hbox{}\par{}
 ある時私は先生の\kana{宅}{うち}で酒を飲まされた。\hbox{}その時奥さんが出て来て\kana{傍}{そば}で\kana{酌}{しゃく}をしてくれた。\hbox{}先生はいつもより愉快そうに見えた。\hbox{}奥さんに「お前も一つお上がり」といって、\hbox{}自分の\kana{呑}{の}み干した\kana{盃}{さかずき}を差した。\hbox{}奥さんは「私は……」と辞退しかけた\kana{後}{あと}、\hbox{}迷惑そうにそれを受け取った。\hbox{}奥さんは\kana{綺麗}{きれい}な\kana{眉}{まゆ}を寄せて、\hbox{}私の半分ばかり\kana{注}{つ}いで上げた盃を、\hbox{}唇の先へ持って行った。\hbox{}奥さんと先生の間に\kana{下}{しも}のような会話が始まった。\hbox{}\par{}
「珍らしい事。\hbox{}私に呑めとおっしゃった事は\kana{滅多}{めった}にないのにね」\par{}
「お前は\kana{嫌}{きら}いだからさ。\hbox{}しかし\kana{稀}{たま}には飲むといいよ。\hbox{}\kana{好}{い}い心持になるよ」\par{}
「ちっともならないわ。\hbox{}苦しいぎりで。\hbox{}でもあなたは大変ご\kana{愉快}{ゆかい}そうね、\hbox{}少しご\kana{酒}{しゅ}を召し上がると」\par{}
「時によると大変愉快になる。\hbox{}しかしいつでもというわけにはいかない」\par{}
「今夜はいかがです」\par{}
「今夜は\kana{好}{い}い心持だね」\par{}
「これから毎晩少しずつ召し上がると\kana{宜}{よ}ござんすよ」\par{}
「そうはいかない」\par{}
「召し上がって下さいよ。\hbox{}その方が\kana{淋}{さむ}しくなくって好いから」\par{}
 先生の\kana{宅}{うち}は夫婦と\kana{下女}{げじょ}だけであった。\hbox{}行くたびに\kana{大抵}{たいてい}はひそりとしていた。\hbox{}高い笑い声などの聞こえる試しはまるでなかった。\hbox{}\kana{或}{あ}る\kana{時}{とき}は宅の中にいるものは先生と私だけのような気がした。\hbox{}\par{}
「子供でもあると好いんですがね」と奥さんは私の方を向いていった。\hbox{}私は「そうですな」と答えた。\hbox{}しかし私の心には何の同情も起らなかった。\hbox{}子供を持った事のないその時の私は、\hbox{}子供をただ\kana{蒼蠅}{うるさ}いもののように考えていた。\hbox{}\par{}
「一人\kana{貰}{もら}ってやろうか」と先生がいった。\hbox{}\par{}
「\kana{貰}{もらい}ッ子じゃ、\hbox{}ねえあなた」と奥さんはまた私の方を向いた。\hbox{}\par{}
「子供はいつまで\kana{経}{た}ったってできっこないよ」と先生がいった。\hbox{}\par{}
 奥さんは黙っていた。\hbox{}「なぜです」と私が代りに聞いた時先生は「天罰だからさ」といって高く笑った。\hbox{}\par{}\par{}     九
\par{}
 \kana{私}{わたくし}の知る限り先生と奥さんとは、\hbox{}仲の\kana{好}{い}い夫婦の\kana{一対}{いっつい}であった。\hbox{}家庭の一員として暮した事のない私のことだから、\hbox{}深い消息は無論\kana{解}{わか}らなかったけれども、\hbox{}座敷で私と\kana{対坐}{たいざ}している時、\hbox{}先生は何かのついでに、\hbox{}\kana{下女}{げじょ}を呼ばないで、\hbox{}奥さんを呼ぶ事があった。\hbox{}(奥さんの名は\kana{静}{しず}といった)。\hbox{}先生は「おい静」といつでも\kana{襖}{ふすま}の方を振り向いた。\hbox{}その呼びかたが私には\kana{優}{やさ}しく聞こえた。\hbox{}返事をして出て来る奥さんの様子も\kana{甚}{はなは}だ素直であった。\hbox{}ときたまご\kana{馳走}{ちそう}になって、\hbox{}奥さんが席へ現われる場合などには、\hbox{}この関係が一層明らかに二人の\kana{間}{あいだ}に\kana{描}{えが}き出されるようであった。\hbox{}\par{}
 先生は時々奥さんを\kana{伴}{つ}れて、\hbox{}音楽会だの芝居だのに行った。\hbox{}それから夫婦づれで一週間以内の旅行をした事も、\hbox{}私の記憶によると、\hbox{}二、\hbox{}三度以上あった。\hbox{}私は\kana{箱根}{はこね}から貰った\kana{絵端書}{えはがき}をまだ持っている。\hbox{}\kana{日光}{にっこう}へ行った時は\kana{紅葉}{もみじ}の葉を一枚封じ込めた郵便も貰った。\hbox{}\par{}
 当時の私の眼に映った先生と奥さんの間柄はまずこんなものであった。\hbox{}そのうちにたった一つの例外があった。\hbox{}ある日私がいつもの通り、\hbox{}先生の玄関から案内を頼もうとすると、\hbox{}座敷の方でだれかの話し声がした。\hbox{}よく聞くと、\hbox{}それが尋常の談話でなくって、\hbox{}どうも\kana{言逆}{いさか}いらしかった。\hbox{}先生の宅は玄関の次がすぐ座敷になっているので、\hbox{}\kana{格子}{こうし}の前に立っていた私の耳にその\kana{言逆}{いさか}いの調子だけはほぼ分った。\hbox{}そうしてそのうちの一人が先生だという事も、\hbox{}時々高まって来る男の方の声で解った。\hbox{}相手は先生よりも低い\kana{音}{おん}なので、\hbox{}誰だか\kana{判然}{はっきり}しなかったが、\hbox{}どうも奥さんらしく感ぜられた。\hbox{}泣いているようでもあった。\hbox{}私はどうしたものだろうと思って玄関先で迷ったが、\hbox{}すぐ決心をしてそのまま下宿へ帰った。\hbox{}\par{}
 妙に不安な心持が私を襲って来た。\hbox{}私は書物を読んでも\kana{呑}{の}み込む能力を失ってしまった。\hbox{}約一時間ばかりすると先生が窓の下へ来て私の名を呼んだ。\hbox{}私は驚いて窓を開けた。\hbox{}先生は散歩しようといって、\hbox{}下から私を誘った。\hbox{}\kana{先刻}{さっき}帯の間へ\kana{包}{くる}んだままの時計を出して見ると、\hbox{}もう八時過ぎであった。\hbox{}私は帰ったなりまだ\kana{袴}{はかま}を着けていた。\hbox{}私はそれなりすぐ表へ出た。\hbox{}\par{}
 その晩私は先生といっしょに\kana{麦酒}{ビール}を飲んだ。\hbox{}先生は元来酒量に乏しい人であった。\hbox{}ある程度まで飲んで、\hbox{}それで酔えなければ、\hbox{}酔うまで飲んでみるという冒険のできない人であった。\hbox{}\par{}
「今日は\kana{駄目}{だめ}です」といって先生は苦笑した。\hbox{}\par{}
「愉快になれませんか」と私は気の毒そうに聞いた。\hbox{}\par{}
 私の腹の中には始終\kana{先刻}{さっき}の事が\kana{引}{ひ}っ\kana{懸}{かか}っていた。\hbox{}\kana{肴}{さかな}の骨が\kana{咽喉}{のど}に刺さった時のように、\hbox{}私は苦しんだ。\hbox{}打ち明けてみようかと考えたり、\hbox{}\kana{止}{よ}した方が\kana{好}{よ}かろうかと思い直したりする動揺が、\hbox{}妙に私の様子をそわそわさせた。\hbox{}\par{}
「君、\hbox{}今夜はどうかしていますね」と先生の方からいい出した。\hbox{}「実は私も少し変なのですよ。\hbox{}君に分りますか」\par{}
 私は何の答えもし得なかった。\hbox{}\par{}
「実は\kana{先刻}{さっき}\kana{妻}{さい}と少し\kana{喧嘩}{けんか}をしてね。\hbox{}それで\kana{下}{くだ}らない神経を\kana{昂奮}{こうふん}させてしまったんです」と先生がまたいった。\hbox{}\par{}
「どうして……」\par{}
 私には喧嘩という言葉が口へ出て来なかった。\hbox{}\par{}
「妻が私を誤解するのです。\hbox{}それを誤解だといって聞かせても承知しないのです。\hbox{}つい腹を立てたのです」\par{}
「どんなに先生を誤解なさるんですか」\par{}
 先生は私のこの問いに答えようとはしなかった。\hbox{}\par{}
「妻が考えているような人間なら、\hbox{}私だってこんなに苦しんでいやしない」\par{}
 先生がどんなに苦しんでいるか、\hbox{}これも私には想像の及ばない問題であった。\hbox{}\par{}\par{}     十
\par{}
 二人が帰るとき歩きながらの沈黙が一\kana{丁}{ちょう}も二丁もつづいた。\hbox{}その\kana{後}{あと}で突然先生が口を\kana{利}{き}き出した。\hbox{}\par{}
「悪い事をした。\hbox{}怒って出たから\kana{妻}{さい}はさぞ心配をしているだろう。\hbox{}考えると女は\kana{可哀}{かわい}そうなものですね。\hbox{}\kana{私}{わたくし}の妻などは私より\kana{外}{ほか}にまるで頼りにするものがないんだから」\par{}
 先生の言葉はちょっとそこで\kana{途切}{とぎ}れたが、\hbox{}別に私の返事を期待する様子もなく、\hbox{}すぐその続きへ移って行った。\hbox{}\par{}
「そういうと、\hbox{}夫の方はいかにも心丈夫のようで少し\kana{滑稽}{こっけい}だが。\hbox{}君、\hbox{}私は君の眼にどう映りますかね。\hbox{}強い人に見えますか、\hbox{}弱い人に見えますか」\par{}
「\kana{中位}{ちゅうぐらい}に見えます」と私は答えた。\hbox{}この答えは先生にとって少し案外らしかった。\hbox{}先生はまた口を閉じて、\hbox{}無言で歩き出した。\hbox{}\par{}
 先生の\kana{宅}{うち}へ帰るには私の下宿のつい\kana{傍}{そば}を通るのが順路であった。\hbox{}私はそこまで来て、\hbox{}曲り角で分れるのが先生に済まないような気がした。\hbox{}「ついでにお\kana{宅}{たく}の前までお\kana{伴}{とも}しましょうか」といった。\hbox{}先生は\kana{忽}{たちま}ち手で私を\kana{遮}{さえぎ}った。\hbox{}\par{}
「もう遅いから早く帰りたまえ。\hbox{}私も早く帰ってやるんだから、\hbox{}\kana{妻君}{さいくん}のために」\par{}
 先生が最後に付け加えた「妻君のために」という言葉は妙にその時の私の心を暖かにした。\hbox{}私はその言葉のために、\hbox{}帰ってから安心して寝る事ができた。\hbox{}私はその\kana{後}{ご}も長い間この「妻君のために」という言葉を忘れなかった。\hbox{}\par{}
 先生と奥さんの間に起った\kana{波瀾}{はらん}が、\hbox{}大したものでない事はこれでも\kana{解}{わか}った。\hbox{}それがまた\kana{滅多}{めった}に起る現象でなかった事も、\hbox{}その後絶えず\kana{出入}{でい}りをして来た私にはほぼ推察ができた。\hbox{}それどころか先生はある時こんな感想すら私に\kana{洩}{も}らした。\hbox{}\par{}
「私は世の中で女というものをたった一人しか知らない。\hbox{}\kana{妻}{さい}以外の女はほとんど女として私に訴えないのです。\hbox{}妻の方でも、\hbox{}私を天下にただ一人しかない男と思ってくれています。\hbox{}そういう意味からいって、\hbox{}私たちは最も幸福に生れた人間の\kana{一対}{いっつい}であるべきはずです」\par{}
 私は今前後の\kana{行}{ゆ}き\kana{掛}{がか}りを忘れてしまったから、\hbox{}先生が何のためにこんな自白を私にして聞かせたのか、\hbox{}\kana{判然}{はっきり}いう事ができない。\hbox{}けれども先生の態度の\kana{真面目}{まじめ}であったのと、\hbox{}調子の沈んでいたのとは、\hbox{}いまだに記憶に残っている。\hbox{}その時ただ私の耳に異様に響いたのは、\hbox{}「最も幸福に生れた人間の一対であるべきはずです」という最後の一句であった。\hbox{}先生はなぜ幸福な人間といい切らないで、\hbox{}あるべきはずであると断わったのか。\hbox{}私にはそれだけが不審であった。\hbox{}ことにそこへ一種の力を入れた先生の語気が不審であった。\hbox{}先生は事実はたして幸福なのだろうか、\hbox{}また幸福であるべきはずでありながら、\hbox{}それほど幸福でないのだろうか。\hbox{}私は心の\kana{中}{うち}で\kana{疑}{うたぐ}らざるを得なかった。\hbox{}けれどもその疑いは一時限りどこかへ\kana{葬}{ほうむ}られてしまった。\hbox{}\par{}
 私はそのうち先生の留守に行って、\hbox{}奥さんと二人\kana{差向}{さしむか}いで話をする機会に出合った。\hbox{}先生はその日\kana{横浜}{よこはま}を\kana{出帆}{しゅっぱん}する汽船に乗って外国へ行くべき友人を\kana{新橋}{しんばし}へ送りに行って留守であった。\hbox{}横浜から船に乗る人が、\hbox{}朝八時半の汽車で新橋を立つのはその\kana{頃}{ころ}の習慣であった。\hbox{}私はある書物について先生に話してもらう必要があったので、\hbox{}あらかじめ先生の承諾を得た通り、\hbox{}約束の九時に訪問した。\hbox{}先生の新橋行きは前日わざわざ告別に来た友人に対する\kana{礼義}{れいぎ}としてその日突然起った出来事であった。\hbox{}先生はすぐ帰るから留守でも私に待っているようにといい残して行った。\hbox{}それで私は座敷へ上がって、\hbox{}先生を待つ間、\hbox{}奥さんと話をした。\hbox{}\par{}\par{}     十一
\par{}
 その時の\kana{私}{わたくし}はすでに大学生であった。\hbox{}始めて先生の\kana{宅}{うち}へ来た\kana{頃}{ころ}から見るとずっと成人した気でいた。\hbox{}奥さんとも\kana{大分}{だいぶ}懇意になった\kana{後}{のち}であった。\hbox{}私は奥さんに対して何の窮屈も感じなかった。\hbox{}\kana{差向}{さしむか}いで色々の話をした。\hbox{}しかしそれは特色のないただの談話だから、\hbox{}今ではまるで忘れてしまった。\hbox{}そのうちでたった一つ私の耳に留まったものがある。\hbox{}しかしそれを話す前に、\hbox{}ちょっと断っておきたい事がある。\hbox{}\par{}
 先生は大学出身であった。\hbox{}これは始めから私に知れていた。\hbox{}しかし先生の何もしないで遊んでいるという事は、\hbox{}東京へ帰って少し\kana{経}{た}ってから始めて分った。\hbox{}私はその時どうして遊んでいられるのかと思った。\hbox{}\par{}
 先生はまるで世間に名前を知られていない人であった。\hbox{}だから先生の学問や思想については、\hbox{}先生と\kana{密切}{みっせつ}の関係をもっている私より\kana{外}{ほか}に敬意を払うもののあるべきはずがなかった。\hbox{}それを私は常に\kana{惜}{お}しい事だといった。\hbox{}先生はまた「私のようなものが世の中へ出て、\hbox{}口を\kana{利}{き}いては済まない」と答えるぎりで、\hbox{}取り合わなかった。\hbox{}私にはその答えが\kana{謙遜}{けんそん}過ぎてかえって世間を冷評するようにも聞こえた。\hbox{}実際先生は時々昔の同級生で今著名になっている\kana{誰彼}{だれかれ}を\kana{捉}{とら}えて、\hbox{}ひどく無遠慮な批評を加える事があった。\hbox{}それで私は露骨にその矛盾を挙げて\kana{云々}{うんぬん}してみた。\hbox{}私の精神は反抗の意味というよりも、\hbox{}世間が先生を知らないで平気でいるのが残念だったからである。\hbox{}その時先生は沈んだ調子で、\hbox{}「どうしても私は世間に向かって働き掛ける資格のない男だから仕方がありません」といった。\hbox{}先生の顔には深い一種の表情がありありと刻まれた。\hbox{}私にはそれが失望だか、\hbox{}不平だか、\hbox{}悲哀だか、\hbox{}\kana{解}{わか}らなかったけれども、\hbox{}何しろ二の句の継げないほどに強いものだったので、\hbox{}私はそれぎり何もいう勇気が出なかった。\hbox{}\par{}
 私が奥さんと話している間に、\hbox{}問題が自然先生の事からそこへ落ちて来た。\hbox{}\par{}
「先生はなぜああやって、\hbox{}宅で考えたり勉強したりなさるだけで、\hbox{}世の中へ出て仕事をなさらないんでしょう」\par{}
「あの人は\kana{駄目}{だめ}ですよ。\hbox{}そういう事が嫌いなんですから」\par{}
「つまり\kana{下}{くだ}らない事だと悟っていらっしゃるんでしょうか」\par{}
「悟るの悟らないのって、\hbox{}――そりゃ女だからわたくしには解りませんけれど、\hbox{}おそらくそんな意味じゃないでしょう。\hbox{}やっぱり何かやりたいのでしょう。\hbox{}それでいてできないんです。\hbox{}だから気の毒ですわ」\par{}
「しかし先生は健康からいって、\hbox{}別にどこも悪いところはないようじゃありませんか」\par{}
「丈夫ですとも。\hbox{}何にも持病はありません」\par{}
「それでなぜ活動ができないんでしょう」\par{}
「それが\kana{解}{わか}らないのよ、\hbox{}あなた。\hbox{}それが解るくらいなら私だって、\hbox{}こんなに心配しやしません。\hbox{}わからないから気の毒でたまらないんです」\par{}
 奥さんの語気には非常に同情があった。\hbox{}それでも口元だけには微笑が見えた。\hbox{}外側からいえば、\hbox{}私の方がむしろ\kana{真面目}{まじめ}だった。\hbox{}私はむずかしい顔をして黙っていた。\hbox{}すると奥さんが急に思い出したようにまた口を開いた。\hbox{}\par{}
「若い時はあんな人じゃなかったんですよ。\hbox{}若い時はまるで違っていました。\hbox{}それが全く変ってしまったんです」\par{}
「若い時っていつ頃ですか」と私が聞いた。\hbox{}\par{}
「書生時代よ」\par{}
「書生時代から先生を知っていらっしゃったんですか」\par{}
 奥さんは急に薄赤い顔をした。\hbox{}\par{}\par{}     十二
\par{}
 奥さんは東京の人であった。\hbox{}それはかつて先生からも奥さん自身からも聞いて知っていた。\hbox{}奥さんは「本当いうと\kana{合}{あい}の\kana{子}{こ}なんですよ」といった。\hbox{}奥さんの父親はたしか\kana{鳥取}{とっとり}かどこかの出であるのに、\hbox{}お母さんの方はまだ江戸といった\kana{時分}{じぶん}の\kana{市ヶ谷}{いちがや}で生れた女なので、\hbox{}奥さんは冗談半分そういったのである。\hbox{}ところが先生は全く方角違いの\kana{新潟}{にいがた}県人であった。\hbox{}だから奥さんがもし先生の書生時代を知っているとすれば、\hbox{}郷里の関係からでない事は明らかであった。\hbox{}しかし薄赤い顔をした奥さんはそれより以上の話をしたくないようだったので、\hbox{}私の方でも深くは聞かずにおいた。\hbox{}\par{}
 先生と知り合いになってから先生の亡くなるまでに、\hbox{}私はずいぶん色々の問題で先生の思想や情操に触れてみたが、\hbox{}結婚当時の状況については、\hbox{}ほとんど何ものも聞き得なかった。\hbox{}私は時によると、\hbox{}それを善意に解釈してもみた。\hbox{}年輩の先生の事だから、\hbox{}\kana{艶}{なま}めかしい回想などを若いものに聞かせるのはわざと\kana{慎}{つつし}んでいるのだろうと思った。\hbox{}時によると、\hbox{}またそれを悪くも取った。\hbox{}先生に限らず、\hbox{}奥さんに限らず、\hbox{}二人とも私に比べると、\hbox{}一時代前の因襲のうちに成人したために、\hbox{}そういう\kana{艶}{つや}っぽい問題になると、\hbox{}正直に自分を開放するだけの勇気がないのだろうと考えた。\hbox{}もっともどちらも推測に過ぎなかった。\hbox{}そうしてどちらの推測の裏にも、\hbox{}二人の結婚の奥に横たわる花やかなロマンスの存在を仮定していた。\hbox{}\par{}
 私の仮定ははたして誤らなかった。\hbox{}けれども私はただ恋の半面だけを想像に\kana{描}{えが}き得たに過ぎなかった。\hbox{}先生は美しい恋愛の裏に、\hbox{}恐ろしい悲劇を持っていた。\hbox{}そうしてその悲劇のどんなに先生にとって\kana{見惨}{みじめ}なものであるかは相手の奥さんにまるで知れていなかった。\hbox{}奥さんは今でもそれを知らずにいる。\hbox{}先生はそれを奥さんに隠して死んだ。\hbox{}先生は奥さんの幸福を破壊する前に、\hbox{}まず自分の生命を破壊してしまった。\hbox{}\par{}
 私は今この悲劇について何事も語らない。\hbox{}その悲劇のためにむしろ生れ出たともいえる二人の恋愛については、\hbox{}\kana{先刻}{さっき}いった通りであった。\hbox{}二人とも私にはほとんど何も話してくれなかった。\hbox{}奥さんは慎みのために、\hbox{}先生はまたそれ以上の深い理由のために。\hbox{}\par{}
 ただ一つ私の記憶に残っている事がある。\hbox{}\kana{或}{あ}る時\kana{花時分}{はなじぶん}に私は先生といっしょに\kana{上野}{うえの}へ行った。\hbox{}そうしてそこで美しい\kana{一対}{いっつい}の\kana{男女}{なんにょ}を見た。\hbox{}彼らは\kana{睦}{むつ}まじそうに寄り添って花の下を歩いていた。\hbox{}場所が場所なので、\hbox{}花よりもそちらを向いて眼を\kana{峙}{そば}だてている人が沢山あった。\hbox{}\par{}
「新婚の夫婦のようだね」と先生がいった。\hbox{}\par{}
「仲が\kana{好}{よ}さそうですね」と私が答えた。\hbox{}\par{}
 先生は苦笑さえしなかった。\hbox{}二人の男女を視線の\kana{外}{ほか}に置くような方角へ足を向けた。\hbox{}それから私にこう聞いた。\hbox{}\par{}
「君は恋をした事がありますか」\par{}
 私はないと答えた。\hbox{}\par{}
「恋をしたくはありませんか」\par{}
 私は答えなかった。\hbox{}\par{}
「したくない事はないでしょう」\par{}
「ええ」\par{}
「君は今あの男と女を見て、\hbox{}\kana{冷評}{ひやか}しましたね。\hbox{}あの\kana{冷評}{ひやかし}のうちには君が恋を求めながら相手を得られないという不快の声が\kana{交}{まじ}っていましょう」\par{}
「そんな\kana{風}{ふう}に聞こえましたか」\par{}
「聞こえました。\hbox{}恋の満足を味わっている人はもっと暖かい声を出すものです。\hbox{}しかし……しかし君、\hbox{}恋は罪悪ですよ。\hbox{}\kana{解}{わか}っていますか」\par{}
 私は急に驚かされた。\hbox{}何とも返事をしなかった。\hbox{}\par{}\par{}     十三
\par{}
 我々は群集の中にいた。\hbox{}群集はいずれも\kana{嬉}{うれ}しそうな顔をしていた。\hbox{}そこを通り抜けて、\hbox{}花も人も見えない森の中へ来るまでは、\hbox{}同じ問題を口にする機会がなかった。\hbox{}\par{}
「恋は罪悪ですか」と\kana{私}{わたくし}がその時突然聞いた。\hbox{}\par{}
「罪悪です。\hbox{}たしかに」と答えた時の先生の語気は前と同じように強かった。\hbox{}\par{}
「なぜですか」\par{}
「なぜだか今に解ります。\hbox{}今にじゃない、\hbox{}もう解っているはずです。\hbox{}あなたの心はとっくの昔からすでに恋で動いているじゃありませんか」\par{}
 私は一応自分の胸の中を調べて見た。\hbox{}けれどもそこは案外に空虚であった。\hbox{}思いあたるようなものは何にもなかった。\hbox{}\par{}
「私の胸の中にこれという目的物は一つもありません。\hbox{}私は先生に何も隠してはいないつもりです」\par{}
「目的物がないから動くのです。\hbox{}あれば落ち付けるだろうと思って動きたくなるのです」\par{}
「今それほど動いちゃいません」\par{}
「あなたは物足りない結果私の所に動いて来たじゃありませんか」\par{}
「それはそうかも知れません。\hbox{}しかしそれは恋とは違います」\par{}
「恋に\kana{上}{のぼ}る\kana{楷段}{かいだん}なんです。\hbox{}異性と抱き合う順序として、\hbox{}まず同性の私の所へ動いて来たのです」\par{}
「私には二つのものが全く性質を\kana{異}{こと}にしているように思われます」\par{}
「いや同じです。\hbox{}私は男としてどうしてもあなたに満足を与えられない人間なのです。\hbox{}それから、\hbox{}ある特別の事情があって、\hbox{}なおさらあなたに満足を与えられないでいるのです。\hbox{}私は実際お気の毒に思っています。\hbox{}あなたが私からよそへ動いて行くのは仕方がない。\hbox{}私はむしろそれを希望しているのです。\hbox{}しかし……」\par{}
 私は変に悲しくなった。\hbox{}\par{}
「私が先生から離れて行くようにお思いになれば仕方がありませんが、\hbox{}私にそんな気の起った事はまだありません」\par{}
 先生は私の言葉に耳を貸さなかった。\hbox{}\par{}
「しかし気を付けないといけない。\hbox{}恋は罪悪なんだから。\hbox{}私の所では満足が得られない代りに危険もないが、\hbox{}――君、\hbox{}黒い長い髪で縛られた時の心持を知っていますか」\par{}
 私は想像で知っていた。\hbox{}しかし事実としては知らなかった。\hbox{}いずれにしても先生のいう罪悪という意味は\kana{朦朧}{もうろう}としてよく\kana{解}{わか}らなかった。\hbox{}その上私は少し不愉快になった。\hbox{}\par{}
「先生、\hbox{}罪悪という意味をもっと\kana{判然}{はっきり}いって聞かして下さい。\hbox{}それでなければこの問題をここで切り上げて下さい。\hbox{}私自身に罪悪という意味が判然解るまで」\par{}
「悪い事をした。\hbox{}私はあなたに\kana{真実}{まこと}を話している気でいた。\hbox{}ところが実際は、\hbox{}あなたを\kana{焦慮}{じら}していたのだ。\hbox{}私は悪い事をした」\par{}
 先生と私とは博物館の裏から\kana{鶯渓}{うぐいすだに}の方角に静かな歩調で歩いて行った。\hbox{}垣の\kana{隙間}{すきま}から広い庭の一部に茂る\kana{熊笹}{くまざさ}が\kana{幽邃}{ゆうすい}に見えた。\hbox{}\par{}
「君は私がなぜ\kana{毎月}{まいげつ}\kana{雑司ヶ谷}{ぞうしがや}の墓地に\kana{埋}{うま}っている友人の墓へ参るのか知っていますか」\par{}
 先生のこの問いは全く突然であった。\hbox{}しかも先生は私がこの問いに対して答えられないという事もよく承知していた。\hbox{}私はしばらく返事をしなかった。\hbox{}すると先生は始めて気が付いたようにこういった。\hbox{}\par{}
「また悪い事をいった。\hbox{}\kana{焦慮}{じら}せるのが悪いと思って、\hbox{}説明しようとすると、\hbox{}その説明がまたあなたを焦慮せるような結果になる。\hbox{}どうも仕方がない。\hbox{}この問題はこれで\kana{止}{や}めましょう。\hbox{}とにかく恋は罪悪ですよ、\hbox{}よござんすか。\hbox{}そうして神聖なものですよ」\par{}
 私には先生の話がますます\kana{解}{わか}らなくなった。\hbox{}しかし先生はそれぎり恋を口にしなかった。\hbox{}\par{}\par{}     十四
\par{}
 年の若い\kana{私}{わたくし}はややともすると\kana{一図}{いちず}になりやすかった。\hbox{}少なくとも先生の眼にはそう映っていたらしい。\hbox{}私には学校の講義よりも先生の談話の方が有益なのであった。\hbox{}教授の意見よりも先生の思想の方が有難いのであった。\hbox{}とどの詰まりをいえば、\hbox{}教壇に立って私を指導してくれる偉い人々よりもただ\kana{独}{ひと}りを守って多くを語らない先生の方が偉く見えたのであった。\hbox{}\par{}
「あんまり\kana{逆上}{のぼせ}ちゃいけません」と先生がいった。\hbox{}\par{}
「\kana{覚}{さ}めた結果としてそう思うんです」と答えた時の私には充分の自信があった。\hbox{}その自信を先生は\kana{肯}{うけ}がってくれなかった。\hbox{}\par{}
「あなたは熱に浮かされているのです。\hbox{}熱がさめると\kana{厭}{いや}になります。\hbox{}私は今のあなたからそれほどに思われるのを、\hbox{}苦しく感じています。\hbox{}しかしこれから先のあなたに起るべき変化を予想して見ると、\hbox{}なお苦しくなります」\par{}
「私はそれほど軽薄に思われているんですか。\hbox{}それほど不信用なんですか」\par{}
「私はお気の毒に思うのです」\par{}
「気の毒だが信用されないとおっしゃるんですか」\par{}
 先生は迷惑そうに庭の方を向いた。\hbox{}その庭に、\hbox{}この間まで重そうな赤い強い色をぽたぽた点じていた\kana{椿}{つばき}の花はもう一つも見えなかった。\hbox{}先生は座敷からこの椿の花をよく\kana{眺}{なが}める癖があった。\hbox{}\par{}
「信用しないって、\hbox{}特にあなたを信用しないんじゃない。\hbox{}人間全体を信用しないんです」\par{}
 その時\kana{生垣}{いけがき}の向うで金魚売りらしい声がした。\hbox{}その\kana{外}{ほか}には何の聞こえるものもなかった。\hbox{}大通りから二\kana{丁}{ちょう}も深く折れ込んだ\kana{小路}{こうじ}は\kana{存外}{ぞんがい}静かであった。\hbox{}\kana{家}{うち}の中はいつもの通りひっそりしていた。\hbox{}私は次の\kana{間}{ま}に奥さんのいる事を知っていた。\hbox{}黙って針仕事か何かしている奥さんの耳に私の話し声が聞こえるという事も知っていた。\hbox{}しかし私は全くそれを忘れてしまった。\hbox{}\par{}
「じゃ奥さんも信用なさらないんですか」と先生に聞いた。\hbox{}\par{}
 先生は少し不安な顔をした。\hbox{}そうして直接の答えを避けた。\hbox{}\par{}
「私は私自身さえ信用していないのです。\hbox{}つまり自分で自分が信用できないから、\hbox{}人も信用できないようになっているのです。\hbox{}自分を\kana{呪}{のろ}うより\kana{外}{ほか}に仕方がないのです」\par{}
「そうむずかしく考えれば、\hbox{}誰だって確かなものはないでしょう」\par{}
「いや考えたんじゃない。\hbox{}やったんです。\hbox{}やった後で驚いたんです。\hbox{}そうして非常に\kana{怖}{こわ}くなったんです」\par{}
 私はもう少し先まで同じ道を\kana{辿}{たど}って行きたかった。\hbox{}すると\kana{襖}{ふすま}の陰で「あなた、\hbox{}あなた」という奥さんの声が二度聞こえた。\hbox{}先生は二度目に「何だい」といった。\hbox{}奥さんは「ちょっと」と先生を次の\kana{間}{ま}へ呼んだ。\hbox{}二人の間にどんな用事が起ったのか、\hbox{}私には\kana{解}{わか}らなかった。\hbox{}それを想像する余裕を与えないほど早く先生はまた座敷へ帰って来た。\hbox{}\par{}
「とにかくあまり私を信用してはいけませんよ。\hbox{}今に後悔するから。\hbox{}そうして自分が\kana{欺}{あざむ}かれた返報に、\hbox{}残酷な\kana{復讐}{ふくしゅう}をするようになるものだから」\par{}
「そりゃどういう意味ですか」\par{}
「かつてはその人の\kana{膝}{ひざ}の前に\kana{跪}{ひざまず}いたという記憶が、\hbox{}今度はその人の頭の上に足を\kana{載}{の}せさせようとするのです。\hbox{}私は未来の侮辱を受けないために、\hbox{}今の尊敬を\kana{斥}{しりぞ}けたいと思うのです。\hbox{}私は今より一層\kana{淋}{さび}しい未来の私を我慢する代りに、\hbox{}淋しい今の私を我慢したいのです。\hbox{}自由と独立と\kana{己}{おの}れとに\kana{充}{み}ちた現代に生れた我々は、\hbox{}その犠牲としてみんなこの淋しみを味わわなくてはならないでしょう」\par{}
 私はこういう覚悟をもっている先生に対して、\hbox{}いうべき言葉を知らなかった。\hbox{}\par{}\par{}     十五
\par{}
 その\kana{後}{ご}\kana{私}{わたくし}は奥さんの顔を見るたびに気になった。\hbox{}先生は奥さんに対しても始終こういう態度に出るのだろうか。\hbox{}もしそうだとすれば、\hbox{}奥さんはそれで満足なのだろうか。\hbox{}\par{}
 奥さんの様子は満足とも不満足とも\kana{極}{き}めようがなかった。\hbox{}私はそれほど近く奥さんに接触する機会がなかったから。\hbox{}それから奥さんは私に会うたびに尋常であったから。\hbox{}最後に先生のいる席でなければ私と奥さんとは\kana{滅多}{めった}に顔を合せなかったから。\hbox{}\par{}
 私の疑惑はまだその上にもあった。\hbox{}先生の人間に対するこの覚悟はどこから来るのだろうか。\hbox{}ただ冷たい眼で自分を内省したり現代を観察したりした結果なのだろうか。\hbox{}先生は\kana{坐}{すわ}って考える\kana{質}{たち}の人であった。\hbox{}先生の頭さえあれば、\hbox{}こういう態度は坐って世の中を考えていても自然と出て来るものだろうか。\hbox{}私にはそうばかりとは思えなかった。\hbox{}先生の覚悟は生きた覚悟らしかった。\hbox{}火に焼けて冷却し切った\kana{石造}{せきぞう}家屋の\kana{輪廓}{りんかく}とは違っていた。\hbox{}私の眼に映ずる先生はたしかに思想家であった。\hbox{}けれどもその思想家の\kana{纏}{まと}め上げた主義の裏には、\hbox{}強い事実が織り込まれているらしかった。\hbox{}自分と切り離された他人の事実でなくって、\hbox{}自分自身が痛切に味わった事実、\hbox{}血が熱くなったり脈が止まったりするほどの事実が、\hbox{}畳み込まれているらしかった。\hbox{}\par{}
 これは私の胸で推測するがものはない。\hbox{}先生自身すでにそうだと告白していた。\hbox{}ただその告白が雲の\kana{峯}{みね}のようであった。\hbox{}私の頭の上に正体の知れない恐ろしいものを\kana{蔽}{おお}い\kana{被}{かぶ}せた。\hbox{}そうしてなぜそれが恐ろしいか私にも\kana{解}{わか}らなかった。\hbox{}告白はぼうとしていた。\hbox{}それでいて明らかに私の神経を\kana{震}{ふる}わせた。\hbox{}\par{}
 私は先生のこの人生観の基点に、\hbox{}\kana{或}{あ}る強烈な恋愛事件を仮定してみた。\hbox{}(無論先生と奥さんとの間に起った)。\hbox{}先生がかつて恋は罪悪だといった事から照らし合せて見ると、\hbox{}多少それが\kana{手掛}{てがか}りにもなった。\hbox{}しかし先生は現に奥さんを愛していると私に告げた。\hbox{}すると二人の恋からこんな\kana{厭世}{えんせい}に近い覚悟が出ようはずがなかった。\hbox{}「かつてはその人の前に\kana{跪}{ひざまず}いたという記憶が、\hbox{}今度はその人の頭の上に足を\kana{載}{の}せさせようとする」といった先生の言葉は、\hbox{}現代一般の\kana{誰彼}{たれかれ}について用いられるべきで、\hbox{}先生と奥さんの間には当てはまらないもののようでもあった。\hbox{}\par{}
 \kana{雑司ヶ谷}{ぞうしがや}にある\kana{誰}{だれ}だか分らない人の墓、\hbox{}――これも私の記憶に時々動いた。\hbox{}私はそれが先生と深い縁故のある墓だという事を知っていた。\hbox{}先生の生活に近づきつつありながら、\hbox{}近づく事のできない私は、\hbox{}先生の頭の中にある\kana{生命}{いのち}の断片として、\hbox{}その墓を私の頭の中にも受け入れた。\hbox{}けれども私に取ってその墓は全く死んだものであった。\hbox{}二人の間にある\kana{生命}{いのち}の扉を開ける\kana{鍵}{かぎ}にはならなかった。\hbox{}むしろ二人の間に立って、\hbox{}自由の往来を妨げる魔物のようであった。\hbox{}\par{}
 そうこうしているうちに、\hbox{}私はまた奥さんと差し向いで話をしなければならない時機が来た。\hbox{}その\kana{頃}{ころ}は日の\kana{詰}{つま}って行くせわしない秋に、\hbox{}誰も注意を\kana{惹}{ひ}かれる\kana{肌寒}{はださむ}の季節であった。\hbox{}先生の\kana{附近}{ふきん}で盗難に\kana{罹}{かか}ったものが三、\hbox{}四日続いて出た。\hbox{}盗難はいずれも宵の口であった。\hbox{}大したものを持って行かれた\kana{家}{うち}はほとんどなかったけれども、\hbox{}はいられた所では必ず何か取られた。\hbox{}奥さんは気味をわるくした。\hbox{}そこへ先生がある晩家を\kana{空}{あ}けなければならない事情ができてきた。\hbox{}先生と同郷の友人で地方の病院に奉職しているものが上京したため、\hbox{}先生は\kana{外}{ほか}の二、\hbox{}三名と共に、\hbox{}ある所でその友人に\kana{飯}{めし}を食わせなければならなくなった。\hbox{}先生は訳を話して、\hbox{}私に帰ってくる間までの留守番を頼んだ。\hbox{}私はすぐ引き受けた。\hbox{}\par{}\par{}     十六
\par{}
 \kana{私}{わたくし}の行ったのはまだ\kana{灯}{ひ}の\kana{点}{つ}くか点かない暮れ方であったが、\hbox{}\kana{几帳面}{きちょうめん}な先生はもう\kana{宅}{うち}にいなかった。\hbox{}「時間に\kana{後}{おく}れると悪いって、\hbox{}つい今しがた出掛けました」といった奥さんは、\hbox{}私を先生の書斎へ案内した。\hbox{}\par{}
 書斎には\kana{洋机}{テーブル}と\kana{椅子}{いす}の\kana{外}{ほか}に、\hbox{}沢山の書物が美しい\kana{背皮}{せがわ}を並べて、\hbox{}\kana{硝子越}{ガラスごし}に\kana{電燈}{でんとう}の光で照らされていた。\hbox{}奥さんは火鉢の前に敷いた\kana{座蒲団}{ざぶとん}の上へ私を\kana{坐}{すわ}らせて、\hbox{}「ちっとそこいらにある本でも読んでいて下さい」と断って出て行った。\hbox{}私はちょうど主人の帰りを待ち受ける客のような気がして済まなかった。\hbox{}私は\kana{畏}{かしこ}まったまま\kana{烟草}{タバコ}を飲んでいた。\hbox{}奥さんが茶の間で何か\kana{下女}{げじょ}に話している声が聞こえた。\hbox{}書斎は茶の間の縁側を突き当って折れ曲った\kana{角}{かど}にあるので、\hbox{}\kana{棟}{むね}の位置からいうと、\hbox{}座敷よりもかえって掛け離れた静かさを\kana{領}{りょう}していた。\hbox{}ひとしきりで奥さんの話し声が\kana{已}{や}むと、\hbox{}\kana{後}{あと}はしんとした。\hbox{}私は泥棒を待ち受けるような心持で、\hbox{}\kana{凝}{じっ}としながら気をどこかに配った。\hbox{}\par{}
 三十分ほどすると、\hbox{}奥さんがまた書斎の入口へ顔を出した。\hbox{}「おや」といって、\hbox{}軽く驚いた時の眼を私に向けた。\hbox{}そうして客に来た人のように\kana{鹿爪}{しかつめ}らしく控えている私をおかしそうに見た。\hbox{}\par{}
「それじゃ窮屈でしょう」\par{}
「いえ、\hbox{}窮屈じゃありません」\par{}
「でも退屈でしょう」\par{}
「いいえ。\hbox{}泥棒が来るかと思って緊張しているから退屈でもありません」\par{}
 奥さんは手に\kana{紅茶茶碗}{こうちゃぢゃわん}を持ったまま、\hbox{}笑いながらそこに立っていた。\hbox{}\par{}
「ここは隅っこだから番をするには\kana{好}{よ}くありませんね」と私がいった。\hbox{}\par{}
「じゃ失礼ですがもっと真中へ出て来て\kana{頂戴}{ちょうだい}。\hbox{}ご\kana{退屈}{たいくつ}だろうと思って、\hbox{}お茶を入れて持って来たんですが、\hbox{}茶の間で\kana{宜}{よろ}しければあちらで上げますから」\par{}
 私は奥さんの\kana{後}{あと}に\kana{尾}{つ}いて書斎を出た。\hbox{}茶の間には\kana{綺麗}{きれい}な\kana{長火鉢}{ながひばち}に\kana{鉄瓶}{てつびん}が鳴っていた。\hbox{}私はそこで茶と菓子のご\kana{馳走}{ちそう}になった。\hbox{}奥さんは\kana{寝}{ね}られないといけないといって、\hbox{}茶碗に手を触れなかった。\hbox{}\par{}
「先生はやっぱり時々こんな会へお\kana{出掛}{でか}けになるんですか」\par{}
「いいえ\kana{滅多}{めった}に出た事はありません。\hbox{}\kana{近頃}{ちかごろ}は段々人の顔を見るのが\kana{嫌}{きら}いになるようです」\par{}
 こういった奥さんの様子に、\hbox{}別段困ったものだという\kana{風}{ふう}も見えなかったので、\hbox{}私はつい大胆になった。\hbox{}\par{}
「それじゃ奥さんだけが例外なんですか」\par{}
「いいえ私も嫌われている一人なんです」\par{}
「そりゃ\kana{嘘}{うそ}です」と私がいった。\hbox{}「奥さん自身嘘と知りながらそうおっしゃるんでしょう」\par{}
「なぜ」\par{}
「私にいわせると、\hbox{}奥さんが好きになったから世間が嫌いになるんですもの」\par{}
「あなたは学問をする\kana{方}{かた}だけあって、\hbox{}なかなかお\kana{上手}{じょうず}ね。\hbox{}\kana{空}{から}っぽな理屈を使いこなす事が。\hbox{}世の中が嫌いになったから、\hbox{}私までも嫌いになったんだともいわれるじゃありませんか。\hbox{}それと\kana{同}{おん}なじ理屈で」\par{}
「両方ともいわれる事はいわれますが、\hbox{}この場合は私の方が正しいのです」\par{}
「議論はいやよ。\hbox{}よく男の方は議論だけなさるのね、\hbox{}面白そうに。\hbox{}\kana{空}{から}の\kana{盃}{さかずき}でよくああ飽きずに\kana{献酬}{けんしゅう}ができると思いますわ」\par{}
 奥さんの言葉は少し\kana{手痛}{てひど}かった。\hbox{}しかしその言葉の\kana{耳障}{みみざわり}からいうと、\hbox{}決して猛烈なものではなかった。\hbox{}自分に頭脳のある事を相手に認めさせて、\hbox{}そこに一種の誇りを\kana{見出}{みいだ}すほどに奥さんは現代的でなかった。\hbox{}奥さんはそれよりもっと底の方に沈んだ心を大事にしているらしく見えた。\hbox{}\par{}\par{}     十七
\par{}
 \kana{私}{わたくし}はまだその\kana{後}{あと}にいうべき事をもっていた。\hbox{}けれども奥さんから\kana{徒}{いたず}らに議論を仕掛ける男のように取られては困ると思って遠慮した。\hbox{}奥さんは飲み干した\kana{紅茶茶碗}{こうちゃぢゃわん}の底を\kana{覗}{のぞ}いて黙っている私を\kana{外}{そ}らさないように、\hbox{}「もう一杯上げましょうか」と聞いた。\hbox{}私はすぐ茶碗を奥さんの手に渡した。\hbox{}\par{}
「いくつ? 一つ? 二ッつ?」\par{}
 妙なもので角砂糖をつまみ上げた奥さんは、\hbox{}私の顔を見て、\hbox{}茶碗の中へ入れる砂糖の\kana{数}{かず}を聞いた。\hbox{}奥さんの態度は私に\kana{媚}{こ}びるというほどではなかったけれども、\hbox{}\kana{先刻}{さっき}の強い言葉を\kana{力}{つと}めて打ち消そうとする\kana{愛嬌}{あいきょう}に\kana{充}{み}ちていた。\hbox{}\par{}
 私は黙って茶を飲んだ。\hbox{}飲んでしまっても黙っていた。\hbox{}\par{}
「あなた大変黙り込んじまったのね」と奥さんがいった。\hbox{}\par{}
「何かいうとまた議論を仕掛けるなんて、\hbox{}\kana{叱}{しか}り付けられそうですから」と私は答えた。\hbox{}\par{}
「まさか」と奥さんが再びいった。\hbox{}\par{}
 二人はそれを\kana{緒口}{いとくち}にまた話を始めた。\hbox{}そうしてまた二人に共通な興味のある先生を問題にした。\hbox{}\par{}
「奥さん、\hbox{}\kana{先刻}{さっき}の続きをもう少しいわせて下さいませんか。\hbox{}奥さんには\kana{空}{から}な理屈と聞こえるかも知れませんが、\hbox{}私はそんな\kana{上}{うわ}の\kana{空}{そら}でいってる事じゃないんだから」\par{}
「じゃおっしゃい」\par{}
「今奥さんが急にいなくなったとしたら、\hbox{}先生は現在の通りで生きていられるでしょうか」\par{}
「そりゃ分らないわ、\hbox{}あなた。\hbox{}そんな事、\hbox{}先生に聞いて見るより\kana{外}{ほか}に仕方がないじゃありませんか。\hbox{}私の所へ持って来る問題じゃないわ」\par{}
「奥さん、\hbox{}私は\kana{真面目}{まじめ}ですよ。\hbox{}だから逃げちゃいけません。\hbox{}正直に答えなくっちゃ」\par{}
「正直よ。\hbox{}正直にいって私には分らないのよ」\par{}
「じゃ奥さんは先生をどのくらい愛していらっしゃるんですか。\hbox{}これは先生に聞くよりむしろ奥さんに伺っていい質問ですから、\hbox{}あなたに伺います」\par{}
「何もそんな事を開き直って聞かなくっても\kana{好}{い}いじゃありませんか」\par{}
「真面目くさって聞くがものはない。\hbox{}分り切ってるとおっしゃるんですか」\par{}
「まあそうよ」\par{}
「そのくらい先生に忠実なあなたが急にいなくなったら、\hbox{}先生はどうなるんでしょう。\hbox{}世の中のどっちを向いても面白そうでない先生は、\hbox{}あなたが急にいなくなったら後でどうなるでしょう。\hbox{}先生から見てじゃない。\hbox{}あなたから見てですよ。\hbox{}あなたから見て、\hbox{}先生は幸福になるでしょうか、\hbox{}不幸になるでしょうか」\par{}
「そりゃ私から見れば分っています。\hbox{}(先生はそう思っていないかも知れませんが)。\hbox{}先生は私を離れれば不幸になるだけです。\hbox{}あるいは生きていられないかも知れませんよ。\hbox{}そういうと、\hbox{}\kana{己惚}{おのぼれ}になるようですが、\hbox{}私は今先生を人間としてできるだけ幸福にしているんだと信じていますわ。\hbox{}どんな人があっても私ほど先生を幸福にできるものはないとまで思い込んでいますわ。\hbox{}それだからこうして落ち付いていられるんです」\par{}
「その信念が先生の心に\kana{好}{よ}く映るはずだと私は思いますが」\par{}
「それは別問題ですわ」\par{}
「やっぱり先生から嫌われているとおっしゃるんですか」\par{}
「私は嫌われてるとは思いません。\hbox{}嫌われる訳がないんですもの。\hbox{}しかし先生は世間が嫌いなんでしょう。\hbox{}世間というより\kana{近頃}{ちかごろ}では人間が嫌いになっているんでしょう。\hbox{}だからその人間の\kana{一人}{いちにん}として、\hbox{}私も好かれるはずがないじゃありませんか」\par{}
 奥さんの嫌われているという意味がやっと私に\kana{呑}{の}み込めた。\hbox{}\par{}\par{}     十八
\par{}
 \kana{私}{わたくし}は奥さんの理解力に感心した。\hbox{}奥さんの態度が旧式の日本の女らしくないところも私の注意に一種の\kana{刺戟}{しげき}を与えた。\hbox{}それで奥さんはその\kana{頃}{ころ}\kana{流行}{はや}り始めたいわゆる新しい言葉などはほとんど使わなかった。\hbox{}\par{}
 私は女というものに深い\kana{交際}{つきあい}をした経験のない\kana{迂闊}{うかつ}な青年であった。\hbox{}男としての私は、\hbox{}異性に対する本能から、\hbox{}\kana{憧憬}{どうけい}の目的物として常に女を夢みていた。\hbox{}けれどもそれは懐かしい春の雲を\kana{眺}{なが}めるような心持で、\hbox{}ただ\kana{漠然}{ばくぜん}と夢みていたに過ぎなかった。\hbox{}だから実際の女の前へ出ると、\hbox{}私の感情が突然変る事が時々あった。\hbox{}私は自分の前に現われた女のために引き付けられる代りに、\hbox{}その場に臨んでかえって変な\kana{反撥力}{はんぱつりょく}を感じた。\hbox{}奥さんに対した私にはそんな気がまるで出なかった。\hbox{}普通\kana{男女}{なんにょ}の間に横たわる思想の不平均という考えもほとんど起らなかった。\hbox{}私は奥さんの女であるという事を忘れた。\hbox{}私はただ誠実なる先生の批評家および同情家として奥さんを眺めた。\hbox{}\par{}
「奥さん、\hbox{}私がこの前なぜ先生が世間的にもっと活動なさらないのだろうといって、\hbox{}あなたに聞いた時に、\hbox{}あなたはおっしゃった事がありますね。\hbox{}元はああじゃなかったんだって」\par{}
「ええいいました。\hbox{}実際あんなじゃなかったんですもの」\par{}
「どんなだったんですか」\par{}
「あなたの希望なさるような、\hbox{}また私の希望するような頼もしい人だったんです」\par{}
「それがどうして急に変化なすったんですか」\par{}
「急にじゃありません、\hbox{}段々ああなって来たのよ」\par{}
「奥さんはその\kana{間}{あいだ}始終先生といっしょにいらしったんでしょう」\par{}
「無論いましたわ。\hbox{}夫婦ですもの」\par{}
「じゃ先生がそう変って行かれる\kana{源因}{げんいん}がちゃんと\kana{解}{わか}るべきはずですがね」\par{}
「それだから困るのよ。\hbox{}あなたからそういわれると実に\kana{辛}{つら}いんですが、\hbox{}私にはどう考えても、\hbox{}考えようがないんですもの。\hbox{}私は今まで\kana{何遍}{なんべん}あの人に、\hbox{}どうぞ打ち明けて下さいって頼んで見たか分りゃしません」\par{}
「先生は何とおっしゃるんですか」\par{}
「何にもいう事はない、\hbox{}何にも心配する事はない、\hbox{}おれはこういう性質になったんだからというだけで、\hbox{}取り合ってくれないんです」\par{}
 私は黙っていた。\hbox{}奥さんも言葉を\kana{途切}{とぎ}らした。\hbox{}\kana{下女部屋}{げじょべや}にいる下女はことりとも音をさせなかった。\hbox{}私はまるで泥棒の事を忘れてしまった。\hbox{}\par{}
「あなたは私に責任があるんだと思ってやしませんか」と突然奥さんが聞いた。\hbox{}\par{}
「いいえ」と私が答えた。\hbox{}\par{}
「どうぞ隠さずにいって下さい。\hbox{}そう思われるのは身を切られるより辛いんだから」と奥さんがまたいった。\hbox{}「これでも私は先生のためにできるだけの事はしているつもりなんです」\par{}
「そりゃ先生もそう認めていられるんだから、\hbox{}大丈夫です。\hbox{}ご安心なさい、\hbox{}私が保証します」\par{}
 奥さんは火鉢の灰を\kana{掻}{か}き\kana{馴}{な}らした。\hbox{}それから\kana{水注}{みずさし}の水を\kana{鉄瓶}{てつびん}に\kana{注}{さ}した。\hbox{}鉄瓶は\kana{忽}{たちま}ち鳴りを沈めた。\hbox{}\par{}
「私はとうとう\kana{辛防}{しんぼう}し切れなくなって、\hbox{}先生に聞きました。\hbox{}私に悪い所があるなら遠慮なくいって下さい、\hbox{}改められる欠点なら改めるからって、\hbox{}すると先生は、\hbox{}お前に欠点なんかありゃしない、\hbox{}欠点はおれの方にあるだけだというんです。\hbox{}そういわれると、\hbox{}私悲しくなって仕様がないんです、\hbox{}涙が出てなおの事自分の悪い所が聞きたくなるんです」\par{}
 奥さんは眼の\kana{中}{うち}に涙をいっぱい\kana{溜}{た}めた。\hbox{}\par{}\par{}     十九
\par{}
 始め\kana{私}{わたくし}は理解のある\kana{女性}{にょしょう}として奥さんに対していた。\hbox{}私がその気で話しているうちに、\hbox{}奥さんの様子が次第に変って来た。\hbox{}奥さんは私の頭脳に訴える代りに、\hbox{}私の\kana{心臓}{ハート}を動かし始めた。\hbox{}自分と夫の間には何の\kana{蟠}{わだか}まりもない、\hbox{}またないはずであるのに、\hbox{}やはり何かある。\hbox{}それだのに眼を\kana{開}{あ}けて\kana{見極}{みきわ}めようとすると、\hbox{}やはり\kana{何}{なん}にもない。\hbox{}奥さんの苦にする要点はここにあった。\hbox{}\par{}
 奥さんは最初世の中を見る先生の眼が\kana{厭世的}{えんせいてき}だから、\hbox{}その結果として自分も嫌われているのだと断言した。\hbox{}そう断言しておきながら、\hbox{}ちっともそこに落ち付いていられなかった。\hbox{}底を割ると、\hbox{}かえってその逆を考えていた。\hbox{}先生は自分を嫌う結果、\hbox{}とうとう世の中まで\kana{厭}{いや}になったのだろうと推測していた。\hbox{}けれどもどう骨を折っても、\hbox{}その推測を突き留めて事実とする事ができなかった。\hbox{}先生の態度はどこまでも\kana{良人}{おっと}らしかった。\hbox{}親切で優しかった。\hbox{}疑いの\kana{塊}{かたま}りをその日その日の\kana{情合}{じょうあい}で包んで、\hbox{}そっと胸の奥にしまっておいた奥さんは、\hbox{}その晩その包みの中を私の前で開けて見せた。\hbox{}\par{}
「あなたどう思って?」と聞いた。\hbox{}「私からああなったのか、\hbox{}それともあなたのいう\kana{人世観}{じんせいかん}とか何とかいうものから、\hbox{}ああなったのか。\hbox{}隠さずいって\kana{頂戴}{ちょうだい}」\par{}
 私は何も隠す気はなかった。\hbox{}けれども私の知らないあるものがそこに存在しているとすれば、\hbox{}私の答えが何であろうと、\hbox{}それが奥さんを満足させるはずがなかった。\hbox{}そうして私はそこに私の知らないあるものがあると信じていた。\hbox{}\par{}
「私には\kana{解}{わか}りません」\par{}
 奥さんは予期の\kana{外}{はず}れた時に見る\kana{憐}{あわ}れな表情をその\kana{咄嗟}{とっさ}に現わした。\hbox{}私はすぐ私の言葉を継ぎ足した。\hbox{}\par{}
「しかし先生が奥さんを嫌っていらっしゃらない事だけは保証します。\hbox{}私は先生自身の口から聞いた通りを奥さんに伝えるだけです。\hbox{}先生は\kana{嘘}{うそ}を\kana{吐}{つ}かない\kana{方}{かた}でしょう」\par{}
 奥さんは何とも答えなかった。\hbox{}しばらくしてからこういった。\hbox{}\par{}
「実は私すこし思いあたる事があるんですけれども……」\par{}
「先生がああいう\kana{風}{ふう}になった\kana{源因}{げんいん}についてですか」\par{}
「ええ。\hbox{}もしそれが源因だとすれば、\hbox{}私の責任だけはなくなるんだから、\hbox{}それだけでも私大変楽になれるんですが、\hbox{}……」\par{}
「どんな事ですか」\par{}
 奥さんはいい渋って\kana{膝}{ひざ}の上に置いた自分の手を眺めていた。\hbox{}\par{}
「あなた判断して下すって。\hbox{}いうから」\par{}
「私にできる判断ならやります」\par{}
「みんなはいえないのよ。\hbox{}みんないうと\kana{叱}{しか}られるから。\hbox{}叱られないところだけよ」\par{}
 私は緊張して\kana{唾液}{つばき}を\kana{呑}{の}み込んだ。\hbox{}\par{}
「先生がまだ大学にいる時分、\hbox{}大変仲の\kana{好}{い}いお友達が一人あったのよ。\hbox{}その\kana{方}{かた}がちょうど卒業する少し前に死んだんです。\hbox{}急に死んだんです」\par{}
 奥さんは私の耳に\kana{私語}{ささや}くような小さな声で、\hbox{}「実は変死したんです」といった。\hbox{}それは「どうして」と聞き返さずにはいられないようないい方であった。\hbox{}\par{}
「それっ切りしかいえないのよ。\hbox{}けれどもその事があってから\kana{後}{のち}なんです。\hbox{}先生の性質が段々変って来たのは。\hbox{}なぜその方が死んだのか、\hbox{}私には解らないの。\hbox{}先生にもおそらく解っていないでしょう。\hbox{}けれどもそれから先生が変って来たと思えば、\hbox{}そう思われない事もないのよ」\par{}
「その人の墓ですか、\hbox{}\kana{雑司ヶ谷}{ぞうしがや}にあるのは」\par{}
「それもいわない事になってるからいいません。\hbox{}しかし人間は親友を一人亡くしただけで、\hbox{}そんなに変化できるものでしょうか。\hbox{}私はそれが知りたくって\kana{堪}{たま}らないんです。\hbox{}だからそこを一つあなたに判断して頂きたいと思うの」\par{}
 私の判断はむしろ否定の方に傾いていた。\hbox{}\par{}\par{}     二十
\par{}
 \kana{私}{わたくし}は私のつらまえた事実の許す限り、\hbox{}奥さんを慰めようとした。\hbox{}奥さんもまたできるだけ私によって慰められたそうに見えた。\hbox{}それで二人は同じ問題をいつまでも話し合った。\hbox{}けれども私はもともと事の\kana{大根}{おおね}を\kana{攫}{つか}んでいなかった。\hbox{}奥さんの不安も実はそこに\kana{漂}{ただよ}う薄い雲に似た疑惑から出て来ていた。\hbox{}事件の真相になると、\hbox{}奥さん自身にも多くは知れていなかった。\hbox{}知れているところでも\kana{悉皆}{すっかり}は私に話す事ができなかった。\hbox{}したがって慰める私も、\hbox{}慰められる奥さんも、\hbox{}共に波に浮いて、\hbox{}ゆらゆらしていた。\hbox{}ゆらゆらしながら、\hbox{}奥さんはどこまでも手を出して、\hbox{}\kana{覚束}{おぼつか}ない私の判断に\kana{縋}{すが}り付こうとした。\hbox{}\par{}
 十時\kana{頃}{ごろ}になって先生の靴の音が玄関に聞こえた時、\hbox{}奥さんは急に今までのすべてを忘れたように、\hbox{}前に\kana{坐}{すわ}っている私をそっちのけにして立ち上がった。\hbox{}そうして\kana{格子}{こうし}を開ける先生をほとんど\kana{出合}{であ}い\kana{頭}{がしら}に迎えた。\hbox{}私は取り残されながら、\hbox{}\kana{後}{あと}から奥さんに\kana{尾}{つ}いて行った。\hbox{}\kana{下女}{げじょ}だけは\kana{仮寝}{うたたね}でもしていたとみえて、\hbox{}ついに出て来なかった。\hbox{}\par{}
 先生はむしろ機嫌がよかった。\hbox{}しかし奥さんの調子はさらによかった。\hbox{}今しがた奥さんの美しい眼のうちに\kana{溜}{たま}った涙の光と、\hbox{}それから黒い\kana{眉毛}{まゆげ}の根に寄せられた八の字を記憶していた私は、\hbox{}その変化を異常なものとして注意深く\kana{眺}{なが}めた。\hbox{}もしそれが\kana{詐}{いつわ}りでなかったならば、\hbox{}(実際それは詐りとは思えなかったが)、\hbox{}今までの奥さんの訴えは\kana{感傷}{センチメント}を\kana{玩}{もてあそ}ぶためにとくに私を相手に\kana{拵}{こしら}えた、\hbox{}\kana{徒}{いたず}らな女性の遊戯と取れない事もなかった。\hbox{}もっともその時の私には奥さんをそれほど批評的に見る気は起らなかった。\hbox{}私は奥さんの態度の急に輝いて来たのを見て、\hbox{}むしろ安心した。\hbox{}これならばそう心配する必要もなかったんだと考え直した。\hbox{}\par{}
 先生は笑いながら「どうもご苦労さま、\hbox{}泥棒は来ませんでしたか」と私に聞いた。\hbox{}それから「来ないんで\kana{張合}{はりあい}が抜けやしませんか」といった。\hbox{}\par{}
 帰る時、\hbox{}奥さんは「どうもお気の毒さま」と会釈した。\hbox{}その調子は忙しいところを暇を\kana{潰}{つぶ}させて気の毒だというよりも、\hbox{}せっかく来たのに泥棒がはいらなくって気の毒だという冗談のように聞こえた。\hbox{}奥さんはそういいながら、\hbox{}\kana{先刻}{さっき}出した西洋菓子の残りを、\hbox{}紙に包んで私の手に持たせた。\hbox{}私はそれを\kana{袂}{たもと}へ入れて、\hbox{}人通りの少ない\kana{夜寒}{よさむ}の\kana{小路}{こうじ}を曲折して\kana{賑}{にぎ}やかな町の方へ急いだ。\hbox{}\par{}
 私はその晩の事を記憶のうちから\kana{抽}{ひ}き抜いてここへ\kana{詳}{くわ}しく書いた。\hbox{}これは書くだけの必要があるから書いたのだが、\hbox{}実をいうと、\hbox{}奥さんに菓子を\kana{貰}{もら}って帰るときの気分では、\hbox{}それほど当夜の会話を重く見ていなかった。\hbox{}私はその\kana{翌日}{よくじつ}\kana{午飯}{ひるめし}を食いに学校から帰ってきて、\hbox{}\kana{昨夜}{ゆうべ}机の上に\kana{載}{の}せて置いた菓子の包みを見ると、\hbox{}すぐその中からチョコレートを塗った\kana{鳶色}{とびいろ}のカステラを出して\kana{頬張}{ほおば}った。\hbox{}そうしてそれを食う時に、\hbox{}\kana{必竟}{ひっきょう}この菓子を私にくれた二人の\kana{男女}{なんにょ}は、\hbox{}幸福な\kana{一対}{いっつい}として世の中に存在しているのだと自覚しつつ味わった。\hbox{}\par{}
 秋が暮れて冬が来るまで格別の事もなかった。\hbox{}私は先生の\kana{宅}{うち}へ\kana{出}{で}はいりをするついでに、\hbox{}衣服の\kana{洗}{あら}い\kana{張}{は}りや\kana{仕立}{した}て\kana{方}{かた}などを奥さんに頼んだ。\hbox{}それまで\kana{繻絆}{じゅばん}というものを着た事のない私が、\hbox{}シャツの上に黒い襟のかかったものを重ねるようになったのはこの時からであった。\hbox{}子供のない奥さんは、\hbox{}そういう世話を焼くのがかえって\kana{退屈凌}{たいくつしの}ぎになって、\hbox{}\kana{結句}{けっく}\kana{身体}{からだ}の薬だぐらいの事をいっていた。\hbox{}\par{}
「こりゃ\kana{手織}{てお}りね。\hbox{}こんな\kana{地}{じ}の\kana{好}{い}い着物は今まで縫った事がないわ。\hbox{}その代り縫い\kana{悪}{にく}いのよそりゃあ。\hbox{}まるで針が立たないんですもの。\hbox{}お\kana{蔭}{かげ}で針を二本折りましたわ」\par{}
 こんな苦情をいう時ですら、\hbox{}奥さんは別に\kana{面倒}{めんどう}くさいという顔をしなかった。\hbox{}\par{}\par{}     二十一
\par{}
 冬が来た時、\hbox{}\kana{私}{わたくし}は偶然国へ帰らなければならない事になった。\hbox{}私の母から受け取った手紙の中に、\hbox{}父の病気の経過が面白くない様子を書いて、\hbox{}今が今という心配もあるまいが、\hbox{}年が年だから、\hbox{}できるなら都合して帰って来てくれと頼むように付け足してあった。\hbox{}\par{}
 父はかねてから\kana{腎臓}{じんぞう}を病んでいた。\hbox{}中年以後の人にしばしば見る通り、\hbox{}父のこの\kana{病}{やまい}は慢性であった。\hbox{}その代り要心さえしていれば急変のないものと当人も家族のものも信じて疑わなかった。\hbox{}現に父は養生のお\kana{蔭}{かげ}一つで、\hbox{}\kana{今日}{こんにち}までどうかこうか\kana{凌}{しの}いで来たように客が来ると\kana{吹聴}{ふいちょう}していた。\hbox{}その父が、\hbox{}母の書信によると、\hbox{}庭へ出て何かしている\kana{機}{はずみ}に突然\kana{眩暈}{めまい}がして引ッ繰り返った。\hbox{}\kana{家内}{かない}のものは軽症の\kana{脳溢血}{のういっけつ}と思い違えて、\hbox{}すぐその手当をした。\hbox{}\kana{後}{あと}で医者からどうもそうではないらしい、\hbox{}やはり持病の結果だろうという判断を得て、\hbox{}始めて卒倒と腎臓病とを結び付けて考えるようになったのである。\hbox{}\par{}
 冬休みが来るにはまだ少し\kana{間}{ま}があった。\hbox{}私は学期の終りまで待っていても\kana{差支}{さしつか}えあるまいと思って一日二日そのままにしておいた。\hbox{}するとその一日二日の間に、\hbox{}父の寝ている様子だの、\hbox{}母の心配している顔だのが時々眼に浮かんだ。\hbox{}そのたびに一種の心苦しさを\kana{嘗}{な}めた私は、\hbox{}とうとう帰る決心をした。\hbox{}国から旅費を送らせる\kana{手数}{てかず}と時間を省くため、\hbox{}私は\kana{暇乞}{いとまご}いかたがた先生の所へ行って、\hbox{}\kana{要}{い}るだけの金を一時立て替えてもらう事にした。\hbox{}\par{}
 先生は少し\kana{風邪}{かぜ}の気味で、\hbox{}座敷へ出るのが\kana{臆劫}{おっくう}だといって、\hbox{}私をその書斎に通した。\hbox{}書斎の\kana{硝子戸}{ガラスど}から冬に\kana{入}{い}って\kana{稀}{まれ}に見るような懐かしい\kana{和}{やわ}らかな日光が\kana{机掛}{つくえか}けの上に\kana{射}{さ}していた。\hbox{}先生はこの日あたりの\kana{好}{い}い\kana{室}{へや}の中へ大きな火鉢を置いて、\hbox{}\kana{五徳}{ごとく}の上に懸けた\kana{金盥}{かなだらい}から立ち\kana{上}{あが}る\kana{湯気}{ゆげ}で、\hbox{}\kana{呼吸}{いき}の苦しくなるのを防いでいた。\hbox{}\par{}
「大病は\kana{好}{い}いが、\hbox{}ちょっとした\kana{風邪}{かぜ}などはかえって\kana{厭}{いや}なものですね」といった先生は、\hbox{}苦笑しながら私の顔を見た。\hbox{}\par{}
 先生は病気という病気をした事のない人であった。\hbox{}先生の言葉を聞いた私は笑いたくなった。\hbox{}\par{}
「私は風邪ぐらいなら我慢しますが、\hbox{}それ以上の病気は\kana{真平}{まっぴら}です。\hbox{}先生だって同じ事でしょう。\hbox{}試みにやってご覧になるとよく\kana{解}{わか}ります」\par{}
「そうかね。\hbox{}私は病気になるくらいなら、\hbox{}死病に\kana{罹}{かか}りたいと思ってる」\par{}
 私は先生のいう事に格別注意を払わなかった。\hbox{}すぐ母の手紙の話をして、\hbox{}金の無心を申し出た。\hbox{}\par{}
「そりゃ困るでしょう。\hbox{}そのくらいなら今手元にあるはずだから持って行きたまえ」\par{}
 先生は奥さんを呼んで、\hbox{}必要の金額を私の前に並べさせてくれた。\hbox{}それを奥の\kana{茶箪笥}{ちゃだんす}か何かの\kana{抽出}{ひきだし}から出して来た奥さんは、\hbox{}白い半紙の上へ\kana{鄭寧}{ていねい}に重ねて、\hbox{}「そりゃご心配ですね」といった。\hbox{}\par{}
「\kana{何遍}{なんべん}も卒倒したんですか」と先生が聞いた。\hbox{}\par{}
「手紙には何とも書いてありませんが。\hbox{}――そんなに何度も引ッ繰り返るものですか」\par{}
「ええ」\par{}
 先生の奥さんの母親という人も私の父と同じ病気で亡くなったのだという事が始めて私に解った。\hbox{}\par{}
「どうせむずかしいんでしょう」と私がいった。\hbox{}\par{}
「そうさね。\hbox{}私が代られれば代ってあげても\kana{好}{い}いが。\hbox{}――\kana{嘔気}{はきけ}はあるんですか」\par{}
「どうですか、\hbox{}何とも書いてないから、\hbox{}\kana{大方}{おおかた}ないんでしょう」\par{}
「吐気さえ来なければまだ大丈夫ですよ」と奥さんがいった。\hbox{}\par{}
 私はその晩の汽車で東京を立った。\hbox{}\par{}\par{}     二十二
\par{}
 父の病気は思ったほど悪くはなかった。\hbox{}それでも着いた時は、\hbox{}\kana{床}{とこ}の上に\kana{胡坐}{あぐら}をかいて、\hbox{}「みんなが心配するから、\hbox{}まあ我慢してこう\kana{凝}{じっ}としている。\hbox{}なにもう起きても\kana{好}{い}いのさ」といった。\hbox{}しかしその\kana{翌日}{よくじつ}からは母が止めるのも聞かずに、\hbox{}とうとう床を上げさせてしまった。\hbox{}母は\kana{不承無性}{ふしょうぶしょう}に\kana{太織}{ふとお}りの\kana{蒲団}{ふとん}を畳みながら「お父さんはお前が帰って来たので、\hbox{}急に気が強くおなりなんだよ」といった。\hbox{}\kana{私}{わたくし}には父の挙動がさして虚勢を張っているようにも思えなかった。\hbox{}\par{}
 私の兄はある職を帯びて遠い九州にいた。\hbox{}これは万一の事がある場合でなければ、\hbox{}容易に\kana{父母}{ちちはは}の顔を見る自由の\kana{利}{き}かない男であった。\hbox{}妹は他国へ\kana{嫁}{とつ}いだ。\hbox{}これも急場の間に合うように、\hbox{}おいそれと呼び寄せられる女ではなかった。\hbox{}\kana{兄妹}{きょうだい}三人のうちで、\hbox{}一番便利なのはやはり書生をしている私だけであった。\hbox{}その私が母のいい付け通り学校の課業を\kana{放}{ほう}り出して、\hbox{}休み前に帰って来たという事が、\hbox{}父には大きな満足であった。\hbox{}\par{}
「これしきの病気に学校を休ませては気の毒だ。\hbox{}お母さんがあまり\kana{仰山}{ぎょうさん}な手紙を書くものだからいけない」\par{}
 父は口ではこういった。\hbox{}こういったばかりでなく、\hbox{}今まで敷いていた\kana{床}{とこ}を上げさせて、\hbox{}いつものような元気を示した。\hbox{}\par{}
「あんまり軽はずみをしてまた\kana{逆回}{ぶりかえ}すといけませんよ」\par{}
 私のこの注意を父は愉快そうにしかし\kana{極}{きわ}めて軽く受けた。\hbox{}\par{}
「なに大丈夫、\hbox{}これでいつものように\kana{要心}{ようじん}さえしていれば」\par{}
 実際父は大丈夫らしかった。\hbox{}家の中を自由に往来して、\hbox{}息も切れなければ、\hbox{}\kana{眩暈}{めまい}も感じなかった。\hbox{}ただ顔色だけは普通の人よりも大変悪かったが、\hbox{}これはまた今始まった症状でもないので、\hbox{}私たちは格別それを気に留めなかった。\hbox{}\par{}
 私は先生に手紙を書いて\kana{恩借}{おんしゃく}の礼を述べた。\hbox{}正月上京する時に持参するからそれまで待ってくれるようにと断わった。\hbox{}そうして父の病状の思ったほど険悪でない事、\hbox{}この分なら当分安心な事、\hbox{}眩暈も\kana{嘔気}{はきけ}も皆無な事などを書き連ねた。\hbox{}最後に先生の\kana{風邪}{ふうじゃ}についても\kana{一言}{いちごん}の見舞を\kana{附}{つ}け加えた。\hbox{}私は先生の風邪を実際軽く見ていたので。\hbox{}\par{}
 私はその手紙を出す時に決して先生の返事を予期していなかった。\hbox{}出した後で父や母と先生の\kana{噂}{うわさ}などをしながら、\hbox{}\kana{遥}{はる}かに先生の書斎を想像した。\hbox{}\par{}
「こんど東京へ行くときには\kana{椎茸}{しいたけ}でも持って行ってお上げ」\par{}
「ええ、\hbox{}しかし先生が干した椎茸なぞを食うかしら」\par{}
「\kana{旨}{うま}くはないが、\hbox{}別に\kana{嫌}{きら}いな人もないだろう」\par{}
 私には椎茸と先生を結び付けて考えるのが変であった。\hbox{}\par{}
 先生の返事が来た時、\hbox{}私はちょっと驚かされた。\hbox{}ことにその内容が特別の用件を含んでいなかった時、\hbox{}驚かされた。\hbox{}先生はただ親切ずくで、\hbox{}返事を書いてくれたんだと私は思った。\hbox{}そう思うと、\hbox{}その簡単な一本の手紙が私には大層な喜びになった。\hbox{}もっともこれは私が先生から受け取った第一の手紙には相違なかったが。\hbox{}\par{}
 第一というと私と先生の間に書信の往復がたびたびあったように思われるが、\hbox{}事実は決してそうでない事をちょっと断わっておきたい。\hbox{}私は先生の生前にたった二通の手紙しか\kana{貰}{もら}っていない。\hbox{}その一通は今いうこの簡単な返書で、\hbox{}あとの一通は先生の死ぬ前とくに私\kana{宛}{あて}で書いた大変長いものである。\hbox{}\par{}
 父は病気の性質として、\hbox{}運動を慎まなければならないので、\hbox{}床を上げてからも、\hbox{}ほとんど\kana{戸外}{そと}へは出なかった。\hbox{}一度天気のごく穏やかな日の午後庭へ下りた事があるが、\hbox{}その時は万一を\kana{気遣}{きづか}って、\hbox{}私が引き添うように\kana{傍}{そば}に付いていた。\hbox{}私が心配して自分の肩へ手を掛けさせようとしても、\hbox{}父は笑って応じなかった。\hbox{}\par{}\par{}     二十三
\par{}
 \kana{私}{わたくし}は退屈な父の相手としてよく\kana{将碁盤}{しょうぎばん}に向かった。\hbox{}二人とも無精な\kana{性質}{たち}なので、\hbox{}\kana{炬燵}{こたつ}にあたったまま、\hbox{}盤を\kana{櫓}{やぐら}の上へ\kana{載}{の}せて、\hbox{}\kana{駒}{こま}を動かすたびに、\hbox{}わざわざ手を\kana{掛蒲団}{かけぶとん}の下から出すような事をした。\hbox{}時々\kana{持駒}{もちごま}を\kana{失}{な}くして、\hbox{}次の勝負の来るまで双方とも知らずにいたりした。\hbox{}それを母が灰の中から\kana{見付}{みつ}け出して、\hbox{}\kana{火箸}{ひばし}で\kana{挟}{はさ}み上げるという\kana{滑稽}{こっけい}もあった。\hbox{}\par{}
「\kana{碁}{ご}だと盤が高過ぎる上に、\hbox{}足が着いているから、\hbox{}炬燵の上では打てないが、\hbox{}そこへ来ると将碁盤は\kana{好}{い}いね、\hbox{}こうして楽に差せるから。\hbox{}無精者には持って来いだ。\hbox{}もう一番やろう」\par{}
 父は勝った時は必ずもう一番やろうといった。\hbox{}そのくせ負けた時にも、\hbox{}もう一番やろうといった。\hbox{}要するに、\hbox{}勝っても負けても、\hbox{}炬燵にあたって、\hbox{}将碁を差したがる男であった。\hbox{}始めのうちは珍しいので、\hbox{}この\kana{隠居}{いんきょ}じみた娯楽が私にも相当の興味を与えたが、\hbox{}少し時日が\kana{経}{た}つに\kana{伴}{つ}れて、\hbox{}若い私の気力はそのくらいな\kana{刺戟}{しげき}で満足できなくなった。\hbox{}私は\kana{金}{きん}や\kana{香車}{きょうしゃ}を握った\kana{拳}{こぶし}を頭の上へ伸ばして、\hbox{}時々思い切ったあくびをした。\hbox{}\par{}
 私は東京の事を考えた。\hbox{}そうして\kana{漲}{みなぎ}る心臓の血潮の奥に、\hbox{}活動活動と打ちつづける\kana{鼓動}{こどう}を聞いた。\hbox{}不思議にもその鼓動の音が、\hbox{}ある微妙な意識状態から、\hbox{}先生の力で強められているように感じた。\hbox{}\par{}
 私は心のうちで、\hbox{}父と先生とを比較して見た。\hbox{}両方とも世間から見れば、\hbox{}生きているか死んでいるか分らないほど\kana{大人}{おとな}しい男であった。\hbox{}\kana{他}{ひと}に認められるという点からいえばどっちも\kana{零}{れい}であった。\hbox{}それでいて、\hbox{}この将碁を差したがる父は、\hbox{}単なる娯楽の相手としても私には物足りなかった。\hbox{}かつて遊興のために\kana{往来}{ゆきき}をした\kana{覚}{おぼ}えのない先生は、\hbox{}歓楽の交際から出る親しみ以上に、\hbox{}いつか私の頭に影響を与えていた。\hbox{}ただ頭というのはあまりに\kana{冷}{ひや}やか過ぎるから、\hbox{}私は胸といい直したい。\hbox{}肉のなかに先生の力が\kana{喰}{く}い込んでいるといっても、\hbox{}血のなかに先生の命が流れているといっても、\hbox{}その時の私には少しも誇張でないように思われた。\hbox{}私は父が私の本当の父であり、\hbox{}先生はまたいうまでもなく、\hbox{}あかの他人であるという明白な事実を、\hbox{}ことさらに眼の前に並べてみて、\hbox{}始めて大きな真理でも発見したかのごとくに驚いた。\hbox{}\par{}
 私がのつそつし出すと前後して、\hbox{}父や母の眼にも今まで珍しかった私が段々\kana{陳腐}{ちんぷ}になって来た。\hbox{}これは夏休みなどに国へ帰る誰でもが一様に経験する心持だろうと思うが、\hbox{}当座の一週間ぐらいは下にも置かないように、\hbox{}ちやほや\kana{歓待}{もてな}されるのに、\hbox{}その峠を\kana{定規通}{ていきどお}り通り越すと、\hbox{}あとはそろそろ家族の熱が冷めて来て、\hbox{}しまいには有っても無くっても構わないもののように粗末に取り扱われがちになるものである。\hbox{}私も滞在中にその峠を通り越した。\hbox{}その上私は国へ帰るたびに、\hbox{}父にも母にも\kana{解}{わか}らない変なところを東京から持って帰った。\hbox{}昔でいうと、\hbox{}\kana{儒者}{じゅしゃ}の家へ\kana{切支丹}{キリシタン}の\kana{臭}{にお}いを持ち込むように、\hbox{}私の持って帰るものは父とも母とも調和しなかった。\hbox{}無論私はそれを隠していた。\hbox{}けれども元々身に着いているものだから、\hbox{}出すまいと思っても、\hbox{}いつかそれが父や母の眼に\kana{留}{と}まった。\hbox{}私はつい面白くなくなった。\hbox{}早く東京へ帰りたくなった。\hbox{}\par{}
 父の病気は幸い現状維持のままで、\hbox{}少しも悪い方へ進む模様は見えなかった。\hbox{}念のためにわざわざ遠くから相当の医者を招いたりして、\hbox{}慎重に診察してもらってもやはり私の知っている以外に異状は認められなかった。\hbox{}私は冬休みの尽きる少し前に国を立つ事にした。\hbox{}立つといい出すと、\hbox{}人情は妙なもので、\hbox{}父も母も反対した。\hbox{}\par{}
「もう帰るのかい、\hbox{}まだ早いじゃないか」と母がいった。\hbox{}\par{}
「まだ四、\hbox{}五日いても間に合うんだろう」と父がいった。\hbox{}\par{}
 私は自分の\kana{極}{き}めた\kana{出立}{しゅったつ}の日を動かさなかった。\hbox{}\par{}\par{}     二十四
\par{}
 東京へ帰ってみると、\hbox{}\kana{松飾}{まつかざり}はいつか取り払われていた。\hbox{}町は寒い風の吹くに任せて、\hbox{}どこを見てもこれというほどの正月めいた景気はなかった。\hbox{}\par{}
 \kana{私}{わたくし}は\kana{早速}{さっそく}先生のうちへ金を返しに行った。\hbox{}例の\kana{椎茸}{しいたけ}もついでに持って行った。\hbox{}ただ出すのは少し変だから、\hbox{}母がこれを差し上げてくれといいましたとわざわざ断って奥さんの前へ置いた。\hbox{}椎茸は新しい菓子折に入れてあった。\hbox{}\kana{鄭寧}{ていねい}に礼を述べた奥さんは、\hbox{}次の\kana{間}{ま}へ立つ時、\hbox{}その折を持って見て、\hbox{}軽いのに驚かされたのか、\hbox{}「こりゃ何の\kana{御菓子}{おかし}」と聞いた。\hbox{}奥さんは懇意になると、\hbox{}こんなところに\kana{極}{きわ}めて\kana{淡泊}{たんぱく}な\kana{小供}{こども}らしい心を見せた。\hbox{}\par{}
 二人とも父の病気について、\hbox{}色々\kana{掛念}{けねん}の問いを繰り返してくれた中に、\hbox{}先生はこんな事をいった。\hbox{}\par{}
「なるほど\kana{容体}{ようだい}を聞くと、\hbox{}今が今どうという事もないようですが、\hbox{}病気が病気だからよほど気をつけないといけません」\par{}
 先生は\kana{腎臓}{じんぞう}の\kana{病}{やまい}について私の知らない事を多く知っていた。\hbox{}\par{}
「自分で病気に\kana{罹}{かか}っていながら、\hbox{}気が付かないで平気でいるのがあの病の特色です。\hbox{}私の知ったある\kana{士官}{しかん}は、\hbox{}とうとうそれでやられたが、\hbox{}全く\kana{嘘}{うそ}のような死に方をしたんですよ。\hbox{}何しろ\kana{傍}{そば}に寝ていた\kana{細君}{さいくん}が看病をする暇もなんにもないくらいなんですからね。\hbox{}夜中にちょっと苦しいといって、\hbox{}細君を起したぎり、\hbox{}\kana{翌}{あく}る朝はもう死んでいたんです。\hbox{}しかも細君は夫が寝ているとばかり思ってたんだっていうんだから」\par{}
 今まで楽天的に傾いていた私は急に不安になった。\hbox{}\par{}
「私の\kana{父}{おやじ}もそんなになるでしょうか。\hbox{}ならんともいえないですね」\par{}
「医者は何というのです」\par{}
「医者は\kana{到底}{とても}治らないというんです。\hbox{}けれども当分のところ心配はあるまいともいうんです」\par{}
「それじゃ\kana{好}{い}いでしょう。\hbox{}医者がそういうなら。\hbox{}私の今話したのは気が付かずにいた人の事で、\hbox{}しかもそれがずいぶん乱暴な軍人なんだから」\par{}
 私はやや安心した。\hbox{}私の変化を\kana{凝}{じっ}と見ていた先生は、\hbox{}それからこう付け足した。\hbox{}\par{}
「しかし人間は健康にしろ病気にしろ、\hbox{}どっちにしても\kana{脆}{もろ}いものですね。\hbox{}いつどんな事でどんな死にようをしないとも限らないから」\par{}
「先生もそんな事を考えてお\kana{出}{いで}ですか」\par{}
「いくら丈夫の私でも、\hbox{}\kana{満更}{まんざら}考えない事もありません」\par{}
 先生の口元には微笑の影が見えた。\hbox{}\par{}
「よくころりと死ぬ人があるじゃありませんか。\hbox{}自然に。\hbox{}それからあっと思う\kana{間}{ま}に死ぬ人もあるでしょう。\hbox{}不自然な暴力で」\par{}
「不自然な暴力って何ですか」\par{}
「何だかそれは私にも\kana{解}{わか}らないが、\hbox{}自殺する人はみんな不自然な暴力を使うんでしょう」\par{}
「すると殺されるのも、\hbox{}やはり不自然な暴力のお\kana{蔭}{かげ}ですね」\par{}
「殺される方はちっとも考えていなかった。\hbox{}なるほどそういえばそうだ」\par{}
 その日はそれで帰った。\hbox{}帰ってからも父の病気はそれほど苦にならなかった。\hbox{}先生のいった自然に死ぬとか、\hbox{}不自然の暴力で死ぬとかいう言葉も、\hbox{}その場限りの浅い印象を与えただけで、\hbox{}\kana{後}{あと}は何らのこだわりを私の頭に残さなかった。\hbox{}私は今まで\kana{幾度}{いくたび}か手を着けようとしては手を引っ込めた卒業論文を、\hbox{}いよいよ本式に書き始めなければならないと思い出した。\hbox{}\par{}\par{}     二十五
\par{}
 その年の六月に卒業するはずの\kana{私}{わたくし}は、\hbox{}ぜひともこの論文を\kana{成規通}{せいきどお}り四月いっぱいに書き上げてしまわなければならなかった。\hbox{}二、\hbox{}三、\hbox{}四と指を折って余る時日を勘定して見た時、\hbox{}私は少し自分の度胸を\kana{疑}{うたぐ}った。\hbox{}\kana{他}{ほか}のものはよほど前から材料を\kana{蒐}{あつ}めたり、\hbox{}ノートを\kana{溜}{た}めたりして、\hbox{}\kana{余所目}{よそめ}にも\kana{忙}{いそが}しそうに見えるのに、\hbox{}私だけはまだ何にも手を着けずにいた。\hbox{}私にはただ年が改まったら大いにやろうという決心だけがあった。\hbox{}私はその決心でやり出した。\hbox{}そうして\kana{忽}{たちま}ち動けなくなった。\hbox{}今まで大きな問題を\kana{空}{くう}に\kana{描}{えが}いて、\hbox{}骨組みだけはほぼでき上っているくらいに考えていた私は、\hbox{}頭を\kana{抑}{おさ}えて悩み始めた。\hbox{}私はそれから論文の問題を小さくした。\hbox{}そうして練り上げた思想を系統的に\kana{纏}{まと}める手数を省くために、\hbox{}ただ書物の中にある材料を並べて、\hbox{}それに相当な結論をちょっと付け加える事にした。\hbox{}\par{}
 私の選択した問題は先生の専門と縁故の近いものであった。\hbox{}私がかつてその選択について先生の意見を尋ねた時、\hbox{}先生は\kana{好}{い}いでしょうといった。\hbox{}\kana{狼狽}{ろうばい}した気味の私は、\hbox{}\kana{早速}{さっそく}先生の所へ出掛けて、\hbox{}私の読まなければならない参考書を聞いた。\hbox{}先生は自分の知っている限りの知識を、\hbox{}快く私に与えてくれた上に、\hbox{}必要の書物を、\hbox{}二、\hbox{}三冊貸そうといった。\hbox{}しかし先生はこの点について\kana{毫}{ごう}も私を指導する任に当ろうとしなかった。\hbox{}\par{}
「\kana{近頃}{ちかごろ}はあんまり書物を読まないから、\hbox{}新しい事は知りませんよ。\hbox{}学校の先生に聞いた方が好いでしょう」\par{}
 先生は一時非常の読書家であったが、\hbox{}その\kana{後}{ご}どういう訳か、\hbox{}前ほどこの方面に興味が働かなくなったようだと、\hbox{}かつて奥さんから聞いた事があるのを、\hbox{}私はその時ふと思い出した。\hbox{}私は論文をよそにして、\hbox{}そぞろに口を開いた。\hbox{}\par{}
「先生はなぜ元のように書物に興味をもち得ないんですか」\par{}
「なぜという訳もありませんが。\hbox{}……つまりいくら本を読んでもそれほどえらくならないと思うせいでしょう。\hbox{}それから……」\par{}
「それから、\hbox{}まだあるんですか」\par{}
「まだあるというほどの理由でもないが、\hbox{}以前はね、\hbox{}人の前へ出たり、\hbox{}人に聞かれたりして知らないと恥のようにきまりが悪かったものだが、\hbox{}近頃は知らないという事が、\hbox{}それほどの恥でないように見え出したものだから、\hbox{}つい無理にも本を読んでみようという元気が出なくなったのでしょう。\hbox{}まあ早くいえば老い込んだのです」\par{}
 先生の言葉はむしろ平静であった。\hbox{}世間に背中を向けた人の\kana{苦味}{くみ}を帯びていなかっただけに、\hbox{}私にはそれほどの\kana{手応}{てごた}えもなかった。\hbox{}私は先生を老い込んだとも思わない代りに、\hbox{}偉いとも感心せずに帰った。\hbox{}\par{}
 それからの私はほとんど論文に\kana{祟}{たた}られた精神病者のように眼を赤くして苦しんだ。\hbox{}私は一年\kana{前}{ぜん}に卒業した友達について、\hbox{}色々様子を聞いてみたりした。\hbox{}そのうちの\kana{一人}{いちにん}は\kana{締切}{しめきり}の日に車で事務所へ\kana{馳}{か}けつけて\kana{漸}{ようや}く間に合わせたといった。\hbox{}他の一人は五時を十五分ほど\kana{後}{おく}らして持って行ったため、\hbox{}\kana{危}{あやう}く\kana{跳}{は}ね付けられようとしたところを、\hbox{}主任教授の好意でやっと受理してもらったといった。\hbox{}私は不安を感ずると共に度胸を\kana{据}{す}えた。\hbox{}毎日机の前で精根のつづく限り働いた。\hbox{}でなければ、\hbox{}薄暗い書庫にはいって、\hbox{}高い本棚のあちらこちらを\kana{見廻}{みまわ}した。\hbox{}私の眼は\kana{好事家}{こうずか}が\kana{骨董}{こっとう}でも掘り出す時のように背表紙の金文字をあさった。\hbox{}\par{}
 梅が咲くにつけて寒い風は段々\kana{向}{むき}を南へ\kana{更}{か}えて行った。\hbox{}それが\kana{一仕切}{ひとしきり}\kana{経}{た}つと、\hbox{}桜の\kana{噂}{うわさ}がちらほら私の耳に聞こえ出した。\hbox{}それでも私は馬車馬のように正面ばかり見て、\hbox{}論文に\kana{鞭}{むち}うたれた。\hbox{}私はついに四月の下旬が来て、\hbox{}やっと予定通りのものを書き上げるまで、\hbox{}先生の敷居を\kana{跨}{また}がなかった。\hbox{}\par{}\par{}     二十六
\par{}
 \kana{私}{わたくし}の自由になったのは、\hbox{}\kana{八重桜}{やえざくら}の散った枝にいつしか青い葉が\kana{霞}{かす}むように伸び始める初夏の季節であった。\hbox{}私は\kana{籠}{かご}を抜け出した小鳥の心をもって、\hbox{}広い天地を\kana{一目}{ひとめ}に見渡しながら、\hbox{}自由に\kana{羽搏}{はばた}きをした。\hbox{}私はすぐ先生の\kana{家}{うち}へ行った。\hbox{}\kana{枳殻}{からたち}の垣が黒ずんだ枝の上に、\hbox{}\kana{萌}{もえ}るような芽を吹いていたり、\hbox{}\kana{柘榴}{ざくろ}の枯れた幹から、\hbox{}つやつやしい茶褐色の葉が、\hbox{}柔らかそうに日光を映していたりするのが、\hbox{}道々私の眼を引き付けた。\hbox{}私は生れて初めてそんなものを見るような珍しさを覚えた。\hbox{}\par{}
 先生は\kana{嬉}{うれ}しそうな私の顔を見て、\hbox{}「もう論文は片付いたんですか、\hbox{}結構ですね」といった。\hbox{}私は「お\kana{蔭}{かげ}でようやく済みました。\hbox{}もう何にもする事はありません」といった。\hbox{}\par{}
 実際その時の私は、\hbox{}自分のなすべきすべての仕事がすでに\kana{結了}{けつりょう}して、\hbox{}これから先は威張って遊んでいても構わないような晴やかな心持でいた。\hbox{}私は書き上げた自分の論文に対して充分の自信と満足をもっていた。\hbox{}私は先生の前で、\hbox{}しきりにその内容を\kana{喋々}{ちょうちょう}した。\hbox{}先生はいつもの調子で、\hbox{}「なるほど」とか、\hbox{}「そうですか」とかいってくれたが、\hbox{}それ以上の批評は少しも加えなかった。\hbox{}私は物足りないというよりも、\hbox{}\kana{聊}{いささ}か拍子抜けの気味であった。\hbox{}それでもその日私の気力は、\hbox{}\kana{因循}{いんじゅん}らしく見える先生の態度に逆襲を試みるほどに\kana{生々}{いきいき}していた。\hbox{}私は青く\kana{蘇生}{よみがえ}ろうとする大きな自然の中に、\hbox{}先生を誘い出そうとした。\hbox{}\par{}
「先生どこかへ散歩しましょう。\hbox{}外へ出ると大変\kana{好}{い}い心持です」\par{}
「どこへ」\par{}
 私はどこでも構わなかった。\hbox{}ただ先生を\kana{伴}{つ}れて郊外へ出たかった。\hbox{}\par{}
 一時間の\kana{後}{のち}、\hbox{}先生と私は目的どおり市を離れて、\hbox{}村とも町とも区別の付かない静かな所を\kana{宛}{あて}もなく歩いた。\hbox{}私はかなめの垣から若い柔らかい葉を\kana{}{も}ぎ取って\kana{芝笛}{しばぶえ}を鳴らした。\hbox{}ある\kana{鹿児島人}{かごしまじん}を友達にもって、\hbox{}その人の\kana{真似}{まね}をしつつ自然に習い覚えた私は、\hbox{}この芝笛というものを鳴らす事が上手であった。\hbox{}私が得意にそれを吹きつづけると、\hbox{}先生は知らん顔をしてよそを向いて歩いた。\hbox{}\par{}
 やがて若葉に\kana{鎖}{と}ざされたように\kana{蓊欝}{こんもり}した小高い\kana{一構}{ひとかま}えの下に細い\kana{路}{みち}が\kana{開}{ひら}けた。\hbox{}門の柱に打ち付けた標札に何々園とあるので、\hbox{}その個人の邸宅でない事がすぐ知れた。\hbox{}先生はだらだら\kana{上}{のぼ}りになっている入口を\kana{眺}{なが}めて、\hbox{}「はいってみようか」といった。\hbox{}私はすぐ「植木屋ですね」と答えた。\hbox{}\par{}
 \kana{植込}{うえこみ}の中を\kana{一}{ひと}うねりして奥へ\kana{上}{のぼ}ると左側に\kana{家}{うち}があった。\hbox{}明け放った\kana{障子}{しょうじ}の内はがらんとして人の影も見えなかった。\hbox{}ただ\kana{軒先}{のきさき}に据えた大きな鉢の中に飼ってある金魚が動いていた。\hbox{}\par{}
「静かだね。\hbox{}断わらずにはいっても構わないだろうか」\par{}
「構わないでしょう」\par{}
 二人はまた奥の方へ進んだ。\hbox{}しかしそこにも人影は見えなかった。\hbox{}\kana{躑躅}{つつじ}が燃えるように咲き乱れていた。\hbox{}先生はそのうちで\kana{樺色}{かばいろ}の\kana{丈}{たけ}の高いのを指して、\hbox{}「これは\kana{霧島}{きりしま}でしょう」といった。\hbox{}\par{}
 \kana{芍薬}{しゃくやく}も\kana{十坪}{とつぼ}あまり一面に植え付けられていたが、\hbox{}まだ季節が来ないので花を着けているのは一本もなかった。\hbox{}この芍薬\kana{畠}{ばたけ}の\kana{傍}{そば}にある古びた縁台のようなものの上に先生は大の字なりに寝た。\hbox{}私はその余った\kana{端}{はじ}の方に腰をおろして\kana{烟草}{タバコ}を吹かした。\hbox{}先生は\kana{蒼}{あお}い\kana{透}{す}き\kana{徹}{とお}るような空を見ていた。\hbox{}私は私を包む若葉の色に心を奪われていた。\hbox{}その若葉の色をよくよく\kana{眺}{なが}めると、\hbox{}一々違っていた。\hbox{}同じ\kana{楓}{かえで}の\kana{樹}{き}でも同じ色を枝に着けているものは一つもなかった。\hbox{}細い杉苗の\kana{頂}{いただき}に投げ\kana{被}{かぶ}せてあった先生の帽子が風に吹かれて落ちた。\hbox{}\par{}\par{}     二十七
\par{}
 \kana{私}{わたくし}はすぐその帽子を取り上げた。\hbox{}\kana{所々}{ところどころ}に着いている赤土を\kana{爪}{つめ}で\kana{弾}{はじ}きながら先生を呼んだ。\hbox{}\par{}
「先生帽子が落ちました」\par{}
「ありがとう」\par{}
 \kana{身体}{からだ}を半分起してそれを受け取った先生は、\hbox{}起きるとも寝るとも片付かないその姿勢のままで、\hbox{}変な事を私に聞いた。\hbox{}\par{}
「突然だが、\hbox{}君の\kana{家}{うち}には財産がよっぽどあるんですか」\par{}
「あるというほどありゃしません」\par{}
「まあどのくらいあるのかね。\hbox{}失礼のようだが」\par{}
「どのくらいって、\hbox{}山と\kana{田地}{でんぢ}が少しあるぎりで、\hbox{}金なんかまるでないんでしょう」\par{}
 先生が私の\kana{家}{いえ}の経済について、\hbox{}問いらしい問いを掛けたのはこれが始めてであった。\hbox{}私の方はまだ先生の暮し向きに関して、\hbox{}何も聞いた事がなかった。\hbox{}先生と知り合いになった始め、\hbox{}私は先生がどうして遊んでいられるかを\kana{疑}{うたぐ}った。\hbox{}その後もこの疑いは絶えず私の胸を去らなかった。\hbox{}しかし私はそんな\kana{露骨}{あらわ}な問題を先生の前に持ち出すのをぶしつけとばかり思っていつでも控えていた。\hbox{}若葉の色で疲れた眼を休ませていた私の心は、\hbox{}偶然またその疑いに触れた。\hbox{}\par{}
「先生はどうなんです。\hbox{}どのくらいの財産をもっていらっしゃるんですか」\par{}
「私は財産家と見えますか」\par{}
 先生は平生からむしろ質素な\kana{服装}{なり}をしていた。\hbox{}それに\kana{家内}{かない}は\kana{小人数}{こにんず}であった。\hbox{}したがって住宅も決して広くはなかった。\hbox{}けれどもその生活の物質的に豊かな事は、\hbox{}内輪にはいり込まない私の眼にさえ明らかであった。\hbox{}要するに先生の暮しは\kana{贅沢}{ぜいたく}といえないまでも、\hbox{}あたじけなく切り詰めた無弾力性のものではなかった。\hbox{}\par{}
「そうでしょう」と私がいった。\hbox{}\par{}
「そりゃそのくらいの金はあるさ、\hbox{}けれども決して財産家じゃありません。\hbox{}財産家ならもっと大きな\kana{家}{うち}でも造るさ」\par{}
 この時先生は起き上って、\hbox{}縁台の上に\kana{胡坐}{あぐら}をかいていたが、\hbox{}こういい終ると、\hbox{}竹の\kana{杖}{つえ}の先で地面の上へ円のようなものを\kana{描}{か}き始めた。\hbox{}それが済むと、\hbox{}今度はステッキを突き刺すように\kana{真直}{まっすぐ}に立てた。\hbox{}\par{}
「これでも元は財産家なんだがなあ」\par{}
 先生の言葉は半分\kana{独}{ひと}り\kana{言}{ごと}のようであった。\hbox{}それですぐ\kana{後}{あと}に\kana{尾}{つ}いて行き損なった私は、\hbox{}つい黙っていた。\hbox{}\par{}
「これでも元は財産家なんですよ、\hbox{}君」といい直した先生は、\hbox{}次に私の顔を見て微笑した。\hbox{}私はそれでも何とも答えなかった。\hbox{}むしろ不調法で答えられなかったのである。\hbox{}すると先生がまた問題を\kana{他}{よそ}へ移した。\hbox{}\par{}
「あなたのお父さんの病気はその後どうなりました」\par{}
 私は父の病気について正月以後何にも知らなかった。\hbox{}月々国から送ってくれる\kana{為替}{かわせ}と共に来る簡単な手紙は、\hbox{}例の通り父の\kana{手蹟}{しゅせき}であったが、\hbox{}病気の訴えはそのうちにほとんど見当らなかった。\hbox{}その上書体も確かであった。\hbox{}この種の病人に見る\kana{顫}{ふる}えが少しも筆の\kana{運}{はこ}びを乱していなかった。\hbox{}\par{}
「何ともいって来ませんが、\hbox{}もう\kana{好}{い}いんでしょう」\par{}
「\kana{好}{よ}ければ結構だが、\hbox{}――病症が病症なんだからね」\par{}
「やっぱり駄目ですかね。\hbox{}でも当分は持ち合ってるんでしょう。\hbox{}何ともいって来ませんよ」\par{}
「そうですか」\par{}
 私は先生が私のうちの財産を聞いたり、\hbox{}私の父の病気を尋ねたりするのを、\hbox{}普通の談話――胸に浮かんだままをその通り口にする、\hbox{}普通の談話と思って聞いていた。\hbox{}ところが先生の言葉の底には両方を結び付ける大きな意味があった。\hbox{}先生自身の経験を持たない私は無論そこに気が付くはずがなかった。\hbox{}\par{}\par{}     二十八
\par{}
「君のうちに財産があるなら、\hbox{}今のうちによく始末をつけてもらっておかないといけないと思うがね、\hbox{}余計なお世話だけれども。\hbox{}君のお父さんが達者なうちに、\hbox{}\kana{貰}{もら}うものはちゃんと貰っておくようにしたらどうですか。\hbox{}万一の事があったあとで、\hbox{}一番面倒の起るのは財産の問題だから」\par{}
「ええ」\par{}
 \kana{私}{わたくし}は先生の言葉に大した注意を払わなかった。\hbox{}私の家庭でそんな心配をしているものは、\hbox{}私に限らず、\hbox{}父にしろ母にしろ、\hbox{}一人もないと私は信じていた。\hbox{}その上先生のいう事の、\hbox{}先生として、\hbox{}あまりに実際的なのに私は少し驚かされた。\hbox{}しかしそこは年長者に対する平生の敬意が私を無口にした。\hbox{}\par{}
「あなたのお父さんが亡くなられるのを、\hbox{}今から予想してかかるような\kana{言葉遣}{ことばづか}いをするのが気に\kana{触}{さわ}ったら許してくれたまえ。\hbox{}しかし人間は死ぬものだからね。\hbox{}どんなに達者なものでも、\hbox{}いつ死ぬか分らないものだからね」\par{}
 先生の\kana{口気}{こうき}は珍しく苦々しかった。\hbox{}\par{}
「そんな事をちっとも気に掛けちゃいません」と私は弁解した。\hbox{}\par{}
「君の\kana{兄弟}{きょうだい}は何人でしたかね」と先生が聞いた。\hbox{}\par{}
 先生はその上に私の家族の\kana{人数}{にんず}を聞いたり、\hbox{}親類の有無を尋ねたり、\hbox{}\kana{叔父}{おじ}や\kana{叔母}{おば}の様子を問いなどした。\hbox{}そうして最後にこういった。\hbox{}\par{}
「みんな\kana{善}{い}い人ですか」\par{}
「別に悪い人間というほどのものもいないようです。\hbox{}大抵\kana{田舎者}{いなかもの}ですから」\par{}
「田舎者はなぜ悪くないんですか」\par{}
 私はこの\kana{追窮}{ついきゅう}に苦しんだ。\hbox{}しかし先生は私に返事を考えさせる余裕さえ与えなかった。\hbox{}\par{}
「田舎者は都会のものより、\hbox{}かえって悪いくらいなものです。\hbox{}それから、\hbox{}君は今、\hbox{}君の\kana{親戚}{しんせき}なぞの\kana{中}{うち}に、\hbox{}これといって、\hbox{}悪い人間はいないようだといいましたね。\hbox{}しかし悪い人間という一種の人間が世の中にあると君は思っているんですか。\hbox{}そんな\kana{鋳型}{いかた}に入れたような悪人は世の中にあるはずがありませんよ。\hbox{}平生はみんな善人なんです。\hbox{}少なくともみんな普通の人間なんです。\hbox{}それが、\hbox{}いざという間際に、\hbox{}急に悪人に変るんだから恐ろしいのです。\hbox{}だから油断ができないんです」\par{}
 先生のいう事は、\hbox{}ここで切れる様子もなかった。\hbox{}私はまたここで何かいおうとした。\hbox{}すると\kana{後}{うし}ろの方で犬が急に\kana{吠}{ほ}え出した。\hbox{}先生も私も驚いて後ろを振り返った。\hbox{}\par{}
 縁台の横から後部へ掛けて植え付けてある杉苗の\kana{傍}{そば}に、\hbox{}\kana{熊笹}{くまざさ}が\kana{三坪}{みつぼ}ほど地を隠すように茂って生えていた。\hbox{}犬はその顔と背を熊笹の上に現わして、\hbox{}盛んに吠え立てた。\hbox{}そこへ\kana{十}{とお}ぐらいの\kana{小供}{こども}が\kana{馳}{か}けて来て犬を\kana{叱}{しか}り付けた。\hbox{}小供は\kana{徽章}{きしょう}の着いた黒い帽子を\kana{被}{かぶ}ったまま先生の前へ\kana{廻}{まわ}って礼をした。\hbox{}\par{}
「叔父さん、\hbox{}はいって来る時、\hbox{}\kana{家}{うち}に\kana{誰}{だれ}もいなかったかい」と聞いた。\hbox{}\par{}
「誰もいなかったよ」\par{}
「姉さんやおっかさんが勝手の方にいたのに」\par{}
「そうか、\hbox{}いたのかい」\par{}
「ああ。\hbox{}叔父さん、\hbox{}\kana{今日}{こんち}はって、\hbox{}断ってはいって来ると\kana{好}{よ}かったのに」\par{}
 先生は苦笑した。\hbox{}\kana{懐中}{ふところ}から\kana{蟇口}{がまぐち}を出して、\hbox{}五銭の\kana{白銅}{はくどう}を小供の手に握らせた。\hbox{}\par{}
「おっかさんにそういっとくれ。\hbox{}少しここで休まして下さいって」\par{}
 小供は\kana{怜悧}{りこう}そうな眼に\kana{笑}{わら}いを\kana{漲}{みなぎ}らして、\hbox{}\kana{首肯}{うなず}いて見せた。\hbox{}\par{}
「今\kana{斥候長}{せっこうちょう}になってるところなんだよ」\par{}
   小供はこう断って、\hbox{}\kana{躑躅}{つつじ}の間を下の方へ駈け下りて行った。\hbox{}犬も\kana{尻尾}{しっぽ}を高く巻いて小供の後を追い掛けた。\hbox{}しばらくすると同じくらいの年格好の小供が二、\hbox{}三人、\hbox{}これも斥候長の下りて行った方へ駈けていった。\hbox{}\par{}\par{}
 私は\kana{妻}{さい}を残して行きます。\hbox{}私がいなくなっても妻に衣食住の心配がないのは\kana{仕合}{しあわ}せです。\hbox{}私は妻に残酷な\kana{驚怖}{きょうふ}を与える事を好みません。\hbox{}私は妻に血の色を見せないで死ぬつもりです。\hbox{}妻の知らない\kana{間}{ま}に、\hbox{}こっそりこの世からいなくなるようにします。\hbox{}私は死んだ後で、\hbox{}妻から\kana{頓死}{とんし}したと思われたいのです。\hbox{}気が狂ったと思われても満足なのです。\hbox{}\par{}
 私が死のうと決心してから、\hbox{}もう十日以上になりますが、\hbox{}その大部分はあなたにこの長い自叙伝の一節を書き残すために使用されたものと思って下さい。\hbox{}始めはあなたに会って話をする気でいたのですが、\hbox{}書いてみると、\hbox{}かえってその方が自分を\kana{判然}{はっきり}\kana{描}{えが}き出す事ができたような心持がして\kana{嬉}{うれ}しいのです。\hbox{}私は\kana{酔興}{すいきょう}に書くのではありません。\hbox{}私を生んだ私の過去は、\hbox{}人間の経験の一部分として、\hbox{}私より\kana{外}{ほか}に誰も語り得るものはないのですから、\hbox{}それを\kana{偽}{いつわ}りなく書き残して置く私の努力は、\hbox{}人間を知る上において、\hbox{}あなたにとっても、\hbox{}外の人にとっても、\hbox{}徒労ではなかろうと思います。\hbox{}\kana{渡辺華山}{わたなべかざん}は\kana{邯鄲}{かんたん}という\kana{画}{え}を\kana{描}{か}くために、\hbox{}死期を一週間繰り延べたという話をつい\kana{先達}{せんだっ}て聞きました。\hbox{}\kana{他}{ひと}から見たら余計な事のようにも解釈できましょうが、\hbox{}当人にはまた当人相応の要求が心の\kana{中}{うち}にあるのだからやむをえないともいわれるでしょう。\hbox{}私の努力も単にあなたに対する約束を果たすためばかりではありません。\hbox{}\kana{半}{なか}ば以上は自分自身の要求に動かされた結果なのです。\hbox{}\par{}
 しかし私は今その要求を果たしました。\hbox{}もう何にもする事はありません。\hbox{}この手紙があなたの手に落ちる\kana{頃}{ころ}には、\hbox{}私はもうこの世にはいないでしょう。\hbox{}とくに死んでいるでしょう。\hbox{}妻は十日ばかり前から\kana{市ヶ谷}{いちがや}の\kana{叔母}{おば}の所へ行きました。\hbox{}叔母が病気で手が足りないというから私が勧めてやったのです。\hbox{}私は妻の留守の\kana{間}{あいだ}に、\hbox{}この長いものの大部分を書きました。\hbox{}時々妻が帰って来ると、\hbox{}私はすぐそれを隠しました。\hbox{}\par{}
 私は私の過去を善悪ともに\kana{他}{ひと}の参考に供するつもりです。\hbox{}しかし妻だけはたった一人の例外だと承知して下さい。\hbox{}私は妻には何にも知らせたくないのです。\hbox{}妻が\kana{己}{おの}れの過去に対してもつ記憶を、\hbox{}なるべく純白に保存しておいてやりたいのが私の\kana{唯一}{ゆいいつ}の希望なのですから、\hbox{}私が死んだ\kana{後}{あと}でも、\hbox{}妻が生きている以上は、\hbox{}あなた限りに打ち明けられた私の秘密として、\hbox{}すべてを腹の中にしまっておいて下さい。\hbox{}」\par{}\par{}\par{}\par{}



        \vspace{1zw plus .1zw minus .4zw}


        

\noindent
        \hfil
        \rule{133.875pt}{.01zw}
        \hfill


\par{}
底本:「こころ」集英社文庫、\hbox{}集英社\par{}
   1991(平成3)年2月25日第1刷\par{}
   1995(平成7)年6月14日第10刷\par{}
初出:「朝日新聞」\par{}
   1914(大正3)年4月20日~8月11日\par{}
※誤植の修正は「漱石全集」岩波書店を参照しました。\hbox{}\par{}
※底本は、\hbox{}物を数える際や地名などに用いる「ヶ」(区点番号5-86)を、\hbox{}大振りにつくっています。\hbox{}\par{}
入力:j.utiyama\par{}
校正:伊藤時也\par{}
1999年7月31日公開\par{}
2010年10月31日修正\par{}
青空文庫作成ファイル:\par{}
このファイルは、\hbox{}インターネットの図書館、\hbox{}青空文庫(http://www.aozora.gr.jp/)で作られました。\hbox{}入力、\hbox{}校正、\hbox{}制作にあたったのは、\hbox{}ボランティアの皆さんです。\hbox{}\par{}\par{}\par{}



        \vspace{1zw plus .1zw minus .4zw}


        

\noindent
        \hfil
        \rule{133.875pt}{.01zw}
        \hfill


\par{}
●表記について\par{}
このファイルは W3C 勧告 XHTML1.1 にそった形式で作成されています。\hbox{}
	[#…]は、\hbox{}入力者による注を表す記号です。\hbox{}
	「くの字点」をのぞくJIS X 0213にある文字は、\hbox{}画像化して埋め込みました。\hbox{}




        \vspace{1zw plus .1zw minus .4zw}


        

\noindent
        \hfil
        \rule{133.875pt}{.01zw}
        \hfill


\par{}●図書カード




\end{document}
