\documentclass[a4j,twocolumn]{tarticle}
\usepackage[dvipdfmx]{graphicx}
\usepackage{wallpaper}
\usepackage{multicol}
\usepackage{otf}

\makeatletter
\renewcommand{\normalsize}{\@setfontsize\normalsize{12pt}{21.0}}
\renewcommand{\tiny}{\@setfontsize\tiny{6.0pt}{10.5}}
\renewcommand{\huge}{\@setfontsize\huge{24.0pt}{42.0}}
\makeatother

\normalsize
\voffset=-1in
\hoffset=-1in
\paperwidth=424.325182059202pt
\paperheight=554.716774462811pt
\textwidth=519.59664pt
\textheight=383.924184pt
\topmargin=17.5600672314055pt
\oddsidemargin=11.200499029601pt
\columnsep=24pt
\headheight=0mm
\headsep=0mm
\topskip=0mm
\footskip=1000mm   % don't use footer
%\kanjiskip=0zw plus .0625zw minus .01zw
\kanjiskip=0zw plus .1zw minus .01zw
\xkanjiskip=0.25em plus 0.15em minus 0.06em
\setlength\parindent{1zw}

\normalsize
\usepackage{furikana}
\usepackage{burasage}
\AtBeginDvi{\special{pdf:docview <</ViewerPreferences <</Direction /R2L>> >>}}

\begin{document}

{\huge 魔王「この我のものとなれ、勇者よ」勇者「断る!」1} \\



魔王「この我のものとなれ、勇者よ」勇者「断る!」1


	

9 :以下、名無しにかわりましてVIPがお送りします :2009/09/03(木) 16:50:11.37 ID:s4r1gUtjP


魔王「どーしてもか?」\par{} 
勇者「アホ云うな。お前のせいでいくつの国が\par{} 
 滅んだと思ってるんだ」



魔王「南の森林皇国のことか?」\par{} 
勇者「空は黒く染まり、人々は貧困にしずんでいった」

 



魔王「考え無しに森林伐採して木炭作りまくって\par{} 
 公害で自滅したんだろう」 



勇者「公害……?」\par{} 
魔王「あー。えーっと。そうか、まだ判らないか」 



勇者「誤魔化すなっ! 聖王国の大臣憑依だって\par{} 
 魔族の仕業じゃないかっ!」 



魔王「欲の皮の突っ張った大臣が政権奪取と\par{} 
 王族の姫君大集合ハーレムを作ろうとして失敗しただけだ。\par{} 
 そもそも逮捕された後に魔族の洗脳とか言い出すのは\par{}
 人間の悪人の悪い習慣だと思うぞ」

	

	

10 :以下、名無しにかわりましてVIPがお送りします :2009/09/03(木) 16:55:18.10 ID:s4r1gUtjP 


勇者「ごまかすのか……許せん……」 \par{}
魔王「誤魔化してない」 



勇者「南部諸王国と戦争はどうなんだ。俺は戦場で\par{} 
 何百という人間が魔族の軍勢に倒されているのを\par{} 
 この目で見てきたんだ」 



魔王「それで?」 



勇者「は? 人間世界を侵略してきた魔王、貴様を\par{} 
 許しはしない!」 



魔王「どちらが侵略したかという点については見解の相違だ。\par{} 
 こちらにはこちらの言い分はあるが、まぁ、戦争してるのは\par{} 
 事実だなー」 



勇者「貴様は悪だ」 



魔王「じゃぁ、悪でも良いけど。当然私を殺した後には\par{} 
 南部諸王国の王族も全部抹殺して回るんだろうな?」

	
     
	

14 :以下、名無しにかわりましてVIPがお送りします :2009/09/03(木) 16:59:14.49 ID:s4r1gUtjP 


勇者「は? 悪はお前だけだ」\par{} 
魔王「人間が魔族を殺していないとでも?\par{} 
 魔族は悪で人間が善だって誰が決めたんだ?」 



勇者「……っ」 



魔王「そこで『俺が法だ!』とか『俺が神だっ!』とか\par{} 
 『俺がガンダムだっ!』とか云えたら、お前も\par{} 
 もうちょっと生きるのが楽なのになぁ……」



勇者「うるさいっ!!」 



魔王「勇者は好きだから、この話はやめてやる」\par{} 
勇者「好きとか云うな」 



魔王「この資料を見ろ」\par{} 
勇者「なんだ、これ……羊皮紙じゃないのか?\par{} 
 薄くて白くてつるつるだ……」\par{} 
魔王「プリンタ用紙だ。それはどうでもいい。書いてある\par{} 
 ことが重要なんだ」 



勇者「……えっと、需要爆発……雇用? 曲線?\par{} 
 消費動向……経済依存率?」

	 
    
	

16 :以下、名無しにかわりましてVIPがお送りします :2009/09/03(木) 17:06:53.97 ID:s4r1gUtjP 


魔王「わかったか?」\par{} 
勇者「なんだこれは。邪神の儀式か?」 



魔王「違う。経済的視点から見た巨大消費市場としての\par{} 
 戦争の効用だ」 



勇者「……効用?」\par{} 
魔王「そうだ」 



勇者「戦争に意味なんてあるものかっ。\par{} 
 貴様ら魔族が人間世界を滅ぼすための侵略だ」 



魔王「勇者がどーシテもと云うなら、ちゃんと戦ってやる」\par{} 
勇者「っ」\par{} 
魔王「話によっては、討たれてやっても良い」\par{} 
勇者「その首差し出せ」\par{} 
魔王「だから、半日ほど話を聞け」\par{} 
勇者「……」 



魔王「これは100年ぶりのチャンスなのだ」

	
     
	

18 :以下、名無しにかわりましてVIPがお送りします :2009/09/03(木) 17:11:55.79 ID:s4r1gUtjP 


勇者「良いだろう、話せ」\par{} 
魔王「じゃぁ、説明する。手元の資料の一ページ目を」 



ぺらっ 



勇者「表だ」\par{} 
魔王「グラフというのだ。……これは中央大陸のこの\par{} 
 50年の消費量と景気を可視化したものだ」\par{} 
勇者「……え」 



魔王「気がついたように、我らが戦争を始めた15年前\par{} 
 から中央大陸の景気は上昇局面に入った」\par{} 
勇者「……嘘だっ」 



魔王「嘘ではない。2ページ目を見るが良い。\par{} 
 こちらには各種統計資料が添付されている」 



勇者「戦争で数多くの死者が……」 



魔王「戦争を始めてから人間世界の人口は順調に増加を始 \par{}
 めている」\par{} 
勇者「そんなのは理屈で考えておかしいだろうっ。\par{} 
 戦争で人が死ぬことはあっても、人が増える道理など\par{} 
 あるものかっ」 

	
    
    

19 :以下、名無しにかわりましてVIPがお送りします :2009/09/03(木) 17:15:50.07 ID:s4r1gUtjP 


魔王「まぁ、一般解はそうだな。\par{} 
 しかし、この世界における戦前の常識では違う。\par{} 
 戦前の――まぁ、この戦前は数百年続いたわけだが\par{} 
 世界では、人間の死因は疫病と飢餓だったのだ」 



勇者「……」 



魔王「この二つは非常に強大な敵で\par{} 
 人間はこの二つを結局500年以上克服できなかった。\par{} 
 人口は増えるどころか、時に疫病が猛威を振るい\par{} 
 国単位で滅亡することも少なくなかった」 



勇者「疫病も飢餓も人間には御し得ないものだ。\par{} 
 神が人間に与えた試練と行ってもいい。\par{} 
 魔族の侵略と一緒にするなっ!」 



魔王「まぁ、降りかかるについてはそうかも知れないな。\par{} 
 しかし、だから克服できないとか、克服してはいけないと\par{} 
 いうものでもなかろう?」 



勇者「それは……」

 
	
    
    

22 :以下、名無しにかわりましてVIPがお送りします :2009/09/03(木) 17:23:52.82 ID:s4r1gUtjP 


魔王「現に戦争が開始されてからこれら二つの\par{}
 原因にする死者は30%まで低下した」 



勇者「理由は? ……なぜ? 
 魔族の暴威を見かねた神の恩寵か」 



魔王「私は結構長生きしてるが、神など見たことはないよ。\par{} 
 理由は明白だ。最大の原因は中央大陸危機会議の設立だよ」 



勇者「……?」 



魔王「つまり、魔族との戦争に対して、人間の王国\par{}
 連合を組んだからだ」\par{} 
勇者「それで、なぜ死者が減るんだ……?」 



魔王「食料の多い国が少ない国へ送ったり、医療の\par{} 
 進歩した国や農業技術の進歩した国が指導を行なった\par{} 
 からだな」 



勇者「それこそ人間の手柄じゃないかっ!」 



魔王「その程度のことも魔族と喧嘩しなければ\par{} 
 実行できない人間が大きな事を言ってはいけないよ」
 

	

	

25 :以下、名無しにかわりましてVIPがお送りします :2009/09/03(木) 17:31:05.78 ID:s4r1gUtjP 


勇者「……」\par{} 
魔王「そんなに悔しがるな。魔族だって大差は\par{} 
 ない事情だった」\par{} 
勇者「そう……なのか……?」 



魔王「戦国だったからな。地方豪族や領主が次々と\par{} 
 王を名乗っては一族郎党血まみれの戦いを送っていた」\par{} 
勇者「……」 



魔王「まぁ、そんな事情で、戦争は人間と魔族を救った」\par{} 
勇者「……」 



魔王「そんなに唇をかむな。血が出てしまうぞ?」\par{} 
勇者「触るなっ!」 



魔王「……君が望まない限り、触れないよ」\par{} 
勇者「……」 



魔王「私の言い分も判ってくれるか?」\par{} 
勇者「……」 

	
    
    

27 :以下、名無しにかわりましてVIPがお送りします :2009/09/03(木) 17:35:13.26 ID:s4r1gUtjP 


勇者「戦争に意味が……結果的にあったかも知れない」\par{} 
魔王「そう言ってもらえるとほっとするよ」 



勇者「だが続けて良い理由にはなってない。\par{} 
 始めて良い理由にもなっていない。\par{} 
 お前は戦争犯罪人だ。いますぐ戦争を中止して \par{}
 戦争犯罪者として法廷に立つんだ」 



魔王「んー」\par{} 
勇者「私利私欲でやった訳じゃないってのは\par{} 
 いまの話で、ちょっとだけ判った。\par{} 
 俺が付き添ってやるから投降しろ」 



魔王「それは難しいな」\par{} 
勇者「なぜだ?」 



魔王「理由は二つある。6ページ目の資料を見てくれ」 



ぺらり 



魔王「ここに消費市場としての『南部諸王国』と\par{} 
 『中央大陸』の物流の関係が記してある」
 
	
    
    

29 :以下、名無しにかわりましてVIPがお送りします :2009/09/03(木) 17:41:53.98 ID:s4r1gUtjP 


勇者「物流……?」\par{} 
魔王「まぁ、早い話、物の流れだ。食べ物や着るものや、\par{} 
 生活用品から武器、鉄、木材に至るまで全てだな」 



勇者「これは『南部諸王国』でどんどん使ってるのか?」\par{} 
魔王「そうだ。戦争は何でも大量に消費されるからな」 



勇者「……『南部諸王国』はどうやって支払ってるんだ?」\par{} 
魔王「ん?」\par{} 
勇者「だって物を買ったらお金が必要だろう?」\par{} 
魔王「ああ、良いところに気がついたな。偉いぞ」 



勇者「撫でようとするなっ」\par{} 
魔王「うっかりだ。そう、許可がない限り触れない。\par{} 
 私は契約を重視するタイプなのだ」 



勇者「どうやって購入してるんだよ」\par{} 
魔王「中央大陸危機会議決議による、戦時支援基金でだ」 



勇者「……?」\par{} 
魔王「わからないか。つまり、全世界が戦争中の \par{}
 『南部諸王国』に義援金を送ってるんだ」 



勇者「そうだったのか!! 人間の善の心に祝福を!\par{} 
 どうだ魔王、これが人間のもつ優しさだ」 

	
    
    

31 :以下、名無しにかわりましてVIPがお送りします :2009/09/03(木) 17:46:14.18 ID:s4r1gUtjP 


魔王「まぁ、そのお金で中央大陸の数多くの国は\par{} 
 自分の国の品物を買ってもらってるんだ。\par{} 
 つまり、お小遣いを上げて自分の店の商品を\par{} 
 買わせているだけさ」 



勇者「……?」 



魔王「このランクの説明は多少難しいかな。……つまり、\par{} 
 富をため込むってのは『お金持ち』にはなれても \par{}
 『豊か』にはなれないんだ。\par{} 
 お金を渡して、使ってもらう。物もお金も流れが \par{}
 よどみなく太いことが豊かなんだよ」 



勇者「……難しい」\par{} 
魔王「まぁ、そう言う物なんだ。\par{} 
 全部を自分でやったりせずに、得意な分野で協力する。\par{} 
 これは理論的に正しいことだ。\par{} 
 麦と塩、木材と鉄を交換することで国も人々の暮らしも \par{}
 豊かになる」 



勇者「それはまぁ、何となく判る。王立広場の市場みたいな\par{}
 もんだろう?」 



魔王「うん、そのとおりだ」

	 

	

33 :以下、名無しにかわりましてVIPがお送りします :2009/09/03(木) 17:49:30.81 ID:s4r1gUtjP 


勇者「でも、この場合、違うだろう」\par{} 
魔王「違うとは?」 



勇者「中央大陸の国家は、戦争で疲弊した『南部諸王国』に \par{}
 善意からお金を送ってるんだ。結果として、その物流?\par{} 
 が良くなったとしても、送ったお金は自分の物じゃないか」\par{} 
魔王「ふむ」 



勇者「つまり特産品同士を交換してるわけじゃない」 



魔王「してるんだよ」 



勇者「与えているだけじゃないのか?」 



魔王「『南部諸王国』は、中央大陸に安全を\par{} 
 輸出しているんだ。つまり、戦争で血を流して\par{} 
 人間世界を防衛することでお金を得ている。 \par{}
 ――見たことがあるんだろう?\par{} 
 人間世界の『全て』が戦火にまみれていたのかい?」 



勇者「……」 



魔王「新しく発明された馬車、豊かな光、豊富なご馳走 \par{}
 毎晩のように舞踏会を開いている国はなかったかい?\par{} 
 ブドウ畑で酔いしれている貴族はいなかったかい?」 

	
    
    

35 :以下、名無しにかわりましてVIPがお送りします :2009/09/03(木) 17:53:30.25 ID:s4r1gUtjP 


勇者「……それは」\par{} 
魔王「つまり、そういうことだ。人間社会は\par{} 
 『南部諸王国』の巨大消費と防衛ラインという存在に\par{} 
 現在依存しているんだ」\par{} 
勇者「い、ぞ……ん?」 



魔王「そうだ、頼っている。溺れているというような意味だな」 



勇者「でも、大多数の人間は戦う力なんて持っていないんだ。\par{} 
 そのためには、『南部諸王国』の戦士団や騎士団に\par{} 
 守ってもらい、せめて食料を送るしかない。\par{} 
 その何処がいけないって云うんだよっ!!」 



魔王「まぁ、感情的にはそれが真実だろう。\par{} 
 そこまで否定したりはしない。\par{} 
 でも同時に、経済的にこの市場が無くなると\par{} 
 人間社会の物流や為替が破滅するのも確かなことだ」 



勇者「破滅……?」\par{} 
魔王「そうさ。その資料にあるだろう? これだけの\par{} 
 巨大消費がなくなったら、中央大陸の生産者は \par{}
 大ダメージを受ける。特に鉄鋼業や造船業がね。\par{} 
 このダメージは波及して、数十万の死者がでる」

	

	

36 :以下、名無しにかわりましてVIPがお送りします :2009/09/03(木) 17:56:58.63 ID:s4r1gUtjP 


勇者「そんな……」\par{} 
魔王「まぁ魔王の云うことだから嘘かも知れないけどね」\par{} 
勇者「嘘なのか?」\par{} 
魔王「少なくとも私は本気だ。もしかしたら避ける方法が \par{}
 あるかも知れないけれど、私は知らない」\par{} 
勇者「……」 



魔王「さて、この物流と依存型のいびつな経済構造が\par{} 
 二つの理由のうち、一つだ」 



勇者「まだ……あるのか」 



魔王「もう一つは比較的簡単に説明できる」\par{} 
勇者「……」\par{} 
魔王「説明が簡単なだけで問題が簡単なわけではないが」 



勇者「どういう理由なんだ?」 



魔王「魔族との大戦争で人間社会は結束した。\par{} 
 物流が改善されて、医療技術もひろまって\par{}
 疫病と飢餓は少なくなったと云っただろう?」 



勇者「ああ、云ってたな」 

	
    
    
    
38 :以下、名無しにかわりましてVIPがお送りします :2009/09/03(木) 18:00:13.15 ID:s4r1gUtjP 


魔王「アレは説明の半分なんだ。確かに物流が以前よりは \par{}
 活発になった。以前は国民の半分が餓死するような国の \par{}
 隣では大豊作の国があり、協力なんてしなかったからね」\par{} 
勇者「うん……」 



魔王「だが、物流が改善されたとはいえ、この世界の\par{} 
 食料生産そのものが劇的に向上した訳じゃない」 



勇者「……?」 



魔王「判らないのかい? ……つまり、まだ餓死者は\par{} 
 いるんだよ」 



勇者「ああ。旅の途中でいくつもの村で、飢えた子供を見たよ」 



魔王「そんな世界で、『大戦争による死者』が居なく\par{} 
 なったらどうなる? 長い戦争で剣を振るう以外に\par{} 
 生きる術を知らない何十万人もの人間が中央大陸に\par{} 
 あふれるんだ。彼らは生きているから食料を必要とする。\par{} 
 人間は増えるぞ? ――でも、食料はそこまで増えない。\par{} 
 この世界にはまだ輪作の概念すらないんだ」 



勇者「そんな……」\par{} 
魔王「それが現実なんだ」

	 
    
	

41 :以下、名無しにかわりましてVIPがお送りします :2009/09/03(木) 18:04:32.19 ID:s4r1gUtjP 


勇者「だって、だって」\par{} 
魔王「だいたい、何で君は1人でここにいるんだ?」\par{} 
勇者「……へ?」 



魔王「腐っても魔王城だぞ。そりゃ警備をくぐり抜けて\par{} 
 こんな所まで来れちゃう君は突然変異というか、\par{} 
 それこそ冗談みたいな奇跡だけど」\par{} 
勇者「何を言ってるんだ?」 



魔王「戦争を終わらせるのは、軍の仕事だろう?\par{} 
 君は勇者じゃないのか?」\par{} 
勇者「勇者だよっ。それが俺の天命だっ」 



魔王「敵の王の命を単身仕留めるのは\par{} 
 暗殺者の仕事じゃないのかい?」 



勇者「……っ!?」 



魔王「多分ね。……人間の王たちも判っているよ」 



魔王「この戦争が終わったら、勝っても負けても\par{} 
 人間は滅びてしまうって」 



勇者「……」 



魔王「だから君を1人で送り出したんだよ」 

	
    
    

44 :以下、名無しにかわりましてVIPがお送りします :2009/09/03(木) 18:07:55.85 ID:s4r1gUtjP 


勇者「……」 



魔王「一方的なことを云ってしまったけれど、それは\par{} 
 魔族の側も事情は一緒でね。知っていると思うけれど\par{} 
 一口に魔族といっても、その内情は様々なんだ。\par{} 
 有角族や飛翼族、鉄蹄族。スライムや遊びコウモリ\par{} 
 なんていう低級なヤツらも沢山いる。\par{} 
 悪戯好きなだけの種族も多いけれど、有力な氏族は \par{}
 ひどく好戦的だし、種族中心主義だ」 



勇者「そうなのか……」 



魔王「うん、そうなんだ。\par{} 
 私はね……見ての通り、腕も細いし、ひ弱で華奢だろ?」 



勇者「魔法で戦うんじゃないのか?」\par{} 
魔王「そりゃまぁ、使えるけれど。\par{} 
 大魔法使いというほどじゃない」 



勇者「なら、どうして魔王になれたんだ?」 



魔王「要領とタイミングと、なんだろう。\par{} 
 ……多分、偶然で」 

	
    
    

49 :以下、名無しにかわりましてVIPがお送りします :2009/09/03(木) 18:18:08.19 ID:s4r1gUtjP 


魔王「私の一族は変わり者が多くて、魔界の端っこで 
 長年研究をしていてね。私の専門は経済なんだ」 



勇者「経済ってなんだ?」 



魔王「信じられないなぁ、人間の文明の程度は」\par{} 
勇者「なんかむかつく」 



魔王「魔族も人のことは云えない。\par{} 
 この戦争が終わったら、たとえ魔族の側が\par{} 
 勝ち残ったとしても、前にも増した乱世が始まるよ。\par{} 
 今度は人間の土地を舞台にして、奴隷を奪い合う\par{} 
 恐ろしい時代が幕を開けるだろう。\par{} 
 有力な魔族は人間の王国を次々と勝手に略奪して\par{} 
 それぞれを自分たちの『植民地』と呼ぶ時代だ。\par{} 
 裕福になった戦闘的な氏族は\par{} 
 その富で弱小氏族を従えたり、より大きな戦力を \par{}
 調えて魔族統一を目指すだろうけれど\par{} 
 いまよりもっと混沌とした魔界はたやすく統一なんか \par{}
 出来るわけが無くて、いまよりずっと多くの \par{}
 血が流れるだろうね」 

	

	

52 :以下、名無しにかわりましてVIPがお送りします :2009/09/03(木) 18:22:11.68 ID:s4r1gUtjP 


勇者「植民地?」 



魔王「他人の土地に攻め込んで支配して、\par{} 
 自分たちの場所であるかのように利益を吸い上げること」 



勇者「許されるわけ、無いっ」 



魔王「人間が勝ったら魔族の土地に同じ事をするだろうね」 



勇者「人間は、そんなことっ」 



魔王「……」\par{} 
勇者「……」 



魔王「しないって、云えないだろう?」 



勇者「……」 



魔王「まぁ、いろんな世界がそうやって滅びていったんだ」\par{} 
勇者「世界?」 



魔王「ああ、それは私たち一族の研究だよ。\par{} 
 気にしないで。でも、私は……」 

	

	

53 :以下、名無しにかわりましてVIPがお送りします :2009/09/03(木) 18:26:43.76 ID:s4r1gUtjP 


魔王「私は、まだ見たことがない物が見たいんだ」\par{} 
勇者「……」 



魔王「勇者になら、判るかも知れないと思ったんだよ」\par{} 
勇者「何を、だよ」 



魔王「言葉では言い表せないけれど」\par{} 
勇者「お前学者なんだろ?」\par{} 
魔王「学者……? ああ、うん、そんな物だ」 



勇者「じゃぁ、説明しろよ」\par{} 
魔王「うーん、つまり」 



勇者「……」 



魔王「『あの丘の向こうに何があるんだろう?』って\par{} 
 思ったことはないかい? 『この船の向かう先には\par{} 
 何があるんだろう?』ってワクワクした覚えは?」 



勇者「そりゃ……あるけど。わりと、沢山」 



魔王「そうだろう? 勇者だものな!」\par{} 
勇者「何でそんなに嬉しそうなんだよ」

	 

	

54 :以下、名無しにかわりましてVIPがお送りします :2009/09/03(木) 18:30:05.56 ID:s4r1gUtjP 


魔王「だから、そう言う物が見たいんだ」\par{} 
勇者「……勇者になりたいのか?」 



魔王「近い。でも、違う。だって私は学者\par{} 
 なのだろうし、いまのこの身は魔王だ……」 



勇者「……」 



魔王「やってて幸せとは云えないけれど\par{} 
 責任を感じるし他の誰かに押しつける気はない。\par{} 
 勇者じゃない私が、勇者になりたいなんて \par{}
 そんな夢物語で時間を浪費するつもりはないんだ。\par{} 
 けれど」 



魔王「見たことがない物は、見てみたい」 



勇者「……そか」 



魔王「だから、もう一度云う。\par{} 
 『この我のものとなれ、勇者よ』\par{} 
 私が望む未だ見ぬ物を探すために\par{} 
 私の瞳、私の明かり、私の剣となって欲しい」 



勇者「断る」 

	
    
    

57 :以下、名無しにかわりましてVIPがお送りします :2009/09/03(木) 18:34:02.22 ID:s4r1gUtjP 


魔王「だめか?」\par{} 
勇者「だめ」 



魔王「絶対か?」\par{} 
勇者「絶対」 



魔王「交渉の余地はないのか?」\par{} 
勇者「ない」 



魔王「……」\par{} 
勇者「……ないぞ? ほんとだぞ」 



魔王「あると見た」 



勇者「くぁ、なんでそこで上目遣いなんだよ。\par{} 
 学者がとって良い態度かよっ」 



魔王「学者であると同時に私は経済屋なんだ。\par{} 
 経済屋は決して諦めない。どんなことにでも \par{}
 妥協して明日を目指すんだ」 



勇者「なんだか俺より勇者っぽい」 

	
    
    

59 :以下、名無しにかわりましてVIPがお送りします :2009/09/03(木) 18:39:17.01 ID:s4r1gUtjP 


魔王「故事によれば『世界の半分』を交渉材料にするらしい」\par{} 
勇者「へー」 



魔王「余裕たっぷりだな」 



勇者「そんなので転ぶ勇者がいるかよ。\par{} 
 もしいたらそいつは勇者でも何でもない。\par{} 
 1から出直せって話だ」 



魔王「うん、私の知っている故事でも \par{}
 結末はそうなっていた」\par{} 
勇者「ださっ」 



魔王「私もそう思う。そもそもその魔王だって\par{} 
 世界征服が終了していた訳じゃないだろうに。\par{} 
 自分が所有していない物件の50%を譲渡するなんて\par{} 
 商道徳にてらしても法的観点から見ても\par{} 
 契約の有効性に疑問を持たざるを得ない」 



勇者「そういう嘘つきだから勇者にふられるんだよ」\par{} 
魔王「仰せの通りだ」 

	
    
    

61 :以下、名無しにかわりましてVIPがお送りします :2009/09/03(木) 18:44:23.58 ID:s4r1gUtjP 


勇者「だからといって魔界の50%譲渡とか云ったって\par{} 
 俺は絶対うんなんて云わないからな。\par{} 
 そんな見知らぬ土地なんかもらったってちっとも\par{} 
 嬉しくない。そもそも賄賂や金品で転ぶなんて\par{} 
 勇者のすることじゃないぞ。\par{} 
 人間ってのは、ベッド一個のスペースと、\par{} 
 毎日腹が満ちる程度の食い物があればそれで充分なんだ」 



魔王「清貧の志だな」 



勇者「貧しいとか云うな。魔王のクセにっ」 



魔王「私としても領土の割譲をテーマに交渉する気はない」\par{} 
勇者「そうなのか?」 



魔王「領土割譲は、後世の統治から見た場合、\par{} 
 民族問題やプライド上の問題があって、紛争の火種に\par{}
 なることもしばしばなのだ。眼前の交渉が重要とはいえ\par{} 
 後世に禍根を残すのは気が進まない」 



勇者「ふぅん……。そういうものなのか」 



魔王「そうなのだ。それにだいたい『50%』という\par{} 
 言い方が良くない。それでは結局『こちらとそちら』が\par{} 
 再生産されるだけではないか」

	 

	

66 :以下、名無しにかわりましてVIPがお送りします :2009/09/03(木) 18:50:17.49 ID:s4r1gUtjP 


勇者「どうゆうこと?」\par{} 
魔王「つまり、世界を分割するのが問題だ、ということだ。\par{} 
 その分割という発想は、結局現在の問題である\par{} 
 『人間世界と魔族世界』という対立を\par{} 
 『勇者の支配地域と魔王の支配地域』という対立に\par{} 
 問題をすり替えただけだという話だ」 



勇者「あー。もっともな話だ」 



魔王「だろう? それは交渉でも妥協でもなく\par{} 
 ただの時間稼ぎに過ぎない」 



勇者「ふむ」 



魔王「故にその種の論法は却下だ」\par{} 
勇者「じゃぁ、交渉も失敗だな。時間もちょうど半日だ。\par{} 
 ……戦闘するような気分じゃなくなっちゃったけれどさ」 



魔王「いや、ちゃんと提案はある」\par{} 
勇者「あるのか?」 



魔王「半分などとけちくさいことは云わない。\par{} 
 でも大地は私の物ではないから差し出せない。\par{} 
 勇者が欲しい。代価は私にはらえる全て。\par{} 
 つまり、私自身だ。\par{} 
 これだけは私の意志で勇者に捧げられる。\par{} 
 お願いだから私の物になってくれ」 

	
    
    

68 :以下、名無しにかわりましてVIPがお送りします :2009/09/03(木) 18:55:56.79 ID:s4r1gUtjP 


勇者「……お、お、おまっ」\par{} 
魔王「口をぱくぱくさせると間抜けに見えるぞ」\par{} 
勇者「なっ何をっ」 



魔王「交渉の提案だ」\par{} 
勇者「何言ってるか判ってるのかっ?」 



魔王「判っている」\par{} 
勇者「しょ、正気かっ!?」 



魔王「もちろんだ」\par{} 
勇者「もうちょっと物考えて発言しろ!\par{} 
 わ、わ、わ、わきまえろっ!!」 



魔王「そんなに驚かないでも良いではないか。\par{} 
 人間世界では15にもなれば農夫の息子だろうが \par{}
 宿屋の娘だろうが、そこら中でこのような睦言と\par{} 
 甘やかな契約を交わして乳繰りあっていると聞く」 



勇者「聞くなよっ」 



魔王「正確には書物で読んだわけだが。\par{} 
 ――読んだだけで実体は不明で経験もない。\par{} 
 これも一つの『未だ見ぬ物』だな」 

	

	

71 :以下、名無しにかわりましてVIPがお送りします :2009/09/03(木) 18:59:54.23 ID:s4r1gUtjP


勇者「何を」\par{} 
魔王「優れた提案とは提案した時点で \par{}
 提案者の目的の一分を達せられるのだ。\par{} 
 『純潔を捧げる願い』を告白するなどと\par{} 
 私の人生に訪れるとは思っていなかった知見だ。\par{} 
 精神的に平静ではいられないなんて貴重だな」 



勇者「まるっきり平静に見えるよ」 



魔王「ダメか?」\par{} 
勇者「だっ、だめっ!!」 



魔王「絶対か?」\par{} 
勇者「ぜ、ぜ、ぜ」 



魔王「前回よりもさらに余地があるように見える」\par{} 
勇者「近寄るなっ」 



魔王「許可無い限り触れたりしない。私は奥手なんだ」\par{} 
勇者「契約主義者ってさっきまでは云ってたじゃないか」 



魔王「それは真実だ。『奥手』というのは\par{} 
 真実にかぶせる演出上の工夫だ」 

	
    
    

75 :以下、名無しにかわりましてVIPがお送りします :2009/09/03(木) 19:05:18.00 ID:s4r1gUtjP 


魔王「勇者」\par{} 
勇者「何だよっ」 



魔王「んぅ。話づらいな」\par{} 
勇者「あんだけ悲惨な未来をぺらぺら解説して \par{}
 やがったじゃないか」\par{} 
魔王「あれは専門分野の講義だったから」 



勇者「どんだけ死にものぐるいな専門分野なんだよ」\par{} 
魔王「経済というのは血が流れない戦争なんだ」 



勇者「おっかないよ。戦闘能力のない魔王を\par{} 
 初めておっかないと思ったよっ」\par{} 
魔王「怖がらせるのは本意ではないし……。\par{} 
 混乱させたら申し訳ないと思うけれど」 



勇者「……」 



魔王「少しセールストークをする」\par{} 
勇者「うー。押されてる」 



魔王「私を独占すると色々便利だぞ?」\par{} 
勇者「たとえば?」 



魔王「家計簿を付けるのが得意だ。完璧を約束できる」 

	

	

80 :以下、名無しにかわりましてVIPがお送りします :2009/09/03(木) 19:09:56.32 ID:s4r1gUtjP


勇者「なんだかなぁ。家計簿か」\par{} 
魔王「それにね」\par{} 
勇者「?」 



魔王「この戦争の向こうに行ける」\par{} 
勇者「それは出来ないって云ってたじゃないか」 



魔王「もちろん、すぐには無理だ。人間の王たちも\par{} 
 納得はしまい。私が投降しても、それは秘密裏に\par{} 
 処理されて、偽魔王が立てられるだろう。\par{} 
 それくらいこの戦争は、人間社会に必要になって\par{} 
 しまっている」 



勇者「……」 



魔王「でも、だからこそ、それが『別の結末』を\par{} 
 迎える事ができるのならば、\par{} 
 それは私にとってだけじゃない。\par{} 
 三千世界にとって『未だ見ぬ物』じゃないだろうか?」 



勇者「……」\par{} 
魔王「どうだ?」
 
	
    
    

88 :以下、名無しにかわりましてVIPがお送りします :2009/09/03(木) 19:17:09.59 ID:s4r1gUtjP


勇者「それが」\par{} 
魔王「ん?」 



勇者「おまえなんだ」\par{} 
魔王「うん、これが私なのだ」 



勇者「ずっとそんなこと考えてきたんだ」\par{} 
魔王「戦争を終わらせるのが軍だとすれば\par{} 
 終わる着地点を模索するのが王の役目だ」 



勇者「そのために俺を欲しがったのか?」\par{} 
魔王「うん、まぁ。そうとも云える」\par{} 
勇者「……」 



魔王「いや、誤解しないでくれっ。\par{} 
 勇者が欲しいのは本当だぞ?\par{} 
 向こう側を一緒に見に行く連れが欲しいだろう?\par{} 
 朝の散歩に一緒に出かけるような気分なんだっ。\par{} 
 それから家計簿も本当だ。偽りはない。\par{} 
 なんだったら賃借対照表や生涯賃金提案書も書けるぞっ。\par{} 
 足りないかっ? 足りないのか?\par{} 
 そうだな。……あまりにも粗末な粗品だが\par{} 
 一緒にいるのも得意だ。私は静かな魔王だからな。\par{} 
 部屋に置いておいても邪魔にはならないことに定評がある\par{} 
 添い寝とかも多分役に立たないほど下手だろうが\par{}
 セット商品についている細々とした備品程度でいいならな\par{}
 付けることが出来る」 

	
    
    

92 :以下、名無しにかわりましてVIPがお送りします :2009/09/03(木) 19:24:09.48 ID:s4r1gUtjP 


勇者「あー」 



魔王「契約詐欺にならないようにあらかじめ\par{} 
 告げておかなくてはならないのだが、\par{} 
 私は料理は不得手だ。料理は科学なのだろう?\par{} 
 わたしにはゲル化や乳化を行なうための \par{}
 洗練された手先の技術が欠けているようで\par{} 
 調理に関して期待してもらっては困る」 



魔王「それから、あー。\par{} 
 世間の、ほら。大多数の一般的な性別において\par{} 
 男性を有する成人の人間が望むような外見的\par{} 
 肉体的な美しさには欠けていると思われる。\par{} 
 運動不足だしな」 



勇者「そうか? そ、そんなことないだろ」 



魔王「いいや、そんなことはあるのだ。\par{} 
 これは侍女たちの用意した魔王のお仕着せであって\par{} 
 欠点が隠れて、と言うか、見えないだけで、\par{} 
 二の腕とかつまめるのだっ」 



勇者「泣きそうになるなよ」 



魔王「ぷるぷるなのだぞ!?」 

	
    
    

97 :以下、名無しにかわりましてVIPがお送りします :2009/09/03(木) 19:29:31.81 ID:s4r1gUtjP 


勇者「いや、その……大変美味しそうに見えます。\par{} 
 ……特に胸とかおっぱいとか」 



魔王「いや、いいんだ。無用な慰めだ。\par{} 
 それに地味すぎて、こればかりは申し訳ない。\par{} 
 思うに我が一族の頭部は長い伝統で\par{} 
 見てくれよりも中身を重視してきたはずで」 



勇者「そうか?」 



魔王「私も娘時代、つまり150年ほど前だが\par{} 
 その時代にもうちょっとこう、何というか\par{} 
 身なりやお手入れに気を配っておけば\par{} 
 こんな一世一代の交渉時、リコールにおびえる経営者の\par{} 
 気分を味合わないで済んだはずなのに……」 



勇者「用語が難しいな」 



魔王「世の中は色々難しいのだ」 



勇者「まったくだな」 

	
    
    

100 :以下、名無しにかわりましてVIPがお送りします :2009/09/03(木) 19:34:36.34 ID:s4r1gUtjP


魔王「とにかく、そう言った外見的な物については\par{} 
 あまり満足のいく案件ではないのだが、経済を\par{} 
 中心にした知識と……知識と……。\par{} 
 それくらいしか誇れないのか、私は」 



勇者「……」 



魔王「あと誇ることが出来るのは、貞淑さくらいだ。\par{} 
 私は長命種で、学者の家系の出だからな。\par{} 
 それに純然たる契約主義者でもある。\par{} 
 私にとっていったん締結されれば\par{} 
 この種の契約は魂にまで食い込み\par{} 
 過去も未来もその一転において鋼に勝る強度を持つだろう。\par{} 
 ――側近く侍る。健やかなる時も病める時も寄り添おう。\par{} 
 それは約束できる」 



勇者「……」 



魔王「どうだ? 私の物にならないか?\par{} 
 私はあんまり我が儘は言わないぞ。\par{} 
 『丘の向こう側』に一緒に行ってくれれば\par{} 
 それだけで満足だ」

	 

	

103 :以下、名無しにかわりましてVIPがお送りします :2009/09/03(木) 19:40:31.38 ID:s4r1gUtjP 


勇者「沢山殺すことになるんだろうな」 



魔王「ああ。……うん。誤魔化さない」 



勇者「河が血で染まるほどかな」 



魔王「うん、終わるまでは。手を血で汚さない約束は\par{} 
 できない。私も、勇者も、非道なことを\par{} 
 沢山することになるだろう」 



勇者「裏切り者と呼ばれるかな」\par{} 
魔王「それは、正体を隠すことも出来る。\par{} 
 私の誇りにかけて、勇者の伝説は綺麗なままに\par{} 
 出来るはずだ」 



勇者「必要なのか?」\par{} 
魔王「それは違う。このままワルツのように戦争を\par{} 
 消耗を繰り返し、屍山血河の平和を享受することも\par{} 
 この世界に許された選択肢の一つだ」 



勇者「それはそれでおびただしい犠牲だろう」\par{} 
魔王「でも勇者が直接的手を汚さないで済む」 

	
    
    

104 :以下、名無しにかわりましてVIPがお送りします :2009/09/03(木) 19:46:44.08 ID:s4r1gUtjP


勇者「なんだよ契約したくないのかよ」\par{} 
魔王「騙して契約したくないんだ」 



勇者「そか」\par{} 
魔王「騙して手に入れたものは、一夜で失われる」\par{} 
勇者「信義に厚いんだな」 



魔王「善悪の話じゃない。ゲーム理論で証明された\par{} 
 商道徳レベルでの話だよ」 



勇者「よし、お前の物になる」\par{} 
魔王「いいのか?」 



勇者「いいんだよ。……あー、いっとくけどなっ」\par{} 
魔王「?」\par{} 
勇者「おっぱいのためじゃないからなっ!」 



魔王「こんなものがいいのか?」 ふにふに\par{} 
勇者「自分で揉むなっ」 



魔王「勇者」\par{} 
勇者「何だよ、魔王っ」 



魔王「触れたい。触って良いですか?」

	 

	

110 :以下、名無しにかわりましてVIPがお送りします :2009/09/03(木) 19:57:12.81 ID:s4r1gUtjP 


勇者「……」\par{} 
魔王「信用してないな?」 



勇者「口調がいきなり丁寧でびっくりしただけだ」\par{} 
魔王「少しだけだから、触らせてくれ」\par{} 
勇者「判った」\par{} 
魔王「……」 おずおず 



勇者「魔王、手がひんやりしてるな」\par{} 
魔王「冷たいか? 済まない」\par{} 
勇者「いや、気持ちよい」 



魔王「私は、勇者のものだ」\par{} 
勇者「俺は魔王のものになる」 



魔王「契約成立だ」 勇者「契約成立だな」 



魔王「嬉しいぞ」にこっ\par{} 
勇者「……。くぁっ。は、はなれろっ」\par{} 
魔王「そうか?」\par{} 
勇者「で、最初の一手はどうするんだよ」 



魔王「そうだな……まずは、麦から手を付ける」\par{} 
勇者「麦か……。長い旅になるな」\par{} 
魔王「もちろん。わたしは勇者と一生離れる気は\par{} 
 ないんだからなっ」 

	
    
    

133 :以下、名無しにかわりましてVIPがお送りします :2009/09/03(木) 23:39:08.74 ID:s4r1gUtjP 


――冬越しの村 



勇者「うぉ、寒くなってきたな」\par{} 
魔王「これから冬に向かっていくからな」 



勇者「こんな中途半端なところで良いのか?」\par{} 
魔王「中途半端とは?」 



勇者「いや、だからさ。魔王の話によれば\par{} 
 いまの人間、中央大陸にとって麦は食料として大事だ。\par{} 
 ってことを基本にしてるんだろう?」\par{} 
魔王「ああ、そうだ」 



勇者「だったら、もっと農業大国の湖の国とか\par{} 
 紫旗女王国とかさ、そっちに行った方が良いんじゃね?」 



魔王「話が聞いてもらえるならな」\par{} 
勇者「あー」 



魔王「勇者もしばらくは姿を見せない方がよいと思う」\par{} 
勇者「そうか?」 



魔王「ああ。あんな風に私の所へよこされたんだ。\par{} 
 誰かが勇者を疎ましく思っていたのかも知れない」\par{} 
勇者「んなことないって」 

	
    
	

137 :以下、名無しにかわりましてVIPがお送りします :2009/09/03(木) 23:45:31.80 ID:s4r1gUtjP


魔王「それに何処に行って紹介するにしたところで\par{} 
 説得力を出すためには実際の実験と資料が必要だ」\par{} 
勇者「そりゃそうだ」 



魔王「この村でそのための手はずを整える」\par{} 
勇者「そう、なのか?」 



魔王「うむ。手の者を忍び込ませているのだ」\par{} 
勇者「手回しが良いな」\par{} 
魔王「何事も根回しと調整が必要だ」 



勇者「この村か」\par{} 
魔王「ああ、何の変哲もない村だろう?」 



勇者「うん、こんな村は沢山知っている」\par{} 
魔王「『南部諸王国』辺境部では典型的な開拓村だな。\par{} 
 複合的な大規模農業を中心に小作農や職人たちが\par{} 
 緩やかな村を形成している」 



勇者「魔王軍との戦いもあるだろうになー」\par{} 
魔王「それでも大地にしがみついて生きてゆくんだ。\par{} 
 人間はそこがすごいところだ」

	
    
    

139 :以下、名無しにかわりましてVIPがお送りします :2009/09/03(木) 23:49:39.59 ID:s4r1gUtjP 


勇者「で、手の者ってのは」\par{} 
魔王「ああ。小さな一軒家を用意してもらってるんだが。\par{} 
 どこなのだろうな。この村としか聞いてないんだが」 



勇者「あー。そうなのか? 探せば判るんだろうけれど\par{} 
 そろそろ陽も暮れるし、こんな寒い中をうろつき回るのは\par{} 
 ぞっとしないなぁ」 



魔王「うむ、もっともな指摘だ」\par{} 
勇者「だよなぁ」 



メイド長「まおー様ぁ~まおー様ぁ~♪」\par{} 
勇者「ば、ば、ばっ」 



魔王「おお。この声は」\par{} 
メイド長「まおー様~♪」 



魔王「おお、紹介しよう。勇者!」\par{} 
勇者「この馬鹿っ。一応人間の村だぞっ! \par{}
 まおーまおー叫んでどうするっ!」 



魔王「む。そう言えばそうだ。以後注意するように」\par{} 
メイド長「はい、まおー様!」 

	
    
    

142 :以下、名無しにかわりましてVIPがお送りします :2009/09/03(木) 23:55:53.42 ID:s4r1gUtjP 


魔王「まぁ、大丈夫だ。そんなに怒ってくれるな」\par{} 
勇者「むー」\par{} 
メイド長「怒りすぎると皺が取れなくなりますよ」なでなで 



勇者「こいつは何なんだ?」 \par{}
魔王「これは私の側近で、メイドを束ねるメイド長だ」 



メイド長「ご紹介にあずかったメイド長です。\par{} 
 まおー様のことは幼い頃から世話をさせていただいて\par{} 
 おりますわ。この度は勇者様とまおー様のご結婚\par{} 
 まことに喜ばしく、お見守りさせていただいてきた\par{} 
 この身、この胸の内も震えんほどです」 



勇者「突っ込みどころはたくさんあるんだけど、\par{} 
 最大のものは結婚なんてしてないって事だぞ」 



メイド長「そうなんですか?」 



魔王「期間無制限の相互所有契約だ」\par{} 
メイド長「あらあら、まさに結婚じゃないですか」 



魔王「白詰草とクローバーほどに違う」\par{} 
メイド長「それじゃ同じものじゃないですか」

	 

	

145 :以下、名無しにかわりましてVIPがお送りします :2009/09/04(金) 00:02:31.05 ID:AIDyRnsCP 


魔王「あえて名前を変えることによる風情の違いがある」 



メイド長「ああ、そうでしたか!\par{} 
 メイドと召使いと側女と寝室奉仕奴隷のようなものですね」\par{} 
魔王「それも風情か?」 



勇者「……」 



メイド長「風情は大事ですよ。男女間の機微の80%は\par{} 
 風情で構成されていると言っても過言ではありません」\par{} 
魔王「なんと! それでは殆ど具材がないではないか?」\par{} 
メイド長「そこがミソなんですよ」 



勇者「あのー。寒いんだけど」 



メイド長「おやおや、まぁ!」\par{} 
魔王「勇者が寒いと云っているんだ。どうにかならんか?」 



メイド長「では、こちらへ。ご案内しますわ」\par{} 
魔王「おお、首尾はどうだ?」 



メイド長「さほど大きくはありませんが、\par{} 
 村はずれの古い館を改修してございます」 



魔王「でかしたぞ、メイド長」 

	
    
    

147 :以下、名無しにかわりましてVIPがお送りします :2009/09/04(金) 00:12:08.05 ID:AIDyRnsCP 


――古びた洋館 



勇者「なんだよなんだよ。充分でかいじゃないか」\par{} 
メイド長「とんでもない。魔王城の1/100以下です」 



勇者「あれはダンジョンだろ。一緒にするな」\par{} 
魔王「いや、私はあそこに住んでたんだがな」\par{} 
勇者「ああ、そっか」 



メイド長「こちらへどうぞ。\par{} 
 ただいまお茶をお持ちしましょう」 



魔王「すまんなー」\par{} 
勇者「側近って、どんな関係なんだ?」 



魔王「うむ。ああ見えてメイド長はわたしの親戚なのだ」\par{} 
勇者「学者なのか?」 



魔王「んー。学者というと語弊があるな。\par{} 
 学者一族と云うより『好奇心を抑えられない一族』なのだ。\par{} 
 しかも専門ジャンルを追求してしまう類の。\par{} 
 彼女は、私から見ると年上なのだが、\par{} 
 なんだか『メイド道』とやらに目覚めてしまってな」 



勇者「ふむ……」 

	
    
    

149 :以下、名無しにかわりましてVIPがお送りします :2009/09/04(金) 00:17:39.88 ID:AIDyRnsCP 


メイド長「殿方における騎士道と似たようなものですわ」\par{} 
魔王「早いな」 



メイド長「紅茶でございますよ。\par{} 
 蜂蜜をたっぷり入れておきましたから」 



魔王「まぁ、そんなわけで、ずっと一緒に育ったし\par{} 
 魔王軍の指揮なんかを任せたこともある」\par{} 
勇者「おいおい、メイドってそんなことも出来るのか!?」 



メイド長「主人のどのような求めにも応える。\par{} 
 それが私の『メイド道』ですわ」 

	
    
    

152 :以下、名無しにかわりましてVIPがお送りします :2009/09/04(金) 00:25:10.69 ID:AIDyRnsCP 


魔王「これは、この館はどういう物件なんだ?」\par{} 
勇者「そうだ、気になってた」\par{} 
メイド長「さる貴族の別邸だったらしい建物でございます。\par{} 
 その貴族……騎士家だったそうですが、その家そのものは\par{} 
 跡継ぎを戦争で亡くされ、この館は手放されたとか」\par{} 
魔王「正規の手段で手に入れたのだろうな?」 



メイド長「ええもちろんです。\par{} 
 修繕を依頼した職工の方々への支払いも現金で行ないました」\par{} 
魔王「うむ」 



勇者「……穏当だな、以外に」\par{} 
魔王「いずれ実力行使でなければ意を通せない相手も出てくる。\par{} 
 穏やかに通るところで無駄に暴力は使いたくない。\par{} 
 ……嫌われると困る」 



勇者「?」  



魔王「なんでもない。\par{} 
 ……で、我らはこの館に逗留して。んー身分は\par{} 
 どうしたものかな」 

	
    
    

153 :以下、名無しにかわりましてVIPがお送りします :2009/09/04(金) 00:32:06.31 ID:AIDyRnsCP 


メイド長「村長以下主立った方々へのご挨拶は\par{} 
 すませております。聖王都の神学院で研究なされた\par{} 
 高名な学者の姫君だと」 



魔王「そんなんで通るのか」\par{} 
勇者「格好さえどうにかすれば余裕だろう?」 



魔王「この白衣とローブではだめかな」\par{} 
勇者「ダメっぽいんじゃね?」\par{} 
メイド長「ダメでしょうね」 



魔王「着心地が良いんだが」\par{} 
勇者「姫君ってのは着心地度外視で服選ぶんじゃねぇかな」\par{} 
メイド長「そうですね。視線を意識せざるを得ない服で\par{} 
 緊張感を維持しているのかと思われます」 



魔王「緊張感がないと!?」\par{} 
メイド長「そうは申しておりません。しかし……」 



魔王「なんじゃ」\par{} 
メイド長「お耳を拝借いたします」 



魔王「ふむふむ……」 

	
    
    

158 :以下、名無しにかわりましてVIPがお送りします :2009/09/04(金) 00:35:41.98 ID:AIDyRnsCP 


魔王「勇者、勇者」\par{} 
勇者「どうした?」 



魔王「緊張感のない肉は駄肉か?」 じわぁ\par{} 
勇者「なんで半べそなんだよ」\par{} 
魔王「そう言われたのだ」\par{} 
勇者「あー」 



魔王「駄肉か? 駄肉なのか!?」\par{} 
勇者「あー。そのー」\par{} 
メイド長「お茶のお代わりはいかがでございましょう?」 



魔王「ゆるんでぷにってしまうのか?」\par{} 
勇者「コメントしづらいな」\par{} 
メイド長「冬場は特に運動が不足しますからね」\par{} 
魔王「うううう」 



勇者「……その、たまには着飾るのも良いんじゃないか?\par{} 
 目先が変わって。その、風情とか云うヤツだ」\par{} 
メイド長「さようでございますよ、まおー様」 



魔王「そ、そうか……」 

	
    
    

160 :以下、名無しにかわりましてVIPがお送りします :2009/09/04(金) 00:44:06.99 ID:AIDyRnsCP 


勇者「で、挨拶がどうとか」\par{} 
メイド長「ああ、そうでした。\par{} 
 その学術の研究と、新しい農作指導のために\par{} 
 この村に興味を持たれてやってくる、というような\par{} 
 説明をいたしております」 



魔王「そうか、話が早くて助かる」\par{} 
勇者「そうだな、農業ともなれば時間がかかるものな」 



メイド長「ええ。……まおー様」\par{} 
魔王「ん?」 



メイド長「魔王軍の方はいかがしましょう」 



魔王「ああ。そうだな」 ちらっ 



勇者「どうかしたか?」 



魔王「勇者と私は、魔王の大広間で決闘をしたとしよう。\par{} 
 そこで両者共に深い傷を負ったという噂を流せ。\par{} 
 私はその傷を癒すために冥界温泉で療養中だ。\par{} 
 勇者は生死不明、一説によると落ち延びたと」 



メイド長「かしこまりました」 

	
    
    

162 :以下、名無しにかわりましてVIPがお送りします :2009/09/04(金) 00:48:41.23 ID:AIDyRnsCP 


魔王「それでいいか? 勇者」 \par{}
勇者「ああ。かまわない。俺はどうせ魔王のものだしな」 



メイド長「あたあた、まぁまぁ」にこっ\par{} 
魔王「からかうな」 



魔王「この噂で、まぁ1年くらいの時間は稼げよう。\par{} 
 魔王軍の急進派も何かの策略ではないかと見て\par{} 
 しばらくは動きを控えるだろう」 



勇者「1年か」 



魔王「その他にも手は打つ。粘るつもりもあるが\par{} 
 まぁ、3年と云ったところだろうな、猶予は」\par{} 
勇者「……」 



魔王「その間に『まだ見たことのない結末』を\par{} 
 見つけなければならない」\par{} 
勇者「見たいだけじゃなかったのか?」 



魔王「……見つけ出さないと、私の持ち主に嫌われる」\par{} 
勇者「あ、その。そ……そっか。そうだな」 

	
    
    

164 :以下、名無しにかわりましてVIPがお送りします :2009/09/04(金) 00:53:49.41 ID:AIDyRnsCP 


魔王「それはいいとして、だ」\par{} 
勇者「お、おう」 



魔王「この地で結果を出したいのは、農法の改善だ」\par{} 
勇者「のーほー?」 



魔王「農業の方法だ」\par{} 
勇者「農業の方法たって、\par{} 
 種撒いて芽が出て収穫するだけだろ?」 



魔王「その方法を改善する」\par{} 
勇者「うーん。改善なんてあるのか?」 



魔王「同じ土地で麦を収穫し続けると \par{}
 どんどん質がわるくなるのはしっているか?」\par{} 
勇者「あー。聞いたことがあるぞ。\par{} 
 大地の恵みみたいな物がなくなっちゃうんだろう?」 



魔王「この辺りで広く行なわれているのは三圃式農業という\par{} 
 もので、畑を3種類に分ける。\par{} 
 それぞれ夏に使う、冬に使う、一年間お休みと分けるんだ。\par{} 
 そしてローテーションさせてゆく」 

	
    
    

165 :以下、名無しにかわりましてVIPがお送りします :2009/09/04(金) 00:59:01.92 ID:AIDyRnsCP 


勇者「工夫してあるんだな。ふむふむ。\par{} 
 いや、まてよ? そもそも何でローテーションするんだ?\par{} 
 どんどん新しい畑を開拓すればいいじゃないか。\par{} 
 新しい場所なら大地の恵みもたくさんある」 



魔王「ばかもの。\par{} 
 誰もが勇者のように化物じみた戦闘能力と体力と\par{} 
 破壊魔法を使えると思ったら大間違いだ」\par{} 
勇者「そ、そか?」 



魔王「開拓には長い時間と労力がかかる。\par{} 
 それにそうやって開拓範囲を広くすれば、\par{} 
 収穫や種まきなどで移動しなければならない\par{} 
 距離も広がるし、魔物や野生動物から防衛しなければ \par{}
 ならない敷地も広がってしまう」 



勇者「そういえばそうか」 



魔王「そこで3分割ローテーションを行なっているのだが」\par{} 
勇者「ふむふむ」 



魔王「この手法をより改善して、\par{} 
 食料の供給量を増やすのが当面の目的だ」 

	
    
    

167 :以下、名無しにかわりましてVIPがお送りします :2009/09/04(金) 01:06:07.19 ID:AIDyRnsCP


勇者「目的は判った。けれど、具体的にはどうするんだ?」 



魔王「輪作……つまりローテーションという概念は良いんだ。\par{} 
 これを4回転式に改善しようと思う」 



勇者「ふむふむ」 



魔王「すなわち、大麦を作る畑、クローバを作る畑、\par{} 
 小麦を作る畑、かぶを作る畑に4分割する。\par{} 
 この四つをセットにして4年周期で畑を活用するんだ」 



勇者「なんか、3ローテと大差がないように聞こえるな」 



魔王「3ローテは1周期に一回お休みがあるだろう?\par{} 
 こちらは4周期にお休みらしいお休みはクローバーだけだ」 



勇者「それがそんなに大きい差なのか?」



魔王「差が出る秘密は、麦以外にある。それがカブだ」 



勇者「シチューに入れると旨いけど、そんなに作っても\par{} 
 飽きるだろう?」 

	

	

169 :以下、名無しにかわりましてVIPがお送りします :2009/09/04(金) 01:10:12.33 ID:AIDyRnsCP 


魔王「もちろん人間が食べても良い。\par{} 
 麦ばかりだと健康に悪いからな。だがこのカブは\par{} 
 畜産の使うんだ。具体的に云うと豚の餌にする」 



メイド長「豚、ですか?」 



魔王「知ってるとおり、この寒くて貧しい地方では\par{} 
 肉食というのは冬場の重要な栄養素だ。\par{} 
 冬には充分な果物も野菜も採れないし、\par{} 
 充分な穀物が無い事の方が多い。\par{} 
 そこで豚を食べるわけだが、豚は生きているから\par{} 
 活かしておくためには食料が必要だ。\par{} 
 冬の間は人間の食糧すら不足するだろう?」 



勇者「ああ、そうだな。だから豚は冬になる前に、\par{} 
 最低限の数を残してしめちゃうんだ。\par{} 
 それでソーセージやベーコンやハムにして、\par{} 
 冬の間の食糧備蓄をする。南部諸王国じゃ\par{} 
 何処でも見られる冬の始まりの風物詩って所だ」



魔王「冬場……つまり農作物が充分では無い時期の\par{} 
 代替え的補助食料なのに、結局は農業と同じように\par{} 
 季節に支配されている。これは非効率的なことだ」 

	
    
    

173 :以下、名無しにかわりましてVIPがお送りします :2009/09/04(金) 01:17:28.72 ID:AIDyRnsCP 


勇者「……云われてみれば、そうだな」 



魔王「そこでクローバーとカブを使う。\par{} 
 クローバーは夏場の間、豚や羊などの牧草になる。\par{} 
 畜産をおこなえば肥料も出してもらえるしな。\par{} 
 カブは冬の間に飼料として役立つ」 



勇者「そうかっ!」 



魔王「そうだ。この方法は『畑を休ませない』、\par{} 
 『畑を痩せさせない』、『農作物以外も豊かにする』\par{} 
 の三つのアイデアで出来ているんだ」 



勇者「そこでこの村な訳か」 



メイド長「どういう事ですか?」 



勇者「この村みたいな、農業も畜産もやって、\par{} 
 ある程度自給自足の体制の方がこの農法には都合が\par{} 
 良いんだよ。データ採りの意味でも。\par{} 
 都市部や、小麦ばっかりを大量生産している、\par{} 
 それが出来ちゃうような温暖な地域では意味が薄い。\par{} 
 東方の米みたいな穀物にも効果が薄い。\par{} 
 『寒くて貧しい地域を救う』アイデアだから」 



メイド長「そういうことでしたか!」 

	
    
    

175 :以下、名無しにかわりましてVIPがお送りします :2009/09/04(金) 01:22:32.70 ID:AIDyRnsCP 


魔王「他にもいくつかある。北氷海の魚による肥料とか\par{} 
 、農機具にも改良の余地がある」 



勇者「へ? 色々あるんだな」 



魔王「難しいのは、農地と村の統廃合だ。\par{} 
 放牧地にした場合、羊などの動物はコントロールに\par{} 
 難があるしな。いまの危険な時勢だと、この種の\par{} 
 『開墾した権利』の縄張り争いは、たやすく利権の \par{}
 争いに変化してしまうのだ」 



勇者「そうだな、それは予想がつくよ」 



メイド長「でも、人間は土地を重視する、\par{} 
 って、まおー様はいってましたよね。\par{} 
 話し合いで解決するんですか?」 



魔王「そこで魔族だ」\par{} 
勇者「?」 



魔王「魔族で村を攻める。\par{} 
 どうせたいした守備軍もいないだろう?\par{} 
 ちょっとした中隊規模で充分だ。\par{} 
 脅して適度に放火でもしてやる。\par{} 
 最悪の場合は主立ったものを殺すこともあるかもしれない」 



勇者「……」 

	
    
    

177 :以下、名無しにかわりましてVIPがお送りします :2009/09/04(金) 01:26:08.44 ID:AIDyRnsCP 


魔王「村人が立ち退いたあとに、魔王軍は駆けつけた\par{} 
 騎士団と一戦して引き上げる。開拓民は戻ってくる \par{}
 だろうが、それは領主の庇護下に入ってのことだろう。\par{} 
 その状況なら、農地の整理や、村と村との統廃合も \par{}
 問題なく行くだろう」 



勇者「……」 



魔王「幻滅したか?」 



メイド長「……まおー様」 



魔王「むろん、手心は加える。そもそもこの手法は\par{} 
 王国側、少なくとも騎士団には内通者というか\par{} 
 こちら側の理解者が必要だ。\par{} 
 だがこの種の作戦には事故がつきものだ。\par{} 
 血が流れないと云うことはあり得ないだろう」 



勇者「……」 



魔王「だが、わたしは何かをすると決めたんだ。\par{} 
 このまま、血まみれの夢に似た消耗戦を100年繰り返し\par{}
 取り返しのつかない停滞を過ごすつもりはない。\par{} 
 わたしは『終わった後の物語』が読みたいんだ」 

	

	

178 :以下、名無しにかわりましてVIPがお送りします :2009/09/04(金) 01:31:23.17 ID:AIDyRnsCP 


魔王「愛想が尽きたか? \par{}
 魔王なんて嫌いになったか?\par{} 
 で、でもダメだぞっ。\par{} 
 勇者は私のものだから。\par{} 
 絶対に手放すつもりはないぞっ」 



メイド長「……」 



魔王「それとも、やっぱり魔王を退治するつもりになったか?\par{} 
 それならそれで仕方がないが……。 \par{}
 私は勇者のものだから\par{} 
 その剣を避けるなんて出来ないけど……」 



勇者「それでも……」\par{} 
魔王「……」 



勇者「それでも、救われる人はいるんだろう?\par{} 
 この3年間の秘密休戦で救われる人はいるんだろう? \par{}
 それにその4ローテーションの工夫が上手く行けば\par{} 
 毎年何万人もの飢えた子供たちが春を迎えられるんだろう? \par{}
 ――そして戦争を終わらせるための余裕が、\par{} 
 人間の手に戻ってくるんだろう?」 

	
    
    

179 :以下、名無しにかわりましてVIPがお送りします :2009/09/04(金) 01:34:26.27 ID:AIDyRnsCP 


魔王「そうだ」\par{}
勇者「なら俺はやっぱり魔王の剣だ」 



メイド長「……」 



勇者「俺は何をすればいい?\par{}  
……まぁ、うっすらと判っちゃいるんだが」 



魔王「予想通りだ。\par{} 
 勇者には『正義の味方』をやってもらう。 \par{}
 攻め込んだ魔王軍を撃退する、南部諸王国領主の\par{} 
 軍事的先鋒だ」 



勇者「茶番だな」 



魔王「うん。悪の首魁とこんな話をしているいま、\par{} 
 それは限りなく茶番だろう」 



勇者「100回地獄へ行っても救われないな」 



魔王「判るなんて云わない」 



勇者「でも『その役が』いないと始まらないもんな」 

	
    
    
    
240 :以下、名無しにかわりましてVIPがお送りします :2009/09/04(金) 14:42:09.44 ID:AIDyRnsCP


――冬越しの村 



魔王「思ったより難航しそうだな」\par{} 
勇者「ああ、そうだな」\par{} 
メイド長「あんなに分からず屋だとは思いませんでした」 



勇者「仕方ないさ。俺たちにとっては挑戦だけど\par{} 
 この村の人たちは、一年不作になっただけで人死が\par{} 
 出るんだ」 



メイド長「それはそうですが……」 



魔王「考えてみると、この地よりも魔界の方が\par{} 
 気候的には恵まれているな」 



勇者「そういやそうだな」 



メイド長「勇者様は魔界のあちこちに旅されたのですか?」\par{} 
勇者「ああ、人間では一番の魔界通だと思うぞ」 



メイド長「しかし、村長があの態度ですと\par{} 
 農業技術の改良ですか? 難しいのでは……」 



魔王「うむ。……露骨に断ってくるわけでもないので\par{} 
 対処に困るな」\par{} 


勇者「村長から見たら、魔王は貴族の令嬢に見えるんだ。\par{} 
 正面から反対はできないさ」 

	
    
    

243 :以下、名無しにかわりましてVIPがお送りします :2009/09/04(金) 14:49:16.30 ID:AIDyRnsCP 


メイド長「それならいっそ……」 



魔王「却下」 
メイド長「しかし、まおー様。目的のために手段は」 



魔王「農地改革のために、細かい利権争いをする\par{} 
 領主や地主を取り除くのはやむを得ない。\par{} 
 生産性を上げるためには、必要なことだ。\par{} 
 でもそれは、農業技術が発展して集約農業が\par{} 
 可能になって初めて出来ることであって\par{} 
 その技術的な先行試験場で、指導者を\par{} 
 取り除くことには百害あって一理もない」 



勇者「だよなぁ」 



メイド長「困りましたね」 



魔王「ん……。やはり教育程度がねっくなのか」\par{} 
勇者「教育?」\par{} 
メイド長「関係あるのですか?」 



魔王「まぁ、いろいろとな。次の段階でと考えていたのだが\par{} 
 あちこちに手を入れなければ物事が進まないようだ」 

	
    
    

245 :以下、名無しにかわりましてVIPがお送りします :2009/09/04(金) 14:54:32.24 ID:AIDyRnsCP 


メイド長「何をするのです?」 



魔王「考えても見ろ。私たちがこの村で生産性を\par{} 
 向上させても。それを他の村や王国にも伝えて\par{} 
 いかなければ、全体的な変化は望めない。\par{} 
 でも私たちが一カ所ずつ教えて回るのは\par{} 
 時間的に見ても経済的に見ても不可能だ」 



勇者「道理だな」 



魔王「だから、新しい技術を覚えて様々な国へ\par{} 
 伝えるため専門の人員、まぁ、部下でも同盟者でも\par{} 
 いいんだが、そういった人材も必要だ」 



勇者「ふむ」\par{} 
メイド長「それで、教育ですか」 



魔王「ところで聞くが人間界の教育とはどうなってるんだ?」\par{} 
勇者「そんなこと云われてもなぁ。そもそも教育って何だよ」 

	
    
    

246 :以下、名無しにかわりましてVIPがお送りします :2009/09/04(金) 15:00:09.67 ID:AIDyRnsCP 


メイド長「……」\par{} 
魔王「えーあー」 



勇者「いやだからさー。当たり前みたいに\par{} 
 難しい言葉を使われても」 



魔王「つまり、知識を子供や年下の存在に伝えると云うことだ」 



勇者「何だ、簡単な説明じゃないか。そんなのは\par{} 
 人間社会にだって当然あるよ。そもそも教えなきゃ\par{} 
 赤ちゃんは話せるようにだってならないだろう?\par{} 
 このあたりじゃ、まずは子供はじいさんか\par{} 
 ばあさんに預けられて、まとめて育てられるな。\par{} 
 両親は畑や狩に出かけるから、同年代の複数の家の\par{} 
 子供がまとめて面倒を見てもらう」 



メイド長「効率的なんですね。お年寄りにも仕事が出来ます」 



勇者「ある程度年甲斐ったら、農作業の手伝いをすることに\par{} 
 なるな。魔王みたいな理屈での話じゃないけれど、\par{} 
 いつ何処に何を作付けするかとか、天気の読み方だとか、\par{} 
 獣の避け方だとか、そう言う知識は数年も働けば自然と\par{} 
 身につくもんだ」 

	
    
    

249 :以下、名無しにかわりましてVIPがお送りします :2009/09/04(金) 15:24:38.66 ID:AIDyRnsCP 


勇者「それから、この地方の冬は深いんだ。\par{} 
 雪がどっさり降るし、冬には嵐がやってくることも多い。\par{} 
 この地方のこんな村では、冬の間は殆ど家の中に\par{} 
 閉じ込められるんだ。そんな間、農民たちは\par{} 
 細かい工芸品を作ったり、羊毛で衣類をこしらえたりする。\par{} 
 子供たちは長い冬の間、村の智慧者たちから\par{} 
 いくつものお話を教わる。英雄譚や王の話、神々の話だな。\par{} 
 あー、なんだ。魔族の話も出てくるぞ。\par{} 
 それから、ちょっと目端の利く若者や村長の子弟は\par{} 
 読み書きを習ったりもする」 



魔王「ふむ」 



勇者「都市部だとまた話が違うな。\par{} 
 都市部で教育と云えば、教会でのミサと家庭教師だ。\par{} 
 家庭教師ってのは、知ってるか?\par{} 
 えーっと、物語や読み書きや算術を教えてくれる\par{} 
 個人用の語り部みたいなものだ。\par{} 
 裕福な商家や貴族の家では両親が家庭教師を雇って、\par{} 
 自分の子供に様々な知識と技術を教えさせる」 



メイド長「勇者様もその口ですか?」 



勇者「まぁ、そうだな。とは云っても、俺の場合は\par{} 
 剣の教師とばっかりつるんで遊んでいたようなものだけど」 

	
    
    

250 :以下、名無しにかわりましてVIPがお送りします :2009/09/04(金) 15:33:52.38 ID:AIDyRnsCP 


勇者「どうだ? 教育ってこういう事の事じゃないのか」 



魔王「いや、まさにそう言うことだ」\par{} 
勇者「人間社会のことも任せておけっ」 



魔王「まぁ、で、その人間社会の教育は未開なので」\par{} 
勇者「未開っ!?」 



魔王「む。表現が悪かった。『発展の過程における\par{} 
 初期的な未発達の状態』だな」\par{} 
勇者「同じじゃねぇか」 



魔王「からかっただけだ」\par{} 
勇者「くぅ」 



魔王「だが、自然な教育だよ。無理がない」\par{} 
勇者「……」 



魔王「とはいえ、こちらは不自然な発展を望んでいるから\par{} 
 不自然で気合いの入った教育をしなければならない。\par{} 
 ……だがなぁ、教育って云うのは金がかかるんだ」 



勇者「金かぁ。俺勇者だし、勇者の装備を売れば\par{} 
 そこそこ金にならないか?」 

	
    
    

252 :以下、名無しにかわりましてVIPがお送りします :2009/09/04(金) 15:40:23.28 ID:AIDyRnsCP 


メイド長「えーっと、勇者の剣と勇者の盾、\par{} 
 勇者の鎧ですか……38万Gくらいにはなりますね」 



勇者「なっ? すげーだろ?」 



メイド長「どれくらい必要なのですか」 



魔王「むぅ。そうだなぁ、今年度の予算計画は\par{} 
 流動的でいくつかのパターンを考えているのだが\par{} 
 切り詰めた案で5600万G程度かな」 



勇者「お、おれの……勇者の……装備が……」\par{} 
メイド長「落ち込まないでください。勇者様」なでなで 



魔王「メイド長。それ以上の接触は禁止だ」\par{} 
メイド長「はい、まおー様♪」 



魔王「なかなかに難儀な話だなぁ」\par{} 
勇者「まったくだな」 



メイド長「まぁ、そろそろ冬になりますし、\par{} 
 どちらにせよ本格的な農業実験は春からでしょう?\par{} 
 この冬を使ってゆっくり手を打てばいいですよ」 

	
    
    

254 :以下、名無しにかわりましてVIPがお送りします :2009/09/04(金) 15:43:06.05 ID:AIDyRnsCP 


魔王「そうは云っても」\par{} 
勇者「気は焦るよな。3年しかないし」 



メイド長「まぁまぁ、今日は豚肉とカブのシチューを\par{} 
 作ってあるんですよ」 



魔王「旨そうだな」\par{} 
勇者「ああ、考えてても仕方ないか」\par{} 
メイド長「では、調理の仕上げがありますので\par{} 
 わたしはこれで。夕食はあと1時間ほどです。\par{} 
 お呼びに上がりますので、暖炉にでも当たっていて\par{} 
 くださいね」 



魔王「ああ、頼んだぞ」 



ぱたり 



勇者「……ふぅー」\par{} 
魔王「疲れたか? 勇者」 \par{}
勇者「んー。身体は何も疲れてないんだけどな。\par{} 
 普段考えないようなことを考えて、頭が疲れたよ」 



魔王「ふふふ。そうだろうな」 

	
    
    

256 :以下、名無しにかわりましてVIPがお送りします :2009/09/04(金) 15:46:39.50 ID:AIDyRnsCP 


魔王「なぁ、勇者」\par{} 
勇者「ん?」\par{} 
魔王「暖炉が暖かいぞ?」\par{} 
勇者「おう。暖かいな。この地方の寒さも、暖炉の前だと\par{} 
 多少許せるよな」 



魔王「なぁ、勇者」\par{} 
勇者「ん?」\par{} 
魔王「そのぅ、良かったらなのだが。隣に来ないか?」\par{} 
勇者「なんで?」 



魔王「この角度が、暖炉が暖かくて気持ちよいのだ」\par{} 
勇者「そなのか?」\par{} 
魔王「そうなのだ。ほら、特等席だぞ」 



勇者「ん。ほんとだ。あったけー」\par{} 
魔王「どうだ?」\par{} 
勇者「自慢そうな顔するなよ、魔王」 



魔王「む。……まぁ、たしかに暖炉が暖かいのは\par{} 
 私の手柄ではないがな」 

	
    
    

257 :以下、名無しにかわりましてVIPがお送りします :2009/09/04(金) 15:50:45.68 ID:AIDyRnsCP 


魔王「……」\par{} 
勇者「……」 



魔王「勇者?」ぺ、ぺとっ\par{} 
勇者「ん?」 



魔王「さ、触って良いか?」\par{} 
勇者「別に良いけど」 



魔王「変な意味はないぞ。勇者の髪は暗い色だから\par{} 
 ちょっと触れてみたくてな。ああ、つんつんしてるな」 



勇者「ご先祖様にサムラーイがいたんだ」\par{} 
魔王「なんだそれは?」 



勇者「東方の戦士だよ。鎧も兜も真っ二つにする技が\par{} 
 使えたんだとさ」 



魔王「そうか、勇猛な戦士殿だったんだな」 



勇者「一族全員戦いの家系なんだ」 



魔王「そうか? それ以外にだって、\par{} 
 勇者には素晴らしくて良いところが沢山だぞ?」 

	
    
    

259 :以下、名無しにかわりましてVIPがお送りします :2009/09/04(金) 15:54:50.43 ID:AIDyRnsCP 


勇者「そうかなー」\par{} 
魔王「そうだぞ、たとえば、私が側にいても怯えないとか」\par{} 
勇者「魔王はひ弱だし、女の人じゃん。運動不足だし」 



魔王「でも魔王だからな」\par{} 
勇者「そう言うものかな」\par{} 
魔王「そう言うものなんだ」 



勇者「なんだかしおらしいな」\par{} 
魔王「さ、作戦なのだ」\par{} 
勇者「?」 



魔王「これから勇者相手に交渉をだな、するのだ」\par{} 
勇者「何の?」 



魔王「それを云っては交渉にならんではないか」\par{} 
勇者「云わないで伝わるものか」 



魔王「事は微妙を要する問題なのだ。出来るだけ誤解を\par{} 
 受けないようにきちんと説明をしたい」\par{} 
勇者「話してみれば?」 

	
    
    

261 :以下、名無しにかわりましてVIPがお送りします :2009/09/04(金) 15:58:09.56 ID:AIDyRnsCP 


魔王「つまり、外は寒いだろう? 冬の嵐到来だ。\par{} 
 そしてここは暖かくて居心地がよい。\par{} 
 じきに夕食のお迎えが来るまでは二人は暇で、\par{} 
 さしあたってやることがないだろう?\par{} 
 もちろん今後やるべき課題は山積みでそれらに対して\par{} 
 考察を加えるという任務が巨大にそびえ立っているわけだが、\par{} 
 勇者は現在頭が疲れているという状況にあり、\par{} 
 効率的な作業は望めないわけだ」 



勇者「あー」 



魔王「もちろんこれはわたしの方で考えた草案というのも\par{} 
 愚かな一私案に過ぎないわけだが勇者の現在の疲労状況を\par{} 
 鑑みるにこのような俗説民間信仰の類も一概にその効能を\par{} 
 否定するわけにも行かないのではないかと思えるのだ時に\par{} 
 戦士はそのような蜜園で疲れを癒すと古書にもある勇者は\par{} 
 そのような爛れた習慣はないというかも知れないが」 



勇者「で、なにをどーしたいんだ?」\par{} 
魔王「勇者に膝枕してみたい」 



勇者「よしきた」ぼふんっ 



魔王「あっ。あわっ。勇者……」\par{} 
勇者「俺は魔王のものなんだ。遠慮は無用だぜ」 

	
    
    

265 :以下、名無しにかわりましてVIPがお送りします :2009/09/04(金) 16:03:32.95 ID:AIDyRnsCP 


魔王「勇者」\par{} 
勇者「んぅ……」 



魔王「勇者の頭、もふもふしてるぞ」\par{} 
勇者「遊ぶなよ」\par{} 
魔王「撫でているだけだ」 



勇者「魔王も良い匂いだぞ?」\par{} 
魔王「毎日ちゃんと湯浴みしてるからな」 



勇者「真面目だな」\par{} 
魔王「うむ、真面目なのだ。\par{} 
 勇者のモノになって以来メイド長が容赦なくてな。\par{} 
 以前は研究続きで一週間着替えないなんて事も\par{} 
 少なくなかったんだが」 



勇者「魔王は太ももすべすべだな」\par{} 
魔王「そ、そうか? 太くないか?」 



勇者「寝心地良いぞ」\par{} 
魔王「そうか。救われる。勇者は本当に\par{} 
 寛大な心の持ち主だな」 



勇者「あーえーうー。……えろ欲求の持ち主ではあるんだが」\par{} 
魔王「?」

	
    
    

272 :以下、名無しにかわりましてVIPがお送りします :2009/09/04(金) 16:12:30.35 ID:AIDyRnsCP 


魔王「その、勇者」\par{} 
勇者「なんだー?」 



魔王「さっきの、遠慮は無用というの」\par{} 
勇者「?」\par{} 
魔王「あれは本当か?」\par{} 
勇者「うん、そうだぞ」 



魔王「そ、そうなのか」おろおろ\par{} 
勇者「……どした?」 



魔王「う、うむ」\par{} 
勇者「押し黙るなよ。怖いから」\par{} 
魔王「私も自分がちょっぴり怖いぞ」\par{} 
勇者「はぁ?」 



魔王「いや、怯えてなどいられるか。\par{} 
 『まだ見ぬものを求める』。\par{} 
 それが私の生涯のテーマだったではないかっ」\par{} 
勇者「??」 



魔王「その、勇者……」じぃっ\par{} 
勇者「何だよ、気軽に云えばいいぞ」 

  

ガ タ ン ッ 

	
    
    

275 :以下、名無しにかわりましてVIPがお送りします :2009/09/04(金) 16:24:44.37 ID:AIDyRnsCP 


魔王「何だ、あの音は」\par{} 
勇者「裏手? ……納屋かっ?」 



ダダダッ 



魔王「えいこれ! 待てと云うのだ!」 




――納屋 



勇者「ここかっ!」\par{} 
魔王「真っ暗ではないか」 



???「……。……」 



勇者(人の気配?)\par{} 
魔王「しばし待て、明かりの呪文を……」ぽわっ 



姉 がたがたがた\par{} 
妹 ぶるぶるぶるぶる 



勇者「……なんだ? 子供だぞ」\par{} 
魔王「どうしたのだ? 殆ど下着ではないか」 

	
    
    

279 :以下、名無しにかわりましてVIPがお送りします :2009/09/04(金) 16:29:26.13 ID:AIDyRnsCP 


メイド長「あらあら、まぁまぁ」\par{} 
魔王「メイド長」 



メイド長「また迷い込んできたのですね」\par{} 
魔王「こやつらは何なのじゃ?」 



メイド長「逃亡奴隷ですわ。この館は長い間無人でした。\par{} 
 無人の割には廃墟ではありませんでしたし、\par{} 
 周辺の村とは違う系統の権力に属していますからね。\par{} 
 そう言う場所は逃げ込む先として使われてしまうんですよ」 



勇者「奴隷?」\par{} 
メイド長「ええ、そうですよ」 



勇者「奴隷なんて、そんなのどこから来るんだよ」いらっ 



メイド長「そこら中にいるではないですか?\par{} 
 ここらにいる人間はその殆どが奴隷でしょう?」 



勇者「ちがうっ。奴隷なんかじゃないっ!」 



メイド長「あら。私の見当違いなのでしょうか」 



勇者「奴隷制は野蛮だ。俺たちは許していない」 



メイド長「そんなことをおっしゃられても」 

	
    
    

282 :以下、名無しにかわりましてVIPがお送りします :2009/09/04(金) 16:33:04.54 ID:AIDyRnsCP 


魔王「この者たちは、農奴なのだな」\par{} 
勇者「農奴……?」 



メイド長「やっぱり奴隷ではないですか」 



魔王「農奴と奴隷は違う」\par{} 
勇者「ほらみろ。この南部諸王国に奴隷はいないんだ」 



メイド長「そうでしょうか?」 



魔王「農奴は奴隷とは違い、個人財産が認められている。\par{} 
 家も持てるし、農機具も自前のものがもてる。\par{} 
 家族と一緒に暮らすことも出来るんだ」 



勇者「まぁ、当たり前だな」 



姉 がたがたがた\par{} 
妹 ぶるぶるぶるぶる 



魔王「その一方で、職業選択や移動の自由はない。\par{} 
 主に荘園……地主の農地で、農作業を行なうための\par{} 
 労働力として、選択の自由無く働かされる存在だ」 



勇者「……」\par{} 
メイド長「……奴隷とは違うのですねぇ。へぇ~」ふんっ

	
    
    

283 :以下、名無しにかわりましてVIPがお送りします :2009/09/04(金) 16:38:20.18 ID:AIDyRnsCP 


勇者「……」ぎりっ 



魔王「メイド長、それ以上は云うな。奴隷制は悲劇的かも\par{} 
 しれないが社会制度の中で経済的にも意味があったのは\par{} 
 事実なのだ」\par{} 
メイド長「そうでしょうか?」 



魔王「メイド長っ」\par{} 
メイド長「……差し出口を。申し訳ありません」 



魔王「……」\par{} 
勇者「……」 



妹 ぶるぶるぶるぶる 



姉「あ、あのぅ……。あ、あさにはでてゆきますっ。 \par{}
 ごめ、ごめいわく……かけませんから…っ…\par{} 
 どうか、ひとばんだけ」 



メイド長「……」 



魔王「メイド長。初めてではないのだろう?\par{}
 いままではどのように対処していたのだ?」 

	
    
    

285 :以下、名無しにかわりましてVIPがお送りします :2009/09/04(金) 16:42:32.64 ID:AIDyRnsCP 


メイド長「逃亡奴隷……農奴でしたか。それは重罪です。\par{} 
 付近の有力者との関係も悪くなります。\par{} 
 すぐさま通報して引き取りに来てもらっていました」 



魔王「そうか」\par{} 
勇者「……」 



メイド長「勇者様、ご気分が優れないようですが?」\par{} 
勇者「……厳しすぎないか?」 



メイド長「自分の運命をつかめない存在は虫です。\par{} 
 私は虫が嫌いです。羽をもがれたようで見るに堪えません」 



魔王「……っ」\par{} 
勇者「あのなぁ!」 



メイド長「大事の前の小事ではありませんか?\par{} 
 このような些末時で付近の有力者の方々の\par{} 
 心証を悪くされても益のないことだと思いますが」 



魔王「そう……だな……」\par{} 
勇者「魔王……」 



メイド長「では」\par{} 
魔王「いや、連絡は明日の朝まで待つ。\par{} 
 湯を沸かせ。もう少しましな衣服もあるだろう。\par{} 
 反論は抜きだ。今回はそうする。もう、決めた」 

	
    
    

286 :以下、名無しにかわりましてVIPがお送りします :2009/09/04(金) 16:47:50.29 ID:AIDyRnsCP 


――小さな部屋 



魔王「あー。なんだ」\par{} 
勇者「……歯切れ悪いな」 



魔王「私は魔王であり経済屋だ。こういうのは苦手なんだ」\par{} 
勇者「俺だって勇者で剣士なんだ。苦手だ」 



姉「あの、ありがとうございます」\par{} 
妹 おどおど 



勇者「おう、気にしないで良いぞ」 



姉「こんなりっぱなふく、はじめてです」\par{} 
妹 こくり 



勇者「そか? なんだかこの屋敷に残されてた古着だけど」\par{} 
魔王「あー。腹はくちくなったか? 寝床は平気か?」 



姉「はい。わらはふかふかで、\par{} 
 あたたかくて、きれいなへやです」 



勇者「こんな小汚い部屋でもか……」\par{} 
魔王「そう言う境遇なのだろう」 

	
    
    

289 :以下、名無しにかわりましてVIPがお送りします :2009/09/04(金) 16:57:08.31 ID:AIDyRnsCP 


姉「その、こんなによくしてもらったのですけど」 



姉「あしたの、あさには、その……」 



魔王「それは……」\par{} 
勇者「……」 



姉「おねがいです、つうほうしないでください。\par{} 
 いえ、そうじゃなくて」\par{} 
妹 じわぁ 



姉「にげますから。ほんのすこしだけ。よあけから\par{} 
 すこしのあいだだけ、まってください」 

 

ガチャリ 



メイド長「何を言ってるんですか。ろくな靴もない。\par{} 
 服も最低限。お金も道具も何もない。\par{} 
 街へ行って乞食でもやるつもりですか?」 



妹「あ、あぅぅぅ」 



勇者「どうにか……。どうにかなんないのか?」 

	
    
    

291 :以下、名無しにかわりましてVIPがお送りします :2009/09/04(金) 17:02:44.67 ID:AIDyRnsCP 


メイド長「なりません。奴隷の生活は惨めですよ。\par{} 
 何も出来ない。何の希望もない。\par{} 
 自分自身の罪でそう居続けるしかできないと\par{} 
 自分で自分に言い聞かせながら生きてゆくのです。\par{} 
 この世の地獄かも知れませんね。\par{} 
 でもね」 



姉「……」 



メイド長「やっていることはメイドと代わりはありませんよ。\par{} 
 主人の意を受けて、主人の言葉なら何でもしたがう。\par{} 
 主人の夢を叶えるため、そのために命を捧げる。\par{} 
 奴隷とたいした違いは有りはしません」 



魔王「メイド長。私はお前を奴隷だなどと思ったことは\par{} 
 有りはしないぞ」 



メイド長「ええ、まおー様。\par{} 
 私もそのような扱い、まおー様より受けた覚えはありません。\par{} 
 でもだからより一層、正視に耐えません。\par{} 
 私と同じ仕事をしながら、\par{} 
 自らの手に運命をつかむことの出来ない\par{} 
 その弱さは、灼かれて死んで償うべきかと思います」 



妹「ちがうよっ!! ちがうもんっ!」 

	
    
    

292 :以下、名無しにかわりましてVIPがお送りします :2009/09/04(金) 17:07:38.70 ID:AIDyRnsCP 


妹「ちがうよっ、めがねのおねえちゃんはいじわるなのっ。\par{} 
 わたしたちはちゃんとにげてきたもんっ。\par{} 
 なにもできないわけじゃないもんっ。\par{} 
 みやこにいって、ふたりで、くらすんだもんっ」 



勇者「……それは」 



メイド長「何を夢物語を」 



妹「でも、やるんだもんっ」 



メイド長「百歩譲ってその熱意を努力と呼んでも\par{} 
 良いでしょう。しかし、それをなすに当たって\par{} 
 他者の家に忍び込み、あまつさえその厚意にすがり。\par{} 
 そればかりか寝床と食事を与えてくれた\par{} 
 その他者の立場を逃亡によってさらに悪くする。\par{} 
 そのような方法を是とする。\par{} 
 それがあなたたち農奴のやりようですか?」 



妹「だって、だってぇ!」 



メイド長「もう一度云います。\par{} 
 自分の運命をつかめない存在は虫です。\par{} 
 私は虫が嫌いです。大嫌いです。\par{} 
 虫で居続けることに甘んじる人を人間だとは思いません」 



姉「……」 

	
    
    

294 :以下、名無しにかわりましてVIPがお送りします :2009/09/04(金) 17:13:49.12 ID:AIDyRnsCP 


メイド長「判りましたか?」\par{} 
姉「はい……」 



メイド長「謝罪を」 



姉「このやかたのみなさま……きぞくさまには\par{} 
 ご、ごめいわくを、かけました。ごめんなさい」 



メイド長「よろしい」 



姉「……」\par{} 
妹「ひっく……う。うううぅ」 



メイド長「……」 じぃっ\par{} 
姉「……」 



メイド長「……それだけですか?」 



妹「やぁ……。もどるの、やだよぅ……こわいよぉ」 



姉「……いもうと、しずかにして」 

	
    
    

297 :以下、名無しにかわりましてVIPがお送りします :2009/09/04(金) 17:17:39.30 ID:AIDyRnsCP 


メイド長「……」 



姉「わたしたちを、ニンゲンにしてください。\par{} 
 わたしは、あなたがうんめい、だと――おもいます」 



メイド長「頭を下げる時は\par{} 
 そのように這いつくばってはいけません。\par{} 
 せっかくスカートをはいているのですから\par{} 
 指先で軽くつまみ、ドレープを美しく見せながら\par{} 
 優雅に一礼するのです」 



姉 ぺ、ぺこり 



メイド長「……魔王様。この館は魔王城に比べれば\par{} 
 掘っ立て小屋も同然ですが私1人ではいささか手が\par{} 
 足りません。メイドを雇ってもよろしいでしょうか?」 



勇者「いいのかっ? メイド長。あんなに嫌いだって\par{} 
 いってたのに。許してくれるのかっ?」 



メイド長「嫌いなのは虫です。メイドを嫌う人は\par{} 
 この世界に存在しません。たとえそのメイドが\par{} 
 新人であってもです」 



魔王「許す。鍛えてやってくれ」 

	
    
    

302 :以下、名無しにかわりましてVIPがお送りします :2009/09/04(金) 17:27:33.55 ID:AIDyRnsCP 


――雪の森の中 



メイド妹「ゆーしゃーさまー。ゆーしゃーさまー」 



メイド妹「ゆーしゃーさまーはどこですかー。\par{} 
 おいしーパンを、おとどけですよ」 



ヒュンッ! 



勇者「おう、お使いお疲れ」 \par{}
メイド妹「わっ。どこにいたの?」 



勇者「転移魔法だよ。声、森の中に響いてたぞ」\par{} 
メイド妹「へへーん♪」 



勇者「まぁ、この辺はすっかり安全になったから\par{} 
 平気だろうけどな。不用心なヤツだ」\par{} 
メイド妹「ゆーしゃさまに、おとどけものです」\par{} 
勇者「お!」 



メイド妹「おひるごはんですー! クルミのパンと、\par{} 
 タマネギとベーコンのオムレツですー」 



勇者「旨そうだな」\par{} 
メイド妹「おねーちゃんがつくりましたっ」 



勇者「順調に仕事覚えてるな。感心感心」 

	
    
    

306 :以下、名無しにかわりましてVIPがお送りします :2009/09/04(金) 17:33:55.68 ID:AIDyRnsCP 


メイド妹「おいしい?」 



勇者「美味いぞ! 温かい紅茶がまた泣かせるな。\par{} 
 走ってきたんだろう?」 



メイド妹「うんっ」 



勇者「一口飲め?」\par{} 
メイド妹「うんっ!」 



勇者「昼間とは云え、屋外作業は辛いなぁ。\par{} 
 なんかこー滅入るよ、まったく」 



メイド妹「あ、そだ。とうしゅのおねーちゃんから\par{} 
 でんごんがあった」 



勇者「何だ、先に云えよ」 



メイド妹「『きょうは、いのしし6とうがのるまだ。\par{} 
 くまならいのしし2とうぶんにかぞえても、よい。\par{} 
 それから、かわのじょうりゅうをみてきてくれ。\par{} 
 はんらんしそうなばしょがあれば、\par{} 
 まほうでこわすか、なおしてくれ』」 



勇者「人使い粗いな」 

	
    
    

307 :以下、名無しにかわりましてVIPがお送りします :2009/09/04(金) 17:39:27.05 ID:AIDyRnsCP 


勇者「そういや、学校はどうした?」\par{} 
メイド妹「ごごは、からだのたんれん」 



勇者「お前は鍛錬良いのか?」 



メイド妹「まだ、せいとすくないから。\par{} 
 おなじ年のこ、いないの。ゆうしゃさまに\par{} 
 ごはんをとどけるのが、ごごのうんどうー」 



勇者「お。『年』っておぼえたのか」\par{} 
メイド妹「とーしゅのおねーちゃんが、\par{} 
 もじおしえてくれるんだよ」 



勇者「忙しいクセにまめだな、魔王」 



メイド妹「あと、さんすー」 



勇者「算数か」\par{} 
メイド妹「うまくやると、儲けでうはうは」\par{} 
勇者「あの経済屋め。『儲け』だけは書けるのか」 



メイド妹「損益分岐点、もかけるよ?」 

	
    
    

310 :以下、名無しにかわりましてVIPがお送りします :2009/09/04(金) 17:43:22.75 ID:AIDyRnsCP 


勇者「あと、2匹か、イノシシ換算で」 



メイド妹「なべー!」\par{} 
勇者「突然どうした」 



メイド妹「いのなべ?」 



勇者「ああ、あれは美味いな!」\par{} 
メイド妹「いのなべにしよー!」 



勇者「お前は食い気ばっかりだな」\par{} 
メイド妹「ゆーしゃさまが、ごはんとってきてくれる」にこっ 



勇者「……ああ、そうだな」\par{} 
メイド妹「ごはんいっぱいは、とても幸せだよ」\par{} 
勇者「そだな」 



メイド妹「けんかないもん。\par{} 
 そんちょーのあとつぎさまにぺとぺととしなくていいの。\par{} 
 まいにちあったかい。おふとんほかほか。\par{} 
 ふくがきれい。おねーちゃんとふたりで\par{} 
 ずっといっしょにいられる。それは幸せ」 



勇者「……」 



メイド妹「どうしたの?」 

	
    
    

312 :以下、名無しにかわりましてVIPがお送りします :2009/09/04(金) 17:47:21.28 ID:AIDyRnsCP 


勇者「いや、勇者って相当に役に立たないなと思って」\par{} 
メイド妹「?」 



勇者「知識があるわけでもなければ、\par{} 
 金勘定が出来るわけでもない。農業も動物の世話も\par{} 
 出来ないし、先生は……出来るって云っても剣くらいだ。\par{} 
 こうなってみるとつくづく思い知る。\par{} 
 俺、口先ばっかり平和とか云ってたけれど\par{} 
 平和ってどういうモノなのか、どうすれば平和になるのか\par{} 
 平和になったらどうすればいいのかなんて\par{} 
 ちっとも考えてなかったよ」 



メイド妹「むずかしーね」\par{} 
勇者「難しいな」 



メイド妹「いのししべーこん、おいしいよ?」 



勇者「あれ、好きなのか?」\par{} 
メイド妹 こくん 



勇者「気に入ったのか?」\par{} 
メイド妹 うんうん 



勇者「じゃ、勇者のお兄ちゃんとしては、もうちょい\par{} 
 イノシシやっつけて、減らしてくるかね~」 

	
    
    

316 :以下、名無しにかわりましてVIPがお送りします :2009/09/04(金) 18:05:28.67 ID:AIDyRnsCP 


――館の広間、授業中 



魔王「……以上が南部諸王国の現在の経済状態から\par{} 
 導かれる戦争の最大規模となる」 



貴族子弟「……」めもめも\par{} 
商人子弟「……えっとえっと」\par{} 
軍人子弟「……」 



魔王「私は専門ではないが、古来軍隊が壊滅すると\par{} 
 されている損耗率は」 



軍人子弟「……最後の一兵まで戦い抜くでござる」 



魔王「おおよそ30%と言われている。3割だな。\par{} 
 ゆえに、この常備軍および恒常的な傭兵戦力によって\par{} 
 戦線維持は難しく、散発的な会戦と拠点防衛が\par{} 
 現在魔王軍との戦闘の要旨となる」 



魔王「ここまでで質問は?」 



メイド姉「聖鍵遠征軍はどうでしょう?」 



魔王「うむ、あれは例外だな」 

	
    
    

317 :以下、名無しにかわりましてVIPがお送りします :2009/09/04(金) 18:17:58.51 ID:AIDyRnsCP 


魔王「聖鍵遠征軍について知っているところを述べよ」 



貴族子弟「あっ。はい。聖鍵遠征軍は中央大陸の\par{} 
 危機会議によって結成された、聖なる遠征軍です。\par{} 
 目的は邪悪なる魔族を殲滅しこの戦争を終結させること。\par{}
 この15年の間に2回おこなわれました。\par{} 
 南の極点にある魔界門から魔界へと侵攻して\par{} 
 魔族の重要都市二つを破壊、魔王の都まで\par{} 
 迫りましたが、神のご加護虚しく魔王の卑劣な\par{} 
 補給線破壊行為により撤退を余儀なくされました」 



魔王「おお。ほぼ満点だな。\par{} 
 ――このような遠征軍は巨大な兵力を背景にして\par{} 
 行なわれる。まず第一に必要なのは世界規模での\par{} 
 戦争終結への熱意だ。ただ一度の大遠征で終戦が\par{} 
 成し遂げられるのならば犠牲を払う価値があると\par{} 
 世界中の人間が望んでこそ、その戦いに身を\par{} 
 投じる参加者が生まれるわけだな」 



魔王「もう一つは経済的バックアップだ。\par{} 
 本講の授業で何度でも扱うが、経済の支援無くして\par{} 
 社会も戦争行為も成り立たない。人間は食べなければ\par{} 
 飢えて死ぬ生き物なのだ」 

	
    
    

321 :以下、名無しにかわりましてVIPがお送りします :2009/09/04(金) 18:21:39.10 ID:AIDyRnsCP 


貴族子弟「時には金や食料よりも重要なことがある」だむんっ\par{} 
軍人子弟「算盤をはじいて戦など出来ないでござる。先生」\par{} 
商人子弟「……そうでしょうか」 



軍人子弟「飢えだなんて精神的な弱者の言い訳だ」\par{} 
貴族子弟「そもそも領主が保護している人民に\par{} 
 飢えなどは存在しないじゃないですか」 



メイド姉「飢えたことがないんですね」 



魔王「……次は、南氷海。すなわち南部諸王国と\par{} 
 極点である、魔王の大陸を取り巻く海についてだ。\par{} 
 この海は軍事的、経済的な意味で非常に大きな意味を\par{} 
 もっている。現在魔王軍との戦いのおよそ25%が\par{} 
 海を舞台に行なわれており……」 

 

ちりーん、ちりーん、ちりーん 



メイド長「お嬢様、授業終了のお時間です」 



魔王「もうそんな時間か。では、本日はこれで終える。\par{} 
 明日はこの続きだ。それから、剣士様が明日の午後は\par{} 
 手ほどきに来るそうだぞ」 



軍人子弟「まっておったでござる」\par{} 
貴族子弟「明日はあたりだな」 



魔王「では、解散。わたしは長老の家に移動して\par{} 
 夜間の農業講座をしなければならないのでな」 

	
    
    

323 :以下、名無しにかわりましてVIPがお送りします :2009/09/04(金) 18:36:03.83 ID:AIDyRnsCP 


――館の廊下 



勇者「よっ。お疲れ」\par{} 
魔王「疲れた」\par{} 
勇者「疲れた顔してるよ」 



魔王「なぜ私は教育などと言い出したんだろう。\par{} 
 人間の子供の相手をするのがあんなにも疲れるとは\par{} 
 思わなかった。あれではまるで動物ではないか。\par{} 
 理非も交渉も通じない」\par{} 
勇者「あー」 



魔王「なぜあの者たちはあんなにもプライドが高いのだ」\par{} 
勇者「貴族や軍人や富裕層だからじゃないか?」 



魔王「いっそ蛙に変えてしまうか」\par{} 
勇者「冗談に聞こえないぞ」\par{} 
魔王「冗談ではない」\par{} 
勇者「止めておけ」 



魔王「そうか」しょぼん\par{} 
勇者「村長の家に向かうんだろう? 付き合うよ」 

	
    
    

332 :以下、名無しにかわりましてVIPがお送りします :2009/09/04(金) 18:43:21.00 ID:AIDyRnsCP 


魔王「む。寒いな」\par{} 
勇者「雪が降ってないだけましだ」 



魔王「寒いぞ、勇者」\par{} 
勇者「俺はその中で一日中イノシシを追っかけてたんだぞ?\par{} 
 魔王は家の中にいたんだから文句言うな」 



魔王「ちがう。寒いのだ」\par{} 
勇者「……」 



魔王「……だめか?」\par{} 
勇者「わかった、ほら」ばふっ「これであったかいか?」\par{} 
魔王「うん、あったかい」 



勇者「ご機嫌か」\par{} 
魔王「ふふん。悪くはない」\par{} 
勇者「偉そうだな」 



魔王「勇者を手に入れて本当に良かった」すりっ\par{} 
勇者「あー。こほん」\par{} 
魔王「?」\par{} 
勇者「おたがいな」 

	
    
    

334 :以下、名無しにかわりましてVIPがお送りします :2009/09/04(金) 18:50:56.62 ID:AIDyRnsCP 


魔王「まぁ、なんとか動き出したのだから\par{} 
 文句を付けるのもおかしいのだろうな」 



勇者「まぁなぁ」 



魔王「悲しいほどに権威が物を言うのだな。\par{} 
 貴族の子弟を受け入れて、箔がついたら農民も\par{} 
 学んでも良いと言い出すのか。新しい農法も\par{} 
 この春からそれなりの規模で実験開始だそうだ」 



勇者「結果が出るのに、時間はかかるだろうな」 



魔王「いや、来年からにでも結果は出す」\par{} 
勇者「出来るのか?」 



魔王「秘密兵器を手に入れたからな」\par{} 
勇者「なんだそれは」 



魔王 ごそごそ 「これだ」\par{} 
勇者「なんだその固まりは?」 



魔王「これは馬鈴薯という。作物だ」\par{} 
勇者「??」\par{} 
魔王「植物なんだ。こうやって掘り出しているが、\par{} 
 この丸い部分は土中に出来る」 

	
    
    

339 :以下、名無しにかわりましてVIPがお送りします :2009/09/04(金) 18:54:44.77 ID:AIDyRnsCP 


勇者「ふぅん……」 



魔王「これはなかなか美味で栄養価の高い食物なのだ。\par{} 
 そのうえ、このような食用部分が地下に出来るために、\par{} 
 鳥害を受けにくい。また、痩せた土地や寒冷地、\par{} 
 固い地面でも成長できるという優れものだ。\par{} 
 そのうえ、土地あたりの収穫量は、ざっと計算した\par{} 
 ところ小麦の3倍に当たる」 



勇者「まじかよっ!?」 



魔王「ああ、大まじめだ」 



勇者「神の食べ物か!?」 



魔王「いいや、魔界の食べ物だ」\par{} 
勇者「……」 



魔王「異文化、異文明の接触というのは、\par{} 
 このように大いなる恩恵を生むことがあるんだ。\par{} 
 たとえ不幸な形の接触であっても、接触は接触だ」 



勇者「複雑だなぁ」 

	
    
    

341 :以下、名無しにかわりましてVIPがお送りします :2009/09/04(金) 18:59:48.00 ID:AIDyRnsCP 


魔王「もっともこの馬鈴薯にだって弱点がない訳じゃない」\par{} 
勇者「なにがあるんだ?」 



魔王「まず、毒性がある」\par{} 
勇者「ダメじゃん!」 



魔王「いや、強い毒ではないし、毒性が発生するのは、\par{} 
 日光に当たって発芽しかけたものだけだ。\par{} 
 収穫や保存においてきちんと管理すれば問題ない。\par{} 
 むしろ冷暗所で保存すれば一年程度は持つはずだ」 



勇者「ふむ……」 



魔王「また、連作障害がある。この馬鈴薯という植物は\par{} 
 条件さえ合えば、年に3回程度収穫できるのだが」 



勇者「聞くだにすごいな。さすが魔界の植物だ」 



魔王「ああ。だが、その分土中の栄養素、つまり\par{} 
 いわゆる『大地の恵み』を多く消費してしまうんだ。\par{} 
 必要な種類の恵だけ使ってしまうから、おなじ場所で\par{} 
 作り続けると出来が悪くなって、病気にもかかりやすくなる」 



勇者「ふむふむ」 

	
    
    

344 :以下、名無しにかわりましてVIPがお送りします :2009/09/04(金) 19:06:00.60 ID:AIDyRnsCP 


魔王「もうちょっと」\par{} 
勇者「へ?」 



魔王「もうちょっとくつくのだ。隙間があると寒い」 



勇者「う、うん……。あー。くっつくとこっちが\par{} 
 いろいろもにょもにょなんだけど」 



魔王「……わたしの身体は気持ち悪いか?」\par{} 
勇者「いやいや、そうじゃないんですが」 



魔王「まぁ、とにかく。この食物も、寒冷地の\par{} 
 飢饉対策に役立つはずなんだ。毒性の部分は \par{}
 気をつけていればさほど大きな問題にはならない。\par{} 
 どちらかというと、連作障害の方が問題だろうな」 



勇者「俺の理性の方にも問題が」 



魔王「大地の恵みは時間がたてば回復するが\par{} 
 それに対してこちら側からも働きかけを行なう方法を\par{} 
 確立しないと、一カ所に留まって生産量を上げることは\par{} 
 限界があるだろうな」 



勇者「大地の神に祈祷でもするのか?」 

	
    
    

346 :以下、名無しにかわりましてVIPがお送りします :2009/09/04(金) 19:10:57.82 ID:AIDyRnsCP 


魔王「そうだな。祈祷の一種だ」\par{} 
勇者「無神論者じゃないのか? 魔王は」 



魔王「私が無神論だろうと何だろうと、\par{} 
 利用できるものは隙無く隈無く躊躇無く\par{} 
 利用したおすのが経済屋というものだ」\par{} 
勇者「ある種の悪魔だな、こいつ」 



魔王「大地そのものに、契約の証として捧げ物をするんだ。\par{} 
 この種の捧げ物は人間社会でも、経験的に行なわれている。\par{} 
 焼いた食物や動物の遺骸、動物の糞尿や、食べかすなどだ」\par{} 
勇者「ふむ。なんか捧げ物ってイメージじゃないけど」 



魔王「期待しているのは南氷海の魚なんだがな」\par{} 
勇者「なぜ?」 



魔王「捧げ物には魚が良いんだよ」\par{} 
勇者「買ってきてやろうか? 転移呪文でひとっ飛びだぞ」 



魔王「ありがたいが、持って帰れる量ではないと思う。\par{} 
 畑一つに月50匹。それも毎年だと云ったら驚くか?」 

	
    
    

350 :以下、名無しにかわりましてVIPがお送りします :2009/09/04(金) 19:17:55.33 ID:AIDyRnsCP 


勇者「うっわ、そりゃ」\par{} 
魔王「無理だろう?」\par{} 
勇者「うん」 



魔王「だが、それも問題が大きくてな」\par{} 
勇者「なんだ? 南氷海に問題でもあるのか?」 



魔王「ああ。二つある。\par{} 
 一つは、勇者も知ってると思うが南氷将軍だ」\par{} 
勇者「……あの親父か」 



魔王「ああ、あの男、魔族の中でも強硬派だからな。\par{} 
 魔王の私が伏せっているこの時期でも略奪行為を\par{} 
 続けていると聞いている。銀鱗族、飛魚族、鉄亀族、\par{} 
 巨大烏賊族、歌姫族を率いる、魔族でも指折りの\par{} 
 実力者だ……」 



勇者「何度か戦りあったことがある。\par{} 
 ばかでかい図体で、すげー銛さばきだった」 



魔王「南氷海で活動するからにはどうあっても\par{} 
 利害が衝突するだろうな」 

	
    
    

355 :以下、名無しにかわりましてVIPがお送りします :2009/09/04(金) 19:27:04.00 ID:AIDyRnsCP 


魔王「もう一つが『同盟』だ」\par{} 
勇者「なんだそりゃ」 



魔王「この話は、もうちょっと伏せておこうかと\par{} 
 思ってたんだがな。良い機会だから説明しておこう」 



勇者「うん」 



魔王「正式には『南部独立都市および自由商人による\par{} 
 経済同盟』と呼ぶ。まぁ、いまでは『同盟』で\par{} 
 何処でも通じるな」\par{} 
勇者「聞いた覚えはあるけど、それって有名なのか?」 



魔王「名前だけは有名だが、実体をする人間は\par{} 
 多くはないな。特に商人でない人間にとっては\par{} 
 意味が薄い」 



勇者「つまり、商人の寄り合い所帯だろう?」 



魔王「まぁ、そうだ。\par{} 
 50年ほど前に南部諸王国中心の街にうまれた団体だ。\par{} 
 交易商人による団体で、団体構成員の交易特権を\par{} 
 守るために生まれたのが発祥の契機だな」 

	
    
    

357 :以下、名無しにかわりましてVIPがお送りします :2009/09/04(金) 19:32:55.13 ID:AIDyRnsCP 


勇者「交易特権?」 



魔王「ああ。ある街から別の街に物資を持っていくと\par{} 
 当然のように、許可が下りたり、降りなかったりするだろう」\par{} 
勇者「あるな」 



魔王「商人たちはその『許可』を求めるし\par{} 
 手に入れれば守りたがったんだ。\par{} 
 当たり前だな、その免許のあるなしで、\par{} 
 商売が出来るかどうかが決まる。死活問題だ」 



勇者「ふむふむ」 



魔王「時代が下ると、税の機構が整備されて、\par{} 
 おなじ許可でも税が重かったり軽かったりするようになった。\par{} 
 こうして王族や貴族は税を通じて経済に接触できるんだ。\par{} 
 しかしそれは逆に、経済の輩、つまり商人が\par{} 
 貴族や王族といった支配階級に接触することをも意味する」 



勇者「うへぇ、なんだか難しい話だ」 



魔王「『同盟』はそういった商人の作った組織の中でも\par{} 
 最大のものだよ。その規模は想像を絶する」 



勇者「へー? どれくらいなんだ? 千人くらいいるのか?」 

	
    
    

359 :以下、名無しにかわりましてVIPがお送りします :2009/09/04(金) 19:38:05.19 ID:AIDyRnsCP 


魔王「この場合、人数は問題じゃないんだ」\par{} 
勇者「そうなのか?」 



魔王「経済的な組織だからな。動かせる富の量や\par{} 
 流通に介入する能力が彼らの武器だ。人数じゃない」 



勇者「理屈で云えばそうなるのか。\par{} 
 ……で、どれくらいなんだ?」 



魔王「その商業範囲は、南部を中心にしてではあるが\par{} 
 この中央大陸全土に及ぶ。\par{} 
 主要な都市に『同盟』の出張機関がない場所はなく、\par{} 
 『同盟』の支店がある場所こそがすなわち主要都市だ。\par{} 
 『同盟』の総資産は誰にも判らないけれど、 \par{}
 いくつかの歴史的な介入から私が試算する限り\par{} 
 その総額は天文学的な規模にあがる」 



勇者「……」 



魔王「少なく見積もっても、南部諸王国全部を\par{} 
 5回売り買いしてもおつりが来ることは確かだ」 



勇者「!?」 

	
    
    

362 :以下、名無しにかわりましてVIPがお送りします :2009/09/04(金) 19:42:11.51 ID:AIDyRnsCP 


魔王「そう言う組織なんだ」\par{} 
勇者「なんだそりゃ!?」 



魔王「こと、この南部地方に限って云えば、都市間の\par{} 
 小麦の流通のおおよそ60%に同盟の息がかかっている。\par{} 
 その気になれば、領主も王の首もすげ替えられる力を\par{} 
 『同盟』はもっていることになるな」 



勇者「化物かよ」\par{} 
魔王「まごうことなき化物だ。\par{} 
 人間の生活は化物の背に乗って行なわれている」 



勇者「俺、そのなんとか同盟ってのに頼まれて\par{} 
 何回か戦意高揚演説したことがあるぞ」 



魔王「そうなのか?」 



勇者「ああ。魔族を倒しに立ち上がろう、えいえいおー!\par{} 
 みたいなやつ。そのあとひらひらしたドレスの \par{}
 姉ちゃんが出てきて、攻城塔の上でうたってたぞ。\par{} 
 キラッ☆とかいって」 



魔王「プロパガンダだな。数百万Gは儲けただろう」\par{} 
勇者「おれには謝礼15Gだったんだぞっ!?」 

	
    
    

368 :以下、名無しにかわりましてVIPがお送りします :2009/09/04(金) 19:47:46.85 ID:AIDyRnsCP 


勇者「うううう。俺は、俺ってヤツは……」\par{} 
魔王「そんなに落ち込むな、勇者」 



勇者「俺はそんなヤツらに騙されて……」\par{} 
魔王「経済は君の専門じゃない。無理もないさ」 



勇者「俺はそのお姉ちゃんに『憧れてますっ!』\par{} 
 なんてキラキラ瞳で云われたせいで \par{}
 それだけで胸がいっぱいになって\par{} 
 魔界へ飛び出しちゃうし」 



魔王「……」 



勇者「帰ってきたら祝勝パレードで\par{} 
 良い感じのパーティーに招待しますからとか\par{} 
 依頼してきた青年に言われちゃったりして、\par{} 
 モテますねとか肘でつつかれて舞い上がったり……。\par{} 
 いま考えるとあの青年も商人だったんだなぁ」 



魔王「……」 



勇者「うううう。俺はダメ勇者だ」\par{} 
魔王「ふんっ。きつい教育が必要だな」 

	
    
    

371 :以下、名無しにかわりましてVIPがお送りします :2009/09/04(金) 19:55:05.85 ID:AIDyRnsCP 


勇者「つまり、敵だな」\par{} 
魔王「あ-。物騒なことを云うな」 



勇者「いいや、敵だ。最上級撃魔封殺雷撃魔法で仕留める」\par{} 
魔王「都市攻略術式を個人相手に使おうと考えるな」 



勇者「だって騙したんだぞ」しくしく 



魔王「子供か、君は。\par{} 
 ……そもそも『同盟』には意志なんてないんだ。\par{} 
 金儲けをするための商人が寄り集まって、 \par{}
 知恵を出し合い、自分たちの身を守り成長することだけを\par{} 
 願った組織。もはや肥大してしまって個人の思惑なんて\par{} 
 欠片も差し挟めないほどに成立してしまった\par{} 
 『概念』に近い存在なんだ。\par{} 
 仮に勇者がだまされたとしても向こうに騙すつもり\par{} 
 なんて無かったし、逆に言えば復讐したって\par{} 
 痛みなんて感じる機能はない」 



勇者「くぁ、余計むかつく」 



魔王「敵でも味方でもない。獣みたいなモノなんだ」\par{} 
勇者「……」 



魔王(それでも、あるいは……) 

	
    
    

372 :以下、名無しにかわりましてVIPがお送りします :2009/09/04(金) 19:59:02.75 ID:AIDyRnsCP 


勇者「むー。知らないことばっかりだ」\par{} 
魔王「ぼやくな」\par{} 
勇者「まぁ、いいけどさ。戦ってばかりいる時よりも\par{} 
 前に進んでいる気がするから」 



魔王「……ずっと側にいる」\par{} 
勇者「うん。俺もだ」 



魔王「あー。あー」あせっ\par{} 
勇者「なんだ?」 



魔王「ほら、もう村長の家だ。きょ、今日は\par{} 
 クローバーによる土中の恵みの結晶化について話すのだっ」\par{} 
勇者「そ、そか」 



魔王「……その」\par{} 
勇者「うん」 



魔王「4時間ほどで、帰るから」\par{} 
勇者「わかった」 



魔王「い、いってくるからな」ぎゅっ\par{} 
勇者「おう! 行ってこい!」ぎゅっ 

	
    
    
    
419 :以下、名無しにかわりましてVIPがお送りします :2009/09/04(金) 22:46:44.67 ID:AIDyRnsCP 


――湖の国、首都郊外 



しゅんっ! 



勇者「……いいぞ」きょろきょろ 



魔王「む。便利なものだな、転移魔法というものは」\par{} 
勇者「魔王だって使えるだろう?」 



魔王「いや、個人長距離移動性能と、目的地の選択の\par{} 
 関係でここまでの汎用性はない」\par{} 
勇者「そうなのか」 



魔王「術式が違うのだ。機会があれば研究したいが」\par{} 
勇者「まぁ、いまは目的が先か」 



魔王「うん。どこだ?」\par{} 
勇者「あの丘の向こうだ。念のために顔は隠してな」 



魔王「心得た。淑女の服に比べれば、変装の方が\par{} 
 ずっと着心地が良いぞ」 \par{}
勇者「あれはあれでよい物なんだがなぁ」 

	
    
    

422 :以下、名無しにかわりましてVIPがお送りします :2009/09/04(金) 22:53:25.47 ID:AIDyRnsCP 


ざっざっ 



魔王「あれか?」\par{} 
勇者「ああ、あの石造りの建物が、このあたりの\par{} 
 修道会を束ねる修道院だ」 



魔王「宗教ばかりは私たちには判りづらいな」\par{} 
勇者「俺だって説明しにくいよ。専門家じゃないんだし。\par{} 
 まぁ、でも『同盟』が化物だとすると\par{} 
 『教会』だって同じくらい化物だって事だ」 



魔王「ふむ。用心するべきなのだな」\par{} 
勇者「ああ、もちろんだ。お前は特に魔王なんだからな。\par{} 
 危険人物リストのぶっちぎりナンバー1だ。\par{} 
 なんせ神の敵だぞ」 



魔王「ははは。神など恐れたことはない」\par{} 
勇者「神の名を叫ぶ人間ってのは怖いんだぞ」\par{} 
魔王「うむ、それは肝に銘じる」ぶるっ 



勇者「さ、いくぞ。一応紹介の連絡だけは\par{} 
 入ってると思うが……」 



魔王「最悪魔法で逃げ出せばいいだろう」 



勇者「悪い意味で場慣れしてきたな、俺たちも」 

	
    
    

423 :以下、名無しにかわりましてVIPがお送りします :2009/09/04(金) 22:59:20.74 ID:AIDyRnsCP 


――湖畔修道会、内部 



修道士「こちらでございます、お客様……」 



魔王「静かだな」\par{} 
勇者「うん」 



修道士「我が修道院はただいま『沈黙の行』の\par{} 
 時間です。どうかお気遣い賜りますよう……」 



魔王「う、うん……」\par{} 
勇者(雰囲気に飲まれてるぞ、魔王) 



かつん、かつん、かつん…… 



勇者(独特の雰囲気があるな、修道会ってのは) 



修道士「こちらが会議のための部屋となっております。\par{} 
 もうしわけありませんが、我が修道院は\par{} 
 午後の祈りを控えております。\par{} 
 しばらくお待たせしてしまうのですが」 



勇者「かまわない。案内ありがとう」 

	
    
    

424 :以下、名無しにかわりましてVIPがお送りします :2009/09/04(金) 23:04:27.16 ID:AIDyRnsCP 


魔王「さて、と。潜入は成功だ」\par{} 
勇者「あとは、院長に面会して交渉か」\par{} 
魔王「うん」 \par{}
勇者「今回は出たとこ勝負って事になるのか」 



魔王「まぁ、いくつか交渉材料は考えてきてあるんだが。\par{} 
 というか、そもそもこれは人間側のために考えた\par{} 
 人間にメリットの多い企画なんだがなぁ」\par{} 
勇者「相手は宗教屋だからな」 



魔王「そういえば、この世界の人間は、なんと言ったっけ? \par{}
 その、光の精霊とかを信じているんだろう?」\par{} 
勇者「ああ、中央大陸の主だった国は全て光の精霊信仰だ」 



魔王「勇者はさっきから聞いていれば、\par{} 
 涜神的な言動が多いが、信仰心は薄いのか?」 



勇者「薄いというか、何というか。\par{} 
 戦場に身を置いて、特に魔物なんかと戦ってると、\par{} 
 精霊様ってのは身近に感じるんだよ」\par{} 
魔王「ふむ」 



勇者「信仰心が薄い訳じゃなく、友達感覚のつきあいなんだ」\par{} 
魔王「そうなのか? それはまた珍しい気がするんだが」 

	
    
    

429 :以下、名無しにかわりましてVIPがお送りします :2009/09/04(金) 23:11:19.24 ID:AIDyRnsCP 


勇者「まぁ、俺は特別だよ。\par{} 
 夢のお告げなんかも聞いたりしちゃったしな」\par{} 
魔王「神は実在するのか!?」 



勇者「神じゃない、光の精霊だ」\par{} 
魔王「ふむ……」 



勇者「すごく善人なだけで、竜とか魔王とかと\par{} 
 似たような存在なのじゃないかな? 光の精霊も。\par{} 
 面倒くさいことが断れない気の弱い性格なんだと思うよ」 



魔王「そんな存在でも、信仰の対象なのだろう?」 



勇者「まぁな。それに信仰以外の所でも、\par{} 
 『教会』ってのは社会の中で大きな意味を持ってるんだ。\par{} 
 こんだけでかい組織だからなぁ。\par{} 
 『同盟』なんか人数だけで云えば比べものにならない」 



魔王「研究や学術の面でも、か」 



勇者「ああ。この世界のそう言った知識は、\par{} 
 殆ど教会の権力の下にあると云っても\par{} 
 良いんじゃないかな。以前にも話しただろう?\par{} 
 都市部の人々は、教会のミサで\par{} 
 お話や読み書きを教えてもらうんだ」 

	
    
    

432 :以下、名無しにかわりましてVIPがお送りします :2009/09/04(金) 23:17:41.52 ID:AIDyRnsCP 


魔王「その組織に期待したいんだがな」 



勇者「まぁ、今日のはとっかかりだし、\par{} 
 失敗しても傷口は浅くて済む。\par{} 
 『教会』は大所帯だから、内部ではいろんな\par{} 
 派閥があるんだ。いまはその派閥が『修道会』という\par{} 
 形で表に出てきている。様々な『修道会』が\par{} 
 入り乱れているのが現状だ」 



魔王「でも、すべて光の精霊を信仰しているのだろう?」 



勇者「そうだよ。だから表向き、全ての『修道会』は\par{} 
 友好的、と言う建前になっている。善の勢力ってことだな。\par{} 
 でも実際には信仰の方法論が違ったり、過激さが違ったり\par{} 
 もっと露骨に云えば信者の奪い合いでライバル関係で\par{} 
 あることも少なくはない」 



魔王「なんだか、魔界の部族の領土争いと変わらないな。\par{} 
 破壊神と煉獄神と暗黒神とにわかれていたほうが、\par{} 
 まだ判りやすいぞ」 



勇者「そう言う宗教があるのか?」\par{} 
魔王「あるぞ。でも、大半はただのファッションだ」 

	
    
    

435 :以下、名無しにかわりましてVIPがお送りします :2009/09/04(金) 23:22:45.56 ID:AIDyRnsCP 


勇者「で、まぁ。この湖畔修道会は修道会の中でも\par{} 
 実利的、かつ穏健でな」\par{} 
魔王「ほう」 



勇者「農民の生活援護みたいな事を主な活動にしているんだ。\par{} 
 労働力の提供とか、ブドウ栽培の指導とか、\par{} 
 戸籍の補完とか、そうそう、病院もやってるよ」 



魔王「病院もか!」 \par{}
勇者「つーか、病院ってのは教会の仕事だろう?\par{} 
 もっとも病人は受け付けない教会も少なくないけどな」 



魔王「……ふむ」\par{} 
勇者「魔王?」 



魔王「どうした?」\par{} 
勇者「魔王は……。なんだかな、そのう。\par{} 
 時々すごく寂しそうな顔をするよな。いまみたいな時」 



魔王「そうか?」\par{} 
勇者「ああ」 



魔王「そんな自覚はないんだがな」\par{} 
勇者「そうなのか? なぁ、魔王。魔王には\par{} 
 どういう風に物事が見えて」 



ガチャリ 

	
    
    

437 :以下、名無しにかわりましてVIPがお送りします :2009/09/04(金) 23:30:35.05 ID:AIDyRnsCP 


魔王「あー。お初にお目にかかる」\par{} 
勇者「はじめまして。紹介書は届いてるかと思いますが」 



魔王「南部辺境で農業を中心に研究生活を送っている。\par{} 
 紅の学士と云います。よろしくお願いしたい」 



勇者「俺はその介添え兼護衛の白の剣士。\par{} 
 修道院に入るのは気後れする粗忽者なのだが 
 ご寛恕ください」 



女騎士「……」 



魔王「湖畔修道会に来たのは初めてですが\par{} 
 立派な建物ですね、びっくりしました」 



勇者「……あ」\par{} 
女騎士「……白の剣士ですって?」 



魔王「へ」\par{} 
勇者「あー。それはな。えっと」 



女騎士「ゆ う し ゃ ! あなたねっ!!」\par{} 
勇者「うぁ」 

	
    
    

441 :以下、名無しにかわりましてVIPがお送りします :2009/09/04(金) 23:38:33.95 ID:AIDyRnsCP 


女騎士「なにが『白の剣士』よっ。\par{} 
 いままで何処ほっつき歩いてたのよ!\par{} 
 もう一年よ!? 一年も音沙汰無しでっ!!」 



魔王「どういうことなのだ?」 \par{}
勇者「いや、その」 



女騎士「あなたがあたしたちを放り出したんでしょっ。\par{} 
 この先に進むのは一人で良いとか何とか\par{} 
 適当なことほざいてっ!! \par{}
 あんな辺境の街で放り出された\par{} 
 私たちの身にもなりなさいよっ。\par{} 
 どんだけ心配したことかっ。\par{} 
 ってか腹立たしかったか!」 



魔王「あー」\par{} 
勇者「だってさぁ」 



女騎士「だってもクソもないのよっ! \par{}
 あっ。す、すみません。精霊様、クソなんて\par{} 
 云ってしまいました。懺悔しますっ」 

	
    
    

442 :以下、名無しにかわりましてVIPがお送りします :2009/09/04(金) 23:42:50.89 ID:AIDyRnsCP 


勇者「ううう」\par{} 
女騎士「私はともかく、弓兵さんも、魔法使いちゃんも\par{} 
 ものすごくへこんでたんだからねっ」 



魔王「攻撃力過多なパーティーだな」\par{} 
勇者「回復は俺と騎士でやりくりをね」 



女騎士「話聞いてるのっ!? 勇者っ」 



勇者「すんません」 



女騎士「……ふぅ。で、いままで何してたの?」 



魔王「あー」\par{} 
勇者「そ、それは」 



女騎士「ああ。済みませんでしたっ。学士様。\par{} 
 席も勧めませんで、今すぐお茶を持ってこさせます」 



魔王「は、はぁ」\par{} 
勇者「どうしたもんかなぁ」 



女騎士「わたしは、元聖銀冠騎士団所属の女騎士。\par{} 
 ゆかりあって、いまはこの湖畔修道会で\par{} 
 みんなの生活の向上のために勤めています」 

	
    
    

444 :以下、名無しにかわりましてVIPがお送りします :2009/09/04(金) 23:48:48.98 ID:AIDyRnsCP 


――湖畔修道会、会議室 



勇者「と、まぁ。そんな訳で。魔王にも手傷は\par{} 
 負わせたんだけどさ。魔物総攻撃みたいな話になっちまって\par{} 
 退却してきたって訳さ」 



女騎士「そうだったの……。まさか、いままでずっと\par{} 
 怪我の療養を?」 



勇者「いや、それはないな。まぁ、色々事情があって\par{} 
 表舞台には顔を出せなかったって云うか……」 



女騎士「諸王国がそこまで手を回したのっ!?」 



魔王「――」じぃっ\par{} 
勇者「いや、なんだそれ?」\par{} 
女騎士「ううん。いいんだけどっ。判ったわ」 



勇者「そっちは何でこんなところで修道院長やってるんだ?」 



女騎士「もうっ。\par{} 
 私は元々の出身がこの辺なのよ。\par{} 
 騎士の叙勲されたのも教会でだったし教会所属の騎士なの」 



勇者「そーいやそうだったなー」 

	
    
    

446 :以下、名無しにかわりましてVIPがお送りします :2009/09/04(金) 23:52:05.81 ID:AIDyRnsCP 


女騎士「……実は、勇者が魔王城に向かってね。\par{} 
 それを諸王国軍本部へと報告して、ひと月たった頃。\par{} 
 特使が来てね。……勇者が出かけて、その身を顧みず\par{} 
 魔王に一矢浴びせたって。そう言って、仲間全員に\par{} 
 恩賞金が出たのよ」 



魔王「ふむ」 



女騎士「勘違いしないでよねっ。\par{} 
 私は受け取ってないんだから。\par{} 
 ……で、そのあとね。\par{} 
 私たち三人はいままで大きな活躍をしてきたから、\par{} 
 王国の要職に取り立てるって……」 



勇者「そうだったのか」 



女騎士「……それって、体の良い引退勧告だよね。\par{} 
 私はイヤだった。勇者をだしにして出世するなんて\par{} 
 イヤだったし。だから故郷に戻って、今度はみんなの\par{} 
 ためになる仕事をしようと思って」 



勇者「立派な志じゃないか。いや、女騎士は以前から\par{} 
 やるときゃやってくれる男気あふれた仲間だと\par{} 
 思ってたんだよ」 



女騎士「…………はぁ」 

	
    
    

450 :以下、名無しにかわりましてVIPがお送りします :2009/09/04(金) 23:55:54.27 ID:AIDyRnsCP 


勇者「で、あとの二人は?」\par{} 
女騎士「……うん」 



勇者「?」 



女騎士「……弓兵さんはね。ほら、元々兵士だったでしょ?」\par{} 
勇者「ああ、そうだな」 



女騎士「だからね、諸王国軍に帰ったの。\par{} 
 恩賞金ももらってた。連合参謀本部の諜報室に\par{} 
 行くんだって云ってたよ。……その、ごめんね」 



勇者「何で謝るんだ? 俺の活躍で報奨金が出たなら\par{} 
 それってすごく良いことじゃないか。出世もしたみたいだし」 



女騎士「……う、うん」 



勇者「で、魔法使いは? あいつも金もらってただろ?\par{} 
 ああ見えて守銭奴だからな。\par{} 
 『東方の、魔道書、買った……』とか\par{} 
 無表情のままぼそぼそーっとか云ってたろ?\par{} 
 あいつは味わいのあるヤツだからなぁ」 



女騎士「魔法使いちゃんは、1人で行っちゃった」\par{} 
勇者「へ?」 



女騎士「勇者を追って、魔界へ」 

	
    
    

453 :以下、名無しにかわりましてVIPがお送りします :2009/09/05(土) 00:07:26.45 ID:5kaffl9OP 


魔王「……」\par{} 
勇者「……」\par{} 
女騎士「……ごっ、ごめんね」 



勇者「止めたんだろ?」\par{} 
女騎士「もちろんだよっ! でも、次の朝。\par{} 
 荷物が無くなってて、多分……」 



勇者「じゃ、仕方ない。気持ちはわからんでも無いけれど\par{} 
 女騎士が気に病む事じゃないさ。もとはといえば\par{} 
 俺が1人で突っ込んだせいなんだろうしな」 



女騎士「……勇者」 



勇者「それより、今日は交渉だの相談だのがあってきたんだ」\par{} 
女騎士「紹介状にも書いてあったけど……」 



勇者「ま……学士」\par{} 
魔王「わたしだな。改めて挨拶させてもらおう。\par{} 
 紅の学士と呼んで欲しい。学者だ」 



女騎士「初めまして、勇者のもと仲間の女騎士です」 

	
    
    

458 :以下、名無しにかわりましてVIPがお送りします :2009/09/05(土) 00:14:23.98 ID:5kaffl9OP 


魔王「今日来たのは、この修道会のお力をお借りするためだ」\par{} 
女騎士「うかがいましょう」 



魔王「まず、これを見て欲しい」 

 

とさっ 



女騎士「これは?」 



魔王「馬鈴薯、と言う植物だ。くわしい情報はこちらの\par{} 
 羊皮紙にもまとめてあるが、要点をまとめると\par{} 
 寒冷地でも耕作可能な農作物で、単位面積あたりの\par{} 
 収穫量は小麦の三倍に達する」 



女騎士「っ!?」 



魔王「もちろん、いくつかの注意点もあるが \par{}
 作物としては多くの優位性がある。栽培もけして\par{} 
 難しくはない。お解りだと思うが」 



女騎士「この作物は、多くの飢餓者を救える」 



魔王「そうだ」こくり 

	

	

461 :以下、名無しにかわりましてVIPがお送りします :2009/09/05(土) 00:18:19.44 ID:5kaffl9OP 


女騎士「どのような助力を当修道会にお望みですか?\par{} 
 金銭ですか? それでしたらどのような手段を用いても、\par{} 
 最大限出来うる限りの謝礼を用意させていただきます」 



魔王「ほら見ろ、勇者。これがこの作物に対する\par{} 
 智慧ある人物の対応だ」 



勇者「わるかったなぁ、反応が鈍くて」 



女騎士「……政治的介入や権力の行使をお望みなのですか?\par{} 
 何らかの爵位や身分を? 申し訳ありませんが、\par{} 
 当修道会は王族や貴族にそこまでの影響力は\par{} 
 保持していないのです。お金の用意できる量も……」 



勇者「いや、それはない。 
 女騎士がそう言うの苦手なのはよく知ってるし」 



女騎士「勇者じゃなくて、学士様と話してるのっ」 



魔王「金銭的な援助は、それはあればあっただけ嬉しいが\par{} 
 当面の目的はそうではない」 



女騎士「どういったことでしょう?」 

	
    
    

467 :以下、名無しにかわりましてVIPがお送りします :2009/09/05(土) 00:26:57.24 ID:5kaffl9OP 


魔王「南部辺境に、冬越しの村という寂れた寒村がある」\par{} 
女騎士「はい」 



魔王「その村に修道院を建てて欲しい」\par{} 
女騎士「そんなことでよろしいのですか?」 



魔王「私ももちろんバックアップをしよう。その修道院を\par{} 
 中心に、この馬鈴薯の栽培方法を農民に指導して欲しいのだ」 



女騎士「それは願ったりというか、我が修道会の理念に\par{} 
 乗っ取った行動ですが……。そんなことで良いのですか?」 



魔王「うん。もちろん、馬鈴薯の栽培が成功した場合、\par{} 
 付近の村や国に修道院を増やして、その栽培方法を\par{} 
 広めてもらえないだろうか」 



女騎士「その過程でこの修道会の影響も増えますから、\par{} 
 それはこちらにとっては得ばかりの話ですが、\par{} 
 学士様にとってはどのような得があるのですか?」 



魔王「実はこちらの目的も、馬鈴薯の栽培方法の伝播でね。\par{} 
 南方寒冷地の食糧事情の改善がされれば目的にかなう」 



女騎士「そう……ですか」 

	
    
    

468 :以下、名無しにかわりましてVIPがお送りします :2009/09/05(土) 00:30:44.67 ID:5kaffl9OP 


魔王「それに栽培したいのは馬鈴薯だけではない。\par{} 
 農業の手法改革研究も進めている。\par{} 
 従来の三圃式農業にかわる、新しい生産性向上の\par{} 
 手法がある」 



女騎士「そうなんですか!?」 



勇者「なかなか優れものだぜ」 



魔王「そう言った手法を実験的に行なっているのが\par{} 
 くだんの冬越し村なのだが、成功したとしても\par{} 
 私達だけでは広く伝えるための組織や人材が\par{} 
 不足しているのだ。\par{} 
 そう言った点で協力しあえればと考えている」 



女騎士「あなたは、光の精霊様に使わされた\par{} 
 御使い様に違いありませんっ」 



勇者「それはどーかなー」\par{} 
魔王「……」げしっ\par{} 
勇者「痛っ!?」 

	
    
    

472 :以下、名無しにかわりましてVIPがお送りします :2009/09/05(土) 00:34:57.42 ID:5kaffl9OP 


女騎士「そのようなことであれば、出来うる限りの。\par{} 
 ええ、私自らが冬越し村へと赴き、修道会の\par{} 
 総力を挙げて助力いたしましょう」 



魔王「ご厚情痛みいる」\par{} 
勇者「いや、それは……」 



女騎士「何か文句あるの? 勇者」 



勇者「いや、なんてーのかなぁ。ほら、えーっと」\par{} 
女騎士「じれったいわね」



勇者「俺って昔から危険をはらんだニヒルな\par{} 
 勇者じゃない? だから、ほら。\par{} 
 近くにいると、無用の火の粉が……」 



女騎士「そんなのずっと前から体験済みよっ。\par{} 
 それとも私が冬越し村に行くと何かまずいわけっ?」 



勇者「えーっと……それは、そのまおーとか……」 

	
    
    

474 :以下、名無しにかわりましてVIPがお送りします :2009/09/05(土) 00:38:15.85 ID:5kaffl9OP 


魔王「協力してくださる修道会の長に\par{} 
 失礼があってはいけないぞ、勇者」 



勇者「ええーっ!?」 



女騎士「……余裕がおありですね」めらっ 



魔王「余裕など無い台所事情ゆえ、こちらの修道会に\par{} 
 協力を求めてきたのだ。わたしは契約至上主義者ゆえ\par{} 
 契約の相手には最大限の敬意を払うことにしている」 



勇者(た、たすけてー) 



女騎士「ともあれ、二度と会えないかと思った……。\par{} 
 いえ、一年ぶりに会うことの出来た勇者と一緒に\par{} 
 このような恩恵の食物をもたらしてくれた学士様も\par{} 
 光の精霊のお導きというものでしょう。\par{} 
 わが修道会の天命かと思います」 



魔王「いいえ、魂持つものの努力です」 



女騎士「……ええ、そうですね。その通りです」 

	
    
    

478 :以下、名無しにかわりましてVIPがお送りします :2009/09/05(土) 00:41:19.55 ID:5kaffl9OP 


――湖畔修道会、前庭 



女騎士「本当い良いの? 見送りは」 



勇者「ああ、かまわない。部屋でも良かったのに。\par{} 
 どうせ転移魔法なんだから」\par{} 
女騎士「そりゃそうだけど」 



魔王「では、冬越し村で会えるのを楽しみにしている」 



女騎士「そうですね、冬の間はさすがに移動できませんから。\par{} 
 この修道院の後任院長を決めて、春一番でそちらへと\par{} 
 向かいましょう。修道院建築に関して、当地の領主や \par{}
 有力者との間に好意的な合意が出来れば良いのですが」 



魔王「そちらに関しては、この冬の間に\par{} 
 出来る限りの根回しをしておこう」\par{} 
女騎士「ありがとうございます」 



勇者「なんだか仲が良さそうに見えて怖い」\par{} 
女騎士「何か言った?」 



勇者「なんでもありません」 



女騎士「では春に!」\par{} 
魔王「ああ、春にお目にかかろう」 

	
    
    

482 :以下、名無しにかわりましてVIPがお送りします :2009/09/05(土) 00:49:00.34 ID:5kaffl9OP 


――冬越し村の春 



小さな村人「うんわぁ、やっとこお日様が顔をだしたなや」\par{} 
痩せた村人「だしたなやぁ。ああ、風がぬるくなってきた」 



村の狩人「ほーい。ほーい」 



小さな村人「どうしたー?」\par{} 
痩せた村人「今日は良い天気だなやー」 



村の狩人「そうだなぁ。今年はなんだか良い事が\par{} 
 起きそうな気がするだなー」 



小さな村人「さっそくかい?」 



村の狩人「ああ、ウサギが4匹も捕れたよ。\par{} 
 1匹は村長さんの所へ持っていく」 



小さな村人「そりゃぁいいな!」\par{} 
痩せた村人「今年はイノシシの塩漬けがまだたくさんあるしな」 



村の狩人「ああ、びっくりしたなや」 



小さな村人「これも村はずれの剣士様のお陰だなー」\par{} 
痩せた村人「うちの息子が、斧を研いでもらっただよ」\par{} 
村の狩人「熊もつぶしてくれたとかで、\par{} 
 森の中も少し風通しが良いみたいだなや」 

	
    
    

485 :以下、名無しにかわりましてVIPがお送りします :2009/09/05(土) 00:56:25.53 ID:5kaffl9OP 


メイド妹 ~♪ ~♪ 



小さな村人「おんや。噂をすれば、村はずれの館の姉妹だなよ」\par{} 
痩せた村人「本当だ。ほーぅい、ほーぅい!」\par{} 
村の狩人「どこへいくんだーい」 



メイド姉「こんにちは、みなさん」ぺこり\par{} 
メイド妹「あのねー。村長さんの所へ、木イチゴの樽漬け\par{} 
 を分けてもらいに行くんだよっ」 



小さな村人「そーかそーか。えらいな」\par{} 
痩せた村人「お客さんでもくるんかい?」 



メイド姉「はい、そのようです」 



村の狩人「そうかそうか。……ふむ。\par{} 
 ようし、このウサギを、当主の学者様へと\par{} 
 お届けしてほしいだなや」 



小さな村人「おんや、太っ腹だな、狩人さん」\par{} 
村の狩人「なんの。森を安全にしてくれた\par{} 
 大恩あるおうちじゃないか。\par{} 
 ウサギなんて春になったのだからまた取れるだな」 

	
    
    

486 :以下、名無しにかわりましてVIPがお送りします :2009/09/05(土) 00:59:34.36 ID:5kaffl9OP 


メイド妹「ありがとー♪」 



小さな村人「それもそうだ。\par{} 
 これは沢で取れたクレソンだなや。\par{} 
 ほら、分けてやるから持っていくと良い」 



メイド姉「ありがとうございます、本当に」 



痩せた村人「雪解けの屋根修理には是非呼んでくれだな」\par{} 
村の狩人「そうだそうだ、是非お世話してやんねと」 



メイド姉「はい。かならず当主に伝えます」 



小さな村人「ええってええって」\par{} 
痩せた村人「なんだ、みんなにこにこしてからに」\par{} 
村の狩人「やぁ。やっぱりお屋敷詰めともなると\par{} 
 本当に2人ともべっぴんさんだねぇ」 



メイド姉「……」 



小さな村人「ああ、本当だ。俺たちとは全然違うだなや。\par{} 
 賢くて優しくてべっぴんで、俺たちは、みんな\par{} 
 2人に憧れてるだなよ」



メイド妹「ありがとー」にこぉっ\par{} 
メイド姉「……ごめんなさい」 

	
    
    

492 :以下、名無しにかわりましてVIPがお送りします :2009/09/05(土) 01:11:23.61 ID:5kaffl9OP 


――村はずれの屋敷、深夜 



勇者「よっ。ほっ」 ぎゅっ、かちっ\par{} 
勇者「こんなもんか? 薬草もあるし、あとは\par{} 
 現地でどうとでも奪えばいいか」 



魔王「こんな深夜に完全武装か」\par{} 
勇者「魔王……」 



魔王「私の物のくせに」\par{} 
勇者「あー。うん。……ごめん」 



魔王「なんだその情けない顔は。勇者だろうに」\par{} 
勇者「後ろめたいとどうしてもこういう顔になるんだよ」 



魔王「私はお前の物なんだぞ。そしてお前は私の物だ」\par{} 
勇者「ああ」 



魔王「止められるとでも思ったか?」\par{} 
勇者「……」 



魔王「見くびらないでもらおう」\par{} 
勇者「え? いいのかっ?」

	
    
    

494 :以下、名無しにかわりましてVIPがお送りします :2009/09/05(土) 01:15:01.77 ID:5kaffl9OP 


魔王「ほら」 



勇者「これは?」ずしっ 



魔王「先々代だったか? の魔王が使ってたという、\par{} 
 黒玉鋼の鎧兜だ。安心して良い。呪いの類は\par{} 
 かかっていない」 



勇者「……?」 



魔王「魔王の私がいなくて、魔界の統治のたがが\par{} 
 緩んできてるんだ。勇者はその粛正を適当にしてきてくれ」\par{} 
勇者「お、おう」 



魔王「こっちの紙に信用できそうな部族の族長のリストと、\par{} 
 紹介状をしたためておいた。人捜しなら助力を仰ぐ\par{} 
 必要もあるだろう」 



勇者「いや、あいつはああみえて、その……。\par{} 
 動じないヤツだから。\par{} 
 きっと平気でけろっとしてると思うんだ」 



魔王「だからといって探していけない道理もあるまい」 

	
    
    

499 :以下、名無しにかわりましてVIPがお送りします :2009/09/05(土) 01:18:26.71 ID:5kaffl9OP 


勇者「魔王……」\par{} 
魔王「私が寛大で感謝するんだぞっ」 



勇者「もちろんだ。ありがとう」 



魔王「……」じぃっ\par{} 
勇者「?」 



魔王「それだけか?」\par{} 
勇者「なにが?」 



魔王「ほら、そのぅ。人間には、その、何だ……\par{} 
 親しい人と……というか親しい男女が別離をする時の\par{} 
 特別な風習があるそうではないか」 



勇者「えー。あ。ああ」\par{} 
魔王「……駄肉だからダメか?」 



勇者「何でこういうタイミングで\par{} 
 じわぁって見上げるかなっ!?」 



魔王「所有契約の項目外なのか?」 じわぁ 

	
    
    

501 :以下、名無しにかわりましてVIPがお送りします :2009/09/05(土) 01:21:29.62 ID:5kaffl9OP 


勇者「えー、あー。その」\par{} 
魔王「やっぱりスキンシップが足りないのか」 



勇者「なんでそうなる」 



魔王「実は毎週メイド長に説教されるんだ。\par{}\par{}

 『まおー様はスキンシップが足りません。\par{} 
 そもそも露出もかわいげも足りてないんですから\par{} 
 スキンシップくらいケチってどうなります?\par{} 
 いいですか? 戦争の基本は物量です。\par{} 
 飽和攻撃で殿方の理性など崩壊させてしまえば\par{} 
 戦術の必要性すらないのです』\par{}\par{}

 そう言われるんだ」 



勇者「戦術論的には正しいんだが」 



魔王「ダメなのか?」\par{} 
勇者「そ、その。照れくさいぞ。\par{} 
 そういうのはさ、ほら。\par{} 
 もっと落ち着いた時にさっ」 

	
    
    

504 :以下、名無しにかわりましてVIPがお送りします :2009/09/05(土) 01:24:44.99 ID:5kaffl9OP 


魔王「それで良く勇者が名乗れるな。\par{} 
 それでは臆病者ではないかっ」 



勇者「ば、ばか云えっ。俺は勇気にかけては\par{} 
 世界公認の第一人者、それゆえ勇者ですよ!?」 



魔王「では覚悟を決めるのだっ」\par{} 
勇者「何で開き直ってるんだよ、魔王っ」 



魔王「半年だぞ!? 雪の中にこもって\par{} 
 生活してればアドバンテージが取れて当然だろうに\par{} 
 なんだか流されるままにずるずると\par{} 
 何の進展もなく半年もの時間を浪費してしまった事実が\par{} 
 私を責めさいなんでるのだ。\par{} 
 そんな状況下でそろそろ修道院の建築も始まり、\par{} 
 夏の間には完成してしまう上に、\par{} 
 私の勇者は昔の女を探しに行ってしまうわけで\par{} 
 精神的に追い詰められない方がおかしいではないかっ」 



勇者「あー」 

	

	

508 :以下、名無しにかわりましてVIPがお送りします :2009/09/05(土) 01:29:15.28 ID:5kaffl9OP 


魔王「……」じぃ\par{} 
勇者「まったくなぁ」 



魔王「……」\par{} 
勇者「……」 ちゅ 



魔王「……むぅ」\par{} 
勇者「なんだよその恨みがましい視線はっ」 



魔王「おでこではないか」 



勇者「おでこで悪いか。気に入らないなら返せ」 



魔王「それはダメだ。勇者の全ては私に所有権がある。\par{} 
 つまりこのおでこも私の私有財産だ。議論の余地はない」 



勇者「むぅ……」\par{} 
魔王「……」 



勇者「残りは帰ってからっ!」\par{} 
魔王「約束だぞ、勇者。かならずだぞっ!」 



しゅんっ! 

	
    
    

578 :以下、名無しにかわりましてVIPがお送りします :2009/09/05(土) 14:14:24.81 ID:5kaffl9OP 


――村はずれの屋敷、中庭 



女騎士「さて、諸君らの手元にあるのは南部諸王国の\par{} 
 軍において用いられる標準的な武器、ロングソードだ。\par{} 
 この武器は威力、間合いにおいてバランスが良く、\par{} 
 鉄の国おいて鋳造された製品で質も良い。\par{} 
 重量バランス配分がこの種の武器の使い勝手を\par{} 
 決めるので、手に持って馴染むかどうか、判断の\par{} 
 参考にして欲しい」 



貴族子弟「……」\par{} 
商人子弟「……」\par{} 
軍人子弟「ばからしーでござる」 



女騎士「何か言ったか?」 



貴族子弟「……」ぷいっ\par{} 
軍人子弟「馬鹿らしいといったでござる。何で拙者が\par{} 
 女如きに剣を教わらないといけないのでござるか」 



女騎士「……」 



軍人子弟「白の剣士殿から剣を教わったのは\par{} 
 別に女に弟子入りするためではないでござるよ。\par{} 
 女は家の中でケーキでも焼いていれば良いでござる」 

	
    
    

581 :以下、名無しにかわりましてVIPがお送りします :2009/09/05(土) 14:20:50.67 ID:5kaffl9OP 


女騎士「おい、そこのデブ」\par{} 
商人子弟「ひゃ、ひゃいっ!? ぼ、ぼく?」 



女騎士「剣を両手に持って構えろ」\par{} 
商人子弟「……ううう」 



女騎士「はっ!!」 ギンッ!! 



貴族子弟「!?」\par{} 
軍人子弟「ッ!!」\par{} 
商人子弟「けけけ、剣がっ!! ま、まっぷた、真っ二つ」 



女騎士「はっ!!」 ギンッ!!\par{} 
商人子弟「み、短くなったっ!?」 



女騎士「その気になれば5cmずつ切り取ることも出来るんだぞ」 



軍人子弟「ど、ど、どうしてっ」 



女騎士「そこのゴザルに云っておく」 

	
    
    

583 :以下、名無しにかわりましてVIPがお送りします :2009/09/05(土) 14:25:44.21 ID:5kaffl9OP 


女騎士「私は、湖の国の女騎士。かつて勇者と共に\par{} 
 魔界で千の戦をくぐり抜けてきた女だ」 



貴族子弟「ゆ、勇者っ勇者様のっ!?」\par{} 
商人子弟「!?」\par{} 
軍人子弟「ま、ま、ま、まさか『鬼面の騎士』!?\par{} 
 『怪力皇女』!? 『石壁しぼりの女夜叉』!?」 



女騎士「色々詳しいじゃないか、ゴザル」 



軍人子弟「……」がくがくぶるぶる 



女騎士「これは別に怪力じゃない。技だ。\par{} 
 刃筋を安定させて、力を強度の低い場所に\par{} 
 集中させれば諸君らでも実行可能だ。\par{} 
 勇……あー。白の剣士は、素質がありすぎでな。\par{} 
 なんでも『なんとなーく』でやってしまうので\par{} 
 教師としては不適当なのだ」 



商人子弟「もしかして、白の剣士殿は、女騎士殿の\par{} 
 弟子だったのですか!?」 



貴族子弟「そ、そうかっ!」\par{} 
軍人子弟「そうでござったか……」 

	
    
    

584 :以下、名無しにかわりましてVIPがお送りします :2009/09/05(土) 14:36:22.66 ID:5kaffl9OP 


女騎士「う、うむ。そういうような……。\par{} 
 そ、そういうことだ。とっ、とにかく。\par{} 
 白の剣士は、勅命を帯びて探索の旅に出ている」 



貴族子弟「勅命……王のご命令ですか」\par{} 
軍人子弟「探索の旅! 男子の本懐でござるな!」 



女騎士「そう言うわけで、週に4回の戦闘訓練は\par{} 
 しばらくのあいだ私が受け持つ」 



商人子弟「は、はヒィ!」 



女騎士「なに。私は白の剣士とちがって\par{} 
 理論的かつ実戦的、基本に即した教練方法を\par{} 
 採用するつもりだ。諸君らの武芸を必ずや\par{} 
 実用の域まで高めよう」 



貴族子弟「勇者の仲間の騎士様に\par{} 
 剣を教授いただけるとは光栄です!」\par{} 
軍人子弟「そこまで言われては仕方ない。\par{} 
 拙者も剣の道を究めるとするでござる」 



女騎士「では、手始めに北の森を、走り込みで三周。\par{} 
 そのあと帯剣して素振りをしながら一周。\par{} 
 小川へと移動したら、腰まで水につかって\par{} 
 ロングソードの素振り500回だ」 



三子弟「「「ひぃぃぃ!?」」」 

	
    
    

587 :以下、名無しにかわりましてVIPがお送りします :2009/09/05(土) 14:41:31.45 ID:5kaffl9OP 


――村はずれの屋敷、初夏 

 

ひいぃぃぃ! ひぃぃぃぃ! 



魔王「今日も元気だな」 



メイド長「まったくです。でも、女騎士さんは\par{} 
 あれで結構楽しそうですよ?」 



魔王「そうなのか? 勇者がいなくなって\par{} 
 お尻に矢が刺さったアナグマのように怒り狂って\par{} 
 いたではないか」



メイド長「頼りにされると張り切ってしまう人\par{} 
 なんでしょう。可愛らしい人ですよ」 



魔王「む」\par{} 
メイド長「まおー様より引き締まった身体ですし」 



魔王「むぅ」\par{} 
メイド長「いえいえ。まおー様もスタイルは\par{} 
 悪くないんですよ? 出るべきところのボリュームは \par{}
 それはたいしたものです。えっちではしたない肉体です」 



魔王「メイド長の言い方の方がはしたない」 

	
    
    

588 :以下、名無しにかわりましてVIPがお送りします :2009/09/05(土) 14:52:32.71 ID:5kaffl9OP 


メイド長「しかし肉体性能は、お色気か癒し系ですのに\par{} 
 ご本人の性格がお色気とも癒しともまるで無関係なのが\par{} 
 まおー様の泣き所でしょうか?」 



魔王「ほうっておけ」 

 

がきょ、がちょ 



メイド長「なんですか? それ」 



魔王「うむ。呼び寄せた職人に依頼していた試作品だ。\par{} 
 実験して手直しして欲しい部分の指示を\par{} 
 書き付けておかないとな」 



メイド長「何に使う物なのですか?」 



魔王「羅針盤といわれているものだ。いま作っているのは\par{} 
 その改良だな。この二軸のシャフトと、大きなガラス球で\par{} 
 内部の羅針盤を水平に保つのだ」 



メイド長「ふむふむ。改良前はどうやっていたんですか?」 



魔王「水の上に磁石を浮かべていたんだ。\par{} 
 ほら、この内側の、内部に浮かんでいるのと\par{} 
 おなじ構造だな」 

	
    
    

589 :以下、名無しにかわりましてVIPがお送りします :2009/09/05(土) 15:00:01.62 ID:5kaffl9OP 


メイド長「だいたい判りました。でも、随分巨大化\par{} 
 してしまったわけですね」 



魔王「仕方ない。これは試作品だからな。\par{} 
 実用化されれば、小型化のめども立つだろう」 



メイド長「どういう改良なのですか」 



魔王「うむ、羅針盤とは方位を知るものだ。\par{} 
 この内部の水の上に浮かべた磁石が回転して\par{} 
 北の方角を教えてくれるわけだが……。\par{} 
 そのためには水面が水平安定する必要があるな」 



メイド長「はぁ」 



魔王「方位を知りたがるのは船乗りだろう?\par{} 
 揺れる船の上で、ましてや嵐なんか来たりした日には\par{} 
 水に浮かべた磁石の方向を安定させるのは至難だ」 



メイド長「じゃぁ、いままでどうやってたんですか!?」 



魔王「根性だろ」 



メイド長「……」\par{} 
魔王「……」 



メイド長「人間ってすごいですね」 

	

	

591 :以下、名無しにかわりましてVIPがお送りします :2009/09/05(土) 15:08:37.94 ID:5kaffl9OP 


魔王「まぁ、この宙づり自由式であれば\par{} 
 設置場所に難があるとは云え、揺れる船の上でも\par{} 
 下部の釣り錘によって水平が保持される」 



メイド長「ふむ。根性が無くても出来るわけですね」 



魔王「いや。人間であるというのは根性は必須だと\par{} 
 女騎士殿は云っていたから、根性はやっぱり\par{} 
 必要なのだろう。\par{} 
 この改良で軽減されるのは技能だ。\par{}
 羅針盤を扱うのは特殊な技術だったからな。\par{} 
 この簡便な装置で技術者が増えるわけだ」 



メイド長「でも、この村には海ありませんよ?」\par{} 
魔王「うむ。この装置は、売りつける」 



メイド長「買ってくれますかね?」\par{} 
魔王「まともな目利きがあれば、家ほどの\par{} 
 黄金でも積むだろうな。これで『同盟』と接触する」 



メイド長「まおー様の専門ですから、お任せします」\par{} 
魔王「まかせておけ」 



メイド長「ところでお昼は馬鈴薯で?」\par{} 
魔王「うむ、まことに馬鈴薯の揚げは美味なるぞ」にこっ 

	
    
    

595 :以下、名無しにかわりましてVIPがお送りします :2009/09/05(土) 15:18:25.86 ID:5kaffl9OP 


――魔界、黒狼砦 



黒狼鬼「うぉろろろ~ん」\par{} 
黒狼鬼「ろろろぉ~ん」 



勇者「うお。何か集まってきたぞ」 



黒狼鬼「うろろ~ん! がうっ! がうがっ!」\par{} 
勇者「おまえらっ。怪我したくなきゃ、引いてろっ!」 

 

ザガッ! ガッ!! 



黒狼鬼「ぎゃんっ!?」\par{} 
黒狼鬼「はっ……はっ……はっ……ギャウッ!」 



羽妖精「黒騎士サマー。コッチコッチ!」 \par{}
勇者「判るのか?」 



羽妖精「女王サマ、コッチコッチ」\par{} 
勇者「まかせろっ! 爆砕呪っ!」 

	
    
    

596 :以下、名無しにかわりましてVIPがお送りします :2009/09/05(土) 15:24:56.97 ID:5kaffl9OP 


羽妖精「上~コノ上~!」\par{} 
黒狼衛兵「行かせぬっ」  



勇者「なんだ、言葉がしゃべれるのもいるのかっ!?」  

 

ギンッ! ギギンッ!  



羽妖精「黒狼族ノ成体ダヨォ。\par{} 
 モット大キナノモ、イルヨォ」  



黒狼衛兵「心配するな、貴様、ここまでだっ」  



勇者「ほあちゃっ!!」  

 

ドビシィッ!!  



羽妖精「デコピン!?」\par{} 
黒狼衛兵「む、無念っ!」  

 

バタリ 

	
    
    

599 :以下、名無しにかわりましてVIPがお送りします :2009/09/05(土) 15:30:57.33 ID:5kaffl9OP 


勇者「切りがないな」\par{} 
羽妖精「一杯来ルヨォ」 



黒狼衛兵×15「ガフッ、ガフッ! オロローン!」 



勇者「面倒くさいぞ、お前ら」\par{} 
羽妖精「ダ、ダメッ! 塔ヲ壊シチャダメ!」 



勇者「む、そうか。上に女王がいるんだっけ。」\par{} 
羽妖精「ウンウンッ」 



勇者「んじゃ、えいっ!」\par{} 
黒狼衛兵「片手で岩扉をっ!?」\par{} 
黒狼衛兵「に、逃げろっ」 



勇者「ちょっと距離が必要なんだ、この技は。\par{} 
 ……あんまりうろちょろするなよ、\par{} 
 急所に当たると死んじまうぞ-。\par{} 
 えっと、たしか、こうやって\par{} 
 背中をひねる感じでぇ……」 



羽妖精「眩シイヨッ」 



勇者「光の精霊直伝、光の封印槍だっ」 

	
    
    

601 :以下、名無しにかわりましてVIPがお送りします :2009/09/05(土) 15:34:49.61 ID:5kaffl9OP 


――魔界、黒狼砦の塔の上 

 

ドッゴォォーン 



羽妖精「ケフッ。ケフッ」\par{} 
勇者「悪いな」 



羽妖精「ヒドイヨォ」 



妖精女王「何事ですっ」\par{} 
勇者「お。この人がそうかな?」 



羽妖精「女王サマッ!」 



妖精女王「羽妖精ではありませんかっ」\par{} 
勇者「こんにちは、手荒な訪問で済みません」 



羽妖精「女王サマ、コレハ人間ノ雄」\par{} 
妖精女王「みれば判ります」 



勇者「人間です」\par{} 
羽妖精「アタシ頭イー♪」 

	
    
    

602 :以下、名無しにかわりましてVIPがお送りします :2009/09/05(土) 15:39:17.34 ID:5kaffl9OP 


妖精女王「速く逃げてくださいっ。\par{} 
 魔狼将軍が来るといけません」 



勇者「倒した」 



妖精女王「まさかっ? 人間にそのような力が。\par{} 
 しかし、それだけではないのです!\par{} 
 魔狼将軍の背後にはさらなる実力を持つ\par{} 
 魔界でも高位の戦士、魔狼元帥が……」 



勇者「それも倒した。先週」 



羽妖精「!? あ、あなたは」\par{} 
妖精女王「黒騎士人間ダヨ」 



勇者「ああ。黒騎士だ。魔王の剣にして、\par{} 
 絶対忠誠を誓う魔界の執行官」 



羽妖精「カックイイヨネ」\par{} 
妖精女王「そうですか、確かにその鎧の紋章は魔王様の物。\par{} 
 いえ、もしやその鎧、魔王様ご自身の物では……?」 



勇者「……その問いに答える言葉はないぞ」 



羽妖精「カッコツケテルー」 

	
    
    

605 :以下、名無しにかわりましてVIPがお送りします :2009/09/05(土) 15:44:54.63 ID:5kaffl9OP 


妖精女王「魔王様の命令に背き、人間をさらっては\par{} 
 無益な殺生と玩弄を繰り返す魔狼族を粛正されに\par{} 
 きたのですね」うるうるっ 



勇者「いや、ついカっとなっ」\par{} 
羽妖精「……」じー 



勇者「ごほん。そうである。魔狼族の横暴、目に余る。\par{} 
 人間族に慈悲を掛けるわけではないが、魔王の\par{} 
 命令は絶対である。逆らうことは許されない」 



妖精女王「元は人間族でしょうに。何という忠誠心でしょう」 



勇者「ふははは。我は黒騎士。絶対不破の魔王の剣」 



妖精女王「魔王様の仰せの通りに」ふかぶかっ 



勇者(なんか気分良いな! 魔王の部下も!!) 



羽妖精「女王サマー」\par{} 
妖精女王「何です?」 



羽妖精「人捜シー」 

	
    
    

606 :以下、名無しにかわりましてVIPがお送りします :2009/09/05(土) 15:48:53.70 ID:5kaffl9OP 


妖精女王「人捜し?」 



勇者「ああ。そういえばそうだった。\par{} 
 あーあー。\par{} 
 魔王の命令により、我は1人の人間をさがすものなり」 



羽妖精「女王サマノトコロニ来テタ人間女ー」 



妖精女王「ああ。あの術士ですか……」\par{} 
勇者「いまは何処に?」 



妖精女王「素晴らしい魔法の素質を秘めていましたからね。\par{} 
 彼女は妖精族の魔法を学ぶと、さらなる奥義を求めると\par{} 
 云って旅に出ました」 



勇者「旅? どこへ」 



妖精女王「それは判りませんが……」 



勇者「一体何処まで努力すれば気が済むんだ、\par{} 
 あの無表情小娘。いまでも人間界最強のクセに」 



妖精女王「そういえば……」 

	
    
    

608 :以下、名無しにかわりましてVIPがお送りします :2009/09/05(土) 15:53:16.00 ID:5kaffl9OP 


勇者「そういえば?」\par{} 
羽妖精「魔界の果て、時の砂の滝が落ちる滝壺に\par{} 
 一つの古いベンチがあると。そのベンチに座った\par{}
 旅人は星の最果て、『外なる図書館』へ行くことが\par{} 
 出来ると云われています。\par{} 
 ――彼女は熱心にその伝承を調べていました」 



勇者「『外なる図書館』だな? 判った」 



妖精女王「しかしそれは伝説の場所。\par{} 
 詳しい場所やたどり着く方法は妖精族でも知りません」 



勇者「そのようなことは問題ではない。\par{} 
 魔王の命にしたがいどのような場所であろうと\par{} 
 必ず見つけ出す」 



羽妖精「カッコイー!」 



妖精女王「ご無事をお祈りいたします」 



勇者「妖精族は元の領地に戻り、いままでと同じく\par{} 
 その民を治めて暮らすようにとの魔王の仰せだ」 



妖精女王「魔界を治める魔王様の治世に幸いあれ」 

	
    
    

610 :以下、名無しにかわりましてVIPがお送りします :2009/09/05(土) 15:57:46.97 ID:5kaffl9OP 


勇者「えー、こほんこほん。\par{} 
 魔狼族の生き残りにはきつく申し渡しておく。\par{} 
 元来魔狼族は誇り高い自由不羈の民のはず。\par{} 
 穏健派を中心に魔王の民として、その誇りを\par{} 
 まもるような生き方にするが良いだろう」 



妖精女王「妖精族は魔狼族からの迫害さえなければ \par{}
 異存はありませぬ。遺恨は伝えぬと誓約しましょう」 



勇者「……その寛容、魔王に伝えよう。\par{} 
 では、時間だ。我は探索の旅に戻らなければ\par{} 
 ならない。縁があればまた逢おう」 



妖精女王「このご恩、けして忘れません」 



しゅんっ!! 



羽妖精「カッコイー!」 



妖精女王「妖精族は救われましたね。\par{} 
 魔王様にあのような部下がいるとは……。\par{} 
 ただのお飾り、柔弱で無能な王と云われてきましたが\par{} 
 何かが変わり始めているのかも知れません。\par{} 
 魔王様と云えば――あっ」 

	
    
    

612 :以下、名無しにかわりましてVIPがお送りします :2009/09/05(土) 16:01:44.28 ID:5kaffl9OP 


羽妖精「ドシタノー?」 



妖精女王「魔王様といえば……」 



(時の砂の滝が落ちる滝壺―― \par{}
 一つの古いベンチ\par{} 
 星の最果て―― \par{}
 『外なる図書館』――) 

妖精女王「『外なる図書館』……」 



羽妖精「?」 



妖精女王「『外なる図書館』に引きこもる、\par{} 
 魔族の中でも変わり者の一族がいると……。\par{} 
 その一族は知識を求め、過去と未来を幻視し\par{} 
 『外なる智慧』を身につけて、憧れに魂を燃やすと……」 



羽妖精「?」 



妖精女王「魔王様って、魔王って……\par{} 
 何なのでしょうか……」 

	
    
    

616 :以下、名無しにかわりましてVIPがお送りします :2009/09/05(土) 16:11:56.87 ID:5kaffl9OP 


――南氷海巨大湾岸都市、商業会館 



青年商人「ふふぅん、こいつはたまげた。\par{} 
 全く度肝を抜かれた、まいったな」 



中年商人「よう。どうした、呼び出して」 



辣腕会計「まだ夕食には早いでしょう?\par{} 
 どうしたんです? \par{}
 湖の国のワインでも暴落しましたか? \par{}
 それとも聖王都の為替変動ですか?」 



青年商人「まぁ、こいつをみてくれ。\par{} 
 午前中に届いて、やっと組み立てたんだな、これが」 



中年商人「――ッ!!」 \par{}
辣腕会計「こ、これは……」 



青年商人「まぁ、一目でわかるか」 



中年商人「これは羅針盤だな? 見たことのない形状だが」\par{} 
辣腕会計「ですが、見ただけで判ります」 

	
    
    

617 :以下、名無しにかわりましてVIPがお送りします :2009/09/05(土) 16:16:42.64 ID:5kaffl9OP 


青年商人「何処のどいつの工夫だかは判らないが\par{} 
 こいつはたいしたものだ。恐ろしいもんだ」 



中年商人「ああ、頭を大石で殴られた気分だ」 



辣腕会計「これは……二つの円環で、どんな場所に\par{} 
 置いても水平が保たれるのですね? さらに\par{} 
 この重りで安定させるわけですか……」 



青年商人「ああ。理屈は見れば判る。\par{} 
 特別な装置が使ってあるわけでもないが、すごい発明だ」 



中年商人「これを見せれば、銅の国の技術士ならば\par{} 
 もっと小型にも出来るだろう。やったな! おい!\par{} 
 何処でこんな物手に入れたんだ。\par{} 
 この功績の価値は、幹部候補生、いや、10人委員会に\par{} 
 入るのも夢じゃないぞ、お前!」 



辣腕会計「ええ、この発明は『同盟』に巨大な利益を\par{} 
 もたらすでしょう、同志よ!」 



青年商人「こいつは世界を変えるな」\par{} 
中年商人「ああ、世界を変えてしまうだろうな」 

	
    
    

620 :以下、名無しにかわりましてVIPがお送りします :2009/09/05(土) 16:24:36.89 ID:5kaffl9OP 


青年商人「さぁて、なかなか」\par{} 
中年商人「ふむ、たしかに」 



辣腕会計「どうしたんです?」 



青年商人「いや、なに。これがここにある、\par{} 
 その意味合いをな」 



中年商人「確かに巨大な利益は目の前だ。\par{} 
 酒樽一杯の蒸留酒のような物。嬉しくてたまらんわな。\par{} 
 しかし、その酒樽にはもう蒸留酒はのこっていないのかな?\par{} 
 あるいは罠の可能性は? \par{}
 俺たちは商人だ。酔っぱらいじゃぁ、無い。\par{} 
 そこんところを頭を使わないとな」 



辣腕会計「そうですね、ふむ」 



青年商人「まず、第一にこれを発明したのは俺じゃない。\par{} 
 俺にこれをとどけた人間がいるんだ。\par{} 
 そいつの思惑を考えなければいけないだろうな」 

	
    
    

622 :以下、名無しにかわりましてVIPがお送りします :2009/09/05(土) 16:28:46.83 ID:5kaffl9OP 


中年商人「身元はわかっているのか?」 



青年商人「まぁ、本人からの手紙にはな。\par{} 
 『紅の学士』とある。送り主は南部諸王国の西の外れ\par{} 
 冬越し村というところだ」 



中年商人「小さな寒村だな」\par{} 
辣腕会計「目立った特産品はなかったと記憶していますが。 
 ――いや、まてよ」 

 

がさごそ 



青年商人「どうした?」\par{} 
辣腕会計「確か、報告にその名前が……。\par{} 
 ああ、ありました。この夏に、湖畔修道会の修道院が \par{}
 その村に建築されたようです」 



中年商人「湖畔修道会? 湖の国の?\par{} 
 もうそんな辺境まで勢力を広げたのか?」 



辣腕会計「いえ、勢力範囲から遠く離れた場所に突然\par{} 
 修道院をつくったようです。教化も進んでいないでしょう。\par{} 
 ですから報告書に特記されていたのでしょうが……」 

	
    
    

623 :以下、名無しにかわりましてVIPがお送りします :2009/09/05(土) 16:32:33.94 ID:5kaffl9OP 


青年商人「ふむ。黒だ」\par{} 
中年商人「関係があると睨んで良いだろうな」 



辣腕会計「接触ですか?」\par{} 
青年商人「それはどうあれ、その必要があるだろう。\par{} 
 『同盟』がこの羅針盤から得られる利益を\par{} 
 最大化するためには、この工夫を独占しなければならない」 



中年商人「だが、この工夫は、一目見ただけで\par{} 
 その革新性が判る。革新性が判りやすいってのは\par{} 
 売る時にはまたとない武器だが、\par{} 
 真似して作るのも簡単だって云う弱点があるな」 



辣腕会計「そのとおりですね」 



青年商人「『同盟』がこの羅針盤を部外秘として\par{} 
 『同盟』所属の船舶だけに装備し、交易優位性を\par{} 
 あげるにしろ、全中央大陸国家に販売して利益を\par{} 
 上げるにしろ。発明元のこの学士と交渉する必要がある」



辣腕会計「真似はできても、あちらが他の様々な\par{} 
 組織や国家に同様の売り込みをしないとも限らない。\par{} 
 ……そうですね?」 



中年商人「場合によっては……」\par{} 
青年商人「そう言うことにはならんで欲しいな。\par{} 
 我らは商人なのだから」 

	
    
    

625 :以下、名無しにかわりましてVIPがお送りします :2009/09/05(土) 16:39:19.13 ID:5kaffl9OP 


――冬越しの村、夏 



小さな村人「ほーぅい、ほーぅい」\par{} 
痩せた村人「ほーぅい」 



小さな村人「なんて良い天気なんだろう」\par{} 
痩せた村人「まったくだなや、大麦さんもそだっとるよ」 



小さな村人「修道院が出来てから、色々教えてもらえるしなや」\par{} 
痩せた村人「おや、修道士さんだべさ」 



修道士「こんにちは、精が出ますね」\par{} 
小さな村人「こんにちは」ぺこり\par{} 
痩せた村人「こんにちはだなや」ぺこり 



修道士「今日はどうされています?」\par{} 
小さな村人「わしは川でマスを釣ってきただぁよ」\par{} 
痩せた村人「わしは薪をつくってただぁ」 



修道士「それは良かった」\par{} 
小さな村人「修道士さんは?」 

	
    
    

627 :以下、名無しにかわりましてVIPがお送りします :2009/09/05(土) 16:42:35.55 ID:5kaffl9OP 


修道士「ははは、実はですね。\par{} 
 試しに作っていた作物が、早くも二回目の 
 収穫を迎えたんですよ!」 



小さな村人「なんだか、修道士さんも嬉しそうだなや!」 



修道士「ええ、嬉しいです。大地が恵んでくださった。\par{} 
 これは光の精霊様が頑張れとおっしゃってくれて\par{} 
 いるわけですよ。それで、この収穫の報告に\par{} 
 学士様への所へ行こうかと思いましてね」 



小さな村人「そうかそうか、そうだったんだべ」 



修道士「ええ。この作物、馬鈴薯というのですが \par{}
 甘くてほくほくして大層美味しいのですよ」 



小さな村人「そうかぁ、一度食べてみたいだなやー」\par{} 
痩せた村人「どんな味なんだろう」 



魔王「招待するぞ?」\par{} 
修道士「ああ、これは学士様!」 



小さな村人「学士様、こんにちはですだよ」\par{} 
痩せた村人「こんにちは、学士様。良い天気ですだ」 

	
    
    

629 :以下、名無しにかわりましてVIPがお送りします :2009/09/05(土) 16:44:34.64 ID:5kaffl9OP 


修道士「いま、ご報告にうかがおうかと」\par{} 
魔王「ああ、ありがとう。そろそろかと思っていたのだ」 



メイド姉妹 ぺこり 



修道士「計画通りに取れました。いやいや、好調ですね。\par{} 
 荷車二台分はたっぷりと取れたかと思います」 



魔王「土壌採集は?」 



修道士「指示通り、六カ所でそれぞれ\par{} 
 樽一杯分づつを保存してあります。それにしても\par{} 
 我が修道会も農業技術の集積は進めてきましたが\par{} 
 前代未聞の方法ですね」 



魔王「結果が出てくれれば嬉しいのだがな。\par{} 
 ふむ、これか」 



修道士「ええ、良く育っています」 



魔王「よし、振る舞いをしよう」\par{} 
修道士「振る舞い、ですか?」 



魔王「こいつを広めるためには、何はともあれ、\par{} 
 皆に食べてもらわねば始まるまい?\par{} 
 それには、宴でも開いて振る舞うのが一番だ」 

	
    
    

630 :以下、名無しにかわりましてVIPがお送りします :2009/09/05(土) 16:47:57.94 ID:5kaffl9OP 


小さな村人「ほんとうですか? 学士様」\par{} 
痩せた村人「良いのでございますか」 



修道士「どうです?」 



魔王「もちろん本当だ。修道士どの、いかがだろう? \par{}
 修道院の前庭を借りることが出来ようか?」 



修道士「もちろんですよ。でも、この馬鈴薯は売って\par{} 
 資金に充てるのかと思っていましたよ」 



魔王「金はもちろん欲しいが、独り占めするつもりはない。\par{} 
 飢えなく、皆が豊かになる方法を考えないと、\par{} 
 先が続かない。そのためには村の皆の手助けが必要だ」 



小さな村人「うわぁ、食べてみたいですだ学士様」 \par{}
痩せた村人「おらのところの畑でも作れるようになるですだ?」 



修道士「ああ。もちろんさ。\par{} 
 作ってみたが、小麦と比べて世話が大変と云うこともない。\par{} 
 もちろんいくつか気をつけなければならないことは\par{} 
 あるけれど、それは修道会で教えてあげることが出来る」 

	
    
    

632 :以下、名無しにかわりましてVIPがお送りします :2009/09/05(土) 16:52:32.61 ID:5kaffl9OP 


小さな村人「さっそく家内におしえてやらにゃぁ!」 



魔王「おお、そうだ。宴の支度に手が足りないかも知れぬ。\par{} 
 奥方の手が空いていれば来ていただけると助かると\par{} 
 思うぞ。なあ、修道士殿」 



小さな村人「あーれ。学士様。奥方なんて照れるだよ。\par{} 
 うちのはただの母ちゃんだよ。でも、そう云われると\par{} 
 なんだか母ちゃんも悪い気はせんかもなぁ。\par{} 
 こっぱずかしいな。でも直ぐに行かせるから!」 



修道士「そうですね、ご報告もしたということにして\par{} 
 私も帰って他の修道士、騎士院長にも伝えて参ります。\par{} 
 ああ、そうだ。その、料理はどうすればよいでしょう」 



魔王「心配ない。いってくれるな?」 



メイド姉「はい」ぺこり\par{} 
メイド妹「いきまーっす」 



修道士「それは助かります。まだこの馬鈴薯の調理方法を\par{} 
 研究した訳じゃありませんからね」 



魔王「あー。くれぐれも云っておくが、\par{} 
 揚げ馬鈴薯だけは必ず作るのだぞ?」 

	
    
    

675 :以下、名無しにかわりましてVIPがお送りします :2009/09/05(土) 20:18:50.82 ID:5kaffl9OP 


――冬の国、王宮 



王子「じぃ、じぃ~」\par{} 
執事「なんでございましょう、若」 



王子「若はやめろ。俺はもう二十歳だ」\par{} 
執事「どうしたのでございます?」 



王子「じぃは馬鈴薯なる物を知ってるか?」\par{} 
執事「ははぁん。若も馬鈴薯を食べたので?」 



王子「ああ、食べた。美味いな!」\par{} 
執事「何でも旅の学者がこの地へもたらしたとか」 



王子「うまいうえに、俺たちの貧しい国でも\par{} 
 もっぎゅもっぎゅ……栽培できるらしいな」\par{} 
執事「さようでございますなぁ」 



王子「情報はあるのか?」\par{} 
執事「ございますとも」 

	
    
    

678 :以下、名無しにかわりましてVIPがお送りします :2009/09/05(土) 20:24:10.25 ID:5kaffl9OP 


王子「ふむふむ」\par{} 
執事「こちらの書類は関連項目でございます」 



ぺらり 



王子「では、湖畔修道会が主導で栽培を\par{} 
 推し進めているのだな?」 



執事「そうなりますな。また、この湖畔修道会は\par{} 
 合わせて様々な改良を施しているようで」 



王子「ふむ、どのような?」 



執事「まずは、四輪作といわれるものですな。触れ込みに\par{} 
 よれば大地の恵みを目減りさせずに、四年周期で麦作を\par{} 
 行なう手法です。以前の三輪作にくらべて、小麦はもと\par{} 
 より豚や羊などを安定して供給できるようですな」 



王子「冬のあいだにもか?」 



執事「冬のあいだには、家畜にカブを食べさせるそうです」 

	
    
    

680 :以下、名無しにかわりましてVIPがお送りします :2009/09/05(土) 20:30:04.11 ID:5kaffl9OP 


執事「それから、えー。農機具の改良、修道学院の設立」\par{} 
王子「学舎か、ふむ」 



執事「さらにこの度作られたのが、『風車』です」\par{} 
王子「それはなんだ?」 



執事「『水車』に似たものですが、川の流れではなく\par{} 
 風の流れをくみとって、動力にしているようですな。\par{} 
 修道会が雇い入れた船大工の一派が工夫して作った\par{} 
 そうですが。我が国北部の高地には、充分な水源が\par{} 
 ありませんから、普及すれば便利でしょう」 



王子「……ふむ」\par{} 
執事「お気になりますか?」 



王子「まぁな。税収が上がっているのは嬉しいが……。\par{} 
 まぁ、それで戦争を終わらせられるわけでなし。\par{} 
 しょせんイモでは我が国を救うことも出来ないが\par{} 
 ……まぁ、なんでも目は通しておかんとな」 



執事「そうですね。税収は荘園ごと、村ごとに納め\par{} 
 させますから、一概にどのくらいの効果があるかは\par{} 
 判りませんが、修道会が関与すると5%ほど税収が\par{} 
 上がるようですな」 



王子「大きいな」\par{} 
執事「小さく考えてはいけませんよ。1年足らずの\par{} 
 あいだにそれだけの改革を見せたわけですから\par{} 
 来年以降どうなるか判りません」 

	
    
    

682 :以下、名無しにかわりましてVIPがお送りします :2009/09/05(土) 20:35:40.50 ID:5kaffl9OP 


王子「冬小麦の収穫はこれからであるしな」\par{} 
執事「その馬鈴薯なる食物は、年に数回収穫できる\par{} 
 そうですな」 



王子「そうなのか?」\par{} 
執事「驚きですが、事実のようです」 



王子「ふむ」\par{} 
執事「税収の形には表れないものの、農民の暮らしには\par{} 
 大いなる恩恵を与えていると云って良いでしょうな」 



王子「じぃの云うことならば信じぬ訳にはいかないな」\par{} 
執事「ありがたいことですなぁ」 



王子「何らかの施策をするべきだろうか?」\par{} 
執事「そうですなぁ。まだ始まったばかりのようですから\par{} 
 傍観していても良いのではないでしょうか」 



王子「ふむふむ」 
執事「修道会はこの運動で、我が国を始め、南部諸王国に\par{} 
 確固たる地盤を築く狙いがあると思います。\par{} 
 運動の結果を出せれば、向こうから王宮に接触を\par{} 
 持ってくるかと思いますな」 

	
    
    

683 :以下、名無しにかわりましてVIPがお送りします :2009/09/05(土) 20:41:15.79 ID:5kaffl9OP 


王子「そうか。修道会の指導者は……」\par{} 
執事「女騎士ですな」 



王子「挨拶くらいしなくて良いのか? 顔見知りではないか」 



執事「まぁ、向こうは現役の時から思い込んだら\par{} 
 動かない高潔なる気位の持ち主でしたからなぁ。\par{} 
 私も恨みに思われているでしょうな。\par{} 
 いわば裏切り者ですから」 



王子「そうか……。すまない」\par{} 
執事「もったいないお言葉ですな、若」 



王子「今年は魔族の動きが鈍い」\par{} 
執事「おそらく、勇者の噂は真実でしょう」 



王子「その勇者を、手を下したわけではないとは云え\par{} 
 死地に追いやったのは我々だ……。 \par{}
 勇者が生還したという情報はないのか?」 



執事「ございませんな」 



王子「この戦争、終わるわけには行かぬのか」 



執事「いま戦争を終えれば、真っ先に消滅するのは\par{} 
 我が国でしょう」 

	
    
    

687 :以下、名無しにかわりましてVIPがお送りします :2009/09/05(土) 20:46:23.82 ID:5kaffl9OP 


王子「……」 



執事「この冬の国、それをいえばおなじ南部諸王国である\par{} 
 氷の国、白夜の国、鉄の国はそれぞれ気候も厳しく、\par{} 
 充分な食料も取れません。最下層の国々です。\par{} 
 いま現在は魔族との大戦争の前線として\par{} 
 中央大陸全土からの資金援助と食料援助がとどいている。\par{} 
 中央大陸の盾と云えば聞こえは良いですが\par{} 
 詰まるところ走狗になっているに過ぎません。\par{} 
 援助がとどこおれば、人々は全て飢えて死ぬでしょうな」 



王子「しかし、それを知らせず、兵をただ消耗させるのは\par{} 
 兵達に対する裏切り行為だ。茶番ではないか」 



執事「ええ、茶番ですとも。\par{} 
 しかし茶番をする存在が、王族です」 



王子「……戦場で雄々しく散るのは良い。\par{} 
 それは氷海の戦士の直系たる我が血にふさわしい。\par{} 
 だが民を欺き、その命を代価にして生を購うのは……」 



執事「若、辛抱ください。\par{} 
 どうか、民を見捨てずにいてください」 

	
    
    

689 :以下、名無しにかわりましてVIPがお送りします :2009/09/05(土) 20:55:15.92 ID:5kaffl9OP 


――魔界、紅玉神殿 



勇者「……うー。疲れた。だるい。腹減った」\par{} 
火竜大公「や、やるな。黒騎士よ」 



勇者「いい加減タフだな、火竜大公」\par{} 
火竜大公「……退くわけには、行かぬっ」 



勇者「おまえ、十回くらいしっぽも腕も切られてるじゃん」\par{} 
火竜大公「何度でも生やすまでだ!」 



勇者「うぁー。どうすれば良いんだよぅ、この変態」 



火竜大公「我が命を絶てば良かろう。\par{} 
 その実力を持っているクセに何を悠長なことをしておる!」 



勇者「別に殺したくてやってるわけじゃない。\par{} 
 編成中の軍勢を退かせてくれれば済む話だろう」 



火竜大公「それは出来ぬ。火竜の勇士によって\par{} 
 『開門都市』は奪還する必要があるのだ」 



勇者「あー。やっぱしそれかよぉ」 

	
    
    

691 :以下、名無しにかわりましてVIPがお送りします :2009/09/05(土) 20:58:39.54 ID:5kaffl9OP 


火竜大公「貴様もだ! 貴様も魔王様直属の執行官で\par{} 
 あるのならば、人間どもに奪われた魔界の都市を\par{} 
 奪い返すのが筋という物であろうにっ!」 



勇者「それは云うとおりなんだけどさ」 



火竜大公「何を躊躇う。人間を皆殺しにすべきではないかっ!」 



勇者「とりあえず、魔王は『開門都市』奪還の命令を\par{} 
 発してはいないんだよ」 



火竜大公「魔王がふぬけなのだ!\par{} 
 わが竜族から魔王が出ていれば、あのような柔弱な弱腰の\par{} 
 魔王などいただかぬでもすんだろうにっ」 



勇者「つまり、魔王に弓引くのか?」\par{} 
火竜大公「……」 



勇者「それは盟約に背くよな。さんざん諸侯が争って\par{} 
 滅亡寸前まで何回も行った魔界が、なんとかやっと\par{} 
 つくった協定らしき協定だもんな」 

	
    
    

694 :以下、名無しにかわりましてVIPがお送りします :2009/09/05(土) 21:07:55.39 ID:5kaffl9OP 


火竜大公「魔王は『開門都市』奪還の命令を発してはいない」\par{} 
勇者「うん」 



火竜大公「だがしかし、禁止の命令を発したわけでもない」\par{} 
勇者「あー。気がついちゃってるよ、このおっさん」 



火竜大公「諸侯に檄を発して、魔王の名をかたり \par{}
 『開門都市』奪還を目指すなら、それは盟約に触れようが\par{} 
 我が部族だけで向かうのであれば、\par{} 
 それは王である私の決定だ。\par{} 
 誰に口を挟まれる云われもないっ!」 



勇者「俺に勝てればな」 



火竜大公「ならば殺すが良いっ!\par{} 
 魔界の溶岩の中で生を受けた火竜大公、逃げも隠れもせんっ!」 



勇者「なんかもー。難しいなぁ。\par{} 
 気に入らないヤツ、刃向かうヤツをかたっぱしから\par{} 
 ぶっ飛ばせた勇者生活が懐かしい……。\par{} 
 あの頃は殺さないように話をまとめる苦労なんて 
 全然しなかったぞ……」 

	
    
    

697 :以下、名無しにかわりましてVIPがお送りします :2009/09/05(土) 21:15:45.78 ID:5kaffl9OP 


勇者「火竜大公」\par{} 
火竜大公「何だ、黒騎士」 



勇者「では、俺があの街に先乗りをしよう」\par{} 
火竜大公「……」 



勇者「あの街、『開門都市』は\par{} 
 魔族があがめる片目の神の聖地だ。\par{} 
 そこを人間に支配されるのは苦痛だろう。\par{} 
 それは判る。しかしまた、その聖地の守りを忘れ\par{} 
 人間世界を攻めるに酔っていた竜族の罪もあると知れ」 



火竜大公「それは……」 



勇者「言い訳無用。……人間が憎いのは判るが\par{} 
 あの都市は彼らが戦争で奪ったのだ。\par{} 
 争いの勝者は神聖だ。その魔界の不文律を忘れるな。\par{} 
 特にその敗北が油断から成されたのなら、なおさらだ」 



火竜大公「……ぐぐ」 

	
    
    

701 :以下、名無しにかわりましてVIPがお送りします :2009/09/05(土) 21:21:37.65 ID:5kaffl9OP 


勇者「それに、火竜大公の軍で攻めたところで\par{} 
 あの街はこの魔界で唯一人間が暮らす街だ。\par{} 
 たやすく奪還できるとはかぎらない。\par{} 
 精鋭たる聖鍵遠征軍が守っているのだからな。\par{} 
 悪くすれば、火竜の民は全滅だ。\par{} 
 それを望むのか、火竜大公」 



火竜大公「そのようなこと、やって見ねば判らぬ!」 



勇者「次の春まで時間をくれ」\par{} 
火竜大公「……」 



勇者「黒騎士が、魔王の名にかけて誓おう。\par{} 
 『開門都市』を取り戻し、魔王の直轄地とする」 



火竜大公「魔王の、直轄地に!?」 



勇者「火竜の一族の関心は誇りだろう? \par{}
 魔王の軍勢が取り戻し、直轄地になるのであれば\par{} 
 問題なかろう。魔王はその柔弱という評判を払拭できる」 

	
    
    

703 :以下、名無しにかわりましてVIPがお送りします :2009/09/05(土) 21:25:30.11 ID:5kaffl9OP 


火竜大公「しかし、もし約束をたがえれば」 



勇者「そのときは魔王がまさに弱腰と云うことだろう」 



火竜大公「容赦はせぬぞ?」 



勇者「ああ、魔王は魔王にふさわしくない。\par{} 
 そのときは魔王の位を譲り渡そう。黒騎士が約束する」 



火竜大公「……」\par{} 
勇者「どうだ?」 



火竜大公「よかろう」\par{} 
勇者「おー! よかった」 



火竜大公「おぬしには男気がある! だれか、だれかある\par{} 
 公女を呼んで参れ!」 



火竜公女「おとうさま、私はここに」\par{} 
勇者「えーっと」 



火竜大公「約束を見事果たした暁には、この公女を\par{} 
 くれてやる! 妻にでも妾にでもするが良い! がはははは」 

	
    
    

718 :以下、名無しにかわりましてVIPがお送りします :2009/09/05(土) 21:34:44.87 ID:5kaffl9OP 


――冬越しの村、村はずれの屋敷 



魔王「こっ、これでいいかの?」\par{} 
メイド長「ええ。あらあら、まぁまぁ。見違えましたね」 



魔王「何だ、そのコメントは」\par{} 
メイド長「勇者様がいなくなってから\par{} 
 まったくお召し物に頓着なさいませんでしたからね」 



魔王「『いなくなって』などと不吉なことを云うな。\par{} 
 ちょっぴり出張しているだけではないか」 



メイド長「ええ、もちろん。まおー様が捨てられた\par{} 
 女であるかのような印象を持たれてしまったのならば\par{} 
 その誤解、このメイド長一生の不覚ですわ」 



魔王「……」 



メイド長「今日は綺麗ですよ、まおー様」\par{} 
魔王「むぅ。釈然としない」 



メイド長「とはいっても、交渉事ですからねぇ。\par{} 
 多少は見栄えを良くしないと」

	
    
    

720 :以下、名無しにかわりましてVIPがお送りします :2009/09/05(土) 21:41:44.31 ID:5kaffl9OP 


魔王「それにしても、なんというか」\par{} 
メイド長「?」 



魔王「ちょっとビラビラしすぎではないか? このドレス」\par{} 
メイド長「素敵ですよ?」 



魔王「それに襟ぐりが随分深いような気がする」\par{} 
メイド長「それくらいがお洒落なんですよ」 



魔王「ううう」\par{} 
メイド長「みっともない駄肉なので恥ずかしいですか?」 



魔王「ええーい、うるさい! そ、そんなに駄肉ではない!\par{} 
 女騎士殿もグラマーですとくれたし、みなそういってる。\par{} 
 ちょっぴり母性的なだけではないかっ」 



メイド長「人格的母性のない肉を駄肉というのです」\par{} 
魔王「ううう。今日のメイド長は厳しい」 



メイド長「ちょっと気が立ってるんですよ。\par{} 
 警備体制を整える関係で」 



魔王「どうなっている?」 

	
    
    

722 :以下、名無しにかわりましてVIPがお送りします :2009/09/05(土) 21:49:36.34 ID:5kaffl9OP 


メイド長「妖霊と夜精霊を配置しています。\par{} 
 まぁ、軍でも出てこなければ充分でしょうが……」 



魔王「心配か?」 



メイド長「相手が貴族や軍人ならばともかく、\par{} 
 『同盟』の商人ですからね。その点に関しては\par{} 
 まおー様におまかせするしかないわけで」 



魔王「信用なさ過ぎだな、わたしなのか」\par{} 
メイド長「いえ、お手伝いできないことが不安なのです」 



魔王「しかたない。これは避けては通れない関門なのだ」\par{} 
メイド長「せめて勇者様がいてくだされば」 



魔王「勇者に役目が渡るとすれば、それは交渉が\par{} 
 失敗した時だからな。そうなったら逃げる段階だ。\par{} 
 だから意味はない」 



メイド長「あんまり強がると殿方は不安だそうですよ」 



魔王「へ?」

	
    
    

725 :以下、名無しにかわりましてVIPがお送りします :2009/09/05(土) 21:55:31.94 ID:5kaffl9OP 


メイド長「まぁ、それはいいですわ」ひらひら\par{} 
魔王「あしらうな」 



メイド長「そろそろでしょうか」 



メイド妹「お客様を客間にお通ししました~。\par{} 
 いまお姉ちゃんがお茶を入れてます~」 



メイド長「語尾を不必要に伸ばさない」\par{} 
メイド妹「はーい♪」 



メイド長「まおー様? 準備はよろしいですか?」\par{} 
魔王「ああ。このボタンをはめてはダメなのか? 



メイド長「そのボタンは飾りボタンです。\par{} 
 はめる目的ではありません」 



魔王「上から実験用の白衣を羽織るとか! 学者らしく!」\par{} 
メイド長「お笑い芸人じゃないんですから」 



メイド妹「当主様、おっぱい格好いいよ♪」 



メイド長「まったくこの娘は。さぁ、まおー様」\par{} 
魔王「ああ、しかたない。出陣だ!」 

	
    
    

727 :以下、名無しにかわりましてVIPがお送りします :2009/09/05(土) 22:02:32.75 ID:5kaffl9OP 


――冬越しの村、客間 

 

かちゃり 



青年商人「やぁ、これは!」\par{} 
辣腕会計「ほほう」 



魔王「お待たせして済まないな。私がこの館の当主\par{} 
 といっても無位無冠の学士だ。紅の学士と呼んでくれ」 



青年商人「はじめまして。私は『同盟』の南氷海西方を\par{} 
 担当しております青年商人です」 



辣腕会計「今回のご挨拶に動向させていただいた\par{} 
 会計でございます。以後、お見知りおきを」 



魔王「いや、ご丁寧なご挨拶、痛みいる」 



青年商人「正直驚きが隠せません! 学士にして発明家\par{} 
 農業への造詣も深いとのことで、\par{} 
 言葉は悪いですが、ご高齢の老師かと想像していたのですが\par{} 
 こんなに麗しいご婦人にお目にかかる事ができるとは!」 

	
    
    

730 :以下、名無しにかわりましてVIPがお送りします :2009/09/05(土) 22:08:18.18 ID:5kaffl9OP 


魔王「そんなに褒められては何を話して良いのか、\par{} 
 言葉を失ってしまいますな」 



青年商人「いえいえ、学士様はその英知だけでなく\par{} 
 美しさでも我らに光を与えてくれるようですよ」にこにこ 



メイド長(商人のお世辞とはいえ、すごい威力ですね) 



魔王「交渉を有利に進めようと思う女の浅知恵だ\par{} 
 どうか笑って許して欲しい」 



メイド長(おお。まおー様。気合いの入った防御ですねー) 



青年商人「いえいえ。……あのような羅針盤を送られては\par{} 
 駆けつけないわけには参りませんよ」 



魔王「それにしては一月もの時間がかかったのは?」 



青年商人「ははは。これはお恥ずかしい。\par{} 
 私のような駆け出しの商人が、『同盟』において\par{} 
 今回のような大規模な案件をこなすにあたっては\par{} 
 様々に根回しが必要でして」 



辣腕会計「お待たせして申し訳ありません」

	
    
    

732 :以下、名無しにかわりましてVIPがお送りします :2009/09/05(土) 22:14:09.35 ID:5kaffl9OP 


魔王「さて、では交渉に入りたいと思うのだが\par{} 
 まずはこれを見て欲しい」 



青年商人「これは……?」\par{} 
辣腕会計「穀物ですか? 見たことはありませんが」 



魔王「これは玉蜀黍という植物だ」\par{} 
青年商人「ほほう」 \par{}
辣腕会計「玉蜀黍、ですか」 



魔王「この食物の特性については、\par{} 
 こちらの書類にまとめてある。\par{} 
 これはお持ち帰りいただいて結構だ。\par{} 
 いまとりあえず、この場では口頭にて説明させて\par{} 
 いただこう」 



青年商人「窺いましょう、学士様」 



魔王「この玉蜀黍は一年草でな。最大の特性は水が\par{} 
 少なくとも順調な生育が望めることだ。むしろ水が多い\par{} 
 場合は生育に悪影響がある。もちろん最低限の水分は\par{} 
 必要だがな。発芽の温度として30度が必要となる」 



青年商人「30度、ですか」\par{} 
辣腕会計「かなり高い温度ですね」 

	
    
    

734 :以下、名無しにかわりましてVIPがお送りします :2009/09/05(土) 22:19:17.89 ID:5kaffl9OP 


魔王「ああ、そうだ。小麦とはまったく栽培の思考を\par{} 
 切り替える必要がある」\par{} 
青年商人「ふむ」 



魔王「つまり、この玉蜀黍は、いままで植物の耕作に\par{} 
 適さないとされていた大陸中央部の荒れ地に\par{} 
 ふさわしい作物なのだ」 



青年商人「……」 



魔王「食用として利用する場合は、完全に完熟させて\par{} 
 乾燥させた粒を製粉してパンのようにすることも出来るし、\par{} 
 饅頭のようにすることも出来よう。\par{} 
 この粉には香ばしくてわずかな甘みがある。\par{} 
 乾燥させることにより、貯蔵、保管にも優れている。\par{} 
 畜産のための飼料としては、\par{} 
 大麦やカブの数倍の効率が見込める」 



魔王「また、食用外への利用も幅広い。\par{} 
 油分の多いこの食物は、油を取り出すことが可能だ」 



青年商人「……油ですか」\par{} 
辣腕会計「……」 



魔王「うむ。玉蜀黍馬車一台あたり、ビン二本ほどだがな。\par{} 
 しかも、この油は製粉するのと同時にとることが出来る。\par{} 
 つまり、両方取れると云うことだ。\par{} 
 油は食用に用いることはもちろん、様々な用途で使えよう」 

	
    
    

736 :以下、名無しにかわりましてVIPがお送りします :2009/09/05(土) 22:25:56.13 ID:5kaffl9OP 


青年商人「たしかに……。油の需要は年々増える\par{} 
 傾向があります。『同盟』でもとりあつかっていますが」 



魔王「商人どの、これは新しい商売の形だと考えて欲しい」 



青年商人「……」 



魔王「確かに巨大な資本が必要だ。\par{} 
 その資本をもちいて、いまは全く役に立っていない荒野に\par{} 
 人を送り、バックアップすることにより開拓を行なう。\par{} 
 玉蜀黍を栽培するための開拓村だ。\par{} 
 まったく開拓されていない場所は確かに開拓に\par{} 
 手間も資本もかかろうが、その分、計画的に物事を\par{} 
 行なうことが可能だ。整地して区画整理を行なった\par{} 
 農地での大規模栽培は、現在中央大陸の各所で見られる\par{} 
 ような小さな農地でモザイクになってしまったような\par{} 
 農場による農業より遙かに集約的な体制での\par{} 
 栽培を可能にするだろう」 



魔王「しかも、そこで新しくできた開拓村は\par{} 
 完全に『同盟』の影響下にある巨大な市場になるだろう。\par{} 
 玉蜀黍以外の作物を始め、木材や塩、鉄、布、\par{} 
 ありとあらゆる消費物を購入する新しい顧客となる」 



青年商人「……」 

	
    
    

739 :以下、名無しにかわりましてVIPがお送りします :2009/09/05(土) 22:31:01.22 ID:5kaffl9OP 


青年商人「それは、つまり……」\par{} 
辣腕会計「……」 



青年商人「商品でも、栽培方法でもなく\par{} 
 『同盟』に、新しい『概念』を売る、と?」 



魔王「そうだ」 



青年商人「判ります。私には。\par{} 
 ……いまの話を聞きましたから、その価値が判ります。\par{} 
 貴女の言葉は……。 \par{}
 この中央大陸の都市の全てより……。\par{} 
 いや、既知世界の全てよりも金になる」 



魔王「あははは。良い顔だな」\par{} 
青年商人「はい?」 



魔王「女におべんちゃらを言っておる時より数段良い」 



青年商人「そうですか? まぁ、しかし。\par{} 
 いまの話を聞いては真面目にならざるを得ませんよ。\par{} 
 しかし良いのですか?」 



魔王「なにがだ?」 

	
    
    

741 :以下、名無しにかわりましてVIPがお送りします :2009/09/05(土) 22:35:38.45 ID:5kaffl9OP 


青年商人「いまの言葉、そして送っていただいた羅針盤。\par{} 
 すべて『考え方』を基本にしたものです。\par{} 
 技術でも品物でもない。 \par{}
 つまり、複製できない物ではない」 



魔王「そうだな」 



青年商人「私たちがそれらを複製して、貴方とは無関係に\par{} 
 計画を進めるとは考えないのですか? 貴方の利益は?\par{} 
 貴方の権利をどうやって守るつもりなのですか?」 



魔王「それについては諦めておる」 



青年商人「はい?」 



魔王「技術も品物も素晴らしい。利益も結構。\par{} 
 私もお金はあれば欲しい。\par{} 
 研究したいことがたくさんあるからな。\par{} 
 しかし、単一技術や独占可能な品物では、\par{} 
 この世界に与える影響は限定されざるを得ない。\par{} 
 必要なのは転換と突破だ」 



辣腕会計「それは神学的な話でしょうか?\par{} 
 複雑すぎて、判らないのですが」 



青年商人「……」 

	
    
    

742 :以下、名無しにかわりましてVIPがお送りします :2009/09/05(土) 22:39:04.44 ID:5kaffl9OP 


魔王「そちらの商人の方は判っているようだ」\par{} 
青年商人「……」 



魔王「どうした?」 
青年商人「だとすれば……貴女は……」 



魔王「……」 



青年商人「選ぶ必要が、あると?」 



魔王「選びに来たのだろう? \par{}
 交渉という言葉の意味はそれだと心得ている」 



青年商人「しかし、それは。貴女は何を望んでいるんですか?」 



魔王「戦争の早期終結」\par{} 
青年商人「……」 



魔王「しかも、その形は勝利でも敗北でもない\par{} 
 形態でなければならない」 



青年商人「……それは」 



魔王「『同盟』が魔族との大戦における、中央大陸最大の\par{} 
 スポンサーだと云うことは心得ている」 

	
    
    

744 :以下、名無しにかわりましてVIPがお送りします :2009/09/05(土) 22:43:56.82 ID:5kaffl9OP 


青年商人「魔族は人類の敵です。魔族との戦いに\par{} 
 人類陣営の一翼である我らが全てをなげうって\par{} 
 協力するのは至極当たり前のことではありませんか」 



魔王「それは公的な見解であろう」\par{} 
青年商人「正式見解です」 



魔王「高きと低きを、北と南を、炎と氷を、 \par{}
 相容れない光と影を仲介し、妥協し、取引することで\par{} 
 利益を上げるのがお主ら商人ではないか?」 



青年商人「あ、貴女は……」 



魔王「なんだろう?」 



青年商人「『同盟』の味方ですか、敵ですか?」 



魔王「取引相手だ」 



青年商人「……っ」 



メイド長(まおー様~っ! がんばって!) 

	
    
    

749 :以下、名無しにかわりましてVIPがお送りします :2009/09/05(土) 22:51:37.42 ID:5kaffl9OP 


青年商人「私は二つの道のあいだで悩んでいます」ぎりっ\par{} 
魔王「なにを?」 



青年商人「人間として、貴女の先ほどの発言は\par{} 
 裏切り行為です。教会の方針においても異端だ。\par{} 
 私は貴女をこの場で断罪し、告発すべきかもしれない」 



メイド長(まおー様、まおー様っ。森の中に\par{} 
 黒装束に黒塗りの剣を持った傭兵団がっ)\par{} 
魔王(控えておれっ) 



魔王「敵と味方の2分割では、\par{} 
 この世界はあまりにも惨めに過ぎよう」 



辣腕会計「……部隊の配置が」\par{} 
青年商人「良い。……試されてるんだね、僕らは」 



魔王「……」 



青年商人「この先もあると?」 



魔王「もちろんだ。大陸中央部の乾燥地帯において\par{} 
 水車の代わりに利用できる『風車』というものも\par{} 
 開発してある。森林資源を消費してしまうが\par{} 
 羊皮紙に変わる新しい筆記資材もめどは立った」 



青年商人「貴女は何を見ているんですか?」 

	
    
    

753 :以下、名無しにかわりましてVIPがお送りします :2009/09/05(土) 22:57:23.24 ID:5kaffl9OP 


魔王「私は学者だが、専門は経済学でな」 



青年商人「経済……?」 



魔王「耳慣れぬ言葉だろうな。\par{} 
 物と金の流れ、利益と損害、\par{} 
 魂持つものが生み出す社会において\par{} 
 たゆまず流れる交流の歴史と未来がその専門だ」 



青年商人「利益と損害、ですか」 



魔王「そうだ、商人殿。\par{} 
 商人殿とおなじものを見ているだけだよ」 



青年商人「それをもって、人類全てを裏切れと?\par{} 
 この戦争を終結させようとする\par{} 
 貴女の見る夢がどのような色をしているか\par{} 
 判らないわたしではないっ」 



魔王「信じている」 



青年商人「わたしの。……『同盟』の。\par{} 
 我ら人間の、何を信じると言うのです?」 



魔王「損得勘定は我らの共通の言葉であることを。\par{} 
 それはこの天と地の間で二番目に強い絆だ」 

	
    
    

756 :以下、名無しにかわりましてVIPがお送りします :2009/09/05(土) 23:00:56.56 ID:5kaffl9OP 


青年商人「あはははははは!」\par{} 
辣腕会計「……商人っ」 



青年商人「いや、いいんだ。\par{} 
 そうだ。まさにその通りだ!\par{} 
 人間である前に商人たれ。\par{} 
 教会の敬虔な信徒である前に商人たれ。\par{} 
 まさに『同盟』の訓辞通りじゃないかっ」 



辣腕会計「それは……」 



魔王「わたしは純粋な契約主義者なのだ」 



青年商人「奇遇ですね、わたしもなんですよ。\par{} 
 作りましょう。我らが未来を照らす光となる\par{} 
 契約書を」 



辣腕会計「……それでは」\par{} 
青年商人「ああ、退かせてかまわない」 



魔王「冷や汗が吹き出る思いであったよ。商人殿」 



青年商人「いやはや。本場の東方商人と渡り合っても、\par{} 
 これほどの緊張感を味わった事はありません。\par{} 
 貴女が学士であり、商人でなくて本当に良かった」 

	
    
    

758 :以下、名無しにかわりましてVIPがお送りします :2009/09/05(土) 23:07:17.43 ID:5kaffl9OP 


魔王「私は無力で腰抜けの存在だよ」 



青年商人「いえいえ、王侯貴族だってあそこまでの\par{} 
 迫力はなかなかにある物じゃない」 



メイド長(あったり前ですよ。まおー様は\par{} 
 これでも王族なんですからねっ!) 



青年商人「それにしても……二番目に強い絆、ね」\par{} 
魔王「……」 



青年商人「玉蜀黍の件はいつうごけます?」\par{} 
魔王「すまないが、いくつかの実験と、苗の\par{} 
 栽培を残している。動けて次の春から、だろう」 



青年商人「充分に早いでしょう。私もこの計画を\par{} 
 聞いたからには『同盟』内部での地盤を\par{} 
 固めなくてはならない。\par{} 
 この巨大利益です、動かすことはたやすいが \par{}
 コントロールが聞いてこその権力ですからね」 



魔王「あの羅針盤が役に立ってくれれば良いな」\par{} 
青年商人「せいぜい、利用させていただきましょう」 

	
    
    

760 :以下、名無しにかわりましてVIPがお送りします :2009/09/05(土) 23:12:13.90 ID:5kaffl9OP 


――冬越しの村、村はずれの館、玄関 



メイド長「日も傾いております、お気をつけを」\par{} 
魔王「近くに隊商をまたせてあるのだろう?」 



青年商人「“隊商”ね。ははは」\par{} 
魔王「それがお互いのためだとしよう。わたしも\par{} 
 警戒はしていた。お互い様だ」 



青年商人「まったく、今日は驚きの連続だ」\par{} 
魔王「心臓に悪い」 



青年商人「そうそう。……二番目に強い、と\par{} 
 おっしゃいましたね。一番はなんなのです?」 



魔王「知れておる。愛情だ」 



辣腕会計「――それは」\par{} 
青年商人「あははははは。ああ! すごい! \par{}
 素晴らしいな。一日に二回も、こんな気持ちに\par{} 
 させられるとはっ!」 



魔王「子供でも知っておることだ」 



青年商人「たしかに! 私はあなたに言いました。\par{} 
 二つの道で迷っていると。 
 あなたを殺すことはすっかり諦めましたからね。\par{} 
 これはもう……求婚するしかありません」 

	
    
    

765 :以下、名無しにかわりましてVIPがお送りします :2009/09/05(土) 23:16:11.13 ID:5kaffl9OP 


魔王「そ、そ、そ、それはなんだっ!?」 



青年商人「なんだって。結婚の申し込みですよ」 



魔王「なんて軽率なことを言うんだ。恥を知れ!」 



青年商人「おやおや。貴女があまりにも明晰な\par{} 
 思考をなさるんで、世間並みのたしなみを\par{} 
 忘れてしまっていました。\par{} 
 たしかに。持参金も贈り物も無しに求婚するなんて\par{} 
 先走りすぎましたね」 



魔王「わ、わ、わたしには、その」 



青年商人「いえいえ。\par{} 
 このようなことは腰をすえて取り組むタイプですからね。\par{} 
 粘り強さは決断力とともに商人の重要な武器なのです」 



魔王「いやっ。いくら時間をかけられてもそんな事はっ」 



青年商人「では、またお会いしましょう!\par{}
 次は都か、船の上か。契約は急ぎお届けします。 \par{}
 愛しの君よ。……そう呼ぶのはかまいませんかね?」 



魔王「だ、ダメダメだーっ!!」 

	
    
    

810 :以下、名無しにかわりましてVIPがお送りします :2009/09/06(日) 02:03:26.96 ID:Lbanm5QNP 


――冬越しの村、村はずれの館、小さな部屋 



メイド妹「~♪ ~♪」\par{} 
メイド姉「ご機嫌だね」 



メイド妹「うんっ。みがくの楽しいねー」\par{} 
メイド姉「そうね。こんなにあったかくて、\par{} 
 きちんとした仕事があって。幸せね」 

 

きゅっきゅっ 



メイド妹「そうだよねー。去年の秋は、毎日、\par{} 
 夜が来るのがこわかったもんねっ」 



メイド姉「うん」 



メイド妹「あたしねー。今度は、せーれー様の本で\par{} 
 勉強するんだよー」 



メイド姉「そうなの? がんばってるね」 



メイド妹「おねーちゃんもやった?」\par{} 
メイド姉「やったわよ、結構難しい単語があるわよ?」 

	
    
    

817 :以下、名無しにかわりましてVIPがお送りします :2009/09/06(日) 02:10:38.23 ID:Lbanm5QNP 


メイド妹「大丈夫だよぉ。\par{} 
 ちゃんとした言葉を覚えるとモテモ? えっと……」 \par{}
メイド姉「『殿方に好意を持っていただける』でしょ?」 



メイド妹「うん、そうそう。それ!\par{} 
 眼鏡のおねーさんがいってた」 



メイド姉「メイド長様は、面倒見が良いから」/

 



メイド妹「怖いよ? すぐ怒るよ」 



メイド姉「怒ってないよ。叱っているだけ。\par{} 
 本当はとっても優しい人だよ?」

 



メイド妹「そうかなぁ? お尻叩かれたとき、\par{} 
 ひりひりして椅子に座れなくなったもん」 



メイド姉「拾い食いなんかするからです」 



メイド妹「昔はおねーちゃんもやってたくせに」\par{} 
メイド姉「ご飯ちゃんと食べさせてもらってるでしょ?」 



メイド妹「うんっ」\par{}
メイド姉「じゃ、恥ずかしいことは、しないの」 

	
    
    

820 :以下、名無しにかわりましてVIPがお送りします :2009/09/06(日) 02:14:40.19 ID:Lbanm5QNP 


メイド妹「おねーちゃんは、年越し祭はどうするの?」\par{} 
メイド姉「どうするって?」 



メイド妹「村の真ん中で、踊るらしいよ?」\par{} 
メイド姉「だれが?」 



メイド妹「村の男の子と、女の子、たくさん」\par{} 
メイド姉「私は良いわ」 



メイド妹「そーなの?」\par{} 
メイド姉「メイドの仕事があるもの」 



メイド妹「でも、踊って来ていいって、\par{} 
 眼鏡のおねーさんがいってたよ?」\par{} 
メイド姉「そう……」 



メイド妹「当主のお姉ちゃんも、元気ないね。\par{} 
 勇者のおにいちゃん、帰ってくればいいのにね」\par{} 
メイド姉「そうね。――そうだ」\par{} 
メイド妹「?」 



メイド姉「年越しの祭には、何かプレゼントを用意\par{} 
 しましょうね。館のみんなに」\par{} 
メイド妹「そうだねっ!」 

	
    
    

822 :以下、名無しにかわりましてVIPがお送りします :2009/09/06(日) 02:18:24.98 ID:Lbanm5QNP 


――冬越しの村、村はずれの館、当主の部屋 



魔王「えー『試験場の数を増やしたく思う。\par{} 
 追加の人員の手配をお願いしたい。\par{} 
 対価は西方貨幣で支払う用意あり』と」 



メイド長「……」さらさら 



魔王「これは蜜蝋で封をしてくれ」\par{} 
メイド長「かしこまりました」 



魔王「んー。これは?」 



メイド長「狩人さんからの手紙ですよ」 



魔王「おー。そうか、そうか。望遠鏡を渡したんだった」\par{} 
メイド長「ええ」 



魔王「なになに。使用するに便利、極めて快適か」\par{} 
メイド長「役立っているようですね」 



魔王「精度が低いかと思ったが、固定観測でないなら\par{} 
 かえって手ごろのようだな」\par{} 
メイド長「はい」 



魔王「よし、では返信だ。『素早い報告、うれしく思う。\par{} 
 森番の仕事、大変かと思うが、当家では付近の地図測量に\par{} 
 興味あり。相談したきことがあるので、一度ご来訪願う』\par{} 
 これで、よしっと」 

	
    
    

823 :以下、名無しにかわりましてVIPがお送りします :2009/09/06(日) 02:23:28.21 ID:Lbanm5QNP 


メイド長「こちらもお願いします」\par{} 
魔王「これは、うん。修道会からの報告か」 



メイド長「あらあら、まぁまぁ」\par{} 
魔王「馬鈴薯の収穫は順調に増加しているらしいな」\par{} 
メイド長「そのようですね」 



魔王「だが、土壌実験によれば\par{} 
 そろそろ栄養枯渇の兆候が出るはずだ。\par{} 
 そうなると抵抗力が低下して虫害が出やすくなる」\par{} 
メイド長「ふむ……」 



魔王「この件では修道会へ、再度警告が必要だな」\par{} 
メイド長「お手紙にしますか?」 



魔王「いや、次行ったときでよかろう。\par{} 
 覚え書きに追加しておいてくれ」 



メイド長「かしこまりました」さらさら 



魔王「どうだ『紙』は」\par{} 
メイド長「羊皮紙よりずっと書きやすいですね」 



魔王「早いところ量産体制を整えないとな」\par{} 
メイド長「作るのは簡単ですけれど、\par{} 
 たくさん作るとなるとまた別問題ですからね……」 

	
    
    

826 :以下、名無しにかわりましてVIPがお送りします :2009/09/06(日) 02:27:54.45 ID:Lbanm5QNP


魔王「うわ、なんだこの束は!?」 



メイド長「そちらの束は、『同盟』からですよ。\par{} 
 納品書、請求書、支払い所、明細書……」 



魔王「あー。銅、鏡、ガラス、海砂?\par{} 
 それに胡椒に、絹に、釘なんていうものもあるな」 



メイド長「みんなまおー様が購入リストに入れたんですよ」 



魔王「判っておる。\par{} 
 ちょっと思い出せなかっただけだ。\par{} 
 必要としているのは誰か判っているか?」 



メイド長「それはまぁ、帳簿につけてありますが」 



魔王「んー。しっちゃかめっちゃかだな、これは」\par{} 
メイド長「まさかここまで仕事量が増えるとは」 



魔王「しかたない。メイド姉にやらせよう」\par{} 
メイド長「彼女にですか?」 



魔王「無理か?」 

	
    
    

828 :以下、名無しにかわりましてVIPがお送りします :2009/09/06(日) 02:31:42.53 ID:Lbanm5QNP 


メイド長「……いえ」\par{} 
魔王「……」 



メイド長「大丈夫です。彼女なら出来ます」 



魔王「そうか」 にこっ\par{} 
 「では、この書類整理は、今日からあやつの仕事だ」 



メイド長「悪のメイド軍団が結成できそうな勢いですね」 



魔王「どうかしたのか?」 



メイド長「いえいえなんでも。……そうだ、\par{} 
 お茶でも入れましょうか? 丁度、聖王都から\par{} 
 オレンジの香りの葉がとどいたんですよ」 



魔王「うむ、疲れた」\par{} 
メイド長「でしょう」 



魔王「私は疲れたのだぞ」\par{} 
メイド長「そんなにつっぷして。どうしたんですか?」 



魔王「むー」 

	
    
    

830 :以下、名無しにかわりましてVIPがお送りします :2009/09/06(日) 02:34:56.75 ID:Lbanm5QNP 


メイド長「膨れているんですか?」 



魔王「もう秋だぞ」 



メイド長「そうですねぇ、実りの季節です。\par{} 
 栗がおいしいですねぇ。今年のベーコンも出来が良いようで」 



魔王「秋なのに」\par{} 
メイド長「はい?」 



魔王「半年も音沙汰無しだぞ」\par{} 
メイド長「あらあら、まぁまぁ」 



魔王「ちょっと応えにくい会話だとすぐその決め台詞で\par{} 
 流そうとするのは止めにしたらどうだ」 



メイド長「これはメイドの特殊技能なんです」 



魔王「連絡くらいくれても良いではないかっ!!」\par{} 
メイド長「来てるじゃないですか」 



魔王「そんなもの、妖精族を助けただの、\par{} 
 鬼腕族を討伐しただの、そんなことばかりではないか」 

	
    
    

833 :以下、名無しにかわりましてVIPがお送りします :2009/09/06(日) 02:39:54.63 ID:Lbanm5QNP 


メイド長「無事で、活躍されているんですよ」 



魔王「勇者なのだぞ。こうしている間にもあっさり\par{} 
 美人が自慢の村娘とか……\par{} 
 いや、歌姫族のハーピーあたりと\par{} 
 いちゃいちゃしているかもしれんではないかっ!?」 



メイド長「そうですか? 勇者様は童貞ですからね。\par{} 
 童貞って言うのは変なところで義理堅くて夢見がち\par{} 
 ですから、きっと大丈夫ですよ」 



魔王「ちっとも安心できんではないかっ」 



メイド長「そんなにいらいらしていると、\par{} 
 眉間のしわが取れなくなってしまいますよ?」 



魔王「ううう、そんなことになったら勇者に噛みついてやる」 



メイド長「さぁさ。談話室の暖炉が暖められています。\par{} 
 今日はこのあたりにして、甘い紅茶をおいれしますから。 
 そちらの方でお待ちになっていてください」 \par{}



魔王「しかしな」 



メイド長「これ以上書類と根をつめていては\par{} 
 それこそお身体を悪くしてしまいますよ?」

	
    
    

837 :以下、名無しにかわりましてVIPがお送りします :2009/09/06(日) 02:45:30.54 ID:Lbanm5QNP 


魔王「むぅ。判った。お茶を頼む」 



メイド長「かしこまりました。まおー様♪」 

 

がちゃん。\par{} 
 とっとっとっ…… 



メイド長「なーんて。……魔王様はおっしゃってますが?」\par{} 
勇者「うわ、ばればれですね」 



メイド長「メイドの勘です」\par{} 
勇者「毎回ばれてるなぁ」 



メイド長「今回のお手紙は?」\par{} 
勇者「ここで書いていきます」 



メイド長「では、こちらにもお茶をお持ちしましょう」\par{} 
勇者「すんません」 



メイド長「いえいえ。メイドの仕事ですから」 



勇者「さってと、インク壷と~羊皮紙あっかな\par{}
 これでいーか」 

	

	

840 :以下、名無しにかわりましてVIPがお送りします :2009/09/06(日) 02:50:51.46 ID:Lbanm5QNP 


――冬越しの村、村はずれの館、当主の部屋 



ガチャリ 



メイド長「おじゃまします。いかがですか?」 



勇者「あ、報告は書き終えました」 



メイド長「そうですか。……こちらはお茶と\par{} 
 簡単な夜食になります。今回は馬鈴薯が\par{} 
 ことのほかよく出来ておりますよ。\par{} 
 これはクリームで甘く煮たものなのですが」 



勇者「旨そうっすねー」 



メイド長「……」 



勇者「わ、熱っ。んまっ! 今回はぁ火竜族と\par{} 
 なんとか手打ちで、でもそのためには『開門都市』\par{} 
 をなんとか奪還しなきゃならなくてですね」 



メイド長「……勇者様」 



勇者「ん?」 



メイド長「やはり今回も?」 

	
    
    

843 :以下、名無しにかわりましてVIPがお送りします :2009/09/06(日) 02:53:59.28 ID:Lbanm5QNP 


勇者「あー。うん」\par{} 
メイド長「魔王様を避けてますよね?」 



勇者「うー」 



メイド長「避けてらっしゃいますよね?」\par{} 
勇者「うー、うん」 



メイド長「……使用人の分際で差し出がましいかと思い\par{} 
 今まで訊ねずに参りましたが、埒が明きません。\par{} 
 魔王様には内緒にしておきますから\par{} 
 何か問題があるのなら相談くださいませ」 



勇者「うん……」 



メイド長「煮えきらない態度ですね。\par{} 
 あれですか。酒場の鳥娘に言い寄られたり\par{} 
 半透明のスライム娘に告白されたり\par{} 
 爆乳自慢の牛娘に婿宣言されたりしたんですか?」 



勇者「うがっ!」 



メイド長「どうなんですか?」 



勇者「そのう、そういうのがないとは言いませんが」 

	
    
    

850 :以下、名無しにかわりましてV\par{} IPがお送りします :2009/09/06(日) 03:01:16.21 ID:Lbanm5QNP 


メイド長「大体転移呪文があるのなら 
 毎日は無理でも、毎週程度には帰ってこられますよね?」 



勇者「うん」 



メイド長「魔王様がそれに気が付かないくらい\par{}  
 お間抜けで今回は助かっていますが……」 



勇者「うん……」 \par{} 
メイド長「どういうことなのですか?」 



勇者「いや、その。さ」\par{}  
メイド長「はい?」 



勇者「……魔王が、あんまりにも俺に頼らないから」\par{}  
メイド長「……」 



勇者「最初にさ、あの魔王の間で『我のものになれ』\par{}  
 なんていわれてさ『まだ見ぬもの』なんていわれたからさ」\par{}  
メイド長「……」 



勇者「てっきり、勇者の力で、魔族の反乱分子を\par{}  
 粛清してさ、たとえばゲートを閉じちゃったりして\par{}  
 そうやって戦争を終わらせると思ってたんだよ」 

	
    
    

853 :以下、名無しにかわりましてVIPがお送りします :2009/09/06(日) 03:05:27.68 ID:Lbanm5QNP 


勇者「そういう意味で、勇者の力が欲しいのだと」\par{} 
メイド長「……」 



勇者「なのに、あいつ、俺の戦闘能力は当てにしないでさ、\par{} 
 それどころか、戦わないように戦わないようにしてさ」\par{} 
メイド長「はい」 



勇者「なんかまるで俺のことが大事みたいに\par{} 
 ……好きみたいにさ。するから」 



メイド長「……」 



勇者「だって所有契約だろう?\par{} 
 俺はあいつのものだしあいつは俺のものだ。\par{} 
 気に入らなかったら命をとられてもいいんだ。\par{} 
 そういう契約じゃないか」 



メイド長「そうですね」 



勇者「それなのにさー。あいつさ。挙動不審だし\par{} 
 言い訳も説明も過剰だし、おっかなびっくりだしさ」 



メイド長「……」 



勇者「……上手く言葉にならねぇや」 

	
    
    

854 :以下、名無しにかわりましてVIPがお送りします :2009/09/06(日) 03:08:45.40 ID:Lbanm5QNP 


メイド長「魔王様は、勇者様のことを――」 



勇者「判ってるんだ。\par{} 
 そこまで馬鹿じゃない。\par{} 
 相手の好意が信じられないから、\par{} 
 自分の好意を与えられないだなんて\par{} 
 そんな腰抜けの言い訳じみたことを言うつもりはないんだ」 



メイド長「では、なぜ?」 



勇者「だって、俺、死んじゃうしさ」 



メイド長「……」 



勇者「今回のことがどう転ぼうがどう成功しようが\par{} 
 それでも俺は人間だから、魔王よりも先に死んじゃうしさ」 



メイド長「それはっ」 



勇者「そんな俺が魔王と想いを重ねるって\par{} 
 それはなんだかすげぇ不実な気もするんだよ」 



メイド長「そんなことはありません」 



勇者「そうかなぁ」 

	
    
    

856 :以下、名無しにかわりましてVIPがお送りします :2009/09/06(日) 03:12:06.98 ID:Lbanm5QNP 


勇者「そりゃ、まぁ。本当かもしれないよ?\par{} 
 終わりがない関係はないけれど\par{} 
 終わるために出会うわけじゃないからさ」 



メイド長「……」 



勇者「でも、なんだかなぁ」 



勇者「俺、最後のときに、魔王の困ったような\par{} 
 泣きそうな顔ばっかり思い出す気がするんだよ」 



メイド長「そんな」 



勇者「これもびびってるっていうのかなぁ。\par{} 
 でも、魔王がそういう顔すると思うとつらい。\par{} 
 勇者って言うのはさ、\par{} 
 もしかしたら幸せになっちゃいけない職業なのかも\par{} 
 しれないって。そう思うんだよ」 



メイド長「……」 



勇者「今の俺は、あんまり勇者って感じじゃないなっ!」 

	
    
    

857 :以下、名無しにかわりましてVIPがお送りします :2009/09/06(日) 03:15:40.56 ID:Lbanm5QNP 


メイド長「メイド如きに口を挟める問題でも\par{} 
 ないのでしょうが、一つだけ」 



勇者「うん」 



メイド長「勇者様は、魔王様のもの。\par{} 
 勇者様のすべては魔王様の、我が主の所有物」 



勇者「ああ、そうだ」 



メイド長「そのことをお忘れなきよう」 



勇者「うん」 



メイド長「だとすれば、\par{} 
 勇者様の感じるためらいも思いやりも、\par{} 
 押し殺している願いや\par{} 
 憧れるような希望も、\par{} 
 触れたいという祈りも。\par{} 
 言葉にならない、魔王様への気持ちさえ。\par{} 
 それらもすべて魔王様のもの」 



勇者「……」 



メイド長「それをお忘れなきように」 

	






	
		INDEX
      /
    	→NEXT
	





 
 
 
 


\end{document}
