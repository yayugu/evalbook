\documentclass[a4j,twocolumn]{tarticle}
\usepackage[dvipdfmx]{graphicx}
\usepackage{wallpaper}
\usepackage{multicol}
\usepackage{otf}

\makeatletter
\renewcommand{\normalsize}{\@setfontsize\normalsize{12pt}{21.0}}
\renewcommand{\tiny}{\@setfontsize\tiny{6.0pt}{10.5}}
\renewcommand{\huge}{\@setfontsize\huge{24.0pt}{42.0}}
\makeatother

\normalsize
\voffset=-1in
\hoffset=-1in
\paperwidth=424.325182059202pt
\paperheight=554.716774462811pt
\textwidth=519.59664pt
\textheight=383.924184pt
\topmargin=17.5600672314055pt
\oddsidemargin=11.200499029601pt
\columnsep=24pt
\headheight=0mm
\headsep=0mm
\topskip=0mm
\footskip=1000mm   % don't use footer
%\kanjiskip=0zw plus .0625zw minus .01zw
\kanjiskip=0zw plus .1zw minus .01zw
\xkanjiskip=0.25em plus 0.15em minus 0.06em
\setlength\parindent{1zw}

\normalsize
\usepackage{furikana}
\usepackage{burasage}
\AtBeginDvi{\special{pdf:docview <</ViewerPreferences <</Direction /R2L>> >>}}

\begin{document}

{\huge こころ} \\
\hfill 夏目漱石\\

\kana[1]{私}{わたくし}はその人を常に先生と呼んでいた\。だからここでもただ先生と書くだけで本名は打ち明けない\。これは世間を\kana[1]{憚}{はば}かる遠慮というよりも\、その方が私にとって自然だからである\。私はその人の記憶を呼び起すごとに\、すぐ「先生」といいたくなる\。筆を\kana[1]{執}{と}っても心持は同じ事である\。よそよそしい\kana[3]{頭}{かしら}\kana[1]{文}{も}\kana[1]{字}{じ}などはとても使う気にならない\。

私が先生と知り合いになったのは\kana[1]{鎌}{かま}\kana[1]{倉}{くら}である\。その時私はまだ若々しい書生であった\。暑中休暇を利用して海水浴に行った友達からぜひ来いという\kana[1]{端}{は}\kana[1]{書}{がき}を受け取ったので\、私は多少の金を\kana[1]{工}{く}\kana[1]{面}{めん}して\、出掛ける事にした\。私は金の工面に\kana[1]{二}{に}\、三\kana[1]{日}{さんち}を費やした\。ところが私が鎌倉に着いて三日と\kana[1]{経}{た}たないうちに\、私を呼び寄せた友達は\、急に国元から帰れという電報を受け取った\。電報には母が病気だからと断ってあったけれども友達はそれを信じなかった\。友達はかねてから国元にいる親たちに\kana[1]{勧}{すす}まない結婚を\kana[1]{強}{し}いられていた\。彼は現代の習慣からいうと結婚するにはあまり年が若過ぎた\。それに\kana[1]{肝}{かん}\kana[1]{心}{じん}の当人が気に入らなかった\。それで夏休みに当然帰るべきところを\、わざと避けて東京の近くで遊んでいたのである\。彼は電報を私に見せてどうしようと相談をした\。私にはどうしていいか分らなかった\。けれども実際彼の母が病気であるとすれば彼は\kana[1]{固}{もと}より帰るべきはずであった\。それで彼はとうとう帰る事になった\。せっかく来た私は一人取り残された\。

学校の授業が始まるにはまだ\kana[1]{大}{だい}\kana[1]{分}{ぶ}\kana[1]{日}{ひ}\kana[1]{数}{かず}があるので鎌倉におってもよし\、帰ってもよいという境遇にいた私は\、当分元の宿に\kana[1]{留}{と}まる覚悟をした\。友達は中国のある資産家の\kana[1]{息}{むす}\kana[1]{子}{こ}で金に不自由のない男であったけれども\、学校が学校なのと年が年なので\、生活の程度は私とそう変りもしなかった\。したがって\kana[1]{一人}{ひとり}ぼっちになった私は別に\kana[1]{恰}{かつ}\kana[1]{好}{こう}な宿を探す面倒ももたなかったのである\。


宿は鎌倉でも\kana[1]{辺}{へん}\kana[1]{鄙}{ぴ}な方角にあった\。\kana[1]{玉}{たま}\kana[1]{突}{つ}きだのアイスクリームだのというハイカラなものには長い\kana[1]{畷}{なわて}を一つ越さなければ手が届かなかった\。車で行っても二十銭は取られた\。けれども個人の別荘はそこここにいくつでも建てられていた\。それに海へはごく近いので海水浴をやるには至極便利な地位を占めていた\。

私は毎日海へはいりに出掛けた\。古い\kana[1]{燻}{くす}ぶり返った\kana[1]{藁}{わら}\kana[1]{葺}{ぶき}の\kana[1]{間}{あいだ}を通り抜けて\kana[1]{磯}{いそ}へ下りると\、この\kana[1]{辺}{へん}にこれほどの都会人種が住んでいるかと思うほど\、避暑に来た男や女で砂の上が動いていた\。ある時は海の中が\kana[1]{銭}{せん}\kana[1]{湯}{とう}のように黒い頭でごちゃごちゃしている事もあった\。その中に知った人を一人ももたない私も\、こういう\kana[1]{賑}{にぎ}やかな景色の中に\kana[1]{裹}{つつ}まれて\、砂の上に\kana[1]{寝}{ね}そべってみたり\、\kana[1]{膝}{ひざ}\kana[3]{頭}{がしら}を波に打たしてそこいらを\kana[1]{跳}{は}ね\kana[1]{廻}{まわ}るのは愉快であった\。

私は実に先生をこの\kana[1]{雑}{ざつ}\kana[1]{沓}{とう}の\kana[1]{間}{あいだ}に見付け出したのである\。その時海岸には\kana[1]{掛}{かけ}\kana[1]{茶}{ぢや}\kana[1]{屋}{や}が二軒あった\。私はふとした\kana[1]{機会}{はずみ}からその一軒の方に行き\kana[1]{慣}{な}れていた\。\kana[1]{長谷}{はせ}\kana[1]{辺}{へん}に大きな別荘を構えている人と違って\、\kana[1]{各自}{めいめい}に専有の\kana[1]{着}{き}\kana[1]{換}{がえ}\kana[1]{場}{ば}を\kana[1]{拵}{こしら}えていないここいらの避暑客には\、ぜひともこうした共同着換所といった\kana[1]{風}{ふう}なものが必要なのであった\。彼らはここで茶を飲み\、ここで休息する\kana[1]{外}{ほか}に\、ここで海水着を洗濯させたり\、ここで\kana[1]{鹹}{しお}はゆい\kana[1]{身体}{からだ}を清めたり\、ここへ帽子や\kana[1]{傘}{かさ}を預けたりするのである\。海水着を持たない私にも持物を盗まれる恐れはあったので\、私は海へはいるたびにその茶屋へ\kana[1]{一}{いつ}\kana[1]{切}{さい}を\kana[1]{脱}{ぬ}ぎ\kana[1]{棄}{す}てる事にしていた\。

二

\kana[1]{私}{わたくし}がその掛茶屋で先生を見た時は\、先生がちょうど着物を脱いでこれから海へ入ろうとするところであった\。私はその時反対に\kana[1]{濡}{ぬ}れた\kana[1]{身体}{からだ}を風に吹かして水から上がって来た\。二人の\kana[1]{間}{あいだ}には目を\kana[1]{遮}{さえぎ}る幾多の黒い頭が動いていた\。特別の事情のない限り\、私はついに先生を見逃したかも知れなかった\。それほど浜辺が混雑し\、それほど私の頭が\kana[1]{放}{ほう}\kana[1]{漫}{まん}であったにもかかわらず\、私がすぐ先生を見付け出したのは\、先生が一人の西洋人を\kana[1]{伴}{つ}れていたからである\。

その西洋人の優れて白い皮膚の色が\、掛茶屋へ入るや\kana[1]{否}{いな}や\、すぐ私の注意を\kana[1]{惹}{ひ}いた\。純粋の日本の\kana[1]{浴衣}{ゆかた}を着ていた彼は\、それを\kana[1]{床}{しよう}\kana[1]{几}{ぎ}の上にすぽりと\kana[1]{放}{ほう}り出したまま\、腕組みをして海の方を向いて立っていた\。彼は我々の\kana[1]{穿}{は}く\kana[1]{猿}{さる}\kana[1]{股}{また}一つの\kana[1]{外}{ほか}何物も肌に着けていなかった\。私にはそれが第一不思議だった\。私はその二日前に\kana[1]{由}{ゆ}\kana[1]{井}{い}が\kana[1]{浜}{はま}まで行って\、砂の上にしゃがみながら\、長い間西洋人の海へ入る様子を\kana[1]{眺}{なが}めていた\。私の\kana[1]{尻}{しり}をおろした所は少し小高い丘の上で\、そのすぐ\kana[1]{傍}{わき}がホテルの裏口になっていたので\、私の\kana[1]{凝}{じっ}としている\kana[1]{間}{あいだ}に\、大\kana[1]{分}{だいぶ}多くの男が塩を浴びに出て来たが\、いずれも胴と腕と\kana[1]{股}{もも}は出していなかった\。女は\kana[1]{殊}{こと}\kana[1]{更}{さら}肉を隠しがちであった\。大抵は頭に\kana[1]{護}{ゴ}\kana[1]{謨}{ム}\kana[1]{製}{せい}の\kana[1]{頭}{ず}\kana[1]{巾}{きん}を\kana[1]{被}{かぶ}って\、\kana[1]{海老}{えび}\kana[1]{茶}{ちや}や\kana[1]{紺}{こん}や\kana[1]{藍}{あい}の色を波間に浮かしていた\。そういう有様を目撃したばかりの私の\kana[1]{眼}{め}には\、猿股一つで済まして\kana[1]{皆}{みん}なの前に立っているこの西洋人がいかにも珍しく見えた\。

彼はやがて自分の\kana[1]{傍}{わき}を顧みて\、そこにこごんでいる日本人に\、\kana[1]{一}{ひと}\kana[1]{言}{こと}\kana[1]{二}{ふた}\kana[1]{言}{こと}\kana[1]{何}{なに}かいった\。その日本人は砂の上に落ちた\kana[1]{手}{て}\kana[1]{拭}{ぬぐい}を拾い上げているところであったが\、それを取り上げるや否や\、すぐ頭を包んで\、海の方へ歩き出した\。その人がすなわち先生であった\。

私は単に好奇心のために\、並んで浜辺を下りて行く二人の\kana[3]{後}{うしろ}\kana[3]{姿}{すがた}を見守っていた\。すると彼らは\kana[1]{真}{まつ}\kana[1]{直}{すぐ}に波の中に足を踏み込んだ\。そうして\kana[1]{遠}{とお}\kana[1]{浅}{あさ}の\kana[1]{磯}{いそ}\kana[1]{近}{ちか}くにわいわい騒いでいる\kana[1]{多}{た}\kana[1]{人}{にん}\kana[1]{数}{ず}の\kana[1]{間}{あいだ}を通り抜けて\、比較的広々した所へ来ると\、二人とも泳ぎ出した\。彼らの頭が小さく見えるまで沖の方へ向いて行った\。それから引き返してまた一直線に浜辺まで戻って来た\。掛茶屋へ帰ると\、井戸の水も浴びずに\、すぐ\kana[1]{身体}{からだ}を\kana[1]{拭}{ふ}いて着物を着て\、さっさとどこへか行ってしまった\。

彼らの出て行った\kana[1]{後}{あと}\、私はやはり元の\kana[1]{床}{しよう}\kana[1]{几}{ぎ}に腰をおろして\kana[1]{烟草}{タバコ}を吹かしていた\。その時私はぽかんとしながら先生の事を考えた\。どうもどこかで見た事のある顔のように思われてならなかった\。しかしどうしてもいつどこで会った人か\kana[1]{想}{おも}い出せずにしまった\。

その時の私は\kana[1]{屈}{くつ}\kana[1]{托}{たく}がないというよりむしろ\kana[1]{無}{ぶ}\kana[3]{聊}{りよう}に苦しんでいた\。それで\kana[1]{翌}{あくる}\kana[1]{日}{ひ}もまた先生に会った時刻を見計らって\、わざわざ\kana[1]{掛}{かけ}\kana[1]{茶}{ぢや}\kana[1]{屋}{や}まで出かけてみた\。すると西洋人は来ないで先生一人\kana[1]{麦}{むぎ}\kana[1]{藁}{わら}\kana[1]{帽}{ぼう}を\kana[1]{被}{かぶ}ってやって来た\。先生は\kana[1]{眼}{め}\kana[1]{鏡}{がね}をとって台の上に置いて\、すぐ\kana[1]{手}{て}\kana[1]{拭}{ぬぐい}で頭を包んで\、すたすた浜を下りて行った\。先生が昨\kana[1]{日}{きのう}のように騒がしい\kana[1]{浴}{よく}\kana[1]{客}{かく}の中を通り抜けて\、一人で泳ぎ出した時\、私は急にその\kana[1]{後}{あと}が追い掛けたくなった\。私は浅い水を頭の上まで\kana[1]{跳}{はね}かして相当の深さの所まで来て\、そこから先生を\kana[1]{目}{め}\kana[3]{標}{じるし}に\kana[1]{抜}{ぬき}\kana[1]{手}{で}を切った\。すると先生は昨日と違って\、一種の\kana[1]{弧}{こ}\kana[1]{線}{せん}を\kana[1]{描}{えが}いて\、妙な方向から岸の方へ帰り始めた\。それで私の目的はついに達せられなかった\。私が\kana[1]{陸}{おか}へ上がって\kana[1]{雫}{しずく}の垂れる手を振りながら掛茶屋に入ると\、先生はもうちゃんと着物を着て入れ違いに外へ出て行った\。





\end{document}
