\documentclass[a4j,twocolumn]{tarticle}
\usepackage[dvipdfmx]{graphicx}
\usepackage{wallpaper}
\usepackage{multicol}
\usepackage{otf}

\makeatletter
\renewcommand{\normalsize}{\@setfontsize\normalsize{12pt}{21.0}}
\renewcommand{\tiny}{\@setfontsize\tiny{6.0pt}{10.5}}
\renewcommand{\huge}{\@setfontsize\huge{24.0pt}{42.0}}
\makeatother

\normalsize
\voffset=-1in
\hoffset=-1in
\paperwidth=424.325182059202pt
\paperheight=554.716774462811pt
\textwidth=519.59664pt
\textheight=383.924184pt
\topmargin=17.5600672314055pt
\oddsidemargin=11.200499029601pt
\columnsep=24pt
\headheight=0mm
\headsep=0mm
\topskip=0mm
\footskip=1000mm   % don't use footer
%\kanjiskip=0zw plus .0625zw minus .01zw
\kanjiskip=0zw plus .1zw minus .01zw
\xkanjiskip=0.25em plus 0.15em minus 0.06em
\setlength\parindent{1zw}

\normalsize
\usepackage{furikana}
\usepackage{burasage}
\AtBeginDvi{\special{pdf:docview <</ViewerPreferences <</Direction /R2L>> >>}}

\begin{document}

{\huge 芥川龍之介 羅生門} \\

羅生門
芥川龍之介
\par{}
\par{}

\par{}
 ある日の暮方の事である。一人の\kana{下人}{げにん}が、\kana{羅生門}{らしょうもん}の下で雨やみを待っていた。\par{}
 広い門の下には、この男のほかに誰もいない。ただ、所々\kana{丹塗}{にぬり}の\kana{剥}{は}げた、大きな\kana{円柱}{まるばしら}に、\kana{蟋蟀}{きりぎりす}が一匹とまっている。羅生門が、\kana{朱雀大路}{すざくおおじ}にある以上は、この男のほかにも、雨やみをする\kana{市女笠}{いちめがさ}や\kana{揉烏帽子}{もみえぼし}が、もう二三人はありそうなものである。それが、この男のほかには誰もいない。\par{}
 何故かと云うと、この二三年、京都には、地震とか\kana{辻風}{つじかぜ}とか火事とか饑饉とか云う\kana{災}{わざわい}がつづいて起った。そこで\kana{洛中}{らくちゅう}のさびれ方は一通りではない。旧記によると、仏像や仏具を打砕いて、その\kana{丹}{に}がついたり、金銀の\kana{箔}{はく}がついたりした木を、路ばたにつみ重ねて、\kana{薪}{たきぎ}の\kana{料}{しろ}に売っていたと云う事である。洛中がその始末であるから、羅生門の修理などは、元より誰も捨てて顧る者がなかった。するとその荒れ果てたのをよい事にして、\kana{狐狸}{こり}が\kana{棲}{す}む。\kana{盗人}{ぬすびと}が棲む。とうとうしまいには、引取り手のない死人を、この門へ持って来て、棄てて行くと云う習慣さえ出来た。そこで、日の目が見えなくなると、誰でも気味を悪るがって、この門の近所へは足ぶみをしない事になってしまったのである。\par{}
 その代りまた\kana{鴉}{からす}がどこからか、たくさん集って来た。昼間見ると、その鴉が何羽となく輪を描いて、高い\kana{鴟尾}{しび}のまわりを啼きながら、飛びまわっている。ことに門の上の空が、夕焼けであかくなる時には、それが\kana{胡麻}{ごま}をまいたようにはっきり見えた。鴉は、勿論、門の上にある死人の肉を、\kana{啄}{ついば}みに来るのである。――もっとも今日は、\kana{刻限}{こくげん}が遅いせいか、一羽も見えない。ただ、所々、崩れかかった、そうしてその崩れ目に長い草のはえた石段の上に、鴉の\kana{糞}{ふん}が、点々と白くこびりついているのが見える。下人は七段ある石段の一番上の段に、洗いざらした紺の\kana{襖}{あお}の尻を据えて、右の頬に出来た、大きな\kana{面皰}{にきび}を気にしながら、ぼんやり、雨のふるのを眺めていた。\par{}
 作者はさっき、「下人が雨やみを待っていた」と書いた。しかし、下人は雨がやんでも、格別どうしようと云う当てはない。ふだんなら、勿論、主人の家へ帰る可き筈である。所がその主人からは、四五日前に暇を出された。前にも書いたように、当時京都の町は一通りならず\kana{衰微}{すいび}していた。今この下人が、永年、使われていた主人から、暇を出されたのも、実はこの衰微の小さな余波にほかならない。だから「下人が雨やみを待っていた」と云うよりも「雨にふりこめられた下人が、行き所がなくて、途方にくれていた」と云う方が、適当である。その上、今日の空模様も少からず、この平安朝の下人の Sentimentalisme に影響した。\kana{申}{さる}の\kana{刻}{こく}\kana{下}{さが}りからふり出した雨は、いまだに上るけしきがない。そこで、下人は、何をおいても差当り\kana{明日}{あす}の暮しをどうにかしようとして――云わばどうにもならない事を、どうにかしようとして、とりとめもない考えをたどりながら、さっきから朱雀大路にふる雨の音を、聞くともなく聞いていたのである。\par{}
 雨は、羅生門をつつんで、遠くから、ざあっと云う音をあつめて来る。夕闇は次第に空を低くして、見上げると、門の屋根が、斜につき出した\kana{甍}{いらか}の先に、重たくうす暗い雲を支えている。\par{}
 どうにもならない事を、どうにかするためには、手段を選んでいる\kana{遑}{いとま}はない。選んでいれば、\kana{築土}{ついじ}の下か、道ばたの土の上で、\kana{饑死}{うえじに}をするばかりである。そうして、この門の上へ持って来て、犬のように棄てられてしまうばかりである。選ばないとすれば――下人の考えは、何度も同じ道を\kana{低徊}{ていかい}した\kana{揚句}{あげく}に、やっとこの局所へ\kana{逢着}{ほうちゃく}した。しかしこの「すれば」は、いつまでたっても、結局「すれば」であった。下人は、手段を選ばないという事を肯定しながらも、この「すれば」のかたをつけるために、当然、その後に来る可き「\kana{盗人}{ぬすびと}になるよりほかに仕方がない」と云う事を、積極的に肯定するだけの、勇気が出ずにいたのである。\par{}
 下人は、大きな\kana{嚔}{くさめ}をして、それから、\kana{大儀}{たいぎ}そうに立上った。夕冷えのする京都は、もう\kana{火桶}{ひおけ}が欲しいほどの寒さである。風は門の柱と柱との間を、夕闇と共に遠慮なく、吹きぬける。\kana{丹塗}{にぬり}の柱にとまっていた\kana{蟋蟀}{きりぎりす}も、もうどこかへ行ってしまった。\par{}
 下人は、\kana{頸}{くび}をちぢめながら、\kana{山吹}{やまぶき}の\kana{汗袗}{かざみ}に重ねた、紺の\kana{襖}{あお}の肩を高くして門のまわりを見まわした。雨風の\kana{患}{うれえ}のない、人目にかかる\kana{惧}{おそれ}のない、一晩楽にねられそうな所があれば、そこでともかくも、夜を明かそうと思ったからである。すると、幸い門の上の楼へ上る、幅の広い、これも丹を塗った\kana{梯子}{はしご}が眼についた。上なら、人がいたにしても、どうせ死人ばかりである。下人はそこで、腰にさげた\kana{聖柄}{ひじりづか}の\kana{太刀}{たち}が\kana{鞘走}{さやばし}らないように気をつけながら、\kana{藁草履}{わらぞうり}をはいた足を、その梯子の一番下の段へふみかけた。\par{}
 それから、何分かの後である。羅生門の楼の上へ出る、幅の広い梯子の中段に、一人の男が、猫のように身をちぢめて、息を殺しながら、上の\kana{容子}{ようす}を窺っていた。楼の上からさす火の光が、かすかに、その男の右の頬をぬらしている。短い鬚の中に、赤く\kana{膿}{うみ}を持った\kana{面皰}{にきび}のある頬である。下人は、始めから、この上にいる者は、死人ばかりだと高を\kana{括}{くく}っていた。それが、梯子を二三段上って見ると、上では誰か火をとぼして、しかもその火をそこここと動かしているらしい。これは、その濁った、黄いろい光が、隅々に\kana{蜘蛛}{くも}の巣をかけた天井裏に、揺れながら映ったので、すぐにそれと知れたのである。この雨の夜に、この羅生門の上で、火をともしているからは、どうせただの者ではない。\par{}
 下人は、\kana{守宮}{やもり}のように足音をぬすんで、やっと急な梯子を、一番上の段まで這うようにして上りつめた。そうして体を出来るだけ、\kana{平}{たいら}にしながら、頸を出来るだけ、前へ出して、恐る恐る、楼の内を\kana{覗}{のぞ}いて見た。\par{}
 見ると、楼の内には、噂に聞いた通り、幾つかの\kana{死骸}{しがい}が、無造作に棄ててあるが、火の光の及ぶ範囲が、思ったより狭いので、数は幾つともわからない。ただ、おぼろげながら、知れるのは、その中に裸の死骸と、着物を着た死骸とがあるという事である。勿論、中には女も男もまじっているらしい。そうして、その死骸は皆、それが、かつて、生きていた人間だと云う事実さえ疑われるほど、土を\kana{捏}{こ}ねて造った人形のように、口を\kana{開}{あ}いたり手を延ばしたりして、ごろごろ床の上にころがっていた。しかも、肩とか胸とかの高くなっている部分に、ぼんやりした火の光をうけて、低くなっている部分の影を一層暗くしながら、永久に\kana{唖}{おし}の如く黙っていた。\par{}
 \kana{下人}{げにん}は、それらの死骸の\kana{腐爛}{ふらん}した臭気に思わず、鼻を\kana{掩}{おお}った。しかし、その手は、次の瞬間には、もう鼻を掩う事を忘れていた。ある強い感情が、ほとんどことごとくこの男の嗅覚を奪ってしまったからだ。\par{}
 下人の眼は、その時、はじめてその死骸の中に\kana{蹲}{うずくま}っている人間を見た。\kana{檜皮色}{ひわだいろ}の着物を着た、背の低い、\kana{痩}{や}せた、\kana{白髪頭}{しらがあたま}の、猿のような老婆である。その老婆は、右の手に火をともした松の\kana{木片}{きぎれ}を持って、その死骸の一つの顔を覗きこむように眺めていた。髪の毛の長い所を見ると、多分女の死骸であろう。\par{}
 下人は、六分の恐怖と四分の好奇心とに動かされて、\kana{暫時}{ざんじ}は\kana{呼吸}{いき}をするのさえ忘れていた。旧記の記者の語を借りれば、「\kana{頭身}{とうしん}の毛も太る」ように感じたのである。すると老婆は、松の木片を、床板の間に挿して、それから、今まで眺めていた死骸の首に両手をかけると、丁度、猿の親が猿の子の\kana{虱}{しらみ}をとるように、その長い髪の毛を一本ずつ抜きはじめた。髪は手に従って抜けるらしい。\par{}
 その髪の毛が、一本ずつ抜けるのに従って、下人の心からは、恐怖が少しずつ消えて行った。そうして、それと同時に、この老婆に対するはげしい憎悪が、少しずつ動いて来た。――いや、この老婆に対すると云っては、\kana{語弊}{ごへい}があるかも知れない。むしろ、あらゆる悪に対する反感が、一分毎に強さを増して来たのである。この時、誰かがこの下人に、さっき門の下でこの男が考えていた、\kana{饑死}{うえじに}をするか\kana{盗人}{ぬすびと}になるかと云う問題を、改めて持出したら、恐らく下人は、何の未練もなく、饑死を選んだ事であろう。それほど、この男の悪を憎む心は、老婆の床に挿した松の\kana{木片}{きぎれ}のように、勢いよく燃え上り出していたのである。\par{}
 下人には、勿論、何故老婆が死人の髪の毛を抜くかわからなかった。従って、合理的には、それを善悪のいずれに片づけてよいか知らなかった。しかし下人にとっては、この雨の夜に、この羅生門の上で、死人の髪の毛を抜くと云う事が、それだけで既に許すべからざる悪であった。勿論、下人は、さっきまで自分が、盗人になる気でいた事なぞは、とうに忘れていたのである。\par{}
 そこで、下人は、両足に力を入れて、いきなり、梯子から上へ飛び上った。そうして\kana{聖柄}{ひじりづか}の太刀に手をかけながら、大股に老婆の前へ歩みよった。老婆が驚いたのは云うまでもない。\par{}
 老婆は、一目下人を見ると、まるで\kana{弩}{いしゆみ}にでも\kana{弾}{はじ}かれたように、飛び上った。\par{}
「おのれ、どこへ行く。」\par{}
 下人は、老婆が死骸につまずきながら、慌てふためいて逃げようとする行手を\kana{塞}{ふさ}いで、こう\kana{罵}{ののし}った。老婆は、それでも下人をつきのけて行こうとする。下人はまた、それを行かすまいとして、押しもどす。二人は死骸の中で、しばらく、無言のまま、つかみ合った。しかし勝敗は、はじめからわかっている。下人はとうとう、老婆の腕をつかんで、無理にそこへ\kana{}{ね}じ倒した。丁度、\kana{鶏}{にわとり}の脚のような、骨と皮ばかりの腕である。\par{}
「何をしていた。云え。云わぬと、これだぞよ。」\par{}
 下人は、老婆をつき放すと、いきなり、太刀の\kana{鞘}{さや}を払って、白い\kana{鋼}{はがね}の色をその眼の前へつきつけた。けれども、老婆は黙っている。両手をわなわなふるわせて、肩で息を切りながら、眼を、\kana{眼球}{めだま}が\kana{}{まぶた}の外へ出そうになるほど、見開いて、唖のように\kana{執拗}{しゅうね}く黙っている。これを見ると、下人は始めて明白にこの老婆の生死が、全然、自分の意志に支配されていると云う事を意識した。そうしてこの意識は、今までけわしく燃えていた憎悪の心を、いつの間にか冷ましてしまった。\kana{後}{あと}に残ったのは、ただ、ある仕事をして、それが円満に成就した時の、安らかな得意と満足とがあるばかりである。そこで、下人は、老婆を見下しながら、少し声を柔らげてこう云った。\par{}
「\kana{己}{おれ}は\kana{検非違使}{けびいし}の庁の役人などではない。今し方この門の下を通りかかった旅の者だ。だからお前に\kana{縄}{なわ}をかけて、どうしようと云うような事はない。ただ、今時分この門の上で、何をして居たのだか、それを己に話しさえすればいいのだ。」\par{}
 すると、老婆は、見開いていた眼を、一層大きくして、じっとその下人の顔を見守った。\kana{}{まぶた}の赤くなった、肉食鳥のような、鋭い眼で見たのである。それから、皺で、ほとんど、鼻と一つになった唇を、何か物でも噛んでいるように動かした。細い喉で、尖った\kana{喉仏}{のどぼとけ}の動いているのが見える。その時、その喉から、\kana{鴉}{からす}の啼くような声が、\kana{喘}{あえ}ぎ喘ぎ、下人の耳へ伝わって来た。\par{}
「この髪を抜いてな、この髪を抜いてな、\kana{鬘}{かずら}にしようと思うたのじゃ。」\par{}
 下人は、老婆の答が存外、平凡なのに失望した。そうして失望すると同時に、また前の憎悪が、冷やかな\kana{侮蔑}{ぶべつ}と一しょに、心の中へはいって来た。すると、その\kana{気色}{けしき}が、先方へも通じたのであろう。老婆は、片手に、まだ死骸の頭から奪った長い抜け毛を持ったなり、\kana{蟇}{ひき}のつぶやくような声で、口ごもりながら、こんな事を云った。\par{}
「成程な、\kana{死人}{しびと}の髪の毛を抜くと云う事は、何ぼう悪い事かも知れぬ。じゃが、ここにいる死人どもは、皆、そのくらいな事を、されてもいい人間ばかりだぞよ。現在、わしが今、髪を抜いた女などはな、蛇を\kana{四寸}{しすん}ばかりずつに切って干したのを、\kana{干魚}{ほしうお}だと云うて、\kana{太刀帯}{たてわき}の陣へ売りに\kana{往}{い}んだわ。\kana{疫病}{えやみ}にかかって死ななんだら、今でも売りに往んでいた事であろ。それもよ、この女の売る干魚は、味がよいと云うて、太刀帯どもが、欠かさず\kana{菜料}{さいりよう}に買っていたそうな。わしは、この女のした事が悪いとは思うていぬ。せねば、饑死をするのじゃて、仕方がなくした事であろ。されば、今また、わしのしていた事も悪い事とは思わぬぞよ。これとてもやはりせねば、饑死をするじゃて、仕方がなくする事じゃわいの。じゃて、その仕方がない事を、よく知っていたこの女は、大方わしのする事も大目に見てくれるであろ。」\par{}
 老婆は、大体こんな意味の事を云った。\par{}
 下人は、太刀を\kana{鞘}{さや}におさめて、その太刀の\kana{柄}{つか}を左の手でおさえながら、冷然として、この話を聞いていた。勿論、右の手では、赤く頬に膿を持った大きな\kana{面皰}{にきび}を気にしながら、聞いているのである。しかし、これを聞いている中に、下人の心には、ある勇気が生まれて来た。それは、さっき門の下で、この男には欠けていた勇気である。そうして、またさっきこの門の上へ上って、この老婆を捕えた時の勇気とは、全然、反対な方向に動こうとする勇気である。下人は、饑死をするか盗人になるかに、迷わなかったばかりではない。その時のこの男の心もちから云えば、饑死などと云う事は、ほとんど、考える事さえ出来ないほど、意識の外に追い出されていた。\par{}
「きっと、そうか。」\par{}
 老婆の話が\kana{完}{おわ}ると、下人は\kana{嘲}{あざけ}るような声で念を押した。そうして、一足前へ出ると、不意に右の手を\kana{面皰}{にきび}から離して、老婆の\kana{襟上}{えりがみ}をつかみながら、噛みつくようにこう云った。\par{}
「では、\kana{己}{おれ}が\kana{引剥}{ひはぎ}をしようと恨むまいな。己もそうしなければ、饑死をする体なのだ。」\par{}
 下人は、すばやく、老婆の着物を剥ぎとった。それから、足にしがみつこうとする老婆を、手荒く死骸の上へ蹴倒した。梯子の口までは、僅に五歩を数えるばかりである。下人は、剥ぎとった\kana{檜皮色}{ひわだいろ}の着物をわきにかかえて、またたく間に急な梯子を夜の底へかけ下りた。\par{}
 しばらく、死んだように倒れていた老婆が、死骸の中から、その裸の体を起したのは、それから間もなくの事である。老婆はつぶやくような、うめくような声を立てながら、まだ燃えている火の光をたよりに、梯子の口まで、這って行った。そうして、そこから、短い\kana{白髪}{しらが}を\kana{倒}{さかさま}にして、門の下を覗きこんだ。外には、ただ、\kana{黒洞々}{こくとうとう}たる夜があるばかりである。\par{}
 下人の\kana{行方}{ゆくえ}は、誰も知らない。\par{}
(大正四年九月)
\par{}
\par{}
\par{}



\par{}
底本:「芥川龍之介全集1」ちくま文庫、筑摩書房\par{}
   1986(昭和61)年9月24日第1刷発行\par{}
   1997(平成9)年4月15日第14刷発行\par{}
底本の親本:「筑摩全集類聚版芥川龍之介全集」筑摩書房\par{}
   1971(昭和46)年3月~1971(昭和46)年11月\par{}
入力:平山誠、野口英司\par{}
校正:もりみつじゅんじ\par{}
1997年10月29日公開\par{}
2010年11月4日修正\par{}
青空文庫作成ファイル:\par{}
このファイルは、インターネットの図書館、青空文庫(http://www.aozora.gr.jp/)で作られました。入力、校正、制作にあたったのは、ボランティアの皆さんです。\par{}
\par{}
\par{}



\par{}
●表記について\par{}

	このファイルは W3C 勧告 XHTML1.1 にそった形式で作成されています。
	「くの字点」をのぞくJIS X 0213にある文字は、画像化して埋め込みました。




\par{}
●図書カード



\end{document}
